\shortdescription{In this activity we explore the areas of lunes and triangles on the sphere.}
\activitytitle{Areas on spheres in euclidean $3$-space}


\subsection*{Lunes}

In the picture we have shaded in an ` $\alpha$-lune' on the $R$-sphere in
euclidean $3$-space.%
\[%
%TCIMACRO{\FRAME{itbpF}{1.2505in}{1.19in}{0in}{}{}{Figure}%
%{\special{ language "Scientific Word";  type "GRAPHIC";
%maintain-aspect-ratio TRUE;  display "USEDEF";  valid_file "T";
%width 1.2505in;  height 1.19in;  depth 0in;  original-width 5.4509in;
%original-height 5.1863in;  cropleft "0";  croptop "1";  cropright "1";
%cropbottom "0";  tempfilename 'MXAJBZ0I.png';tempfile-properties "XPR";}}}%
%BeginExpansion
{\includegraphics[
natheight=5.186300in,
natwidth=5.450900in,
height=1.19in,
width=1.2505in
]%
{MXAJBZ0I.png}%
}%
%EndExpansion
\]


The lune has two vertices. They are at opposite (antipodal) points on the
$R$-sphere, that is, the line in Eucludean $3$-space that joins the two
vertices runs through the center of the sphere. The angle at a vertex of the
lune is $\alpha$ radians.

\begin{exercise}
\label{67}(\textbf{SG}) Explain why the area of the $\alpha$-lune is $2\alpha
$\textperiodcentered$R^{2}$.
\end{exercise}


\subsection*{Spherical triangles}

If a triangle on the sphere of radius $R$ has interior angles with radian
measures $\alpha$, $\beta$, and $\gamma$, it can be covered three times by
lunes as shown in the figure below.%
\[%
%TCIMACRO{\FRAME{itbpF}{1.1865in}{1.4425in}{0pt}{}{}{Figure}%
%{\special{ language "Scientific Word";  type "GRAPHIC";
%maintain-aspect-ratio TRUE;  display "USEDEF";  valid_file "T";
%width 1.1865in;  height 1.4425in;  depth 0pt;  original-width 5.0548in;
%original-height 6.154in;  cropleft "0";  croptop "1";  cropright "1";
%cropbottom "0";  tempfilename 'MXAJBZ0J.png';tempfile-properties "XPR";}}}%
%BeginExpansion
\raisebox{-0pt}{\includegraphics[
natheight=6.154000in,
natwidth=5.054800in,
height=1.4425in,
width=1.1865in
]%
{MXAJBZ0J.png}%
}%
%EndExpansion
\]
Notice that each lune has one vertex at a vertex of the triangle and angle
equal to that interior angle of the triangle. The other vertices of each lune
are vertices of an `opposite' triangle that has the same area as the given one
since it is just the image of the given one under the rigid motion%
\[
\left(  \underline{\hat{x}},\underline{\hat{y}},\underline{\hat{z}}\right)
=\left(  \hat{x},\hat{y},\hat{z}\right)  \cdot\left(
\begin{array}
[c]{ccc}%
-1 & 0 & 0\\
0 & -1 & 0\\
0 & 0 & -1
\end{array}
\right)  .
\]
(See, for example, the formula $\left(  \ref{68}\right)  $.) The three lunes
cover the triangle three times. The three opposite lunes cover the opposite
triangle three times. If you take all six lunes together, they cover each of
the two triangles three times and everything else exactly once.

\begin{exercise}
(\textbf{SG}) a) Show that the area of the spherical triangle is given by the
formula%
\[
R^{2}\left(  \left(  \alpha+\beta+\gamma\right)  -\pi\right)  ,
\]
that is,%
\[
\left\vert K\right\vert ^{-1}\left(  \left(  \alpha+\beta+\gamma\right)
-\pi\right)  .
\]


Hint: Use Exercise \ref{67} to turn the sentence just preceding the Exercise
into an equation.

b) Give a formula for the area of any spherical $n$-gon.

Hint: Divide the spherical $n$-gon into spherical triangles.
\end{exercise}
