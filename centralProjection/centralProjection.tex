\documentclass{ximera}

\usepackage{microtype}
\usepackage{tikz}
\usepackage{tkz-euclide}
\usetkzobj{all}
\tikzstyle geometryDiagrams=[ultra thick,color=blue!50!black]

\renewcommand{\epsilon}{\varepsilon}



\title{Central projection}
\begin{document}
\begin{abstract}
Here we start to develop a model for our geometry.
\end{abstract}
\maketitle



\subsection*{Central projection coordinates}

Let's project $K$-geometry, that is, the set
\begin{equation}
1=K\left(  x^{2}+y^{2}\right)  +z^{2} \label{30}%
\end{equation}
onto the set%
\[
z=1
\]
using the origin%
\[
O=\left(  0,0,0\right)
\]
as the center of projection:%

\begin{tabular}
[c]{cc}%
\includegraphics[
natheight=2.648100in,
natwidth=5.274500in,
height=1.3543in,
width=2.6852in
]%
{MXAJBZ0K.jpg}%
 &
\includegraphics[
natheight=3.472200in,
natwidth=5.736300in,
height=1.6345in,
width=2.6913in
]%
{MXAJBZ0L.jpg}%
\end{tabular}


That is,%
\begin{equation}
r\text{\textperiodcentered}\left(  x_{c},y_{c},1\right)  =\left(
x,y,z\right)  . \label{131}%
\end{equation}
So%
\[
r=z
\]
and, from the equation $\left(  \ref{30}\right)  $%
\begin{gather*}
K\left(  \left(  rx_{c}\right)  ^{2}+\left(  ry_{c}\right)  ^{2}\right)
+r^{2}=1\\
r^{2}=\frac{1}{K\left(  x_{c}^{2}+y_{c}^{2}\right)  +1}.
\end{gather*}
Notice that, when $K<0$ this last formula only makes sense when%
\begin{gather}
K\left(  x_{c}^{2}+y_{c}^{2}\right)  >-1\label{75}\\
x_{c}^{2}+y_{c}^{2}<\frac{-1}{K}.\nonumber
\end{gather}


\begin{exercise}\label{31}\hfil
\begin{enumerate}
\item  For the projection of the set $\left(  \ref{30}\right)  $ onto
the $z=1$ plane with center of projection $O$, write $\left(  x_{c}%
,y_{c}\right)  $ as a function of $\left(  x,y,z\right)  $.

\item For the projection of the set $\left(  \ref{30}\right)  $ onto the $z=1$
plane with center of projection $O$, write $\left(  x,y,z\right)  $ as a
function of $\left(  x_{c},y_{c}\right)  $.
\end{enumerate}
\end{exercise}

\subsection*{Rigid motion in central projection coordinates}

Suppose now we have a $K$-rigid motion%
\[
\left(  \underline{x},\underline{y},\underline{z}\right)  =\left(
x,y,z\right)  \cdot M
\]
of $K$-geometry, given by a $K$-orthogonal matrix%
\[
M=\left(
\begin{array}
[c]{ccc}%
m_{11} & m_{12} & m_{13}\\
m_{21} & m_{22} & m_{23}\\
m_{31} & m_{32} & m_{33}%
\end{array}
\right)  .
\]
To see what this $K$-rigid motion looks like in central projection coordinates
we simply do the matrix multiplication%
\begin{gather*}
\left(  \underline{x},\underline{y},\underline{z}\right)  =\left(
x,y,z\right)  \cdot\left(
\begin{array}
[c]{ccc}%
m_{11} & m_{12} & m_{13}\\
m_{21} & m_{22} & m_{23}\\
m_{31} & m_{32} & m_{33}%
\end{array}
\right) \\
=\left(  \left(  m_{11}x+m_{21}y+m_{31}z\right)  ,\left(  m_{12}%
x+m_{22}y+m_{32}z\right)  ,\left(  m_{13}x+m_{23}y+m_{33}z\right)  \right)  .
\end{gather*}
Then%
\begin{align}
\underline{x_{c}}  &  =\frac{\underline{x}}{\underline{z}}\label{77}\\
&  =\frac{m_{11}x+m_{21}y+m_{31}z}{m_{13}x+m_{23}y+m_{33}z}\nonumber\\
&  =\frac{m_{11}\left(  x/z\right)  +m_{21}\left(  y/z\right)  +m_{31}}%
{m_{13}\left(  x/z\right)  +m_{23}\left(  y/z\right)  +m_{33}}\nonumber\\
&  =\frac{m_{11}x_{c}+m_{21}y_{c}+m_{31}}{m_{13}x_{c}+m_{23}y_{c}+m_{33}%
}\nonumber
\end{align}
and similarly%
\[
\underline{y_{c}}=\frac{m_{12}x_{c}+m_{22}y_{c}+m_{32}}{m_{13}x_{c}%
+m_{23}y_{c}+m_{33}}.
\]
So we write%
\begin{gather*}
\left(  \underline{x_{c}},\underline{y_{c}}\right)  =M_{c}\left(  x_{c}%
,y_{c}\right) \\
=\left(  \frac{m_{11}x_{c}+m_{21}y_{c}+m_{31}}{m_{13}x_{c}+m_{23}y_{c}+m_{33}%
},\frac{m_{12}x_{c}+m_{22}y_{c}+m_{32}}{m_{13}x_{c}+m_{23}y_{c}+m_{33}%
}\right)  .
\end{gather*}




\subsection*{Length and angle in central projection coordinates}

\begin{exercise}
\label{33}For the $K$-geometry coordinates%
\[
X=\left(  x,y,z\right)
\]
use the formulas you derived in Exercise \ref{31}b) to calculate%
\[
dX=\left(  \frac{\partial X}{\partial x_{c}}\right)  dx_{c}+\left(
\frac{\partial X}{\partial y_{c}}\right)  dy_{c}%
\]
That is, calculate the $2\times3$ matrix%
\[
D_{c}=\left(
\begin{array}
[c]{ccc}%
\frac{\partial x}{\partial x_{c}} & \frac{\partial y}{\partial x_{c}} &
\frac{\partial z}{\partial x_{c}}\\
\frac{\partial x}{\partial y_{c}} & \frac{\partial y}{\partial y_{c}} &
\frac{\partial z}{\partial y_{c}}%
\end{array}
\right)  =\left(
\begin{array}
[c]{c}%
\left(  \frac{\partial X}{\partial x_{c}}\right) \\
\left(  \frac{\partial X}{\partial y_{c}}\right)
\end{array}
\right)  .
\]
Hint: Use logarithmic differentiation:%
\begin{align*}
dx  &  =d\left(  rx_{c}\right)  =x_{c}dr+rdx_{c}\\
r^{-1}dx  &  =x_{c}d\ln\left(  r\right)  +dx_{c}%
\end{align*}
and similarly for $y$ and $z$ since it is easier to compute $r^{-1}\left(
\frac{dx}{dt},\frac{dy}{dt},\frac{dz}{dt}\right)  $ than $\left(  \frac
{dx}{dt},\frac{dy}{dt},\frac{dz}{dt}\right)  $. Next use that%
\begin{align*}
2d\ln\left(  r\right)   &  =d\ln\left(  r^{2}\right)
=-d\ln\left(  K\left(  x_{c}^{2}+y_{c}^{2}\right)  +1\right) \\
&  =-\frac{1}{K\left(  x_{c}^{2}+y_{c}^{2}\right)  +1}d\left(  K\left(
x_{c}^{2}+y_{c}^{2}\right)  +1\right) \\
&  =-r^{2}K\left(  2x_{c}dx_{c}+2y_{c}dy_{c}\right)  .
\end{align*}

\end{exercise}

\begin{exercise}
\label{prev}Now suppose we have a path,%
\[
\left(  x_{c}\left(  t\right)  ,y_{c}\left(  t\right)  \right)  ,\;a\leq t\leq
b
\]
in the $\left(  x_{c},y_{c}\right)  $-plane, that is, in the central
projection plane%
\[
\left(  x_{c},y_{c},1\right)  .
\]
Use the formula you derived in Exercise \ref{31}b) to write the corresponding
path%
\[
x\left(  x_{c}\left(  t\right)  ,y_{c}\left(  t\right)  \right)  ,y\left(
x_{c}\left(  t\right)  ,y_{c}\left(  t\right)  \right)  ,z\left(  x_{c}\left(
t\right)  ,y_{c}\left(  t\right)  \right)
\]
in the $K$-geometry space of $\left(  x,y,z\right)  $ such that $K\left(
x^{2}+y^{2}\right)  +z^{2}=1$.
\end{exercise}

\begin{exercise}
For the path $\left(  x\left(  t\right)  ,y\left(  t\right)  ,z\left(
t\right)  \right)  $ in Exercise \ref{prev} lying on the set $\left(
\ref{30}\right)  $, use the Chain Rule from calculus of several variables to
compute%
\[
\left(  \frac{dx}{dt},\frac{dy}{dt},\frac{dz}{dt}\right)  =\left(
\frac{dx_{c}\left(  t\right)  }{dt},\frac{dy_{c}\left(  t\right)  }%
{dt}\right)  \cdot D_{c}.
\]

\end{exercise}

This last Exercise allows us to do something very nice. Namely now, not only
can we use the coordinates $\left(  x_{c},y_{c}\right)  $ for our geometry but
we can also compute the $K$-dot product in terms of these coordinates. By the
Chain Rule from calculus of several variables%
\[
\left(  \frac{dx}{dt},\frac{dy}{dt},\frac{dz}{dt}\right)  =\left(
\frac{dx_{c}}{dt},\frac{dy_{c}}{dt}\right)  \cdot D_{c}.
\]
So%
\begin{align*}
\left(  \frac{dx}{dt},\frac{dy}{dt},\frac{dz}{dt}\right)  \bullet_{K}\left(
\frac{dx}{dt},\frac{dy}{dt},\frac{dz}{dt}\right)   &  =\left(
\begin{array}
[c]{ccc}%
\frac{dx}{dt} & \frac{dy}{dt} & \frac{dz}{dt}%
\end{array}
\right)  \left(
\begin{array}
[c]{ccc}%
1 & 0 & 0\\
0 & 1 & 0\\
0 & 0 & K^{-1}%
\end{array}
\right)  \left(
\begin{array}
[c]{c}%
\frac{dx}{dt}\\
\frac{dy}{dt}\\
\frac{dz}{dt}%
\end{array}
\right) \\
&  =\left(
\begin{array}
[c]{cc}%
\frac{dx_{c}}{dt} & \frac{dy_{c}}{dt}%
\end{array}
\right)  \cdot D_{c}\cdot\left(
\begin{array}
[c]{ccc}%
1 & 0 & 0\\
0 & 1 & 0\\
0 & 0 & K^{-1}%
\end{array}
\right)  \cdot D_{c}^{t}\cdot\left(
\begin{array}
[c]{c}%
\frac{dx_{c}}{dt}\\
\frac{dy_{c}}{dt}%
\end{array}
\right)  ,
\end{align*}


\begin{exercise}
\label{32}Compute the $2\times2$ matrix%
\[
P_{c}=D_{c}\cdot\left(
\begin{array}
[c]{ccc}%
1 & 0 & 0\\
0 & 1 & 0\\
0 & 0 & K^{-1}%
\end{array}
\right)  \cdot D_{c}^{t},
\]
that that gives the $K$-dot product in $\left(  x_{c},y_{c}\right)
$-coordinates. That is, use matrix multiplication to show that%
\[
P_{c}=\left(
\begin{array}
[c]{cc}%
r^{2}\left(  1-r^{2}Kx_{c}^{2}\right)  & -r^{4}Kx_{c}y_{c}\\
-r^{4}Kx_{c}y_{c} & r^{2}\left(  1-r^{2}Ky_{c}^{2}\right)
\end{array}
\right)  .
\]


Hint: For example%
\begin{align*}
\frac{\partial x}{\partial x_{c}}  &  =r\left(  x_{c}\frac{\partial \ln\left(
r\right)  }{\partial x_{c}}+1\right)  =-r^{3}Kx_{c}^{2}+r\\
\frac{\partial y}{\partial x_{c}}  &  =r\left(  y_{c}\frac{\partial \ln\left(
r\right)  }{\partial x_{c}}\right)  =-r^{3}Kx_{c}y_{c}\\
\frac{\partial z}{\partial x_{c}}  &  =r\left(  \frac{\partial \ln\left(
r\right)  }{\partial x_{c}}\right)  =-r^{3}Kx_{c}%
\end{align*}
so that%
\begin{align*}
&  \left(  \frac{\partial x}{\partial x_{c}},\frac{\partial y}{\partial x_{c}%
},\frac{\partial z}{\partial x_{c}}\right)  \bullet_{K}\left(  \frac{\partial
x}{\partial x_{c}},\frac{\partial y}{\partial x_{c}},\frac{\partial
z}{\partial x_{c}}\right) \\
&  =r^{6}K^{2}x_{c}^{4}-2r^{4}Kx_{c}^{2}+r^{2}+r^{6}K^{2}x_{c}^{2}y_{c}%
^{2}+r^{6}Kx_{c}^{2}\\
&  =\left(  r^{6}K^{2}x_{c}^{4}+r^{6}K^{2}x_{c}^{2}y_{c}^{2}+r^{6}Kx_{c}%
^{2}\right)  -2r^{4}Kx_{c}^{2}+r^{2}\\
&  =r^{4}Kx_{c}^{2}-2r^{4}Kx_{c}^{2}+r^{2}=r^{2}\left(  1-r^{2}Kx_{c}%
^{2}\right)  .
\end{align*}

\end{exercise}

So, if, if $K>0$ and you have a path on the sphere of radius $R=K^{-1/2}$ in
euclidean $3$-space given in $\left(  x_{c},y_{c}\right)  $-coordinates as
$\left(  x_{c}\left(  t\right)  ,y_{c}\left(  t\right)  \right)  $ for
$t\in\left[  b,e\right]  $, you can trace back everything we have done with
coordinate changes to see that the length of the path on the sphere of radius
$R=K^{-1/2}$ in euclidean $3$-space is given by%
\[%
%TCIMACRO{\dint \nolimits_{b}^{e}}%
%BeginExpansion
{\displaystyle\int\nolimits_{b}^{e}}
%EndExpansion
l\left(  t\right)  dt
\]
where%
\begin{align*}
l\left(  t\right)  ^{2}  &  =\left(  \frac{dx_{c}}{dt},\frac{dy_{c}}%
{dt}\right)  \bullet_{c}\left(  \frac{dx_{c}}{dt},\frac{dy_{c}}{dt}\right) \\
&  =\left(
\begin{array}
[c]{cc}%
\frac{dx_{c}}{dt} & \frac{dy_{c}}{dt}%
\end{array}
\right)  \cdot P_{c}\cdot\left(
\begin{array}
[c]{c}%
\frac{dx_{c}}{dt}\\
\frac{dy_{c}}{dt}%
\end{array}
\right)  .
\end{align*}
Notice that the matrix $P_{c}$ still makes sense when $K=0$ and when $K$
becomes negative. So we do have%
\[{\renewcommand{\arraystretch}{1.6}
\frame{%
\begin{tabular}
[c]{c|c|c}%
\textit{Spherical} ($K>0$) & \textit{Euclidean} ($K=0$) & \textit{Hyperbolic}
($K<0$)\\\hline
$\hat{x}^{2}+\hat{y}^{2}+\hat{z}^{2}=R^{2}$ &  & \\ \hline
$\hat{V}\bullet\hat{V}$ &  & \\ \hline
$1=K\left(  x^{2}+y^{2}\right)  +z^{2}$ & $1=K\left(  x^{2}+y^{2}\right)
+z^{2}$ & $1=K\left(  x^{2}+y^{2}\right)  +z^{2}$\\ \hline
$V_{1}\bullet_{K}V_{2}$ &  & $V_{1}\bullet_{K}V_{2}$\\ \hline
$V_{1}^{c}\bullet_{c}V_{2}^{c}$ & $V_{1}^{c}\bullet_{c}V_{2}^{c}$ & $V_{1}%
^{c}\bullet_{c}V_{2}^{c}$%
\end{tabular}
}}
\]
where%
\[
V_{1}^{c}\bullet_{c}V_{2}^{c}=\left(  V_{1}^{c}\right)  \cdot P_{c}%
\cdot\left(  V_{2}^{c}\right)  ^{t}.
\]
Of course if $K>0$, we again have euclidean angles $\theta$ between vectors
$\hat{V}_{1}$ and $\hat{V}_{2}$ tangent to the $R$-sphere at some point
computed by%
\begin{align*}
\hat{V}_{1}\bullet\hat{V}_{2}  &  =\left\vert \hat{V}_{1}\right\vert
\text{\textperiodcentered}\left\vert \hat{V}_{2}\right\vert
\text{\textperiodcentered\textrm{cos}}\left(  \theta\right) \\
&  =V_{1}^{c}\bullet_{c}V_{2}^{c}.
\end{align*}


\subsection*{Area in central projection coordinates}

Suppose you were given a region $G_{c}$ in the $\left(  x_{c},y_{c}\right)
$-coordinate plane. Also suppose that $K>0$. If you trace back everything we
have done with coordinate changes, you can see how $G_{c}$ gives you a region
$\hat{G}$ on the sphere of radius $R=K^{-1/2}$ in euclidean $3$-space via the
formulas%
\begin{align*}
\left(  \hat{x},\hat{y},\hat{z}\right)   &  =\left(  x,y,Rz\right) \\
&  =r\text{\textperiodcentered}\left(  x_{c},y_{c},R\right) \\
&  =\left(  \frac{x_{c}}{\sqrt{K\left(  x_{c}^{2}+y_{c}^{2}\right)  +1}}%
,\frac{y_{c}}{\sqrt{K\left(  x_{c}^{2}+y_{c}^{2}\right)  +1}},\frac{R}%
{\sqrt{K\left(  x_{c}^{2}+y_{c}^{2}\right)  +1}}\right)  .
\end{align*}
Now there is a formula in several variable calculus for computing the area of
the region $\hat{G}$ on the sphere of radius $R$ in euclidean $3$-space in
terms of the parameters $\left(  x_{c},y_{c}\right)  $. [DS,49,231]. It is
\begin{equation}%
%TCIMACRO{\dint \nolimits_{G_{c}}}%
%BeginExpansion
{\displaystyle\int\nolimits_{G_{c}}}
%EndExpansion
\hat{a}\left(  \frac{d\hat{X}}{dx_{c}},\frac{d\hat{X}}{dy_{c}}\right)
dx_{c}dy_{c} \label{68}%
\end{equation}
where $\hat{a}\left(  \frac{d\hat{X}}{dx_{c}},\frac{d\hat{X}}{dy_{c}}\right)
$ is the (euclidean) area of the parallelogram spanned by the two vectors
$\frac{d\hat{X}}{dx_{c}}$ and $\frac{d\hat{X}}{dy_{c}}$ in euclidean
$3$-space. Thus%
\[
\hat{a}\left(  \frac{d\hat{X}}{dx_{c}},\frac{d\hat{X}}{dy_{c}}\right)
=\left\vert \frac{d\hat{X}}{dx_{c}}\right\vert \text{\textperiodcentered
}\left\vert \frac{d\hat{X}}{dy_{c}}\right\vert \text{\textperiodcentered
}\mathrm{sin}\left(  \theta\right)
\]
where $\theta$ is the angle between the two vectors.

\begin{exercise}
Using Exercise 7.\ref{9} and Exercise 8.\ref{222} show that%
\begin{align*}
\hat{a}\left(  \frac{d\hat{X}}{dx_{c}},\frac{d\hat{X}}{dy_{c}}\right)  ^{2}
&  =\left\vert
\begin{array}
[c]{cc}%
\frac{d\hat{X}}{dx_{c}}\bullet\frac{d\hat{X}}{dx_{c}} & \frac{d\hat{X}}%
{dy_{c}}\bullet\frac{d\hat{X}}{dx_{c}}\\
\frac{d\hat{X}}{dx_{c}}\bullet\frac{d\hat{X}}{dy_{c}} & \frac{d\hat{X}}%
{dy_{c}}\bullet\frac{d\hat{X}}{dy_{c}}%
\end{array}
\right\vert \\
&  =\left\vert
\begin{array}
[c]{cc}%
\frac{dX}{dx_{c}}\bullet_{K}\frac{dX}{dx_{c}} & \frac{dX}{dy_{c}}\bullet
_{K}\frac{dX}{dx_{c}}\\
\frac{dX}{dx_{c}}\bullet_{K}\frac{dX}{dy_{c}} & \frac{dX}{dy_{c}}\bullet
_{K}\frac{dX}{dy_{c}}%
\end{array}
\right\vert \\
&  =\left\vert \left(
\begin{array}
[c]{c}%
\left(  \frac{dX}{dx_{c}}\right) \\
\left(  \frac{dX}{dy_{c}}\right)
\end{array}
\right)  \left(
\begin{array}
[c]{ccc}%
1 & 0 & 0\\
0 & 1 & 0\\
0 & 0 & K^{-1}%
\end{array}
\right)  \left(
\begin{array}
[c]{cc}%
\left(  \frac{dX}{dx_{c}}\right)  ^{t} & \left(  \frac{dX}{dy_{c}}\right)
^{t}%
\end{array}
\right)  \right\vert \\
&  =\left\vert P_{c}\right\vert .
\end{align*}

\end{exercise}

\begin{exercise}
\label{79}Use Exercise \ref{32} to show that%
\[
\hat{a}\left(  \frac{d\hat{X}}{dx_{c}},\frac{d\hat{X}}{dy_{c}}\right)
^{2}=r^{6}=\frac{1}{\left(  K\left(  x_{c}^{2}+y_{c}^{2}\right)  +1\right)
^{3}}%
\]
as a function of $\left(  x_{c},y_{c}\right)  $.

Hint: Notice that the matrix $D_{c}$ in Exercise \ref{33} is simply the
$2\times3$ matrix whose rows are the vectors $\frac{dX}{dx_{c}}$ and $\frac
{dX}{dy_{c}}$. So referring to Exercise \ref{32}, we know that%
\[
\left(
\begin{array}
[c]{cc}%
\frac{d\hat{X}}{dx_{c}}\bullet\frac{d\hat{X}}{dx_{c}} & \frac{d\hat{X}}%
{dy_{c}}\bullet\frac{d\hat{X}}{dx_{c}}\\
\frac{d\hat{X}}{dx_{c}}\bullet\frac{d\hat{X}}{dy_{c}} & \frac{d\hat{X}}%
{dy_{c}}\bullet\frac{d\hat{X}}{dy_{c}}%
\end{array}
\right)  =\left(
\begin{array}
[c]{cc}%
r^{2}\left(  1-r^{2}Kx_{c}^{2}\right)  & -r^{4}Kx_{c}y_{c}\\
-r^{4}Kx_{c}y_{c} & r^{2}\left(  1-r^{2}Ky_{c}^{2}\right)
\end{array}
\right)  .
\]

\end{exercise}

Since all these computations can be extended to $K$-geometry for all $K$, we
define the $K$-area of a region $G_{c}$ in the $\left(  x_{c},y_{c}\right)
$-coordinate plane by first computing the $K$-area of the parallelogram
spanned by $\frac{dX}{dx_{c}}$ and $\frac{dX}{dy_{c}}$ at each point of
$G_{c}$ as%
\begin{align*}
a_{K}\left(  \frac{dX}{dx_{c}},\frac{dY}{dy_{c}}\right)   &  =\left\vert
\frac{dX}{dx_{c}}\right\vert _{K}\cdot\left\vert \frac{dX}{dy_{c}}\right\vert
_{K}\cdot\mathrm{sin}\left(  \theta_{K}\right) \\
&  =\sqrt{\left\vert
\begin{array}
[c]{cc}%
\frac{dX}{dx_{c}}\bullet_{K}\frac{dX}{dx_{c}} & \frac{dX}{dy_{c}}\bullet
_{K}\frac{dX}{dx_{c}}\\
\frac{dX}{dx_{c}}\bullet_{K}\frac{dX}{dy_{c}} & \frac{dX}{dy_{c}}\bullet
_{K}\frac{dX}{dy_{c}}%
\end{array}
\right\vert }%
\end{align*}
and then integrating this area over $G_{c}$ to get%
\begin{align*}
A_{K}\left(  G_{c}\right)   &  =%
%TCIMACRO{\dint \nolimits_{G_{c}}}%
%BeginExpansion
{\displaystyle\int\nolimits_{G_{c}}}
%EndExpansion
a_{K}\left(  \frac{dX}{dx_{c}},\frac{dY}{dy_{c}}\right)  dx_{c}dy_{c}\\
&  =%
%TCIMACRO{\dint \nolimits_{G_{c}}}%
%BeginExpansion
{\displaystyle\int\nolimits_{G_{c}}}
%EndExpansion
\sqrt{\left\vert
\begin{array}
[c]{cc}%
\frac{dX}{dx_{c}}\bullet_{K}\frac{dX}{dx_{c}} & \frac{dX}{dy_{c}}\bullet
_{K}\frac{dX}{dx_{c}}\\
\frac{dX}{dx_{c}}\bullet_{K}\frac{dX}{dy_{c}} & \frac{dX}{dy_{c}}\bullet
_{K}\frac{dX}{dy_{c}}%
\end{array}
\right\vert }dx_{c}dy_{c}.
\end{align*}

\end{document}
