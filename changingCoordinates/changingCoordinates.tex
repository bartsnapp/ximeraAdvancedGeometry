\documentclass{ximera}

\usepackage{microtype}
\usepackage{tikz}
\usepackage{tkz-euclide}
\usetkzobj{all}
\tikzstyle geometryDiagrams=[ultra thick,color=blue!50!black]

\renewcommand{\epsilon}{\varepsilon}



\title{Changing coordinates}
\begin{document}
\begin{abstract}
Now we will change coordinates.
\end{abstract}
\maketitle


\subsection*{Bringing the North Pole of the $R$-sphere to $\left(
0,0,1\right)  $}

We are now ready to introduce a slightly different set of coordinates for
$\mathbb{R}^{3}$, three-dimensional euclidean space. To see why we do this,
suppose we are standing at the North Pole%
\[
N=\left(0,0,R\right)
\]
of the sphere%
\[
\hat{x}^{2}+\hat{y}^{2}+\hat{z}^{2}=R^{2} %\label{4}%
\]
of radius $R$. As $R$ increases, but we stay our same size, the sphere
around us becomes more and more like a flat, plane surface. However it
can never get completely flat because we are zooming out the positive
$\hat{z}$-axis and we would have to be `at infinity' for our surface
to become exactly flat. We remedy that unfortunate situation by
considering another copy of $\mathbb{R}^{3}$, an \dfn{embedding
  space}, whose coordinates we denote as $\left( x,y,z\right) $.  We
make the following rule in order to pass between the two
$\mathbb{R}^{3}$'s:%
\begin{align*}
\hat{x}  &  =x\\
\hat{y}  &  =y\\
\hat{z}  &  =Rz.
\end{align*}
We think of the $\left( x,y,z\right) $-coordinates as simply being a
different set of addresses for the points in euclidean space. For
example,
\[
\left(x,y,z\right)  =\left(0,0,1\right)
\]
tells me that the point in euclidean space that I'm referring to is%
\[
\left(\hat{x},\hat{y},\hat{z}\right) =\left( 0,0,R\right)= N.
\]
Continuing with this ``change of addresses'' the sphere of radius $R$
in euclidean space is given by
\begin{align*}
R^{2} & =\hat{x}^{2}+\hat{y}^{2}+\hat{z}^{2}\\ &
=x^{2}+y^{2}+R^{2}z^{2}
\end{align*}
that is, by the equation
\[
1=\frac{1}{R^{2}}\left(  x^{2}+y^{2}\right)  +z^{2}. %\label{5}%
\]
\begin{definition}
  For the surface defined by
  \[
  1=\frac{1}{R^{2}}\left(  x^{2}+y^{2}\right)  +z^{2}. %\label{5%}
  \]
The quantity $K=\frac{1}{R^{2}}$ is called the \dfn{curvature} of the
$R$-sphere.
\end{definition}

\begin{problem}
  What happens to the surface when $K$ goes to $0$? Does this make
  sense in terms of curvature?
\end{problem}

\begin{problem}\hfil
\begin{enumerate}
\item Sketch the solution set in $\left(  x,y,z\right)  $-coordinates
representing the sphere%
\[
R^{2}=\hat{x}^{2}+\hat{y}^{2}+\hat{z}^{2}=1.
\]

\item Sketch the solution set in $\left(  x,y,z\right)  $-coordinates
representing the sphere%
\[
R^{2}=\hat{x}^{2}+\hat{y}^{2}+\hat{z}^{2}=10^{2}.
\]

\item Sketch the solution set in $\left(  x,y,z\right)  $-coordinates
representing the sphere%
\[
R^{2}=\hat{x}^{2}+\hat{y}^{2}+\hat{z}^{2}=10^{-2}.
\]
\end{enumerate}
\end{problem}

\subsection*{Formulas for euclidean lengths and angles in terms
of $\left(  x,y,z\right)  $-coordinates}

To prepare ourselves to do hyperbolic geometry, which (in some sense)
has no satisfactory model in euclidean space, we will `practice' by
doing spherical geometry (which \textit{does} have a completely
satisfactory model in euclidean space) using these `slightly strange'
$\left( x,y,z\right) $-coordinates. Gradually throughout this course
we will discover that the same rules that govern spherical geometry,
expressed in $\left( x,y,z\right) $-coordinates, also govern flat and
hyperbolic geometry! In all three cases, the surface in $\left(
x,y,z\right) $-coordinates that we will study is%
\[
1=K(x^{2}+y^{2})+z^{2}.
\]
If $K>0$, the geometry we will be studying is the geometry of the the
euclidean sphere of radius%
\[
R=\frac{1}{\sqrt{K}}.
\]
If $K=0$ we will be studying flat (plane) geometry. If $K<0$, we will be
studying hyperbolic geometry. 

In short, we want to use $\left(  x,y,z\right)  $-coordinates to compute with,
but we want lengths and angles to be the usual euclidean ones in $\left(
\hat{x},\hat{y},\hat{z}\right)  $-coordinates.

\begin{problem}\hfil
\begin{enumerate}
\item Suppose we have
\[
\hat{X}=\left(  \hat{x}\left(  x,y,z\right)  ,\hat{y}\left(  x,y,z\right)  ,\hat
{z}\left(  x,y,z\right)  \right)
\]
where%
\begin{align*}
x  &  =f\left(  t\right) \\
y  &  =g\left(  t\right) \\
z  &  =h\left(  t\right)  .
\end{align*}
State the chain rule for%
\begin{align*}
\frac{d\hat{x}}{dt}  &  =\\
\frac{d\hat{y}}{dt}  &  =\\
\frac{d\hat{z}}{dt}  &  =
\end{align*}


\item Rewrite the chain rule in matrix notation and find $M$ such that
\[
\begin{bmatrix}
\frac{d\hat{x}}{dt} & \frac{d\hat{y}}{dt} & \frac{d\hat{z}}{dt}%
\end{bmatrix}
=
\begin{bmatrix}
\frac{dx}{dt} & \frac{dy}{dt} & \frac{dz}{dt}%
\end{bmatrix}\cdot M.
\]
\end{enumerate}
\end{problem}

\begin{problem}
  \label{8} Recalling that $R$ is a positive constant, use
\begin{align*}
\hat{x}  &=x\\
\hat{y}  &=y\\
\hat{z}  &=Rz.
\end{align*}
and the chain rule to show that, for any path
$\hat{X}(t)=(\hat{x}(t),\hat{y}(t),\hat{z}(t))$ in euclidean space,%
\begin{align*}
\frac{d\hat{x}}{dt}  &  =\frac{dx}{dt}\\
\frac{d\hat{y}}{dt}  &  =\frac{dy}{dt}\\
\frac{d\hat{z}}{dt}  &  =R\frac{dz}{dt}.
\end{align*}
\end{problem}

\begin{problem}
  Use matrix multiplication %[DS,307]
  and the previous problem to show that%
\begin{align*}
\frac{d\hat{X}(t)}{dt}  &  =\frac{dX(t)}{dt}
\begin{bmatrix}
1 & 0 & 0\\
0 & 1 & 0\\
0 & 0 & R
\end{bmatrix} \\
\frac{dX(t)}{dt}  &=\frac{d\hat{X}(t)
}{dt}
\begin{bmatrix}
1 & 0 & 0\\
0 & 1 & 0\\
0 & 0 & R^{-1}%
\end{bmatrix}.
\end{align*}

\end{problem}

This last computation shows that, if
\begin{align*}
\hat{V}_{1}  &  =\left(  \hat{a}_{1},\hat{b}_{1},\hat{c}_{1}\right) \\
\hat{V}_{2}  &  =\left(  \hat{a}_{2},\hat{b}_{2},\hat{c}_{2}\right)
\end{align*}
are tangent vectors in $\left(  \hat{x},\hat{y},\hat{z}\right)  $-coordinates
and%
\begin{align*}
V_{1}  &  =\left(  a_{1},b_{1},c_{1}\right) \\
V_{2}  &  =\left(  a_{2},b_{2},c_{2}\right)
\end{align*}
are their transformations into $\left(  x,y,z\right)  $-coordinates, then%
\begin{align*}
\hat{V}_{1}  &  =V_{1}
\begin{bmatrix}
1 & 0 & 0\\
0 & 1 & 0\\
0 & 0 & R
\end{bmatrix}
 \\
\hat{V}_{2}  &  =V_{2} 
\begin{bmatrix}
1 & 0 & 0\\
0 & 1 & 0\\
0 & 0 & R
\end{bmatrix}
\end{align*}
and%
\begin{align*}
\hat{V}_{1}\bullet\hat{V}_{2} &= \hat{V}_{1}\cdot \hat{V}_{2}^\transpose\\
&=V_{1}
\begin{bmatrix}
1 & 0 & 0\\
0 & 1 & 0\\
0 & 0 & R
\end{bmatrix}
\left(V_{2}
\begin{bmatrix}
1 & 0 & 0\\
0 & 1 & 0\\
0 & 0 & R
\end{bmatrix}\right)^\transpose\\
&=V_{1}
\begin{bmatrix}
1 & 0 & 0\\
0 & 1 & 0\\
0 & 0 & R
\end{bmatrix}
\begin{bmatrix}
1 & 0 & 0\\
0 & 1 & 0\\
0 & 0 & R
\end{bmatrix}
V_{2}^\transpose\\
&=V_{1}
\begin{bmatrix}
1 & 0 & 0\\
0 & 1 & 0\\
0 & 0 & K^{-1}%
\end{bmatrix}
V_{2}^\transpose.
\end{align*}
This last computation says that we can compute the euclidean dot $\hat{V}%
_{1}\bullet\hat{V}_{2}$ without ever referring to euclidean coordinates. We
incorporate that fact into the following definition.

\begin{definition}
The \textbf{$\boldsymbol{K}$-dot-product} of vectors:%
\begin{align*}
V_{1}\bullet_{K}V_{2}  &=V_{1} 
\begin{bmatrix}
1 & 0 & 0\\
0 & 1 & 0\\
0 & 0 & K^{-1}%
\end{bmatrix}
V_{2}^\transpose\\
&=
\begin{bmatrix}
a_{1} & b_{1} & c_{1}%
\end{bmatrix}
\begin{bmatrix}
1 & 0 & 0\\
0 & 1 & 0\\
0 & 0 & K^{-1}%
\end{bmatrix}
\begin{bmatrix}
a_{2}\\
b_{2}\\
c_{2}%
\end{bmatrix}.
\end{align*}

\end{definition}

So, suppose we have a curve on the $R$-sphere in euclidean space
but it is given to us in $X(t)=(x(t),y(t),z(t))$-coordinates. Then the
length of that curve in euclidean space is
\[
\int_{b}^{e}\sqrt{\frac{dX}{dt}\bullet_{K}\frac{dX}{dt}}\,dt.
\]


\begin{problem}
Show that, if we have any two vectors in euclidean three-space that
are tangent to the $R$-sphere at some point on it, but the two vectors
are given to us in $(x,y,z)$-coordinates as%
\begin{align*}
V_{1}  &  =\left(  a_{1},b_{1},c_{1}\right) \\
V_{2}  &  =\left(  a_{2},b_{2},c_{2}\right)  ,
\end{align*}
then the area of the parallelogram spanned by those two vectors in euclidean
space is%
\[
\sqrt{\det
\begin{bmatrix}
V_{1}\bullet_{K}V_{1} & V_{2}\bullet_{K}V_{1}\\
V_{1}\bullet_{K}V_{2} & V_{2}\bullet_{K}V_{2}%
\end{bmatrix}}
=\sqrt{\det\left( 
\begin{bmatrix}
V_{1} \\
V_{2}
\end{bmatrix}
\begin{bmatrix}
1 & 0 & 0\\
0 & 1 & 0\\
0 & 0 & K^{-1}%
\end{bmatrix}
\begin{bmatrix}
V_{1}^\transpose & V_{2}^\transpose%
\end{bmatrix}
\right) }.
\]

\end{problem}

\textit{Moral of the story:} The dot-product rules! That is, if you
know the dot-product you know everything there is to know about a
geometry, lengths, areas, angles, everything. And the set
\[
1=K(x^2+y^2)+z^2
\]
continues to make sense even when $K$ is
negative. And as we will see later on, the definition of the $K$-dot
product also makes sense for tangent vectors to that set when $K$ is
negative.The geometry we get, when the constant $K$ is chosen to be
negative is called a hyperbolic geometry. The geometry we get, when
the constant $K$ is just chosen to be non-zero is called a
non-euclidean geometry.  In fact all the non-euclidean $2$-dimensional
geometries are either spherical or hyperbolic.

%% \textit{Coming attractions:} In hyperbolic geometry, when $K^{-1}$ in
%% \[
%% V_{1}\bullet_{K}V_{2}  =V_{1} 
%% \begin{bmatrix}
%% 1 & 0 & 0\\
%% 0 & 1 & 0\\
%% 0 & 0 & K^{-1}%
%% \end{bmatrix}
%% V_{2}^\transpose
%% \]
%% becomes negative, so that the third coordinate of velocity, that is,
%% the $c$-direction, actually \textit{contracts} lengths. It was the
%% understanding of this mysterious fact that allowed Einstein to
%% discover (special) relativity.


\begin{problem}
Summarize the results from this section. In particular, indicate which
results follow from the others.
\begin{freeResponse}
\end{freeResponse}
\end{problem}


\end{document}
