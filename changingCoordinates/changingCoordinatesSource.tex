\shortdescription{Now we will look at rigid motions in three-space.}
\activitytitle{Changing coordinates}

\subsection*{Bringing the North Pole of the $R$-sphere to $\left(
0,0,1\right)  $}

We are now ready to introduce a slightly different set of coordinates for
$\mathbb{R}^{3}$, three-dimensional euclidean space. To see why we do this,
suppose we are standing at the North Pole%
\[
N=\left(  0,0,R\right)
\]
of the sphere%
\begin{equation}
\hat{x}^{2}+\hat{y}^{2}+\hat{z}^{2}=R^{2} \label{4}%
\end{equation}
of radius $R$. As $R$ increases (but we stay our same size, the sphere around
us becomes more and more like a flat, plane surface. However it can never get
completely flat because we are zooming out the positive $\hat{z}$-axis and we
would have to be `at infinity' for our surface to become exactly flat. We
remedy that unfortunate situation by considering another copy of
$\mathbb{R}^{3}$, whose coordinates we denote as $\left(  x,y,z\right)  $ and
make the following rule in order to pass between the two $\mathbb{R}^{3}$'s:%
\begin{align}
\hat{x}  &  =x\label{105}\\
\hat{y}  &  =y\nonumber\\
\hat{z}  &  =Rz.\nonumber
\end{align}
We think of the $\left(  x,y,z\right)  $-coordinates as simply being a
different set of addresses for the points in euclidean $3$-space, for example,%
\[
\left(  x,y,z\right)  =\left(  0,0,1\right)
\]
tells me that the point in euclidean $3$-space that I'm referring to is%
\[
\left(  \hat{x},\hat{y},\hat{z}\right)  =\left(  0,0,R\right)  =N,
\]
and the sphere of radius $R$ in euclidean $3$-space is given by%
\begin{align*}
R^{2}  &  =\hat{x}^{2}+\hat{y}^{2}+\hat{z}^{2}\\
&  =x^{2}+y^{2}+R^{2}z^{2}%
\end{align*}
that is, by the equation%
\begin{equation}
1=\frac{1}{R^{2}}\left(  x^{2}+y^{2}\right)  +z^{2}. \label{5}%
\end{equation}
The quantity%
\[
K=\frac{1}{R^{2}}%
\]
is called the curvature of the $R$-sphere. So in $\left(  x,y,z\right)
$-coordinates, as $R$ goes to infinity, $K$ goes to $0$. The formula $\left(
\ref{5}\right)  $ is rewritten as%
\begin{equation}
1=K\left(  x^{2}+y^{2}\right)  +z^{2}, \label{7}%
\end{equation}
and so goes to%
\[
1=z^{2}%
\]
as $R$ goes to infinity. So, in the $\left(  x,y,z\right)  $-coordinates, our
`$R$-geometry' does indeed go to something finite and flat as $R$ goes to
infinity, namely the set given by the formula%
\[
z=\pm1
\]
which is in fact (two copies of) a plane!

\begin{exercise}\hfil
\begin{enumerate}
\item Sketch the solution set in $\left(  x,y,z\right)  $-coordinates
representing the sphere%
\[
R^{2}=\hat{x}^{2}+\hat{y}^{2}+\hat{z}^{2}=2^{2}%
\]
of radius $2$ in euclidean three-space.

\item Sketch the solution set in the same $\left(  x,y,z\right)  $-coordinates
representing the sphere%
\[
R^{2}=\hat{x}^{2}+\hat{y}^{2}+\hat{z}^{2}=10^{2}%
\]
of radius $10$ in euclidean three-space.

\item Sketch the solution set in the same $\left(  x,y,z\right)  $-coordinates
representing the sphere%
\[
R^{2}=\hat{x}^{2}+\hat{y}^{2}+\hat{z}^{2}=10^{-2}%
\]
of radius $10^{-1}$ in euclidean three-space.
\end{enumerate}
\end{exercise}

\subsection*{$K$-geometry: Formulas for euclidean lengths and angles in terms
of $\left(  x,y,z\right)  $-coordinates}

To prepare ourselves to do hyperbolic geometry, which has no satisfactory
model in euclidean three-space, we will `practice' by doing spherical geometry
(which \textit{does} have a completely satisfactory model in euclidean
three-space) using these `slightly strange' $\left(  x,y,z\right)
$-coordinates. Gradually throughout this course we will discover that the same
rules that govern spherical geometry, expressed in $\left(  x,y,z\right)
$-coordinates, also govern flat and hyperbolic geometry! In all three cases,
the space in $\left(  x,y,z\right)  $-coordinates that we will study is%
\begin{equation}
1=K\left(  x^{2}+y^{2}\right)  +z^{2}. \label{11}%
\end{equation}
If $K>0$, the geometry we will be studying is the geometry of the the
euclidean sphere of radius%
\[
R=\frac{1}{\sqrt{K}}.
\]
If $K=0$ we will be studying flat (plane) geometry. If $K<0$, we will be
studying hyperbolic geometry. The number $K$, in all cases, is called the
\textit{curvature} of the geometry.

In short, we want to use $\left(  x,y,z\right)  $-coordinates to compute with,
but we want lengths and angles to be the usual euclidean ones in $\left(
\hat{x},\hat{y},\hat{z}\right)  $-coordinates.

\begin{exercise}
a) Suppose we have functions%
\[
\left(  \hat{x}\left(  x,y,z\right)  ,\hat{y}\left(  x,y,z\right)  ,\hat
{z}\left(  x,y,z\right)  \right)
\]
where%
\begin{align*}
x  &  =f\left(  t\right) \\
y  &  =g\left(  t\right) \\
z  &  =h\left(  t\right)  .
\end{align*}
State the Chain Rule for%
\begin{align*}
\frac{d\hat{x}}{dt}  &  =\\
\frac{d\hat{y}}{dt}  &  =\\
\frac{d\hat{z}}{dt}  &  =.
\end{align*}


b) Rewrite the Chain Rule in matrix notation%
\[
\left(
\begin{array}
[c]{ccc}%
\frac{d\hat{x}}{dt} & \frac{d\hat{y}}{dt} & \frac{d\hat{z}}{dt}%
\end{array}
\right)  =\left(
\begin{array}
[c]{ccc}%
\frac{dx}{dt} & \frac{dy}{dt} & \frac{dz}{dt}%
\end{array}
\right)  \cdot\left(
\begin{array}
[c]{ccc}
&  & \\
&  & \\
&  &
\end{array}
\right)  .
\]

\end{exercise}

\begin{exercise}
\label{8} Recalling that $R$ is a positive constant, use $\left(
\ref{105}\right)  $ and the Chain Rule to show that, for any path $\hat
{X}\left(  t\right)  =\left(  \hat{x}\left(  t\right)  ,\hat{y}\left(
t\right)  ,\hat{z}\left(  t\right)  \right)  $ in euclidean $3$-space,%
\begin{align*}
\frac{d\hat{x}}{dt}  &  =\frac{dx}{dt}\\
\frac{d\hat{y}}{dt}  &  =\frac{dy}{dt}\\
\frac{d\hat{z}}{dt}  &  =R\frac{dz}{dt}.
\end{align*}

\end{exercise}

\begin{exercise}
Use matrix multiplication [DS,307] and Exercise \ref{8} to show that%
\begin{align*}
\frac{d\hat{X}\left(  t\right)  }{dt}  &  =\left(  \frac{dX\left(  t\right)
}{dt}\right)  \left(
\begin{array}
[c]{ccc}%
1 & 0 & 0\\
0 & 1 & 0\\
0 & 0 & R
\end{array}
\right) \\
\frac{dX\left(  t\right)  }{dt}  &  =\left(  \frac{d\hat{X}\left(  t\right)
}{dt}\right)  \left(
\begin{array}
[c]{ccc}%
1 & 0 & 0\\
0 & 1 & 0\\
0 & 0 & R^{-1}%
\end{array}
\right)  .
\end{align*}

\end{exercise}

\textit{NB}: This last computation shows that, if
\begin{align*}
\hat{V}_{1}  &  =\left(  \hat{a}_{1},\hat{b}_{1},\hat{c}_{1}\right) \\
\hat{V}_{2}  &  =\left(  \hat{a}_{2},\hat{b}_{2},\hat{c}_{2}\right)
\end{align*}
are tangent vectors in $\left(  \hat{x},\hat{y},\hat{z}\right)  $-coordinates
and%
\begin{align*}
V_{1}  &  =\left(  a_{1},b_{1},c_{1}\right) \\
V_{2}  &  =\left(  a_{2},b_{2},c_{2}\right)
\end{align*}
are their transformations into $\left(  x,y,z\right)  $-coordinates, then%
\begin{align*}
\hat{V}_{1}  &  =\left(  V_{1}\right)  \left(
\begin{array}
[c]{ccc}%
1 & 0 & 0\\
0 & 1 & 0\\
0 & 0 & R
\end{array}
\right) \\
\hat{V}_{2}  &  =\left(  V_{2}\right)  \left(
\begin{array}
[c]{ccc}%
1 & 0 & 0\\
0 & 1 & 0\\
0 & 0 & R
\end{array}
\right)
\end{align*}
and%
\begin{align*}
\hat{V}_{1}\bullet\hat{V}_{2}  &  =\left(  \hat{V}_{1}\right)  \cdot\left(
\hat{V}_{2}\right)  ^{t}\\
&  =\left(  V_{1}\right)  \left(
\begin{array}
[c]{ccc}%
1 & 0 & 0\\
0 & 1 & 0\\
0 & 0 & R
\end{array}
\right)  \left(  \left(  V_{2}\right)  \left(
\begin{array}
[c]{ccc}%
1 & 0 & 0\\
0 & 1 & 0\\
0 & 0 & R
\end{array}
\right)  \right)  ^{t}\\
&  =\left(  V_{1}\right)  \left(
\begin{array}
[c]{ccc}%
1 & 0 & 0\\
0 & 1 & 0\\
0 & 0 & R
\end{array}
\right)  \left(
\begin{array}
[c]{ccc}%
1 & 0 & 0\\
0 & 1 & 0\\
0 & 0 & R
\end{array}
\right)  \left(  V_{2}\right)  ^{t}\\
&  =\left(  V_{1}\right)  \left(
\begin{array}
[c]{ccc}%
1 & 0 & 0\\
0 & 1 & 0\\
0 & 0 & K^{-1}%
\end{array}
\right)  \left(  V_{2}\right)  ^{t}.
\end{align*}
This last computation says that we can compute the euclidean dot $\hat{V}%
_{1}\bullet\hat{V}_{2}$ without ever referring to euclidean coordinates. We
incorporate that fact into the following definition.

\begin{definition}
\textquotedblleft$K$-dot-product\textquotedblright\ of vectors:%
\begin{align}
V_{1}\bullet_{K}V_{2}  &  =\left(  V_{1}\right)  \left(
\begin{array}
[c]{ccc}%
1 & 0 & 0\\
0 & 1 & 0\\
0 & 0 & K^{-1}%
\end{array}
\right)  \left(  V_{2}\right)  ^{t}\label{10}\\
&  =\left(
\begin{array}
[c]{ccc}%
a_{1} & b_{1} & c_{1}%
\end{array}
\right)  \left(
\begin{array}
[c]{ccc}%
1 & 0 & 0\\
0 & 1 & 0\\
0 & 0 & K^{-1}%
\end{array}
\right)  \left(
\begin{array}
[c]{c}%
a_{2}\\
b_{2}\\
c_{2}%
\end{array}
\right)  .\nonumber
\end{align}

\end{definition}

So, suppose we have a curve on the $R$-sphere in euclidean $3$-space but it is
given to us in $X\left(  t\right)  =\left(  x\left(  t\right)  ,y\left(
t\right)  ,z\left(  t\right)  \right)  $-coordinates. Then the length of that
curve in euclidean $3$-space is%
\[%
%TCIMACRO{\dint \nolimits_{b}^{e}}%
%BeginExpansion
{\displaystyle\int\nolimits_{b}^{e}}
%EndExpansion
\sqrt{\frac{dX}{dt}\bullet_{K}\frac{dX}{dt}}dt.
\]


\begin{exercise}
\label{222}Use Exercise \ref{9} to show that, if we have any two vectors in
euclidean three-space that are tangent to the $R$-sphere at some point on it,
but the two vectors are given to us in $\left(  x,y,z\right)  $-coordinates as%
\begin{align*}
V_{1}  &  =\left(  a_{1},b_{1},c_{1}\right) \\
V_{2}  &  =\left(  a_{2},b_{2},c_{2}\right)  ,
\end{align*}
then the area of the parallelogram spanned by those two vectors in euclidean
$3$-space is%
\[
\sqrt{\left\vert
\begin{array}
[c]{cc}%
V_{1}\bullet_{K}V_{1} & V_{2}\bullet_{K}V_{1}\\
V_{1}\bullet_{K}V_{2} & V_{2}\bullet_{K}V_{2}%
\end{array}
\right\vert }=\sqrt{\left\vert \left(
\begin{array}
[c]{c}%
\left(  V_{1}\right) \\
\left(  V_{2}\right)
\end{array}
\right)  \cdot\left(
\begin{array}
[c]{ccc}%
1 & 0 & 0\\
0 & 1 & 0\\
0 & 0 & K^{-1}%
\end{array}
\right)  \cdot\left(
\begin{array}
[c]{cc}%
\left(  V_{1}\right)  ^{t} & \left(  V_{2}\right)  ^{t}%
\end{array}
\right)  \right\vert }.
\]

\end{exercise}

\textit{Moral of the story:} The dot-product rules! That is, if you know the
dot-product you know everything there is to know about a geometry, lengths,
areas, angles, everything. And the set $\left(  \ref{11}\right)  $ continues
to make sense even when $K$ is negative. And as we will see later on, the
definition of the $K$-dot product also makes sense for tangent vectors to that
set when $K$ is negative.The geometry we get, when the constant $K$ is chosen
to be negative is called a hyperbolic geometry. The geometry we get, when the
constant $K$ is just chosen to be non-zero is called a non-euclidean geometry.
In fact all the non-euclidean $2$-dimensional geometries are either spherical
or hyperbolic.

\textit{Coming attractions:} A big idea is that in hyperbolic geometry
$K^{-1}$ in $\left(  \ref{10}\right)  $ becomes negative, so that the third
coordinate of velocity, that is, the $c$-direction, actually
\textit{contracts} lengths. It was the understanding of this mysterious fact
that allowed Einstein to discover (special) relativity.
