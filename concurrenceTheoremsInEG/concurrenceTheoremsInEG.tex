\documentclass[newpage,hints,handout]{ximera}

\usepackage{microtype}
\usepackage{tikz}
\usepackage{tkz-euclide}
\usetkzobj{all}
\tikzstyle geometryDiagrams=[ultra thick,color=blue!50!black]

\renewcommand{\epsilon}{\varepsilon}



\title{Concurrence theorems in euclidean geometry}

\begin{document}
\begin{abstract}
Now we explore when three lines meet at a point.
\end{abstract}
\maketitle

Let's look at a some \textit{concurrence} theorems. Concurrence
theorems deal with situations when three or more lines (or curves)
pass through the same point.

\begin{problem}
\label{25} Denote the measure or area of a triangle $\triangle
ABC$ as $\left\vert \triangle ABC\right\vert $. Show that, in the
diagram below,
\[
\frac{\left\vert AF\right\vert }{\left\vert FB\right\vert }=\frac{\left\vert
\triangle AFC\right\vert }{\left\vert \triangle CFB\right\vert }%
=\frac{\left\vert \triangle AFX\right\vert }{\left\vert \triangle
XFB\right\vert }.
\]
\begin{image}
\begin{tikzpicture}[geometryDiagrams]
\tkzDefPoint(0,0){A}
\tkzDefPoint(5,3){C}
\tkzDefPoint(8,0){B}
\tkzDefPoint(4.79,1.33){X}
\tkzDefPoint(4.63,0){F}
\tkzDefPoint(6.26,1.74){D}
\tkzDefPoint(3.27,1.96){E}
\draw (A)--(B)--(C)--cycle;
\draw (A)--(D);
\draw (F)--(C);
\draw (B)--(E);


\tkzLabelPoints[above](C)
\tkzLabelPoints[above right](X)
\tkzLabelPoints[below](A,B,F)
\tkzLabelPoints[above left](E)
\tkzLabelPoints[right](D)
%\draw[step=.5cm] (0,0) grid (10,5);
\end{tikzpicture}
\end{image}
\begin{hint}
Mark the height of the relevant triangles.
\end{hint}
\begin{freeResponse}
Start by marking the height of $\triangle AFC$ and $\triangle CFB$:
\begin{image}
\begin{tikzpicture}[geometryDiagrams]
\tkzDefPoint(0,0){A}
\tkzDefPoint(5,3){C}
\tkzDefPoint(8,0){B}
\tkzDefPoint(5,0){D}
\tkzDefPoint(4.63,0){F}
\draw (A)--(B)--(C)--cycle;
\draw (F)--(C);
\draw[thin] (C)--(5,0);
\tkzMarkRightAngle[thin](B,D,C)

\node[right] at (5,1.5) {$h$};

\tkzLabelPoints[above](C)
\tkzLabelPoints[below](A,B,F)
%\draw[step=.5cm] (0,0) grid (10,5);
\end{tikzpicture}
\end{image}
Now 
\[
|\triangle AFC| = \frac{1}{2} |AF|\cdot h\qquad\text{and}\qquad |\triangle CFB| = \frac{1}{2} |FB|\cdot h,
\]
and so
\[
\frac{|\triangle AFC|}{|\triangle CFB|} = \frac{\frac{1}{2} |AF|\cdot h}{\frac{1}{2} |FB|\cdot h} = \frac{|AF|}{|FB|}.
\]

To get the second equality, again add the height of the triangles in problem:
\begin{image}
\begin{tikzpicture}[geometryDiagrams]
\tkzDefPoint(0,0){A}
\tkzDefPoint(5,3){C}
\tkzDefPoint(8,0){B}
\tkzDefPoint(4.79,1.33){X}
\tkzDefPoint(4.63,0){F}
\tkzDefPoint(6.26,1.74){D}
\tkzDefPoint(3.27,1.96){E}
\tkzDefPoint(4.79,0){G}

\draw (A)--(B)--(C)--cycle;
\draw (A)--(D);
\draw (F)--(C);
\draw (B)--(E);
\draw[thin] (G)--(X);

\tkzMarkRightAngle[thin](B,D,C)

\node[right] at (4.79,.66) {$g$};

\tkzLabelPoints[above](C)
\tkzLabelPoints[above right](X)
\tkzLabelPoints[below](A,B,F)
\tkzLabelPoints[above left](E)
\tkzLabelPoints[right](D)
%\draw[step=.5cm] (0,0) grid (10,5);
\end{tikzpicture}
\end{image}
Now 
\[
|\triangle AFX| = \frac{1}{2} |AF|\cdot g\qquad\text{and}\qquad |\triangle XFB| = \frac{1}{2} |FB|\cdot g,
\]
and so
\[
\frac{|\triangle AFX|}{|\triangle XFB|} = \frac{\frac{1}{2} |AF|\cdot g}{\frac{1}{2} |FB|\cdot g} = \frac{|AF|}{|FB|}.
\]
\end{freeResponse}
\end{problem}

\begin{problem}
\label{26} Use the previous problem to show using pure algebra that%
\[
\frac{|AF|}{|FB|}=\frac{|\triangle AXC|}{|\triangle CXB|}. %\label{27}%
\]
\begin{freeResponse}
Write
\[
\frac{|\triangle AXC|}{|\triangle CXB|} = \frac{|\triangle AFC|-|\triangle AFX|}{|\triangle CFB|-|\triangle XFB|}.
\]
From our work above, we know that
\[
|\triangle AFC| = \frac{|AF|\cdot |\triangle CFB|}{|FB|} \qquad\text{and}\qquad |\triangle AFX| = \frac{|AF|\cdot |\triangle XFB|}{|FB|}.
\]
Substituting we find
\begin{align*}
\frac{|\triangle AXC|}{|\triangle CXB|} &= \frac{|\triangle AFC|-|\triangle AFX|}{|\triangle CFB|-|\triangle XFB|}\\
&=\frac{ \frac{|AF|\cdot |\triangle CFB|}{|FB|}- \frac{|AF|\cdot |\triangle XFB|}{|FB|}}{|\triangle CFB|-|\triangle XFB|}\\
&=\frac{|AF|}{|FB|} \frac{|\triangle CFB|- |\triangle XFB|}{|\triangle CFB|-|\triangle XFB|}\\
&=\frac{|AF|}{|FB|}.
\end{align*}
\end{freeResponse}
\end{problem}


Now we will present, and you will prove, \textit{Ceva's Theorem}. %[MJG,287-288]

\textbf{Historical note:} Giovanni Ceva (\emph{CHEH-vah}) was an Italian
mathematician who proved this theorem in the 1600s.  In fact, it was previously
proved in the 1000s by Yusuf al-Mu'taman ibn-H\=ud, the Muslim king of Zaragoza
in Spain.  But we still call it Ceva's theorem.

\begin{theorem}[Ceva's Theorem]
Three segments $\bar{AD}$, $\bar{BE}$, and $\bar{CF}$
\begin{image}
\begin{tikzpicture}[geometryDiagrams]
\tkzDefPoint(0,0){A}
\tkzDefPoint(5,3){C}
\tkzDefPoint(8,0){B}
\tkzDefPoint(4.79,1.33){X}
\tkzDefPoint(4.63,0){F}
\tkzDefPoint(6.26,1.74){D}
\tkzDefPoint(3.27,1.96){E}
\draw (A)--(B)--(C)--cycle;
\draw (A)--(D);
\draw (F)--(C);
\draw (B)--(E);


\tkzLabelPoints[above](C)
\tkzLabelPoints[above right](X)
\tkzLabelPoints[below](A,B,F)
\tkzLabelPoints[above left](E)
\tkzLabelPoints[right](D)
%\draw[step=.5cm] (0,0) grid (10,5);
\end{tikzpicture}
\end{image}
are concurrent if and only if 
\[
\frac{|AF|}{|FB|}\cdot\frac{|BD|}{|DC|}\cdot\frac{|CE|}{|EA|}=1.
\]
\end{theorem}

\begin{problem}
Prove the forward direction of Ceva's theorem: For three concurrent
segments $\bar{AD}$, $\bar{BE}$ and $\bar{CF}$
\begin{image}
\begin{tikzpicture}[geometryDiagrams]
\tkzDefPoint(0,0){A}
\tkzDefPoint(5,3){C}
\tkzDefPoint(8,0){B}
\tkzDefPoint(4.79,1.33){X}
\tkzDefPoint(4.63,0){F}
\tkzDefPoint(6.26,1.74){D}
\tkzDefPoint(3.27,1.96){E}
\draw (A)--(B)--(C)--cycle;
\draw (A)--(D);
\draw (F)--(C);
\draw (B)--(E);


\tkzLabelPoints[above](C)
\tkzLabelPoints[above right](X)
\tkzLabelPoints[below](A,B,F)
\tkzLabelPoints[above left](E)
\tkzLabelPoints[right](D)
%\draw[step=.5cm] (0,0) grid (10,5);
\end{tikzpicture}
\end{image}
show that
\[
\frac{|AF|}{|FB|}\cdot\frac{|BD|}{|DC|}\cdot\frac{|CE|}{|EA|}=1.
\]
\begin{hint}
Use the previous problem repeatedly.
\end{hint}
\begin{freeResponse}
From the previous problem, we see that
\[
\frac{|AF|}{|FB|}=\frac{|\triangle AXC|}{|\triangle CXB|},\qquad
\frac{|BD|}{|DC|}=\frac{|\triangle AXB|}{|\triangle AXC|},\qquad
\frac{|CE|}{|EA|}=\frac{|\triangle CXB|}{|\triangle AXB|}.
\]
So we may write
\[
\frac{|AF|}{|FB|}\cdot\frac{|BD|}{|DC|}\cdot\frac{|CE|}{|EA|} = 
\frac{|\triangle AXC|}{|\triangle CXB|}\cdot 
\frac{|\triangle AXB|}{|\triangle AXC|}\cdot
\frac{|\triangle CXB|}{|\triangle AXB|} = 1.
\]
\end{freeResponse}
\end{problem}



\begin{problem}
Prove the reverse direction of Ceva's Theorem: If%
\[
\frac{|AF|}{|FB|}\cdot\frac{|BD|}{|DC|}\cdot\frac{|CE|}{|EA|}=1
\]
then the lines $AD$, $BE$, and $CF$ pass through a common point.
\begin{hint}
Suppose that they do not pass through a common point.
\begin{image}
\begin{tikzpicture}[geometryDiagrams]
\tkzDefPoint(0,0){A}
\tkzDefPoint(5,3){C}
\tkzDefPoint(8,0){B}
\tkzDefPoint(5,0){F}
\tkzDefPoint(6.26,1.74){D}
\tkzDefPoint(2.75,1.65){E}
\draw (A)--(B)--(C)--cycle;
\draw (A)--(D);
\draw (F)--(C);
\draw (B)--(E);


\tkzLabelPoints[above](C)
\tkzLabelPoints[below](A,B,F)
\tkzLabelPoints[above left](E)
\tkzLabelPoints[right](D)
%\draw[step=.5cm] (0,0) grid (10,5);
\end{tikzpicture}
\end{image}
\end{hint}
\begin{hint}
Notice that if, for example, $F$ moves along the segment $\bar{AB}$,
then $\frac{|AF|}{|FB|}$ is a strictly increasing function of
$|AF|$. Now use a previous problem to determine a position $F'$ for
$F$ along the segment $\bar{AB}$ at which
\[
\frac{|AF'|}{|F'B|}\cdot\frac{|BD|}{|DC|}\cdot\frac{|CE|}{|EA|}=1.
\]
\end{hint}
\begin{freeResponse}
Seeking a contradiction, suppose that 
\[
\frac{|AF|}{|FB|}\cdot\frac{|BD|}{|DC|}\cdot\frac{|CE|}{|EA|}=1
\]
and we have the situation where the lines in problem do not meet at a point:
\begin{image}
\begin{tikzpicture}[geometryDiagrams]
\tkzDefPoint(0,0){A}
\tkzDefPoint(5,3){C}
\tkzDefPoint(8,0){B}
\tkzDefPoint(5,0){F}
\tkzDefPoint(6.26,1.74){D}
\tkzDefPoint(2.75,1.65){E}
\draw (A)--(B)--(C)--cycle;
\draw (A)--(D);
\draw (F)--(C);
\draw (B)--(E);


\tkzLabelPoints[above](C)
\tkzLabelPoints[below](A,B,F)
\tkzLabelPoints[above left](E)
\tkzLabelPoints[right](D)
%\draw[step=.5cm] (0,0) grid (10,5);
\end{tikzpicture}
\end{image}
There is an $F'$ such that we have the following diagram:
\begin{image}
\begin{tikzpicture}[geometryDiagrams]
\tkzDefPoint(0,0){A}
\tkzDefPoint(5,3){C}
\tkzDefPoint(8,0){B}
\tkzDefPoint(5,0){F}
\tkzDefPoint(3.76,0){F'}
\tkzDefPoint(6.26,1.74){D}
\tkzDefPoint(2.75,1.65){E}
\draw (A)--(B)--(C)--cycle;
\draw (A)--(D);
\draw (F)--(C);
\draw (B)--(E);
\draw[dashed] (F')--(C);


\tkzLabelPoints[above](C)
\tkzLabelPoints[below](A,B,F,F')
\tkzLabelPoints[above left](E)
\tkzLabelPoints[right](D)
%\draw[step=.5cm] (0,0) grid (10,5);
\end{tikzpicture}
\end{image}
Note, $F'$ might be on the other side of $F$. Regardless, since
$\frac{|AF|}{|FB|}$ increases as $|AF|$ increases, we see that
\[
\frac{|AF|}{|FB|} \ne \frac{|AF'|}{|F'B|}.
\]
But this means that 
\[
\frac{|AF'|}{|F'B|}\cdot\frac{|BD|}{|DC|}\cdot\frac{|CE|}{|EA|}\ne\frac{|AF|}{|FB|}\cdot\frac{|BD|}{|DC|}\cdot\frac{|CE|}{|EA|}=1
\]
contradicting our previous result! Hence the lines $AD$, $BE$, and $CF$ must
pass through a common point.
\end{freeResponse}
\end{problem}

\begin{problem}
A \textbf{median} of a triangle is a line segment from a vertex
to the midpoint of the opposite side. Show that the medians of any triangle
meet in a common point.

\begin{hint}
Use Ceva's Theorem.
\end{hint}
\begin{freeResponse}
Since medians connect the midpoint of a side of a triangle to the
opposite angle, we have
\begin{align*}
\frac{|AF|}{|FB|}\cdot\frac{|BD|}{|DC|}\cdot\frac{|CE|}{|EA|} &= 
\frac{|AB|/2}{|AB|/2}\cdot\frac{|BC|/2}{|BC|/2}\frac{|CA|/2}{|CA|/2}\\
&= 1.
\end{align*}
Hence by Ceva's Theorem, we see that the medians are concurrent. 
\end{freeResponse}
\end{problem}

\begin{definition}\index{altitude} 
An \textbf{altitude} of a triangle is a line segment originating at a
vertex of the triangle that meets the line containing the opposite
side at a right angle.
\end{definition}


\begin{problem}
Use Ceva's theorem to show that the three lines containing altitudes
of a triangle are concurrent.
\begin{image}
\begin{tikzpicture}[geometryDiagrams]
\tkzDefPoint(0,0){A}
\tkzDefPoint(5,4){C}
\tkzDefPoint(6.5,0){B}
\tkzDefPoint(5,0){F}
\tkzDefPoint(5.54,2.56){D}
\tkzDefPoint(3.98,3.18){E}
\draw (A)--(B)--(C)--cycle;
\draw (A)--(D);
\draw (F)--(C);
\draw (B)--(E);


\tkzLabelPoints[above](C)
\tkzLabelPoints[below](A,B,F)
\tkzLabelPoints[above left](E)
\tkzLabelPoints[right](D)
%\draw[step=.5cm] (0,0) grid (10,5);
\end{tikzpicture}
\end{image}
\begin{hint}
Use all three similarities of the form $\triangle CEB\sim\triangle
CDA$ and then apply Ceva's theorem.
\end{hint}
\begin{freeResponse}
Since altitudes intersect lines containing the opposite sites at right
angles, see that 
\[
\triangle CEB\sim\triangle CDA,\qquad 
\triangle BFC \sim \triangle BDA,\qquad
\triangle AFC \sim \triangle AEB.
\]
From this we see that 
\[
\frac{|AF|}{|EA|}=\frac{|AC|}{|AB|},\qquad
\frac{|BD|}{|FB|}=\frac{|AB|}{|BC|},\qquad
\frac{|CE|}{|DC|}=\frac{|BC|}{|AC|}.
\]
Now we see that 
\begin{align*}
\frac{|AF|}{|FB|}\cdot\frac{|BD|}{|DC|}\cdot\frac{|CE|}{|EA|} &= 
\frac{|AF|}{|EA|}\cdot\frac{|BD|}{|FB|}\cdot\frac{|CE|}{|DC|}\\
&=\frac{|AC|}{|AB|}\cdot\frac{|AB|}{|BC|}\cdot\frac{|BC|}{|AC|}\\
&= 1.
\end{align*}
Hence by Ceva's Theorem, we see that the altitudes of a triangle are
concurrent.
\end{freeResponse}
\end{problem}

\begin{problem}
Summarize the results from this section. In particular, indicate which
results follow from the others.
\begin{freeResponse}
\end{freeResponse}
\end{problem}
\end{document}
