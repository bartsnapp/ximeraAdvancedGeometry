\documentclass{ximera}

\usepackage{microtype}
\usepackage{tikz}
\usepackage{tkz-euclide}
\usetkzobj{all}
\tikzstyle geometryDiagrams=[ultra thick,color=blue!50!black]

\renewcommand{\epsilon}{\varepsilon}



\title{Congruences, that is, rigid motions}
\begin{document}
\begin{abstract}
We explore rigid motions as matrix multiplication.
\end{abstract}
\maketitle

\subsection*{Transformations of euclidean space}

Consider the following mapping of euclidean space to itself:%
\[
\begin{bmatrix}
\underline{\hat{x}} & \underline{\hat{y}} & \underline{\hat{z}}%
\end{bmatrix}
=
\begin{bmatrix} \hat{x} & \hat{y} & \hat{z}%
\end{bmatrix}
\cdot\hat{M}
\]
where $\hat{M}$ is an invertible $3\times3$ matrix. Since the matrix
is invertible, this mapping is one-to-one and onto.

\begin{definition}
  A mapping of euclidean space to itself given by
  \[
\begin{bmatrix}
\underline{\hat{x}} & \underline{\hat{y}} & \underline{\hat{z}}%
\end{bmatrix}
=
\begin{bmatrix} \hat{x} & \hat{y} & \hat{z}%
\end{bmatrix}
\cdot\hat{M}
  \]
  is called a \dfn{rigid motion} if the distance between any two
points in euclidean space is left unchanged by the mapping, that is, for
any two points, $\hat{X}_{1}$ and $\hat{X}_{2}$ in euclidean space%
\[
d\left( \hat{X}_{1}  \cdot\hat{M},\hat{X}_{2}
\cdot\hat{M}\right)  =d\left(  \hat{X}_{1},\hat{X}_{2}\right).
\]
\end{definition}

It turns out that there is a special class of matrices that give rise
to rigid motions.

\begin{definition}
  A matrix $\hat{M}$ satisfying
  \[
  \hat{M}\cdot\hat{M}^\transpose=I=\begin{bmatrix}
  1 & 0 & 0\\
  0 & 1 & 0\\
  0 & 0 & 1
  \end{bmatrix}.
  \]
  is called an \dfn{orthogonal} matrix.
\end{definition}

\begin{problem}
  Prove that a matrix $\hat{M}$ defines a rigid motion (a congruence)
  via
  \[
\begin{bmatrix}
\underline{\hat{x}} & \underline{\hat{y}} & \underline{\hat{z}}%
\end{bmatrix}
=
\begin{bmatrix} \hat{x} & \hat{y} & \hat{z}%
\end{bmatrix}
\cdot\hat{M}
  \]
  if and only if it is orthogonal.

  \begin{hint}
    Note that the square of the distance between $\hat{X}_{1}$ and
    $\hat{X}_{2}$ is just the dot-product of the vector%
    \[
    \hat{V}=\hat{X}_{2}-\hat{X}_{1}%
    \]
    with itself. 
  \end{hint}
  \begin{hint}
    If $\hat{M}$ is orthogonal, write
    \[
    \left(  \hat{V}  \cdot\hat{M} \right) \bullet\left(
    \hat{V}  \cdot\hat{M}\right)
    \]
    and deduce that this equals $\hat{V}\bullet\hat{V}$. 
  \end{hint}
  \begin{hint}
  Now suppose that $\hat{M}$ is a rigid motion. Explain why this means that:
  \[
  \left(  \hat{V}  \cdot\hat{M} \right) \bullet\left(
  \hat{V}  \cdot\hat{M}\right)  =\hat{V}\bullet\hat{V}
  \]
  Now rewrite as:
  \[
  \hat{V}\cdot\hat{V}^\transpose= \left(  \hat{V}  \cdot\hat{M} \right) \cdot\left(
    \hat{V}  \cdot\hat{M}\right)^\transpose 
  \]
  \end{hint}
  \begin{hint} Set
    \[
    \hat{V} =
    \begin{bmatrix}
      a & b & c
    \end{bmatrix}
    \]
    and
    \[
    \hat{M} =
    \begin{bmatrix}
      m_{11} & m_{12} & m_{13}\\
      m_{21} & m_{22} & m_{23}\\
      m_{31} & m_{32} & m_{33}
    \end{bmatrix}
    \]
    and view the equation 
    \[
    \hat{V}\cdot\hat{V}^\transpose= \left(  \hat{V}  \cdot\hat{M} \right) \cdot\left(
    \hat{V}  \cdot\hat{M}\right)^\transpose 
    \]
    as an equation of polynomials in the variables $a$, $b$, and $c$. 
  \end{hint}
  \begin{hint}
    Polynomials are equal if and only if their coefficients are equal. 
  \end{hint}
  
  
\begin{freeResponse}
For the first direction we assume that $\hat{M}$ is orthogonal and we want to show that for any two points,  $\hat{X}_{1}$ and $\hat{X}_{2}$,%
\[
   d\left( \hat{X}_{1}  \cdot\hat{M},\hat{X}_{2}
   \cdot\hat{M}\right)  =d\left(  \hat{X}_{1},\hat{X}_{2}\right).
\]
Let $ \hat{V}=\hat{X}_{2}-\hat{X}_{1}$.

\begin{align*}
   d\left( \hat{X}_{1}  \cdot\hat{M},\hat{X}_{2}
   \cdot\hat{M}\right) &= \sqrt{\left(\hat{X}_{2}  \cdot\hat{M}-\hat{X}_{1}  \cdot\hat{M}\right) \bullet \left(\hat{X}_{2}  	\cdot\hat{M}-\hat{X}_{1}  \cdot\hat{M}\right)} \\
   & = \sqrt{\left( \left( \hat{X}_{2}-\hat{X}_{1} \right) \cdot\hat{M}\right) \bullet \left( \left( \hat{X}_{2}-\hat{X}_{1} 		\right) \cdot\hat{M}\right)} \\
   & = \sqrt{ \left(\hat{V}\cdot\hat{M} \right) \bullet \left(\hat{V}\cdot\hat{M} \right)}\\
   & = \sqrt{
       \begin{bmatrix}
          \hat{V}\cdot\hat{M}
       \end{bmatrix} \cdot
       \begin{bmatrix}
          \hat{V}\cdot\hat{M}
       \end{bmatrix}^\transpose}\\
   & = \sqrt{ \hat{V}\cdot\hat{M} \cdot \hat{M}^\transpose \cdot \hat{V}^\transpose} \\
   & = \sqrt{ \hat{V} \cdot \hat{V}^\transpose} \\
   & = \sqrt{\hat{V} \bullet \hat{V}} = \sqrt{d\left(  \hat{X}_{1},\hat{X}_{2}\right)^2} = d\left( \hat{X}_{1},\hat{X}_{2}\right)
\end{align*}

For the other direction we suppose $\hat{M}$ is a rigid motion, so
\[
 d\left( \hat{X}_{1}  \cdot\hat{M},\hat{X}_{2}
   \cdot\hat{M}\right) = d\left(  \hat{X}_{1},\hat{X}_{2}\right).
\]
Therefore, 
\[
	\left(  \hat{V}  \cdot\hat{M} \right) \bullet\left(
    \hat{V}  \cdot\hat{M}\right)  =\hat{V}\bullet\hat{V}.
 \]
Following the hint, write,
\[
\hat{V}\cdot\hat{V}^\transpose= \left(  \hat{V}  \cdot\hat{M} \right) \cdot\left(
    \hat{V}  \cdot\hat{M}\right)^\transpose  = \hat{V} \cdot\hat{M} \cdot \hat{M}^\transpose \cdot \hat{V}^\transpose .
\]
Then, 
\begin{align*}
a^2 + b^2 +c^2 &= 
\begin{bmatrix}
      a & b & c
    \end{bmatrix}
\begin{bmatrix}
      a \\
      b \\
      c
    \end{bmatrix} \\ &=
\begin{bmatrix}
      a & b & c
    \end{bmatrix}
\begin{bmatrix}
	m_{11} & m_{12} & m_{13}\\
	m_{21} & m_{22} & m_{23}\\
	m_{31} & m_{32} & m_{33}
\end{bmatrix}
\begin{bmatrix}
	m_{11} & m_{21} & m_{31}\\
	m_{12} & m_{22} & m_{32}\\
	m_{13} & m_{23} & m_{33}
\end{bmatrix}
\begin{bmatrix}
      a \\
      b \\
      c
    \end{bmatrix}. \\
 \end{align*}
 Let $\hat{W} = \hat{M} \cdot \hat{M}^\transpose$ then,
 \begin{align*}
 w_{11} &= m_{11}^2 + m_{12}^2 +  m_{13}^2 \\
 w_{12} &= m_{11}m_{21} + m_{12}m_{22} +m_{13}m_{23}\\
 w_{13} &= m_{11}m_{31} + m_{12}m_{32} +m_{13}m_{33}\\
 w_{21} &= m_{21}m_{11} + m_{22}m_{12} +m_{23}m_{13} \\
 w_{22} &= m_{21}^2 + m_{22}^2 +  m_{23}^2 \\
 w_{23} &=  m_{21}m_{31} + m_{22}m_{32} +m_{23}m_{33} \\
 w_{31} &= m_{31}m_{11} + m_{32}m_{12} +m_{33}m_{13} \\
 w_{32} &= m_{31}m_{11} + m_{32}m_{12} +m_{33}m_{13}\\
 w_{33} &= m_{31}^2 + m_{32}^2 +  m_{33}^2.
 \end{align*}    
   
 Note that $w_{12} =  w_{21}$, $w_{13} =  w_{31}$, and $w_{23} =  w_{32}$. Now,
\[
 a^2 + b^2 +c^2 =  
a^2 w_{11} + 2abw_{21} + b^2 w_{22} + 2ac w_{13} + c^2 w_{33} + 2bc w_{23}.
\]
Therefore,  $0 = w_{12} =  w_{21} =w_{13} =  w_{31} = w_{23} =  w_{32}$, and $1 =  w_{11} =  w_{22} =  w_{33}$ so $\hat{M}$ is orthogonal. 
 
\end{freeResponse}

\end{problem}


\begin{problem}
  Show that, if $\hat{M}$ is orthogonal, then the transformation
  \[
  \begin{bmatrix}
    \underline{\hat{x}} & \underline{\hat{y}} & \underline{\hat{z}}
  \end{bmatrix}
  =
  \begin{bmatrix}
    \hat{x} & \hat{y} & \hat{z}
  \end{bmatrix}
  \cdot \hat{M}.
  \]
  takes the set of points $\left(\hat{x},\hat{y},\hat{z}\right)$ such
  that
\[
\hat{x}^2 + \hat{y}^2 + \hat{z}^2 = R^2
\]
to the set of points
$\left(\underline{\hat{x}},\underline{\hat{y}},\underline{\hat{z}}\right)$
such that
\[
\underline{\hat{x}}^2 + \underline{\hat{y}}^{2} + \underline{\hat{z}}^{2}=R^2.
\]
That is, $\hat{M}$ gives a one-to-one and onto mapping of the $R$-sphere to
itself.
\begin{hint}
  Write the equation
  \[
  \underline{\hat{x}}^2 + \underline{\hat{y}}^{2} + \underline{\hat{z}}^{2}=R^2
  \]
  as
  \[
  \begin{bmatrix}
    \underline{\hat{x}} & \underline{\hat{y}} & \underline{\hat{z}}%
  \end{bmatrix}  
  \cdot
  \begin{bmatrix}
    \underline{\hat{x}}\\
    \underline{\hat{y}}\\
    \underline{\hat{z}}
  \end{bmatrix}  =R^2.
  \]
\end{hint}

\begin{freeResponse}
\begin{align*}
R^2 = \underline{\hat{x}}^2 + \underline{\hat{y}}^2 +\underline{\hat{z}}^2 
	&= \begin{bmatrix} 
	 \underline{\hat{x}} & \underline{\hat{y}} & \underline{\hat{z}}
  	\end{bmatrix}  
  		\cdot
  	\begin{bmatrix}
    		\underline{\hat{x}}\\
    		\underline{\hat{y}}\\
    		\underline{\hat{z}}
  	\end{bmatrix}\\
	&= \begin{bmatrix} 
		 \underline{\hat{x}} & \underline{\hat{y}} & \underline{\hat{z}}
  		\end{bmatrix}  
  		\cdot
		\begin{bmatrix} 
		 \underline{\hat{x}} & \underline{\hat{y}} & \underline{\hat{z}}
  		\end{bmatrix}^\transpose \\
	&= \begin{bmatrix}
   		 \hat{x} & \hat{y} & \hat{z}
 	        \end{bmatrix}
	       \cdot \hat{M} \cdot \hat{M}^\transpose \cdot 
	      \begin{bmatrix}
	      	{\hat{x}}\\
    		{\hat{y}}\\
    		{\hat{z}}
		\end{bmatrix}\\
	&= \begin{bmatrix}
   		 \hat{x} & \hat{y} & \hat{z}
 	        \end{bmatrix}
	       \cdot
	       \begin{bmatrix}
	      	{\hat{x}}\\
    		{\hat{y}}\\
    		{\hat{z}}
		\end{bmatrix}
	= \hat{x}^2 + \hat{y}^2 + \hat{z}^2 .
\end{align*}
\end{freeResponse}

\end{problem}


\begin{problem}
Explain how the following diagram ``proves'' that functional
composition is \textit{always} associative. Note, we are writing our
functions on the \textit{right} of the argument.
\begin{image}
  \begin{tikzpicture}[scale=.5,geometryDiagrams]

    \node at (-2,8) {$X$};
    \draw[->] (-1.5,8)--(-.5,8);
    \draw[fill=blue!10!white] (0,8) ellipse (0.166 and 0.5);% ellipse
    \draw (.5,7) rectangle (2.5,9);
    \draw (3,8) [partial ellipse=270:450:0.166 and 0.5];
    \node at (1.5,8) {\Large $M_1$};
    \draw (.5,8.5) [partial ellipse=180:270:.5 and 0.166];
    \draw (.5,7.5) [partial ellipse=90:180:.5 and 0.166];
    \draw (2.5,8.5) [partial ellipse=270:360:.5 and 0.166];
    \draw (2.5,7.5) [partial ellipse=0:90:.5 and 0.166];

    %\node at (5,8.6) {$X M_1$};
    \draw[->] (3.5,8)--(6.5,8);
    
    \draw[fill=blue!10!white] (7,8) ellipse (0.166 and 0.5);% ellipse
    \draw (7.5,7) rectangle (9.5,9);
    \draw (10,8) [partial ellipse=270:450:0.166 and 0.5];
    \node at (8.5,8) {\Large $M_2$};
    \draw (7.5,8.5) [partial ellipse=180:270:.5 and 0.166];
    \draw (7.5,7.5) [partial ellipse=90:180:.5 and 0.166];
    \draw (9.5,8.5) [partial ellipse=270:360:.5 and 0.166];
    \draw (9.5,7.5) [partial ellipse=0:90:.5 and 0.166];

    \node at (12,8.6) {$X M_1 M_2$};
    \draw[->] (10.5,8)--(13.5,8);
    
    \draw[fill=blue!10!white] (14,8) ellipse (0.166 and 0.5);% ellipse
    \draw (14.5,7) rectangle (16.5,9);
    \draw (17,8) [partial ellipse=270:450:0.166 and 0.5];
    \node at (15.5,8) {\Large $M_3$};
    \draw (14.5,8.5) [partial ellipse=180:270:.5 and 0.166];
    \draw (14.5,7.5) [partial ellipse=90:180:.5 and 0.166];
    \draw (16.5,8.5) [partial ellipse=270:360:.5 and 0.166];
    \draw (16.5,7.5) [partial ellipse=0:90:.5 and 0.166];

    \node at (20.5,8) {$X(M_1 M_2) M_3$};
    \draw[->] (17.5,8)--(18.5,8);

    \draw[dashed,gray] (-.4,6.6) rectangle (10.35,9.4);
    \draw[dashed,gray] (6.6,2.6) rectangle (17.35,5.4);    
    
    %%%%%%%%%%%%%%%%%
    
    \node at (-2,4) {$X$};
    \draw[->] (-1.5,4)--(-.5,4);
    \draw[fill=blue!10!white] (0,4) ellipse (0.166 and 0.5);% ellipse
    \draw (.5,3) rectangle (2.5,5);
    \draw (3,4) [partial ellipse=270:450:0.166 and 0.5];
    \node at (1.5,4) {\Large $M_1$};
    \draw (.5,4.5) [partial ellipse=180:270:.5 and 0.166];
    \draw (.5,3.5) [partial ellipse=90:180:.5 and 0.166];
    \draw (2.5,4.5) [partial ellipse=270:360:.5 and 0.166];
    \draw (2.5,3.5) [partial ellipse=0:90:.5 and 0.166];

    \node at (5,4.6) {$X M_1$};
    \draw[->] (3.5,4)--(6.5,4);
    
    \draw[fill=blue!10!white] (7,4) ellipse (0.166 and 0.5);% ellipse
    \draw (7.5,3) rectangle (9.5,5);
    \draw (10,4) [partial ellipse=270:450:0.166 and 0.5];
    \node at (8.5,4) {\Large $M_2$};
    \draw (7.5,4.5) [partial ellipse=180:270:.5 and 0.166];
    \draw (7.5,3.5) [partial ellipse=90:180:.5 and 0.166];
    \draw (9.5,4.5) [partial ellipse=270:360:.5 and 0.166];
    \draw (9.5,3.5) [partial ellipse=0:90:.5 and 0.166];

    %\node at (12,4.6) {$X M_1 M_2$};
    \draw[->] (10.5,4)--(13.5,4);
    
    \draw[fill=blue!10!white] (14,4) ellipse (0.166 and 0.5);% ellipse
    \draw (14.5,3) rectangle (16.5,5);
    \draw (17,4) [partial ellipse=270:450:0.166 and 0.5];
    \node at (15.5,4) {\Large $M_3$};
    \draw (14.5,4.5) [partial ellipse=180:270:.5 and 0.166];
    \draw (14.5,3.5) [partial ellipse=90:180:.5 and 0.166];
    \draw (16.5,4.5) [partial ellipse=270:360:.5 and 0.166];
    \draw (16.5,3.5) [partial ellipse=0:90:.5 and 0.166];

    \node at (20.5,4) {$X M_1 (M_2 M_3)$};
    \draw[->] (17.5,4)--(18.5,4);

    %%%%%%%%%%%%%%%%%

    \node at (-2,0) {$X$};
    \draw[->] (-1.5,0)--(-.5,0);
    \draw[fill=blue!10!white] (0,0) ellipse (0.166 and 0.5);% ellipse
    \draw (.5,-1) rectangle (2.5,1);
    \draw (3,0) [partial ellipse=270:450:0.166 and 0.5];
    \node at (1.5,0) {\Large $M_1$};
    \draw (.5,.5) [partial ellipse=180:270:.5 and 0.166];
    \draw (.5,-.5) [partial ellipse=90:180:.5 and 0.166];
    \draw (2.5,.5) [partial ellipse=270:360:.5 and 0.166];
    \draw (2.5,-.5) [partial ellipse=0:90:.5 and 0.166];

    \node at (5,.6) {$X M_1$};
    \draw[->] (3.5,0)--(6.5,0);
    
    \draw[fill=blue!10!white] (7,0) ellipse (0.166 and 0.5);% ellipse
    \draw (7.5,-1) rectangle (9.5,1);
    \draw (10,0) [partial ellipse=270:450:0.166 and 0.5];
    \node at (8.5,0) {\Large $M_2$};
    \draw (7.5,.5) [partial ellipse=180:270:.5 and 0.166];
    \draw (7.5,-.5) [partial ellipse=90:180:.5 and 0.166];
    \draw (9.5,.5) [partial ellipse=270:360:.5 and 0.166];
    \draw (9.5,-.5) [partial ellipse=0:90:.5 and 0.166];

    \node at (12,.6) {$X M_1 M_2$};
    \draw[->] (10.5,0)--(13.5,0);
    
    \draw[fill=blue!10!white] (14,0) ellipse (0.166 and 0.5);% ellipse
    \draw (14.5,-1) rectangle (16.5,1);
    \draw (17,0) [partial ellipse=270:450:0.166 and 0.5];
    \node at (15.5,0) {\Large $M_3$};
    \draw (14.5,.5) [partial ellipse=180:270:.5 and 0.166];
    \draw (14.5,-.5) [partial ellipse=90:180:.5 and 0.166];
    \draw (16.5,.5) [partial ellipse=270:360:.5 and 0.166];
    \draw (16.5,-.5) [partial ellipse=0:90:.5 and 0.166];

    \node at (20.5,0) {$X M_1 M_2 M_3$};
    \draw[->] (17.5,0)--(18.5,0);

    %%%%%%%%%%%%%%%%%

    

  \end{tikzpicture}
\end{image}

\end{problem}






\begin{problem}
Show that the set of orthogonal matrices $\hat{M}$ form a group.  That
is, show that
\begin{enumerate}
\item multiplication of orthogonal matrices is associative, 
 	\begin{hint}
	Use the previous problem. 
	\end{hint}
\item the product of two orthogonal matrices is orthogonal,
\item the identity matrix is orthogonal,
\item the inverse matrix $\hat{M}^{-1}$ of a orthogonal matrix $\hat{M}$ is orthogonal.
\end{enumerate}

\begin{freeResponse}
\begin{enumerate}
\item Matrices are functions and in the previous problem we showed that function composition is associative. Therefore, matrix multiplication is associative.

\item Let $\hat{M}_{1}$ and $\hat{M}_{2}$ be orthogonal matrices.
\[
\left( \hat{M}_{1} \cdot \hat{M}_{2} \right) \left( \hat{M}_{1} \cdot \hat{M}_{2} \right)^\transpose 
	= \left( \hat{M}_{1} \cdot \hat{M}_{2} \right) \left( \hat{M}_{2}^\transpose \cdot \hat{M}_{1}^\transpose \right)  
\]
Since $\hat{M}_{1}$ and $\hat{M}_{2}$ are orthogonal,
\[
\left( \hat{M}_{1} \cdot \hat{M}_{2} \right) \left( \hat{M}_{2}^\transpose \cdot \hat{M}_{1}^\transpose \right) 
	= \hat{M}_{2} \cdot I \cdot \hat{M}_{2}^\transpose = I.
\]

\item Since $ I^\transpose = I$, then $I \cdot I^\transpose = I$

\item Since $\left( \hat{M}^{-1} \cdot \hat{M} \right)^\transpose = I^\transpose = I$, then 
\[
\hat{M}^\transpose \cdot \left( \hat{M}^{-1} \right)^\transpose = I.
\]
Therefore, because $\hat{M}$ is orthogonal $\left( \hat{M}^{-1} \right)^\transpose = \hat{M}$, so
\[
\hat{M}^{-1} \cdot \left( \hat{M}^{-1} \right)^\transpose = \hat{M}^{-1} \cdot \hat{M} = I.
\]

\end{enumerate}
\end{freeResponse}

\end{problem}




\begin{problem}
  Consider
  \[
  \hat{M}_\theta=\begin{bmatrix}
  \cos\theta & \sin\theta & 0\\
  -\sin\theta & \cos\theta & 0\\
  0 & 0 & 1
  \end{bmatrix}
  \]
  Can you describe geometrically what this mapping is doing
  to the points in euclidean space?

%rotationAboutZaxis Geogebra

\begin{freeResponse}
$\hat{M}_\theta$ rotates points in euclidean space around the $z$-axis. 
\end{freeResponse}
\end{problem}


\begin{problem}
\label{14} Show that the matrix%
\[
\hat{M}_\theta=\begin{bmatrix}
\cos\theta & \sin\theta & 0\\
-\sin\theta & \cos\theta & 0\\
0 & 0 & 1
\end{bmatrix}
\]
is orthogonal. 

\begin{freeResponse}
\begin{align*}
\hat{M}_\theta \cdot \hat{M}_\theta^\transpose 
	&= \begin{bmatrix}
	\cos\theta & \sin\theta & 0 \\
	-\sin\theta & \cos\theta & 0 \\
	0 & 0 & 1
	\end{bmatrix} \cdot
	\begin{bmatrix}
	\cos\theta & -\sin\theta & 0 \\
	\sin\theta & \cos\theta & 0 \\
	0 & 0 & 1
	\end{bmatrix} \\
	&= \begin{bmatrix}
	\cos^{2}\theta + \sin^{2} \theta & 0 & 0 \\
	0 &  \sin^{2} \theta + \cos^{2}\theta  & 0 \\
	0 & 0 & 1
	\end{bmatrix}
	=  \begin{bmatrix}
	1 & 0 & 0 \\
	0 & 1  & 0 \\
	0 & 0 & 1
	\end{bmatrix}
\end{align*}
\end{freeResponse}
\end{problem}


\begin{problem}
  Consider
  \[
  \hat{N}_\psi=\begin{bmatrix}
  \cos\psi & 0 & \sin\psi\\
  0 & 1 & 0\\
  -\sin\psi & 0 & \cos\psi
  \end{bmatrix}
  \]
  Can you describe geometrically what this mapping is doing
  to the points in euclidean space?
  
 %rotationAboutYaxis geogebra 
  
\begin{freeResponse}
$\hat{N}_\psi$ rotates points in euclidean space around the $y$-axis.
\end{freeResponse}

\end{problem}


\begin{problem}
Show that the matrix
\[
\hat{N}_\psi=\begin{bmatrix}
\cos\psi & 0 & \sin\psi\\
0 & 1 & 0\\
-\sin\psi & 0 & \cos\psi
\end{bmatrix}
\]
is orthogonal.

\begin{freeResponse}
\begin{align*}
\hat{N}_\psi \cdot \hat{N}_\psi^\transpose
&= \begin{bmatrix}
	\cos\psi & 0 & \sin\psi \\
	0 & 1 & 0 \\
	-\sin\theta & 0 & \cos\psi 
	\end{bmatrix} \cdot
	\begin{bmatrix}
	\cos\psi & 0 & -\sin\psi \\
	0 & 1 & 0 \\
	\sin\theta & 0 & \cos\psi 
	\end{bmatrix}  \\
	&= \begin{bmatrix}
	\cos^{2}\psi + \sin^{2} \psi & 0 & -\sin\psi \cos\psi + \sin\psi \cos\psi \\
	0 &  1 & 0 \\
	 -\sin\psi \cos\psi + \sin\psi \cos\psi & 0 & \cos^{2}\psi + \sin^{2} \psi 
	\end{bmatrix}
	=  \begin{bmatrix}
	1 & 0 & 0 \\
	0 & 1  & 0 \\
	0 & 0 & 1
	\end{bmatrix}
\end{align*}
\end{freeResponse}
\end{problem}


\begin{problem}
If a curve
\[
\hat{X}(t) = \left(\hat{x}(t),\hat{y}(t),\hat{z}(t)\right),\qquad b\le t\le e
\]
is moved by a transformation given by an orthogonal matrix $\hat{M}$,
show that its length is unchanged.
\begin{hint}
  Recall that the length of a curve is given by
\[
\int_{b}^{e} \sqrt{\frac{d\hat{X}}{dt} \bullet \frac{d\hat{X}}{dt}}\d t.
\]  
\end{hint}

\begin{freeResponse}
Let $\hat{X}_{2}(t) = \hat{X}(t) \cdot \hat{M}$, then the length of $\hat{X}_{2}$ is given by,
\begin{align*}
\int_{b}^{e} \sqrt{\frac{d\hat{X}_{2}}{dt} \bullet \frac{d\hat{X}_{2}}{dt}}\d t 
	&= \int_{b}^{e} \sqrt{ \left( \frac{d\hat{X}}{dt} \cdot \hat{M} \right) \bullet 
	\left( \frac{d\hat{X}}{dt} \cdot \hat{M} \right)}\d t \\
	&= \int_{b}^{e} \sqrt{ \left( \frac{d\hat{X}}{dt} \cdot \hat{M} \right) \cdot 
	\left( \frac{d\hat{X}}{dt} \cdot \hat{M} \right)^\transpose}\d t \\
	&= \int_{b}^{e} \sqrt{ \frac{d\hat{X}}{dt} \cdot \hat{M} \cdot 
	 \hat{M}^\transpose \cdot \frac{d\hat{X}}{dt}^\transpose}\d t \\
	 &= \int_{b}^{e} \sqrt{ \frac{d\hat{X}}{dt} \bullet \frac{d\hat{X}}{dt}}\d t
\end{align*}
\end{freeResponse}
\end{problem}


\begin{problem}
Summarize the results from this section. In particular, indicate which
results follow from the others.
\begin{freeResponse}
In this section we learned the definition of a \textit{rigid motion}. Next we gave a necessary and sufficient condition for a matrix to be a rigid motion: 
A matrix $\hat{M}$ is a rigid motion if and only if 
\[
\hat{M}\cdot\hat{M}^\transpose = I.
\]
In other words, a matrix $\hat{M}$ is a rigid motion if and only if it is orthogonal.
We proved that the set of orthogonal matrices (and hence the rigid motions defined by matrices) form a group. 
We also showed that the dot product is preserved under a transformation given by an orthogonal matrix $\hat{M}$ which we used to show that length is preserved under a transformation given by an orthogonal matrix $\hat{M}$.
%In this section we defined a rigid motion and determined that a matrix $\hat{M}$ defines a rigid motion if and only if it is orthogonal.  We proved that the set of rigid motions form a group. We also showed that the dot product is preserved by transformations, which helped us show that length is preserved by transformations. 
\end{freeResponse}
\end{problem}



\end{document}
