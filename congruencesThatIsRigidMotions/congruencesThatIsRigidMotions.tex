\documentclass{ximera}

\usepackage{microtype}
\usepackage{tikz}
\usepackage{tkz-euclide}
\usetkzobj{all}
\tikzstyle geometryDiagrams=[ultra thick,color=blue!50!black]

\renewcommand{\epsilon}{\varepsilon}



\title{Congruences, that is, rigid motions}
\begin{document}
\begin{abstract}
We explore rigid motions as matrix multiplication.
\end{abstract}
\maketitle

\subsection*{Transformations of euclidean space}

Consider the following mapping of euclidean space to itself:%
\[
\begin{bmatrix}
\underline{\hat{x}} & \underline{\hat{y}} & \underline{\hat{z}}%
\end{bmatrix}
=
\begin{bmatrix} \hat{x} & \hat{y} & \hat{z}%
\end{bmatrix}
\cdot\hat{M}
\]
where $\hat{M}$ is an invertible $3\times3$ matrix. Since the matrix
is invertible, this mapping is one-to-one and onto.

\begin{definition}
  A mapping of euclidean space to itself given by
  \[
\begin{bmatrix}
\underline{\hat{x}} & \underline{\hat{y}} & \underline{\hat{z}}%
\end{bmatrix}
=
\begin{bmatrix} \hat{x} & \hat{y} & \hat{z}%
\end{bmatrix}
\cdot\hat{M}
  \]
  is called a \dfn{rigid motion} if the distance between any two
points in euclidean space is left unchanged by the mapping, that is, for
any two points, $\hat{X}_{1}$ and $\hat{X}_{2}$ in euclidean space%
\[
d\left( \hat{X}_{1}  \cdot\hat{M},\hat{X}_{2}
\cdot\hat{M}\right)  =d\left(  \hat{X}_{1},\hat{X}_{2}\right).
\]
\end{definition}

It turns out that there is a special class of matrices that give rise
to rigid motions.

\begin{definition}
  A matrix $\hat{M}$ satisfying
  \[
  \hat{M}\cdot\hat{M}^\transpose=I=\begin{bmatrix}
  1 & 0 & 0\\
  0 & 1 & 0\\
  0 & 0 & 1
  \end{bmatrix}.
  \]
  is called an \dfn{orthogonal} matrix.
\end{definition}

\begin{problem}
  Prove that a matrix $\hat{M}$ defines a rigid motion via
  \[
\begin{bmatrix}
\underline{\hat{x}} & \underline{\hat{y}} & \underline{\hat{z}}%
\end{bmatrix}
=
\begin{bmatrix} \hat{x} & \hat{y} & \hat{z}%
\end{bmatrix}
\cdot\hat{M}
  \]
  if and only if it is orthogonal.

  \begin{hint}
    Note that the square of the distance between $\hat{X}_{1}$ and
    $\hat{X}_{1}$ is just the dot-product of the vector%
    \[
    \hat{V}=\hat{X}_{2}-\hat{X}_{1}%
    \]
    with itself. 
  \end{hint}
  \begin{hint}
    Hence we must show that $\hat{M}$ is orthogonal if
    \[
    \left(  \hat{V}  \cdot\hat{M} \right) \bullet\left(
    \hat{V}  \cdot\hat{M}\right)  =\hat{V}\bullet\hat{V}%
    \]
  \end{hint}
\end{problem}


\begin{problem}
  Consider
  \[
  \hat{M}=\begin{bmatrix}
  \cos\theta & \sin\theta & 0\\
  -\sin\theta & \cos\theta & 0\\
  0 & 0 & 1
  \end{bmatrix}
  \]
  Can you describe geometrically what this mapping is doing
  to the points in euclidean space?
\end{problem}


\begin{problem}
\label{14} Show that the matrix%
\[
\hat{M}=\begin{bmatrix}
\cos\theta & \sin\theta & 0\\
-\sin\theta & \cos\theta & 0\\
0 & 0 & 1
\end{bmatrix}
\]
is orthogonal. 
\end{problem}


\begin{problem}
  Consider
  \[
  \hat{M}=\begin{bmatrix}
  %
  \cos\psi & 0 & \sin\psi\\
  0 & 1 & 0\\
  -\sin\psi & 0 & \cos\psi
  \end{bmatrix}
  \]
  Can you describe geometrically what this mapping is doing
  to the points in euclidean space?
\end{problem}


\begin{problem}
Show that the matrix%
\[
\hat{M}=\begin{bmatrix}
%
\cos\psi & 0 & \sin\psi\\
0 & 1 & 0\\
-\sin\psi & 0 & \cos\psi
\end{bmatrix}
\]
is orthogonal.
\end{problem}


\begin{problem}
If a curve
\[
\hat{X}(t) = \left(\hat{x}(t),\hat{y}(t),\hat{z}(t)\right),\qquad b\le t\le e
\]
is moved by a transformation given by an orthogonal matrix $\hat{M}$,
show that its length is unchanged.
\begin{hint}
  Recall that the length of a curve is given by
\[
\int_{b}^{e} \sqrt{\left( \frac{d\hat{X}}{dt}\right) \cdot\left(
  \frac{d\hat{X}}{dt}\right)^\transpose}\d t.
\]  
\end{hint}
\end{problem}


\subsection*{Formula in $(x,y,z)$-coordinates for rigid motions
of euclidean space}

We now wish to figure out how to convert a transformation
\[
\begin{bmatrix}

\underline{\hat{x}} & \underline{\hat{y}} & \underline{\hat{z}}%
\end{bmatrix}
  =\begin{bmatrix}
\hat{x} & \hat{y} & \hat{z}%
\end{bmatrix}
          \cdot\hat{M} \label{100}%
\]
from euclidean space to $(x,y,z)$-coordinates. Recall to convert a
point from $(x,y,z)$-coordinates to euclidean coordinates, we write
\[
\begin{bmatrix}
\hat{x} & \hat{y} & \hat{z}%
\end{bmatrix}
=\begin{bmatrix}
x & y & Rz
\end{bmatrix}
=\begin{bmatrix}
x & y & z
\end{bmatrix}
\cdot\begin{bmatrix}
1 & 0 & 0\\
0 & 1 & 0\\
0 & 0 & R
\end{bmatrix}.
\]
So we have the diagram%
\begin{image}
\begin{tikzpicture}
  \node (xyzHat) {$\begin{bmatrix}\hat{x} & \hat{y} & \hat{z}\end{bmatrix}\in\mathbb{R}^3$};
  \node (xyzHatBar) [node distance=2cm,below of=xyzHat] {$\begin{bmatrix}\underline{\hat{x}} & \underline{\hat{y}} & \underline{\hat{z}}\end{bmatrix}$};
  \node (xyz) [node distance=6cm,right of=xyzHat] {$\begin{bmatrix}x & y & z\end{bmatrix}\in\mathbb{R}^3$};
  \node (xyzBar) [node distance=6cm,right of=xyzHatBar] {$\begin{bmatrix}\underline{x} & \underline{y} & \underline{z}\end{bmatrix}$};
  \draw[->] (xyz) to node[above] {$\cdot\left[\begin{smallmatrix}1 & 0 & 0\\ 0 & 1 & 0\\ 0 & 0 & R\end{smallmatrix}\right]$} (xyzHat);
  \draw[->] (xyz) to node[right] {$\cdot M = \text{?}$} (xyzBar);
  \draw[->] (xyzHat) to node[right] {$\cdot\hat{M}$} (xyzHatBar);
  \draw[->] (xyzBar) to node [above] {$\cdot\left[\begin{smallmatrix}1 & 0 & 0\\ 0 & 1 & 0\\ 0 & 0 & R\end{smallmatrix}\right]$} (xyzHatBar);
\end{tikzpicture}
\end{image}
\begin{problem}
Using the diagram above, explain why%
\[
\begin{bmatrix}
\underline{x} & \underline{y} & \underline{z}%
\end{bmatrix}
=\begin{bmatrix}
x & y & z
\end{bmatrix}
  \cdot\begin{bmatrix}
%
1 & 0 & 0\\
0 & 1 & 0\\
0 & 0 & R
\end{bmatrix}
  \cdot\hat{M}\cdot\begin{bmatrix}
%
1 & 0 & 0\\
0 & 1 & 0\\
0 & 0 & R^{-1}%
\end{bmatrix}.
\]

\end{problem}

%% So, if we let%
%% \[
%% M=\begin{bmatrix}
%% %
%% 1 & 0 & 0\\
%% 0 & 1 & 0\\
%% 0 & 0 & R
%% \end{bmatrix}
%%   \cdot\hat{M}\cdot\begin{bmatrix}
%% %
%% 1 & 0 & 0\\
%% 0 & 1 & 0\\
%% 0 & 0 & R^{-1}%
%% \end{bmatrix}
%%   ,
%% \]
%% then%
%% \begin{equation}
%% \begin{bmatrix}
%% %
%% \underline{x} & \underline{y} & \underline{z}%
%% \end{bmatrix}
%%   =\begin{bmatrix}
%% %
%% x & y & z
%% \end{bmatrix}
%%   \cdot M, \label{15}%
%% \end{equation}
%% that is $M$ is the matrix that gives the transformation $\left(
%% \ref{12}\right)  $ in $\left(  x,y,z\right)  $-coordinates.

%% So how would we check whether a transformation given in $\left(  x,y,z\right)
%% $-coordinates by a matrix $M$ preserves distances in euclidean space?
%% Again, starting from $\left(  \ref{16}\right)  $ this is just a substitution
%% problem:%
%% \begin{gather*}
%% \hat{M}\cdot\hat{M}^\transpose=I\\
%% M=\begin{bmatrix}
%% %
%% 1 & 0 & 0\\
%% 0 & 1 & 0\\
%% 0 & 0 & R
%% \end{bmatrix}
%%   \cdot\hat{M}\cdot\begin{bmatrix}
%% %
%% 1 & 0 & 0\\
%% 0 & 1 & 0\\
%% 0 & 0 & R^{-1}%
%% \end{bmatrix}
%%  \\
%% \begin{bmatrix}
%% %
%% 1 & 0 & 0\\
%% 0 & 1 & 0\\
%% 0 & 0 & R^{-1}%
%% \end{bmatrix}
%%   \cdot M\cdot\begin{bmatrix}
%% %
%% 1 & 0 & 0\\
%% 0 & 1 & 0\\
%% 0 & 0 & R
%% \end{bmatrix}
%%   =\hat{M}%
%% \end{gather*}


\begin{problem}
Show that the condition that a transformation $M$ in
$\left(x,y,z\right)$-coordinates preserves distances in euclidean
space is the condition that%
\[
M\cdot\begin{bmatrix}
1 & 0 & 0\\
0 & 1 & 0\\
0 & 0 & K^{-1}
\end{bmatrix}
  \cdot M^\transpose=\begin{bmatrix}
1 & 0 & 0\\
0 & 1 & 0\\
0 & 0 & K^{-1}
\end{bmatrix}. 
\]

\end{problem}

This is the condition (in $\left(  x,y,z\right)  $-coordinates) which affirms
that the transformation which takes the path $\left(  x(t),y(t),z(t)\right)  $
to the path $\left(  x(t),y(t),z(t)\right)  \cdot M$ preserves lengths of
tangent vectors at corresponding points. Therefore, by integrating, the
(total) length of the curve $\left\{  \left(  x(t),y(t),z(t)\right)  \cdot
M:b\leq t\leq e\right\}  $ is the same as the total length of the curve
$\left\{  \left(  x(t),y(t),z(t)\right)  :b\leq t\leq e\right\}  $.

\begin{problem}
Verify that your condition above is the correct condition by showing
that any $3\times3$ matrix $M$ that satisfying your condition also satisfies%
\[
\left(   V  \cdot M\right)  \bullet_{K}\left(   V
\cdot M\right)  =V\bullet_{K}V
\]
where%
\[
V=X_{2}-X_{1}.
\]
That is, the transformation given in $\left( x,y,z\right)
$-coordinates by a matrix $M$ that satisfies your condition preserves
the $K$-dot product.
\end{problem}

\begin{problem}
Summarize the results from this section. In particular, indicate which
results follow from the others.
\begin{freeResponse}
\end{freeResponse}
\end{problem}



\end{document}
