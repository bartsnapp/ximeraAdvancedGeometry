\documentclass{ximera}

\usepackage{microtype}
\usepackage{tikz}
\usepackage{tkz-euclide}
\usetkzobj{all}
\tikzstyle geometryDiagrams=[ultra thick,color=blue!50!black]

\renewcommand{\epsilon}{\varepsilon}



\title{Congruences, that is, rigid motions}
\begin{document}
\begin{abstract}
We explore rigid motions as matrix multiplication.
\end{abstract}
\maketitle

\subsection*{Transformations of Euclidean $3$-space}

Consider the following mapping of Euclidean $3$-space to itself:%
\begin{equation}
\begin{bmatrix}
%
\underline{\hat{x}} & \underline{\hat{y}} & \underline{\hat{z}}%
\end{bmatrix}
  =\begin{bmatrix}
%
\hat{x} & \hat{y} & \hat{z}%
\end{bmatrix}
  \cdot\hat{M} \label{12}%
\end{equation}
where $\hat{M}$ is an invertible $3\times3$ matrix. Then by matrix
multiplication%
\[
\begin{bmatrix}
%
\hat{x} & \hat{y} & \hat{z}%
\end{bmatrix}
  =\begin{bmatrix}
%
\underline{\hat{x}} & \underline{\hat{y}} & \underline{\hat{z}}%
\end{bmatrix}
  \cdot\hat{M}^{-1}%
\]
so that this mapping is $1-1$ and onto.

\begin{definition}
The mapping of Euclidean $3$-space to itself given by the rule $\left(
\ref{12}\right)  $ is called a rigid motion if the distance between any two
points in Euclidean $3$-space is left unchanged by the mapping, that is, for
any two points, $\hat{X}_{1}$ and $\hat{X}_{2}$ in Euclidean $3$-space%
\[
d\left( \hat{X}_{1}  \cdot\hat{M},\hat{X}_{2}
\cdot\hat{M}\right)  =d\left(  \hat{X}_{1},\hat{X}_{2}\right)  .
\]

\end{definition}

We saw in formula $\left(  \ref{13}\right)  $ that the square of the distance
between $\hat{X}_{1}$ and $\hat{X}_{1}$ is just the dot-product of the vector%
\[
\hat{V}=\hat{X}_{2}-\hat{X}_{1}%
\]
with itself. So the transformation given by the matrix $\hat{M}$ will leave
distances unchanged if and only if, for all vectors $\hat{V}$,%
\[
\left(  \hat{V}  \cdot\hat{M} \right) \bullet\left(
\hat{V}  \cdot\hat{M}\right)  =\hat{V}\bullet\hat{V}%
\]
that is%
\[
\left( \hat{V}  \cdot\hat{M} \right) \cdot\left(
\hat{V}  \cdot\hat{M}\right)  ^\transpose=\hat{V}  \cdot
\hat{V}  ^\transpose.
\]
We can rewrite this requirement as%
\begin{equation}
\hat{V}  \cdot\hat{M}\cdot\hat{M}^\transpose\cdot  \hat
{V}  ^\transpose=\hat{V}  \cdot\hat{V}  ^\transpose
\label{116}%
\end{equation}
for all vectors $\hat{V}$. Condition $\left(  \ref{116}\right)  $ is certainly
satisfied for all vectors $\hat{V}$ if
\begin{equation}
\hat{M}\cdot\hat{M}^\transpose=I=\begin{bmatrix}
%
1 & 0 & 0\\
0 & 1 & 0\\
0 & 0 & 1
\end{bmatrix}
  . \label{16}%
\end{equation}


\begin{problem}
Suppose $\hat{M}$ is such that%
\[
\hat{M}\cdot\hat{M}^\transpose=\begin{bmatrix}
%
1 & 2 & 0\\
2 & 1 & 0\\
0 & 0 & 1
\end{bmatrix}
  .
\]
Find a vector $\hat{V}=\left(  \hat{a},\hat{b},\hat{c}\right)  $ such that%
\[
\hat{V} \cdot\hat{M}\cdot\hat{M}^\transpose\cdot\hat{V}  ^\transpose\neq\hat{V}  \cdot\hat{V}
^\transpose.
\]

\end{problem}

In fact, reasoning as in this last Problem, one can show that, if a matrix
$\hat{M}$ satisfies the condition $\left(  \ref{116}\right)  $ for
\textit{all} vectors $\hat{V}$, then the matrix $\hat{M}$ also satisfies
$\left(  \ref{16}\right)  $. A matrix $\hat{M}$ satisfying $\left(
\ref{16}\right)  $ is called an \textit{orthogonal matrix}. [DS,316-321]

\begin{problem}
\label{14} Show that the matrix%
\[
\hat{M}=\begin{bmatrix}
\cos\theta & \sin\theta & 0\\
-\sin\theta & \cos\theta & 0\\
0 & 0 & 1
\end{bmatrix}
\]
is orthogonal. Can you describe geometrically what this rigid motion is doing
to the points in Euclidean $3$-space?
\end{problem}

\begin{problem}
Show that the matrix%
\[
\hat{M}=\begin{bmatrix}
%
\cos\psi & 0 & \sin\psi\\
0 & 1 & 0\\
-\sin\psi & 0 & \cos\psi
\end{bmatrix}
\]
is orthogonal. Can you describe geometrically what this rigid motion is doing
to the points in Euclidean $3$-space?
\end{problem}

For any curve%
\[
\hat{X}\left(  t\right)  =\left(  \hat{x}\left(  t\right)  ,\hat{y}\left(
t\right)  ,\hat{z}\left(  t\right)  \right)  ,\;b\leq t\leq e,
\]
its length is%
\[%
%TCIMACRO{\dint \nolimits_{b}^{e}}%
%BeginExpansion
\int\nolimits_{b}^{e}
%EndExpansion
\sqrt{\left(  \frac{d\hat{X}}{dt}\right)  \cdot\left(  \frac{d\hat{X}}%
{dt}\right)  ^\transpose}\,dt.
\]
Suppose now that the curve is moved by a transformation given by an orthogonal
matrix $\hat{M}$. After it is moved, its length is given by%
\[%
%TCIMACRO{\dint \nolimits_{b}^{e}}%
%BeginExpansion
\int\nolimits_{b}^{e}
\sqrt{\left(  \frac{d\left(  \hat{X}\cdot\hat{M}\right)  }{dt}\right)
\cdot\left(  \frac{d\left(  \hat{X}\cdot\hat{M}\right)  }{dt}\right)  ^\transpose%
}\,dt.
\]
But%
\begin{align*}%
%TCIMACRO{\dint \nolimits_{b}^{e}}%
%BeginExpansion
\int_{b}^{e}
%EndExpansion
\sqrt{\left(  \frac{d\left(  \hat{X}\cdot\hat{M}\right)  }{dt}\right)
\cdot\left(  \frac{d\left(  \hat{X}\cdot\hat{M}\right)  }{dt}\right)  ^\transpose}\,dt
&  =%
%TCIMACRO{\dint \nolimits_{b}^{e}}%
%BeginExpansion
\int_{b}^{e}
%EndExpansion
\sqrt{\left(  \frac{d\hat{X}}{dt}\cdot\hat{M}\right)  \cdot\left(  \frac
{d\hat{X}}{dt}\cdot\hat{M}\right)  ^\transpose}\,dt\\
&  =%
%TCIMACRO{\dint \nolimits_{b}^{e}}%
%BeginExpansion
\int_{b}^{e}
%EndExpansion
\sqrt{\left(  \frac{d\hat{X}}{dt}\cdot\hat{M}\right)  \cdot\left(  \hat{M}%
^\transpose\cdot\frac{d\hat{X}}{dt}^\transpose\right)  }dt\\
&  =%
%TCIMACRO{\dint \nolimits_{b}^{e}}%
%BeginExpansion
\int\nolimits_{b}^{e}
%EndExpansion
\sqrt{\left(  \frac{d\hat{X}}{dt}\right)  \cdot\left(  \frac{d\hat{X}}%
{dt}\right)  ^\transpose}\,dt.
\end{align*}


\begin{corollary}
If a curve
\[
\hat{X}\left(  t\right)  =\left(  \hat{x}\left(  t\right)  ,\hat{y}\left(
t\right)  ,\hat{z}\left(  t\right)  \right)  ,\;b\leq t\leq e
\]
is moved by a transformation given by an orthogonal matrix $\hat{M}$, its
length is unchanged.\pagebreak
\end{corollary}

\subsection*{Formula in $\left(  x,y,z\right)  $-coordinates for rigid motions
of Euclidean $3$-space}

We now wish to figure out how to write the transformation $\left(
\ref{12}\right)  $ in $\left(  x,y,z\right)  $-coordinates. This is a simple
substitution problem:%
\begin{equation}
\begin{bmatrix}
%
\underline{\hat{x}} & \underline{\hat{y}} & \underline{\hat{z}}%
\end{bmatrix}
  =\begin{bmatrix}
%
\hat{x} & \hat{y} & \hat{z}%
\end{bmatrix}
          \cdot\hat{M} \label{100}%
\end{equation}%
\begin{gather*}
\begin{bmatrix}
%
\hat{x} & \hat{y} & \hat{z}%
\end{bmatrix}
=\begin{bmatrix}
%
x & y & Rz
\end{bmatrix}
=\begin{bmatrix}
%
x & y & z
\end{bmatrix}
\cdot\begin{bmatrix}
%
1 & 0 & 0\\
0 & 1 & 0\\
0 & 0 & R
\end{bmatrix}\\
\begin{bmatrix}
%
\underline{\hat{x}} & \underline{\hat{y}} & \underline{\hat{z}}%
\end{bmatrix}
=\begin{bmatrix}
%
\underline{x} & \underline{y} & R\underline{z}%
\end{bmatrix}
=\begin{bmatrix}
%
\underline{x} & \underline{y} & \underline{z}%
\end{bmatrix}
\cdot\begin{bmatrix}
%
1 & 0 & 0\\
0 & 1 & 0\\
0 & 0 & R
\end{bmatrix}
\end{gather*}
So we have the diagram%
\begin{image}
\begin{tikzpicture}
  \node (xyzHat) {$\begin{bmatrix}\hat{x} & \hat{y} & \hat{z}\end{bmatrix}\in\mathbb{R}^3$};
  \node (xyzHatBar) [node distance=2cm,below of=xyzHat] {$\begin{bmatrix}\underline{\hat{x}} & \underline{\hat{y}} & \underline{\hat{z}}\end{bmatrix}$};
  \node (xyz) [node distance=6cm,right of=xyzHat] {$\begin{bmatrix}x & y & z\end{bmatrix}\in\mathbb{R}^3$};
  \node (xyzBar) [node distance=6cm,right of=xyzHatBar] {$\begin{bmatrix}\underline{x} & \underline{y} & \underline{z}\end{bmatrix}$};
  \draw[->] (xyz) to node[above] {$\cdot\left[\begin{smallmatrix}1 & 0 & 0\\ 0 & 1 & 0\\ 0 & 0 & R\end{smallmatrix}\right]$} (xyzHat);
  \draw[->] (xyz) to node[right] {$\cdot M = \text{?}$} (xyzBar);
  \draw[->] (xyzHat) to node[right] {$\cdot\hat{M}$} (xyzHatBar);
  \draw[->] (xyzHatBar) to node [above] {$\cdot\left[\begin{smallmatrix}1 & 0 & 0\\ 0 & 1 & 0\\ 0 & 0 & R^{-1}\end{smallmatrix}\right]$} (xyzBar);
\end{tikzpicture}
\end{image}
\begin{problem}
Starting from the equality $\left(\ref{100}\right)$ describing the
transformation in Euclidean coordinates, explain why%
\[
\begin{bmatrix}
\underline{x} & \underline{y} & \underline{z}%
\end{bmatrix}
=\begin{bmatrix}
x & y & z
\end{bmatrix}
  \cdot\begin{bmatrix}
%
1 & 0 & 0\\
0 & 1 & 0\\
0 & 0 & R
\end{bmatrix}
  \cdot\hat{M}\cdot\begin{bmatrix}
%
1 & 0 & 0\\
0 & 1 & 0\\
0 & 0 & R^{-1}%
\end{bmatrix}.
\]

\end{problem}

So, if we let%
\[
M=\begin{bmatrix}
%
1 & 0 & 0\\
0 & 1 & 0\\
0 & 0 & R
\end{bmatrix}
  \cdot\hat{M}\cdot\begin{bmatrix}
%
1 & 0 & 0\\
0 & 1 & 0\\
0 & 0 & R^{-1}%
\end{bmatrix}
  ,
\]
then%
\begin{equation}
\begin{bmatrix}
%
\underline{x} & \underline{y} & \underline{z}%
\end{bmatrix}
  =\begin{bmatrix}
%
x & y & z
\end{bmatrix}
  \cdot M, \label{15}%
\end{equation}
that is $M$ is the matrix that gives the transformation $\left(
\ref{12}\right)  $ in $\left(  x,y,z\right)  $-coordinates.

So how would we check whether a transformation given in $\left(  x,y,z\right)
$-coordinates by a matrix $M$ preserves distances in Euclidean $3$-space?
Again, starting from $\left(  \ref{16}\right)  $ this is just a substitution
problem:%
\begin{gather*}
\hat{M}\cdot\hat{M}^\transpose=I\\
M=\begin{bmatrix}
%
1 & 0 & 0\\
0 & 1 & 0\\
0 & 0 & R
\end{bmatrix}
  \cdot\hat{M}\cdot\begin{bmatrix}
%
1 & 0 & 0\\
0 & 1 & 0\\
0 & 0 & R^{-1}%
\end{bmatrix}
 \\
\begin{bmatrix}
%
1 & 0 & 0\\
0 & 1 & 0\\
0 & 0 & R^{-1}%
\end{bmatrix}
  \cdot M\cdot\begin{bmatrix}
%
1 & 0 & 0\\
0 & 1 & 0\\
0 & 0 & R
\end{bmatrix}
  =\hat{M}%
\end{gather*}


\begin{problem}
Finish the matrix algebra computations just above to show that the condition
that a transformation $M$ in $\left(  x,y,z\right)  $-coordinates preserves
distances in Euclidean $3$-space is the condition that%
\begin{equation}
M\cdot\begin{bmatrix}
%
1 & 0 & 0\\
0 & 1 & 0\\
0 & 0 & K^{-1}%
\end{bmatrix}
  \cdot M^\transpose=\begin{bmatrix}
%
1 & 0 & 0\\
0 & 1 & 0\\
0 & 0 & K^{-1}%
\end{bmatrix}
  . \label{17}%
\end{equation}

\end{problem}

This is the condition (in $\left(  x,y,z\right)  $-coordinates) which affirms
that the transformation which takes the path $\left(  x(t),y(t),z(t)\right)  $
to the path $\left(  x(t),y(t),z(t)\right)  \cdot M$ preserves lengths of
tangent vectors at corresponding points. Therefore, by integrating, the
(total) length of the curve $\left\{  \left(  x(t),y(t),z(t)\right)  \cdot
M:b\leq t\leq e\right\}  $ is the same as the total length of the curve
$\left\{  \left(  x(t),y(t),z(t)\right)  :b\leq t\leq e\right\}  $.

\begin{problem}
Check that $\left(  \ref{17}\right)  $ is the correct condition by showing
that any $3\times3$ matrix $M$ that satisfies $\left(  \ref{17}\right)  $ also
satisfies%
\[
\left(   V  \cdot M\right)  \bullet_{K}\left(   V
\cdot M\right)  =V\bullet_{K}V
\]
where%
\[
V=X_{2}-X_{1}.
\]
That is, the transformation given in $\left(  x,y,z\right)  $-coordinates by a
matrix $M$ that satisfies $\left(  \ref{17}\right)  $ preserves the $K$-dot product.
\end{problem}

\begin{problem}
Summarize the results from this section. In particular, indicate which
results follow from the others.
\begin{freeResponse}
\end{freeResponse}
\end{problem}



\end{document}
