\documentclass{ximera}

\usepackage{microtype}
\usepackage{tikz}
\usepackage{tkz-euclide}
\usetkzobj{all}
\tikzstyle geometryDiagrams=[ultra thick,color=blue!50!black]

\renewcommand{\epsilon}{\varepsilon}



\title{Congruences, that is, rigid motions}
\begin{document}
\begin{abstract}
We explore rigid motions as matrix multiplication.
\end{abstract}
\maketitle

\subsection*{Transformations of euclidean space}

Consider the following mapping of euclidean space to itself:%
\[
\begin{bmatrix}
\underline{\hat{x}} & \underline{\hat{y}} & \underline{\hat{z}}%
\end{bmatrix}
=
\begin{bmatrix} \hat{x} & \hat{y} & \hat{z}%
\end{bmatrix}
\cdot\hat{M}
\]
where $\hat{M}$ is an invertible $3\times3$ matrix. Since the matrix
is invertible, this mapping is one-to-one and onto.

\begin{definition}
  A mapping of euclidean space to itself given by
  \[
\begin{bmatrix}
\underline{\hat{x}} & \underline{\hat{y}} & \underline{\hat{z}}%
\end{bmatrix}
=
\begin{bmatrix} \hat{x} & \hat{y} & \hat{z}%
\end{bmatrix}
\cdot\hat{M}
  \]
  is called a \dfn{rigid motion} if the distance between any two
points in euclidean space is left unchanged by the mapping, that is, for
any two points, $\hat{X}_{1}$ and $\hat{X}_{2}$ in euclidean space%
\[
d\left( \hat{X}_{1}  \cdot\hat{M},\hat{X}_{2}
\cdot\hat{M}\right)  =d\left(  \hat{X}_{1},\hat{X}_{2}\right).
\]
\end{definition}

It turns out that there is a special class of matrices that give rise
to rigid motions.

\begin{definition}
  A matrix $\hat{M}$ satisfying
  \[
  \hat{M}\cdot\hat{M}^\transpose=I=\begin{bmatrix}
  1 & 0 & 0\\
  0 & 1 & 0\\
  0 & 0 & 1
  \end{bmatrix}.
  \]
  is called an \dfn{orthogonal} matrix.
\end{definition}

\begin{problem}
  Prove that a matrix $\hat{M}$ defines a rigid motion (a conrguence)
  via
  \[
\begin{bmatrix}
\underline{\hat{x}} & \underline{\hat{y}} & \underline{\hat{z}}%
\end{bmatrix}
=
\begin{bmatrix} \hat{x} & \hat{y} & \hat{z}%
\end{bmatrix}
\cdot\hat{M}
  \]
  if and only if it is orthogonal.

  \begin{hint}
    Note that the square of the distance between $\hat{X}_{1}$ and
    $\hat{X}_{1}$ is just the dot-product of the vector%
    \[
    \hat{V}=\hat{X}_{2}-\hat{X}_{1}%
    \]
    with itself. 
  \end{hint}
  \begin{hint}
    Hence we must show that $\hat{M}$ is orthogonal if
    \[
    \left(  \hat{V}  \cdot\hat{M} \right) \bullet\left(
    \hat{V}  \cdot\hat{M}\right)  =\hat{V}\bullet\hat{V}%
    \]
  \end{hint}
\end{problem}


\begin{problem}
  Show that, if $\hat{M}$ is orthagonal, then the transformation
  \[
  \begin{bmatrix}
    \underline{\hat{x}} & \underline{\hat{y}} & \underline{\hat{z}}
  \end{bmatrix}
  =
  \begin{bmatrix}
    \hat{x} & \hat{y} & \hat{z}
  \end{bmatrix}
  \cdot \hat{M}.
  \]
  takes the set of points $\left(\hat{x},\hat{y},\hat{z}\right)$ such
  that
\[
\hat{x}^2 + \hat{y}^2 + \hat{z}^2 = R^2
\]
to the set of points
$\left(\underline{\hat{x}},\underline{\hat{y}},\underline{\hat{z}}\right)$
such that
\[
\underline{\hat{x}}^2 + \underline{\hat{y}}^{2} + \underline{\hat{z}}^{2}=R^2.
\]
That is, $\hat{M}$ gives a one-to-one and onto mapping of the $R$-sphere to
itself.
\begin{hint}
  Write the equation
  \[
  \underline{\hat{x}}^2 + \underline{\hat{y}}^{2} + \underline{\hat{z}}^{2}=R^2
  \]
  as
  \[
  \begin{bmatrix}
    \underline{\hat{x}} & \underline{\hat{y}} & \underline{\hat{z}}%
  \end{bmatrix}  
  \cdot
  \begin{bmatrix}
    \underline{\hat{x}}\\
    \underline{\hat{y}}\\
    \underline{\hat{z}}
  \end{bmatrix}  =R^2.
  \]
\end{hint}
\end{problem}





\begin{problem}
Show that the set of orthogonal matrices $\hat{M}$ form a group.  That
is, show that
\begin{enumerate}
\item the product of two orthogonal matrices is orthogonal,
\item the identity matrix is orthogonal,
\item the inverse matrix $\hat{M}^{-1}$ of a orthogonal matrix $\hat{M}$ is orthogonal.
\end{enumerate}
\end{problem}




\begin{problem}
  Consider
  \[
  \hat{M}=\begin{bmatrix}
  \cos\theta & \sin\theta & 0\\
  -\sin\theta & \cos\theta & 0\\
  0 & 0 & 1
  \end{bmatrix}
  \]
  Can you describe geometrically what this mapping is doing
  to the points in euclidean space?
\end{problem}


\begin{problem}
\label{14} Show that the matrix%
\[
\hat{M}=\begin{bmatrix}
\cos\theta & \sin\theta & 0\\
-\sin\theta & \cos\theta & 0\\
0 & 0 & 1
\end{bmatrix}
\]
is orthogonal. 
\end{problem}


\begin{problem}
  Consider
  \[
  \hat{M}=\begin{bmatrix}
  %
  \cos\psi & 0 & \sin\psi\\
  0 & 1 & 0\\
  -\sin\psi & 0 & \cos\psi
  \end{bmatrix}
  \]
  Can you describe geometrically what this mapping is doing
  to the points in euclidean space?
\end{problem}


\begin{problem}
Show that the matrix%
\[
\hat{M}=\begin{bmatrix}
%
\cos\psi & 0 & \sin\psi\\
0 & 1 & 0\\
-\sin\psi & 0 & \cos\psi
\end{bmatrix}
\]
is orthogonal.
\end{problem}


\begin{problem}
If a curve
\[
\hat{X}(t) = \left(\hat{x}(t),\hat{y}(t),\hat{z}(t)\right),\qquad b\le t\le e
\]
is moved by a transformation given by an orthogonal matrix $\hat{M}$,
show that its length is unchanged.
\begin{hint}
  Recall that the length of a curve is given by
\[
\int_{b}^{e} \sqrt{\frac{d\hat{X}}{dt} \bullet \frac{d\hat{X}}{dt}}\d t.
\]  
\end{hint}
\end{problem}


\end{document}
