\documentclass{ximera}

\usepackage{microtype}
\usepackage{tikz}
\usepackage{tkz-euclide}
\usetkzobj{all}
\tikzstyle geometryDiagrams=[ultra thick,color=blue!50!black]

\renewcommand{\epsilon}{\varepsilon}



\title{Euclidean three-space as a metric space}

\begin{document}
\begin{abstract}
In this activity we will work in three dimensional space and see the
importance of the dot product to geometry.
\end{abstract}
\maketitle

\subsection*{Points and vectors in euclidean 3-space}

%In this book, we will study plane, spherical and hyperbolic geometry in \textit{two}-dimensions. Each one of these geometries looks the same at each of its points and it also looks  the same in every direction emanating from any of its points. But to study them all at the same time and in a uniform way we will need to visualize them all as different surfaces lying in some common \textit{three}-dimensional space. Let's examine this ``ambient'' space.

Up to this point in this book, we have studied two and three-dimensional shapes, the \textit{objects} found in two (and three) dimensional geometry. Now we will study geometries as objects unto themselves. In particular, we will study plane, spherical, and hyperbolic geometry in two-dimensions. What does it mean to study a ``whole geometry?'' Well, geometry is the study of 
\begin{itemize}
\item length,
\item angle, and
\item area.
\end{itemize}
Hence, if we can explain how to compute lengths, angles, and areas, we
understand something about the geometry as a whole. Moreover, the
congruences found in geometry are simply the transformations that
preserve length, angle, and area; the \textit{rigid motions}. Thus we
now seek to understand how to compute length, angle, and area, along
with a description of the congruences for euclidean, spherical, and
hyperbolic geometry.  Our plan of attack is as follows: we will
visualize each of the geometries as different surfaces lying in some
common three-dimensional space. This will allow us to use the
techniques from calculus and linear algebra to describe euclidean,
spherical, and hyperbolic geometry in a unified way. We start by
introducing the concepts we need in the more familiar setting in
ordinary $3$-dimensional euclidean space:
\[
\mathbb{R}^{3}=\left\{  \left(  \hat{x},\hat{y},\hat{z}\right)  :\hat{x}%
,\hat{y},\hat{z}\in\mathbb{R}\right\}.
\]
We reserve the notation $\left( x,y,z\right) $ for some new
coordinates that we will put on the `same' objects later in this
course. In euclidean space, there is a standard way to measure
distance between two points%
\begin{align*}
\hat{X}_{1}  &  =\left(  \hat{x}_{1},\hat{y}_{1},\hat{z}_{1}\right) \\
\hat{X}_{2}  &  =\left(  \hat{x}_{2},\hat{y}_{2},\hat{z}_{2}\right)  ,
\end{align*}
namely%
\[
d\left(  \hat{X}_{1},\hat{X}_{2}\right)  =\sqrt{\left(  \hat{x}_{2}-\hat
{x}_{1}\right)  ^{2}+\left(  \hat{y}_{2}-\hat{y}_{1}\right)  ^{2}+\left(
\hat{z}_{2}-\hat{z}_{1}\right)  ^{2}}. \label{0}%
\]

When you see two point $\hat{X}_1$ and $\hat{X}_2$ in what follows,
the `hats' mean that distance between points is measured by the
formula above. One more thing, in euclidean three-space it will be
important throughout to make the distinction between \textbf{points}
and \textbf{vectors}: Although each will be represented by a triple of
real numbers we will use%
\[
\hat{X}=\left(  \hat{x},\hat{y},\hat{z}\right)
\]
to denote \textbf{points}, that is, \textbf{position} in euclidean $3$-space,
and%
\[
\hat{V}=\left(  \hat{a},\hat{b},\hat{c}\right)
\]
to denote \textbf{vectors}, that is, \textbf{displacement} by which we mean
the amount and direction a given point is being moved. So vectors always
indicate \textit{motion} from an explicit (or implicit) \textit{point} of
reference. 

\subsection*{The dot product determines length, angle, and area}

There are various operations we can perform on one or more vectors when we
think of them as emanating from the same point in euclidean $3$-space. The
first is the dot product of two vectors.

\begin{definition}
The dot product of two vectors%
\begin{align*}
\hat{V}_{1}  &  =\left(  \hat{a}_{1},\hat{b}_{1},\hat{c}_{1}\right), \\
\hat{V}_{2}  &  =\left(  \hat{a}_{2},\hat{b}_{2},\hat{c}_{2}\right),
\end{align*}
emanating from the same point in 3-dimensional euclidean space is
defined as the real number given by the formula%
\[
\hat{a}_{1}\hat{a}_{2}+\hat{b}_{1}\hat{b}_{2}+\hat{c}_{1}\hat{c}_{2}%
\]
or in matrix notation as%
\[
\begin{bmatrix}
\hat{a}_{1} & \hat{b}_{1} & \hat{c}_{1}%
\end{bmatrix}
\begin{bmatrix}
\hat{a}_{2}\\
\hat{b}_{2}\\
\hat{c}_{2}%
\end{bmatrix}.
\]
It is also denoted as%
\[
\hat{V}_{1}\bullet\hat{V}_{2}%
\]
or in matrix notation as%
\[
\hat{V}_{1}\cdot \hat{V}_{2}^\transpose.
\]

\end{definition}

\begin{problem}
Give the formula for the length $\left\vert \hat{V}\right\vert $ of a vector
$\hat{V}=\left(  \hat{a},\hat{b},\hat{c}\right)  $ in 3-dimensional euclidean
space in terms of dot product.

\begin{onlineOnly}
\geogebra[sdz,stb,smb]{xZrB7Cck}{640}{640}
\end{onlineOnly}

\begin{freeResponse}
Let 
\[
\hat{V}=\left(  \hat{a},\hat{b},\hat{c}\right)
\]
be a vector with the origin as it's point of reference. The length of $\hat{V}$ is

\[
\left\vert \hat{V}\right\vert  = \sqrt{\left(  \hat{a}-\hat
{0}\right)  ^{2}+\left(  \hat{b}-\hat{0}\right)  ^{2}+\left(
\hat{c}-\hat{0}\right)  ^{2}} \label{0} = \sqrt{\hat{a} ^2+\hat{b}^2+\hat{c} ^2} = \sqrt{\hat{V}\bullet \hat{V}}
\]
\end{freeResponse}
\end{problem}

\begin{problem}
  Prove that
  \[
  \hat{x}^2 + \hat{y}^2 + \hat{z}^2 = R^2
  \]
  is a sphere in euclidean three-space.
  
\begin{hint}
Remember, a sphere in euclidean three-space is the set of points equidistant from the origin.
\end{hint}

\begin{freeResponse}
Let $\hat{V}= \left(\hat{x},\hat{y}, \hat{z}\right)$ be a vector with it's tail on the origin and tip on the surface above.
\[
\left\vert \hat{V}\right\vert = \sqrt{\hat{V}\bullet \hat{V}} = \sqrt{ \hat{x}^2+\hat{y}^2 + \hat{z}^2}= \sqrt{R^2} = \left\vert R \right\vert
\] 
Therefore, every point on the surface is equidistant from the origin.
\end{freeResponse}  
\end{problem}


\begin{lemma}[Law of Cosines]
\label{110} The (smaller) angle $\theta$ between two
vectors $\hat{V}_{1}$ and $\hat{V}_{2}$ emanating from $O=\left(
0,0,0\right)  $ satisfies the relation%
\[
\left\vert \hat{V}_{2}-\hat{V}_{1}\right\vert ^{2}=\left\vert \hat{V}%
_{1}\right\vert ^{2}+\left\vert \hat{V}_{2}\right\vert ^{2}-2\left\vert
\hat{V}_{1}\right\vert \cdot\left\vert \hat{V}_{2}\right\vert \cdot
\cos\theta.
\]
\end{lemma}

Now we will prove this lemma. Your task is to fill-in the details of
the proof below.

\begin{problem}
Start by noting that without loss of generality we can assume that
$\left\vert \hat{V} _{1}\right\vert \le\left\vert
\hat{V}_{2}\right\vert$. Consider the triangle with one side given by
the segment from $O=\left( 0,0,0\right) $ to the endpoint $P_{1}$ of
$\hat{V}_{1}$, with a second side $S_{2}$ given by the segment from
$O$ to the endpoint $P_{2}$ of $\hat{V}_{2}$ and with the third side
given by the segment joining $P_{1}$ and $P_{2}$. Let $P$ be the
point on $S_{2}$ so that the segment between $P_{1}$ and $P$ is
perpendicular to $S_{2}$.
\begin{itemize}
\item Illustrate the diagram described above. 
%\begin{freeResponse}

%lawOfCosines Geogebra

%\end{freeResponse}
\item Now explain how the Pythagorean theorem gives:
\begin{align*}
|P_{1}P_{2}|^{2} -  |P_{2}P|^{2} &= |PP_{1}|^{2}\\
&= |OP_{1}|^{2} - |OP|^{2}\\
\end{align*}

\begin{freeResponse}
$\triangle PP_{2}P_{1}$ is a right triangle, so by the pythagorean theorem:

\begin{align*}
|P_{1}P_{2}|^{2} - |P_{2}P|^{2} = |PP_{1}|^{2}
\end{align*}
Also by the pythagorean theorem, $ \triangle PPP_{1}$ gives:
\begin{align*}  |PP_{1}|^{2} = |OP_{1}|^{2} - |OP|^{2}
\end{align*}

\end{freeResponse}

\item Explain the following lines:
\begin{align*}
|P_{1}P_{2}|^{2} &= |OP_{1}|^{2}+\left(|P_{2}P|^{2}-|OP|^{2}\right) \\
&=|OP_{1}|^{2}+\left(|P_{2}P|+|OP|\right)\left(|P_{2}P|-|OP| \right)
\end{align*}

\begin{freeResponse}
Add $|P_{1}P_{2}|^2$ to both sides of the previous equation and then use the difference of squares to factor $|P_{2}P|^{2}- |OP|^2$.
\end{freeResponse}

\item Explain the following lines:
\begin{align*}
&=|OP_{1}|^{2}+|OP_{2}| \left(|P_{2}P|-|OP|\right)\\
&=|OP_{1}|^{2}+|OP_{2}| \left(|OP_{2}| -2| OP| \right)
\end{align*}

\begin{freeResponse}
The length $|P_{2}P|+|OP|$ is the same as $|OP_{2}|$, then $|OP_{2}| - 2|OP| = |P_{2}P|-|OP|$.
\end{freeResponse}
\item Now consider,
\[
|OP| =|OP_{1}| \cdot\cos\theta.
\]
and explain how this completes the proof.

\begin{freeResponse}
We have $|\hat{V}_{1}| = |OP_{1}|$ and $|\hat{V}_{2}| = |OP_{2}|$, then using the work above and substitution:
\begin{align*} 
|\hat{V}_{2} - \hat{V}_{1}|^2 &= |P_{1}P_{2}|^{2}\\
&=|OP_{1}|^{2}+|OP_{2}| \left(|OP_{2}| -2| OP| \right)\\
&=|\hat{V}_{1}|^2 + |\hat{V}_{2}|\left(|\hat{V}_{2}| -2|\hat{V}_{1}| \cdot\cos\theta \right) \\
&=|\hat{V}_{1}|^2 + |\hat{V}_{2}|^2 -2|\hat{V}_{1}| \cdot |\hat{V}_{2}| \cdot\cos\theta
\end{align*}
\end{freeResponse}

\end{itemize}
\end{problem}



%% \begin{problem}
%% What can you say about the cosine of the larger of the two angles between two
%% vectors $\hat{V}_{1}$ and $\hat{V}_{2}$, that is about $\left(  360^{\circ
%% }-\theta\right)  $?
%% \end{problem}

\begin{theorem}
\label{111}The angle $\theta$ between two vectors $\hat{V}_{1}$ and
$\hat{V}_{2}$ emanating from the same point in euclidean $3$-space satisfies
the relation
\begin{equation}
\hat{V}_{1}\bullet\hat{V}_{2}=\left\vert \hat{V}_{1}\right\vert \cdot
\left\vert \hat{V}_{2}\right\vert \cdot\cos\theta. \label{2}%
\end{equation}
%[DS,30ff]
\end{theorem}

Now we will prove this theorem. Your task is to fill-in the details of
the proof below.

%angleBetweenDotProduct Geogebra

\begin{problem}
Multiplying out using the definition and algebraic properties of dot product,%
\begin{align*}
\left\vert \hat{V}_{2}-\hat{V}_{1}\right\vert ^{2}  &  =\left(  \hat{V}%
_{2}-\hat{V}_{1}\right)  \bullet\left(  \hat{V}_{2}-\hat{V}_{1}\right) \\
&  =\left\vert \hat{V}_{1}\right\vert ^{2}+\left\vert \hat{V}_{2}\right\vert
^{2}-2\left(  \hat{V}_{1}\bullet\hat{V}_{2}\right)  .
\end{align*}
Now apply the previous lemma.

Explain how this proves the theorem.
\begin{freeResponse}
The previous lemma gives:
\[
|\hat{V}_{1}|\cdot|\hat{V}_{2}|\cdot\cos\theta = -\frac{1}{2}\left(|\hat{V}_{2}-\hat{V}_{1}|^2 -|\hat{V}_{1}|^2 - |\hat{V}_{2}|^2\right)
\]
Therefore, by the definition and algebraic properties of dot product,
\[
\hat{V}_{1}\bullet\hat{V}_{2} = |\hat{V}_{1}|\cdot|\hat{V}_{2}|\cdot\cos\theta
\]
\end{freeResponse} 
\end{problem}

The significance of the theorem above is that the measure of angles between
vectors depends only on the definition of the dot product.

\begin{corollary}
The formula for the angle $\theta$ between two vectors $\hat{V}_{1}=\left(
\hat{a}_{1},\hat{b}_{1},\hat{c}_{1}\right)  $ and $\hat{V}_{2}=\left(  \hat
{a}_{2},\hat{b}_{2},\hat{c}_{2}\right)  $ in $3$-dimensional euclidean space
depends only on the dot products of the two vectors with themselves and with
each other. Namely%
\[
\theta=\arccos\left(  \frac{\hat{V}_{1}\bullet\hat{V}_{2}%
}{\left\vert \hat{V}_{1}\right\vert \cdot\left\vert \hat{V}_{2}\right\vert
}\right)  .
\]

\end{corollary}

In fact it is also true that the formula for the area of the parallelogram
determined by two vectors $\hat{V}_{1}$ and $\hat{V}_{2}$ depends only on the
dot products of the two vectors with themselves and with each other. You will
see this by answering the following problems.

\begin{problem}
Show that the area of the parallelogram determined by $\hat{V}_{1}$ and
$\hat{V}_{2}$ emanating from the same point in euclidean $3$-space is given by%
\[
|\hat{V}_{1}|\cdot|\hat{V}_{2}|\cdot\sin\theta.
\]

%areaOfParallelogram Geogebra

\begin{freeResponse}

Let $|\hat{V}_{1}|$ be the base and $h$ be the height of the parallelogram determined by  $\hat{V}_{1}$ and $\hat{V}_{2}$.  If $\theta$ is the angle between $\hat{V}_{1}$ and $\hat{V}_{2}$ then $|\hat{V}_{2}|\cdot \sin\theta = h$. Therefore, $A = b \cdot h = |\hat{V}_{1}| \cdot |\hat{V}_{2}| \cdot\sin\theta$.

\end{freeResponse}
\end{problem}

\begin{problem}
\label{9}Show that the area of the parallelogram determined by $\hat{V}_{1}$
and $\hat{V}_{2}$ emanating from the same point in euclidean $3$-space is also
given by%
\[
\sqrt{\det
\begin{bmatrix}
\hat{V}_{1}\bullet\hat{V}_{1} & \hat{V}_{2}\bullet\hat{V}_{1}\\
\hat{V}_{1}\bullet\hat{V}_{2} & \hat{V}_{2}\bullet\hat{V}_{2}%
\end{bmatrix}}.
\]

\begin{hint}
Start from the square of
$|\hat{V}_{1}|\cdot|\hat{V}_{2}|\cdot\sin\theta$, substitute $\left(
1-\cos^{2}\theta\right)$ for $\sin^{2}\theta$, and use the theorem
above. Alternatively start from
$|\hat{V}_{1}|\cdot|\hat{V}_{2}|\cdot\sin\theta$ and show that%
\[
\sin\left(  \arccos\left(  \frac{\hat{V}_{1}\bullet\hat{V}%
_{2}}{\left\vert \hat{V}_{1}\right\vert \cdot\left\vert \hat{V}_{2}\right\vert
}\right)  \right)  
=\frac{\sqrt{\det
    \begin{bmatrix}
      \hat{V}_{1}\bullet\hat{V}_{1} & \hat{V}_{2}\bullet\hat{V}_{1}\\
      \hat{V}_{1}\bullet\hat{V}_{2} & \hat{V}_{2}\bullet\hat{V}_{2}
    \end{bmatrix}
  }
}{\left\vert\hat{V}_{1}\right\vert \cdot\left\vert
  \hat{V}_{2}\right\vert }.
\]
\end{hint}

\begin{freeResponse}
By the theorem above:
\begin{align*}
A^2 = |\hat{V}_{1}|^2 \cdot |\hat{V}_{2} |^2 \cdot\sin^{2}\theta 
&= |\hat{V}_{1}|^2 \cdot |\hat{V}_{2} |^2 \cdot \left(1- \cos^{2}\theta\right)\\
&= |\hat{V}_{1}|^2 \cdot |\hat{V}_{2} |^2 - |\hat{V}_{1}|^2 \cdot |\hat{V}_{2} |^2 \cdot\cos^{2}\theta \\
&= |\hat{V}_{1}|^2 \cdot |\hat{V}_{2} |^2 - |\hat{V}_{1}|^2 \cdot |\hat{V}_{2} |^2 \left(  \frac{\hat{V}_{1}\bullet\hat{V}_{2}}{\left\vert \hat{V}_{1}\right\vert \cdot\left\vert \hat{V}_{2}\right\vert}\right)^2\\
&= |\hat{V}_{1}|^2 \cdot |\hat{V}_{2} |^2 - \left(\hat{V}_{1}\bullet\hat{V}_{2}\right)^2\\
&= \left(\hat{V}_{1}\bullet\hat{V}_{1}\right) \left(\hat{V}_{2}\bullet\hat{V}_{2}\right) - \left(\hat{V}_{1}\bullet\hat{V}_{2}\right)^2\\
&=\det
    \begin{bmatrix}
      \hat{V}_{1}\bullet\hat{V}_{1} & \hat{V}_{2}\bullet\hat{V}_{1}\\
      \hat{V}_{1}\bullet\hat{V}_{2} & \hat{V}_{2}\bullet\hat{V}_{2}
    \end{bmatrix}
\end{align*}
Therefore, the area of the parallelogram is also given by
$\sqrt{\det
    \begin{bmatrix}
      \hat{V}_{1}\bullet\hat{V}_{1} & \hat{V}_{2}\bullet\hat{V}_{1}\\
      \hat{V}_{1}\bullet\hat{V}_{2} & \hat{V}_{2}\bullet\hat{V}_{2}
    \end{bmatrix}}$.

Alternatively, given that $A = |\hat{V}_{1}| \cdot |\hat{V}_{2}| \cdot\sin\theta$, we know
\[
\theta = \arccos \left( 
	\frac{\hat{V}_{1}\bullet\hat{V}_{2}}{\left\vert \hat{V}_{1}\right\vert \cdot\left\vert \hat{V}_{2}\right\vert}.
		 \right)
\]
Then, using a right triangle whose ``adjacent" side is $\hat{V}_{1}\bullet\hat{V}_{2}$ and hypotenuse is $\left\vert \hat{V}_{1}\right\vert \cdot\left\vert \hat{V}_{2}\right\vert$,

\begin{align*}
\sin \theta = \sin\left(  \arccos\left(  \frac{\hat{V}_{1}\bullet\hat{V}_{2}}{\left\vert \hat{V}_{1}\right\vert \cdot\left\vert 	\hat{V}_{2}\right\vert} \right)  \right) 
	&= \frac{\sqrt{
	 \left( \left\vert \hat{V}_{1}\right\vert \cdot\left\vert \hat{V}_{2}\right\vert \right)^2 
	- \left( \hat{V}_{1}\bullet\hat{V}_{2} \right)^2}}{\left\vert \hat{V}_{1}\right\vert \cdot\left\vert \hat{V}_{2}\right\vert}  \\
	&= \frac{\sqrt{ 
	\left( \hat{V}_{1}\bullet\hat{V}_{1} \right) \cdot \left(\hat{V}_{2}\bullet\hat{V}_{2} \right)
	- \left( \hat{V}_{1}\bullet\hat{V}_{2} \right)^2}}{\left\vert\hat{V}_{1}\right\vert \cdot\left\vert \hat{V}_{2}\right\vert}\\
	&=\frac{\sqrt{\det
    	\begin{bmatrix}
      		\hat{V}_{1}\bullet\hat{V}_{1} & \hat{V}_{2}\bullet\hat{V}_{1}\\
      		\hat{V}_{1}\bullet\hat{V}_{2} & \hat{V}_{2}\bullet\hat{V}_{2}
    	\end{bmatrix}}}{\left\vert\hat{V}_{1}\right\vert \cdot\left\vert \hat{V}_{2}\right\vert}
\end{align*}

Therefore, the area of the parallelogram is also given by
$\sqrt{\det
    \begin{bmatrix}
      \hat{V}_{1}\bullet\hat{V}_{1} & \hat{V}_{2}\bullet\hat{V}_{1}\\
      \hat{V}_{1}\bullet\hat{V}_{2} & \hat{V}_{2}\bullet\hat{V}_{2}
    \end{bmatrix}}$.

\end{freeResponse}

\end{problem}

\begin{problem}
Show that we have the following equality:
\[
\sqrt{
  \det
\begin{bmatrix}
\hat{V}_{1}\bullet\hat{V}_{1} & \hat{V}_{2}\bullet\hat{V}_{1}\\
\hat{V}_{1}\bullet\hat{V}_{2} & \hat{V}_{2}\bullet\hat{V}_{2}%
\end{bmatrix}}
=
\sqrt{
  \det\left(
\begin{bmatrix}
\hat{V}_{1} \\
\hat{V}_{2}
\end{bmatrix}
\cdot
\begin{bmatrix}
\hat{V}_{1}^\transpose  & \hat{V}_{2}^\transpose 
\end{bmatrix}\right)}.
\]

\begin{freeResponse}
\[
\sqrt{
  \det
\begin{bmatrix}
\hat{V}_{1}\bullet\hat{V}_{1} & \hat{V}_{2}\bullet\hat{V}_{1}\\
\hat{V}_{1}\bullet\hat{V}_{2} & \hat{V}_{2}\bullet\hat{V}_{2}%
\end{bmatrix}}
=\sqrt{
  \det
\begin{bmatrix}
\hat{V}_{1}\cdot \hat{V}_{1}^\transpose & \hat{V}_{1}\cdot \hat{V}_{2}^\transpose\\
\hat{V}_{2}\cdot \hat{V}_{1}^\transpose & \hat{V}_{2}\cdot \hat{V}_{2}^\transpose%
\end{bmatrix}}
=\sqrt{
  \det\left(
\begin{bmatrix}
\hat{V}_{1} \\
\hat{V}_{2}
\end{bmatrix}
\cdot
\begin{bmatrix}
\hat{V}_{1}^\transpose  & \hat{V}_{2}^\transpose 
\end{bmatrix}\right)}.
\]
\end{freeResponse}
\end{problem}

Again, the significance of the previous problems is that, to compute
areas, we only need to know how to compute dot-products. Hence, it is
the definition of the dot-product of the vectors completely determines
the calculation of the area of the parallelogram they generate.


%% We finish this section with one other related fact.

%% \begin{lemma}
%% The area of the parallelogram determined by two vectors $\hat{V}_{1}=\left(
%% \hat{a}_{1},\hat{b}_{1},\hat{c}_{1}\right)  $ and $\hat{V}_{2}=\left(  \hat
%% {a}_{2},\hat{b}_{2},\hat{c}_{2}\right)  $ emanating from the same point in
%% euclidean $3$-space is given by the length of the cross-product%
%% \[
%% \hat{V}_{1}\times\hat{V}_{2}=\left(
%% \det\begin{bmatrix}
%% \hat{b}_{1} & \hat{c}_{1}\\
%% \hat{b}_{2} & \hat{c}_{2}%
%% \end{bmatrix},
%% \det\begin{bmatrix}
%% \hat{c}_{1} & \hat{a}_{1}\\
%% \hat{c}_{2} & \hat{a}_{2}%
%% \end{bmatrix},
%% \det\begin{bmatrix}
%% \hat{a}_{1} & \hat{b}_{1}\\
%% \hat{a}_{2} & \hat{b}_{2}%
%% \end{bmatrix}
%% \right).
%% \]

%% \end{lemma}

%% Now we will prove this lemma. Your task is to fill-in the details of
%% the proof below.

%% \begin{problem}
%% We use some facts from linear algebra. First of all, develop the determinant%
%% \[
%% \det
%% \begin{bmatrix}
%% \hat{V}_{1} \\
%% \hat{V}_{2} \\
%% \hat{V}_{1}\times\hat{V}_{2}
%% \end{bmatrix}
%% \]
%% along the third row.
%% \begin{itemize}
%% \item Conclude that 
%% \[
%% \left\vert
%% \det
%% \begin{bmatrix}
%% \hat{V}_{1} \\
%% \hat{V}_{2} \\
%% \hat{V}_{1}\times\hat{V}_{2}
%% \end{bmatrix}\right\vert
%% =\left\vert \hat{V}_{1}\times\hat{V}_{2}\right\vert ^{2}
%% \]
%% \item On the other hand by developing the left-hand determinant along
%%   the third row, show that
%% \[
%% \det
%% \begin{bmatrix}
%% \hat{V}_{1} \\
%% \hat{V}_{2} \\
%% \hat{V}_{1}
%% \end{bmatrix}
%% =0
%% \]
%% \end{itemize}
%% Hence we conclude that $\hat{V}_{1}$ is perpendicular to
%% $\hat{V}_{1}\times\hat{V}_{2}$. Similarly $\hat{V}_{2}$ is
%% perpendicular to $\hat{V}_{1}\times\hat{V}_{2}$. Finally, the absolute
%% value of the determinant of a $3\times3$ matrix%
%% \[
%% \left\vert
%% \det
%% \begin{bmatrix}
%% \hat{V}_{1} \\
%% \hat{V}_{2} \\
%% \hat{V}_{1}\times\hat{V}_{2}
%% \end{bmatrix}
%% \right\vert
%% \]
%% is the volume of the parallelepiped determined by the row vectors of the
%% matrix. But that volume is%
%% \[
%% \left( \left\vert \hat{V}_{1}\right\vert \cdot\left\vert \hat{V}%
%% _{2}\right\vert \cdot\sin\theta\right)  \cdot\left\vert \hat{V}%
%% _{1}\times\hat{V}_{2}\right\vert
%% \]
%% since $\left(  \left\vert \hat{V}_{1}\right\vert \cdot\left\vert \hat{V}%
%% _{2}\right\vert \cdot\sin\theta\right)  $ is the area of the base
%% of the parallelepiped and $\hat{V}_{1}\times\hat{V}_{2}$ is perpendicular to
%% both $V_{1}$ and $V_{2}$. So we see
%% \[
%% \left(  \left\vert \hat{V}_{1}\right\vert \cdot\left\vert \hat{V}%
%% _{2}\right\vert \cdot\sin\theta\right)  \cdot\left\vert \hat{V}%
%% _{1}\times\hat{V}_{2}\right\vert =\left\vert
%% \det\begin{bmatrix}
%% \hat{V}_{1} \\
%% \hat{V}_{2} \\
%% \hat{V}_{1}\times\hat{V}_{2}
%% \end{bmatrix}
%% \right\vert =\left\vert \hat{V}_{1}\times\hat{V}_{2}\right\vert ^{2}.
%% \]
%% So%
%% \[
%% \left(  \left\vert \hat{V}_{1}\right\vert \cdot\left\vert \hat{V}%
%% _{2}\right\vert \cdot\sin\theta\right)  =\left\vert \hat{V}%
%% _{1}\times\hat{V}_{2}\right\vert .
%% \]
%% %[DS,42-47]
%% \begin{itemize}
%% \item Explain how this completes the proof. 
%% \end{itemize}
%% \end{problem}


\subsection*{Curves in euclidean $3$-space and vectors tangent to them}

\begin{definition}
A \textbf{smooth curve in }$\mathbf{3}$\textbf{-dimensional euclidean space}
is given by a differentiable mapping%
\begin{gather*}
\hat{X}:\left[  b,e\right]  \rightarrow\mathbb{R}^{3}\\
t\mapsto\left(  \hat{x}\left(  t\right)  ,\hat{y}\left(  t\right)  ,\hat
{z}\left(  t\right)  \right)
\end{gather*}
from an interval $\left[ b,e\right] $ on the real line. We shall
use the notation%
\[
\hat{X}(t) = \begin{cases}
  \hat{x}(t),\\
  \hat{y}(t),\\
  \hat{z}(t).
\end{cases}
\]
The mapping $\hat{X}(t)$ must have the additional property that
the tangent vector
\[
\dd[\hat{X}]{t} =\left(
  \dd[\hat{x}]{t},
  \dd[\hat{y}]{t},
  \dd[\hat{z}]{t}
\right)
\]
is not the zero vector for any $t$ in $[b,e]$.
\end{definition}

\begin{problem}\hfil
\begin{enumerate}
\label{1}\item Give two examples of smooth curves,
\begin{align*}
\hat{X}_{1}(s) &=\left(\hat{x}_{1}(s),\hat{y}_{1}(s),\hat{z}_{1}(s)\right) \\
\hat{X}_{2}(t) &=\left(\hat{x}_{2}(t),\hat{y}_{2}(t),\hat{z}_{2}(t)\right)
\end{align*}
where
\begin{itemize}
\item neither $\hat{X}_1$ nor $\hat{X}_2$ are a straight line,
\item the two curves $\hat{X}_1$ and $\hat{X}_2$ pass through a common
  point \textbf{and} go in distinct tangent directions at that point.
\item None of the components of $\hat{X}_1$ and $\hat{X}_2$ are constant functions.
\end{itemize}

\item Compute the tangent vectors of each of the two curves at each of their points.

\item For the two curves you defined in a), what are the coordinates of the point
in euclidean $3$-space at which the two curves intersect?

\item Use the dot product formula to compute the angle $\theta$ between (the
tangent vectors to) your two example curves in a) at the point at which the
curves intersect. %[DS,20-21]
\end{enumerate}

%smoothCurvesExample geogebra

\begin{freeResponse} 
\begin{enumerate}
\label{1}\item Two examples of smooth curves:
\begin{align*}
\hat{X}_{1}(s) &=\left(s^3, \frac{\sqrt{6}}{2}s^2, s\right) \qquad 1 \leq s\leq3 \\
\hat{X}_{2}(s) &=\left(s, \frac{\sqrt{6}}{2}s^2, s^3\right) \qquad 1 \leq s\leq4
\end{align*}

\item  $\dd[\hat{X}_{1}]{s} = \left(3s^2, \sqrt{6}s, 1 \right)$ and 
$\dd[\hat{X}_{2}]{s} =  \left(1, \sqrt{6}s,  3s^2 \right)$

\item $\left(1,1,1 \right)$

\item 
$\theta = \arccos \left( 
   \frac{ \left(3, \sqrt{6} , 1 \right) \bullet \left(1, \sqrt{6} , 3 \right)}
   { |\left(3, \sqrt{6} , 1 \right)| \cdot |\left(1, \sqrt{6} , 3 \right)|} \right)
= \arccos \left( \frac{12}{16} \right) = 41.41^{\circ}.$

\end{enumerate}
\end{freeResponse}

\end{problem}

Sometimes displacement is measured by showing how a given point is displaced,
as in%
\[
\hat{V}=\hat{X}_{2}-\hat{X}_{1}=\left(  \hat{x}_{2}-\hat{x}_{1},\hat{y}%
_{2}-\hat{y}_{1},\hat{z}_{2}-\hat{z}_{1}\right)  ,
\]
and sometimes displacement is expressed as the instantaneous velocity of a
point moving along a curve as in%
\[
\hat{V}=\frac{d\hat{X}\left(  t\right)  }{dt}=\left(  \frac{d\hat{x}\left(
t\right)  }{dt},\frac{d\hat{y}\left(  t\right)  }{dt},\frac{d\hat{z}\left(
t\right)  }{dt}\right)  .
\]
%[DS,30ff].

In matrix notation we can think of
\[
\hat{X}_{2}-\hat{X}_{1}=\left(  \hat{x}_{2}-\hat{x}_{1},\hat{y}_{2}-\hat
{y}_{1},\hat{z}_{2}-\hat{z}_{1}\right)
\]
as a $1\times3$ matrix $\left[\hat{X}_{2}-\hat{X}_{1}\right]$. Then we can
write the formula for the distance between two points $\hat{X}_{1}$ and
$\hat{X}_{2}$ in euclidean $3$-space in terms of the dot-product%
\begin{equation}
d\left(  \hat{X}_{1},\hat{X}_{2}\right)  =\sqrt{\left(  \hat{X}_{2}-\hat
{X}_{1}\right)  \bullet\left(  \hat{X}_{2}-\hat{X}_{1}\right)  } \label{13}%
\end{equation}
or in terms of the matrix product%
\[
d\left( \hat{X}_{1},\hat{X}_{2}\right) =\sqrt{\left[
  \hat{X}_{2}-\hat{X}_{1} \right] \cdot\left[\hat{X}_{2}-\hat{X}%
  _{1}\right]^\transpose }.
\]


\subsection*{Length of a smooth curve in euclidean $3$-space}

\begin{problem}
Compute the length of the tangent vector
\[
\l(t)=\sqrt{\frac{d\hat{X}}{dt}\bullet\frac{d\hat{X}}{dt}}%
\]
to each of your two example curves in the previous problem at each of
their points.

%lengthOfTangentVectorsSmoothCurvesExamples Geogebra


\begin{freeResponse}
\[
\l_{1}(s) = \sqrt{\frac{d\hat{X}_{1}}{ds}\bullet\frac{d\hat{X}_{1}}{ds}} = \sqrt{9s^4 + 6s^2 +1}
\] and
\[ 
\l_{2}(s) = \sqrt{\frac{d\hat{X}_{2}}{ds}\bullet\frac{d\hat{X}_{2}}{ds}} = \sqrt{1 + 6s^2 +9s^4}
\]
\end{freeResponse}

\end{problem}

\begin{definition}
The length $L$ of the curve $\hat{X}(t)$, $t\in[b,e] $, in euclidean
$3$-space is obtained by integrating the length of the tangent vector
to the curve, that is,%
\[
L=\int_{b}^{e} \l(t)  \,dt.
\]
%[DS,82]
  Notice that the length of any curve only depends on the definition of
the dot-product. That is, if we know the formula for the dot-product, we know
(the formula for) the length of any curve.
\end{definition}

Our first example is the path%
\[
\left(\hat{x}(t),\hat{y}(t),\hat{z}(t)\right)=\left(R\cdot\sin(t),0,R\cdot\cos(t)\right) \qquad
0 \leq t\leq\pi.
\]


Notice that this path lies on the sphere of radius $R$.

\begin{problem}
Write the formula for the tangent vector to the path above at each
point using
$\left(\hat{x}(t),\hat{y}(t),\hat{z}(t)\right)$-coordinates. Show that
the length of this path is $R\pi$.

\begin{freeResponse}
\[
\left( \dd[\hat{x}]{t}, \dd[\hat{y}]{t}, \dd[\hat{z}]{t} \right) 
    = \left(R\cdot\cos(t),0,-R\cdot\sin(t) \right)
\]
\[
\l(t) = \sqrt{
   R^2 \cdot\cos^{2}(t) + R^2 \cdot\sin^{2}(t)} 
   = \sqrt{R^2} = |R|
\]
\[
L=\int_{0}^{\pi} R \,dt = R\cdot\pi
\]
\end{freeResponse}

\end{problem}

\begin{problem}
Compute the length of each of your two example curves in the previous
problem.

\begin{freeResponse}
\[
L_{1}= \int_{1}^{3}  \sqrt{9s^4 + 6s^2 +1} \,ds = \int_{1}^{3} |3s^2 +1| \, ds = 28
\]
\[
L_{2}= \int_{1}^{4}  \sqrt{9s^4 + 6s^2 +1} \,ds = \int_{1}^{4} |3s^2 +1| \, ds = 66
\]
\end{freeResponse}

\end{problem}

\begin{remark}
In this last problem, you may easily be confronted with an integral
that you cannot compute. For example, if your curve $\hat{X}_{1}\left(
t\right) $ happens to describe an ellipse that is not circular, it was
proved in the 19th century that no formula involving only the
standard functions from calculus will give you the length of your path
from a fixed beginning point to a variable ending point on the
ellipse. If that kind of thing occurs, go back and change the
definitions of your curves in the previous problem until you get two
curves for which you can compute length of your path from a fixed
beginning point to a fixed ending point. %[DS,81-82]
\end{remark}





To study plane geometry, spherical geometry, and the hyperbolic
geometries in a uniform way we will have to \textit{change} the
coordinate system we use. In essence, this means that we will have to
change the distance formula slightly for each geometry. These new
coordinates will be chosen to keep the north and south poles from
going to infinity as the radius $R$ of a sphere increases without
bound. This change of viewpoint will eventually let us go
non-euclidean or, in the language of Buzz Lightyear ``to infinity and
beyond.'' The idea will be like the change from rectangular to polar
coordinates for the plane that you encountered in calculus, only
easier.


\begin{problem}
Summarize the results from this section. In particular, indicate which
results follow from the others.
\begin{freeResponse}
We defined the dot product for Euclidean space. We calculated the length of a vector, angle between two vectors, and the area of a parallelogram determined by two vectors, determining that they all depend only on the dot product. 
\end{freeResponse}
\end{problem}

\end{document}
