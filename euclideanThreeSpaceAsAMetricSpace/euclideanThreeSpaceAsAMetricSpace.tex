\documentclass{ximera}

\preambleinput{../preamble.tex}

\title{Euclidean three-space as a metric space}

\begin{document}
\begin{abstract}
In this activity we will work in three dimensional space.
\end{abstract}
\maketitle

\subsection*{Points and vectors in Euclidean 3-space}

In this book, we will study all the \textit{two}-dimensional geometries
(spheres, the plane and hyperbolic spaces). Each one of these geometries looks
the same at each of its points and it also looks the same in every direction
emanating from any of its points. But to study them all at the same time and
in a uniform way we will need to visualize them all as different surfaces
lying in some common \textit{three}-dimensional space. We start with the most
familiar cases, namely the spherical geometries.

For those we begin with three-dimensional Euclidean space%
\[
\mathbb{R}^{3}=\left\{  \left(  \hat{x},\hat{y},\hat{z}\right)  :\hat{x}%
,\hat{y},\hat{z}\in\mathbb{R}\right\}  ,
\]
where there is a standard way to measure distance between two points%
\begin{align*}
\hat{X}_{1}  &  =\left(  \hat{x}_{1},\hat{y}_{1},\hat{z}_{1}\right) \\
\hat{X}_{2}  &  =\left(  \hat{x}_{2},\hat{y}_{2},\hat{z}_{2}\right)  ,
\end{align*}
namely%
\begin{equation}
d\left(  \hat{X}_{1},\hat{X}_{2}\right)  =\sqrt{\left(  \hat{x}_{2}-\hat
{x}_{1}\right)  ^{2}+\left(  \hat{y}_{2}-\hat{y}_{1}\right)  ^{2}+\left(
\hat{z}_{2}-\hat{z}_{1}\right)  ^{2}}. \label{0}%
\end{equation}


As we will see, the formula $\left(  \ref{0}\right)  $ is compatible with
distances on such objects as spheres%
\[
\left\{  \left(  \hat{x},\hat{y},\hat{z}\right)  \in\mathbb{R}^{3}:\hat{x}%
^{2}+\hat{y}^{2}+\hat{z}^{2}=R^{2}\right\}
\]
of a fixed radius $R$, since these can be faithfully represented in ordinary
Euclidean three-space. However, there is one disconcerting fact about studying
the geometry of spheres in this way. Namely, as $R$ approaches infinity, the
geometry of the $R$-sphere at any point looks more and more like plane
geometry, but on the other hand, that `limit' plane geometry is `out at
infinity.' So in order to study all the two-dimensional geometries, including
plane geometry and the hyperbolic geometries, in a uniform way we will have to
\textit{change} the coordinate system we use, or, what will turn out to be the
same thing, we will have to change the distance formula slightly for each
geometry. We will do that in later sections, but first we want to review some
of the basic properties of ordinary Euclidean three-space you learned about it
in calculus.

We write $\left(  \hat{x},\hat{y},\hat{z}\right)  $ for our ordinary Euclidean
coordinates. When you see $\left(  \hat{x},\hat{y},\hat{z}\right)  $ in what
follows, that means that distance between points is measured by the formula
$\left(  \ref{0}\right)  $. One more thing--in Euclidean three-space it will
be important thoughout to make the distinction between \textbf{points} and
\textbf{vectors}: Although each will be represented by a triple of real
numbers we will use%
\[
\hat{X}=\left(  \hat{x},\hat{y},\hat{z}\right)
\]
to denote \textbf{points}, that is, \textbf{position} in Euclidean $3$-space,
and%
\[
\hat{V}=\left(  \hat{a},\hat{b},\hat{c}\right)
\]
to denote \textbf{vectors}, that is, \textbf{displacement} by which we mean
the amount and direction a given point is being moved. So vectors always
indicate \textit{motion} from an explicit (or implicit) \textit{point} of
reference. 

\subsection*{Dot product of vectors emanating from a point}

There are various operations we can perform on one or more vectors when we
think of them as emanating from the same point in Euclidean $3$-space. The
first is the dot product of two vectors.

\begin{definition}
The dot product of two vectors%
\begin{align*}
\hat{V}_{1}  &  =\left(  \hat{a}_{1},\hat{b}_{1},\hat{c}_{1}\right) \\
\hat{V}_{2}  &  =\left(  \hat{a}_{2},\hat{b}_{2},\hat{c}_{2}\right)
\end{align*}
emanating from the same point in 3-dimensional Euclidean space is
defined as the real number given by the formula%
\[
\hat{a}_{1}\hat{a}_{2}+\hat{b}_{1}\hat{b}_{2}+\hat{c}_{1}\hat{c}_{2}%
\]
or in matrix notation as%
\[
\left(
\begin{array}
[c]{ccc}%
\hat{a}_{1} & \hat{b}_{1} & \hat{c}_{1}%
\end{array}
\right)  \cdot\left(
\begin{array}
[c]{c}%
\hat{a}_{2}\\
\hat{b}_{2}\\
\hat{c}_{2}%
\end{array}
\right)  .
\]
It is also denoted as%
\[
\hat{V}_{1}\bullet\hat{V}_{2}%
\]
or in matrix notation as%
\[
\hat{V}_{1}\cdot\left(  \hat{V}_{2}\right)  ^{t}.
\]

\end{definition}

\begin{exercise}
Give the formula for the length $\left\vert \hat{V}\right\vert $ of a vector
$\hat{V}=\left(  \hat{a},\hat{b},\hat{c}\right)  $ in 3-dimensional Euclidean
space in terms of dot product.
\end{exercise}

\begin{exercise}
As you work through the proof of the Law of Cosines in the following Lemma,
construct a diagram or picture for each step.
\end{exercise}

\begin{lemma}[Law of Cosines]
\label{110} The (smaller) angle $\vartheta$ between two
vectors $\hat{V}_{1}$ and $\hat{V}_{2}$ emanating from $O=\left(
0,0,0\right)  $ satisfies the relation%
\[
\left\vert \hat{V}_{2}-\hat{V}_{1}\right\vert ^{2}=\left\vert \hat{V}%
_{1}\right\vert ^{2}+\left\vert \hat{V}_{2}\right\vert ^{2}-2\left\vert
\hat{V}_{1}\right\vert \cdot\left\vert \hat{V}_{2}\right\vert \cdot
\cos\vartheta.
\]

\end{lemma}

\begin{proof}
Without loss of generality we can assume that $\left\vert \hat{V}%
_{1}\right\vert \leq\left\vert \hat{V}_{2}\right\vert $. Consider the triangle
with one side given by the interval from $O=\left(  0,0,0\right)  $ to the
endpoint $P_{1}$ of $\hat{V}_{1}$, with a second side $S_{2}$ given by the
interval from $O$ to the endpoint $P_{2}$ of $\hat{V}_{2}$ and with the third
side given by the interval joining $P_{1}$ and $P_{2}$. Let $P$ be the point
on $S_{2}$ so that the interval between $P_{1}$ and $P$ is perpendicular to
$S_{2}$. By the Pythagorean theorem%
\begin{align*}
\left\vert P_{1}P_{2}\right\vert ^{2}-\left\vert P_{2}P\right\vert ^{2}  &
=\left\vert PP_{1}\right\vert ^{2}\\
&  =\left\vert OP_{1}\right\vert ^{2}-\left\vert OP\right\vert ^{2}\\
\left\vert P_{1}P_{2}\right\vert ^{2}  &  =\left\vert OP_{1}\right\vert
^{2}+\left(  \left\vert P_{2}P\right\vert ^{2}-\left\vert OP\right\vert
^{2}\right) \\
&  =\left\vert OP_{1}\right\vert ^{2}+\left(  \left\vert P_{2}P\right\vert
+\left\vert OP\right\vert \right)  \left(  \left\vert P_{2}P\right\vert
-\left\vert OP\right\vert \right) \\
&  =\left\vert OP_{1}\right\vert ^{2}+\left\vert OP_{2}\right\vert \left(
\left\vert P_{2}P\right\vert -\left\vert OP\right\vert \right) \\
&  =\left\vert OP_{1}\right\vert ^{2}+\left\vert OP_{2}\right\vert \left(
\left\vert OP_{2}\right\vert -2\left\vert OP\right\vert \right)
\end{align*}
But%
\[
\left\vert OP\right\vert =\left\vert OP_{1}\right\vert \cdot\cos\vartheta.
\]

\end{proof}

\begin{exercise}
What can you say about the cosine of the larger of the two angles between two
vectors $\hat{V}_{1}$ and $\hat{V}_{2}$, that is about $\left(  360^{\circ
}-\vartheta\right)  $?
\end{exercise}

\begin{lemma}
\label{111}The angle $\vartheta$ between two vectors $\hat{V}_{1}$ and
$\hat{V}_{2}$ emanating from the same point in Euclidean $3$-space satisfies
the relation
\begin{equation}
\hat{V}_{1}\bullet\hat{V}_{2}=\left\vert \hat{V}_{1}\right\vert \cdot
\left\vert \hat{V}_{2}\right\vert \cdot\cos\vartheta. \label{2}%
\end{equation}
[DS,30ff]
\end{lemma}

\begin{proof}
Multipying out using the definition and algebraic properties of dot product,%
\begin{align*}
\left\vert \hat{V}_{2}-\hat{V}_{1}\right\vert ^{2}  &  =\left(  \hat{V}%
_{2}-\hat{V}_{1}\right)  \bullet\left(  \hat{V}_{2}-\hat{V}_{1}\right) \\
&  =\left\vert \hat{V}_{1}\right\vert ^{2}+\left\vert \hat{V}_{2}\right\vert
^{2}-2\left(  \hat{V}_{1}\bullet\hat{V}_{2}\right)  .
\end{align*}
Now apply Lemma \ref{110}.
\end{proof}

The significance of Lemma \ref{111} is that the measure of angles between
vectors depends only on the definition of the dot product.

\begin{corollary}
The formula for the angle $\vartheta$ between two vectors $\hat{V}_{1}=\left(
\hat{a}_{1},\hat{b}_{1},\hat{c}_{1}\right)  $ and $\hat{V}_{2}=\left(  \hat
{a}_{2},\hat{b}_{2},\hat{c}_{2}\right)  $ in $3$-dimensional Euclidean space
depends only on the dot products of the two vectors with themselves and with
each other. Namely%
\[
\vartheta=\arccos\left(  \frac{\hat{V}_{1}\bullet\hat{V}_{2}%
}{\left\vert \hat{V}_{1}\right\vert \cdot\left\vert \hat{V}_{2}\right\vert
}\right)  .
\]

\end{corollary}

In fact it is also true that the formula for the area of the parallelogram
determined by two vectors $\hat{V}_{1}$ and $\hat{V}_{2}$ depends only on the
dot products of the two vectors with themselves and with each other. You will
see this by answering the following Exercises.

\begin{exercise}
Show that the area of the parallelogram determined by $\hat{V}_{1}$ and
$\hat{V}_{2}$ emanating from the same point in Euclidean $3$-space is given by%
\begin{equation}
\left\vert \hat{V}_{1}\right\vert \cdot\left\vert \hat{V}_{2}\right\vert
\cdot\sin\vartheta. \label{3}%
\end{equation}

\end{exercise}

\begin{exercise}
\label{9}Show that the area of the parallelogram determined by $\hat{V}_{1}$
and $\hat{V}_{2}$ emanating from the same point in Euclidean $3$-space is also
given by%
\[
\sqrt{\left\vert
\begin{array}
[c]{cc}%
\hat{V}_{1}\bullet\hat{V}_{1} & \hat{V}_{2}\bullet\hat{V}_{1}\\
\hat{V}_{1}\bullet\hat{V}_{2} & \hat{V}_{2}\bullet\hat{V}_{2}%
\end{array}
\right\vert }.
\]


Hint: Start from the square of $\left(  \ref{3}\right)  $, substitute $\left(
1-\cos^{2}\vartheta\right)  $ for $\sin^{2}\vartheta$, and use
$\left(  \ref{2}\right)  $. Alternatively start from $\left(  \ref{3}\right)
$ and show that%
\[
\sin\left(  \arccos\left(  \frac{\hat{V}_{1}\bullet\hat{V}%
_{2}}{\left\vert \hat{V}_{1}\right\vert \cdot\left\vert \hat{V}_{2}\right\vert
}\right)  \right)  =\frac{\sqrt{\left\vert
\begin{array}
[c]{cc}%
\hat{V}_{1}\bullet\hat{V}_{1} & \hat{V}_{2}\bullet\hat{V}_{1}\\
\hat{V}_{1}\bullet\hat{V}_{2} & \hat{V}_{2}\bullet\hat{V}_{2}%
\end{array}
\right\vert }}{\left\vert \hat{V}_{1}\right\vert \cdot\left\vert \hat{V}%
_{2}\right\vert }.
\]

\end{exercise}

\begin{exercise}
Show that we have the following equality of matrices%
\[
\left(
\begin{array}
[c]{cc}%
\hat{V}_{1}\bullet\hat{V}_{1} & \hat{V}_{2}\bullet\hat{V}_{1}\\
\hat{V}_{1}\bullet\hat{V}_{2} & \hat{V}_{2}\bullet\hat{V}_{2}%
\end{array}
\right)  =\left(
\begin{array}
[c]{c}%
\left(  \hat{V}_{1}\right) \\
\left(  \hat{V}_{2}\right)
\end{array}
\right)  \cdot\left(
\begin{array}
[c]{cc}%
\left(  \hat{V}_{1}\right)  ^{t} & \left(  \hat{V}_{2}\right)  ^{t}%
\end{array}
\right)  .
\]

\end{exercise}

Again, the significance of Exercise \ref{9} is that, to compute areas, we only
need to know how to compute dot-products--the definition of the dot-product of
the vectors completely determines the calculation of the area of the
parallelogram they generate.

We finish this section with one other related fact.

\begin{lemma}
The area of the parallelogram determined by two vectors $\hat{V}_{1}=\left(
\hat{a}_{1},\hat{b}_{1},\hat{c}_{1}\right)  $ and $\hat{V}_{2}=\left(  \hat
{a}_{2},\hat{b}_{2},\hat{c}_{2}\right)  $ emanating from the same point in
Euclidean $3$-space is given by the length of the cross-product%
\[
\hat{V}_{1}\times\hat{V}_{2}=\left(  \left\vert
\begin{array}
[c]{cc}%
\hat{b}_{1} & \hat{c}_{1}\\
\hat{b}_{2} & \hat{c}_{2}%
\end{array}
\right\vert ,\left\vert
\begin{array}
[c]{cc}%
\hat{c}_{1} & \hat{a}_{1}\\
\hat{c}_{2} & \hat{a}_{2}%
\end{array}
\right\vert ,%
\begin{array}
[c]{cc}%
\hat{a}_{1} & \hat{b}_{1}\\
\hat{a}_{2} & \hat{b}_{2}%
\end{array}
\right)  .
\]

\end{lemma}

\begin{proof}
We use some facts from linear algebra. First of all, develop the determinant%
\[
\left\vert
\begin{array}
[c]{c}%
\left(  \hat{V}_{1}\right) \\
\left(  \hat{V}_{2}\right) \\
\left(  \hat{V}_{1}\times\hat{V}_{2}\right)
\end{array}
\right\vert
\]
along the third row. Then writing out both sides of the equation%
\begin{equation}
\left\vert
\begin{array}
[c]{c}%
\left(  \hat{V}_{1}\right) \\
\left(  \hat{V}_{2}\right) \\
\left(  \hat{V}_{1}\times\hat{V}_{2}\right)
\end{array}
\right\vert =\left\vert \hat{V}_{1}\times\hat{V}_{2}\right\vert ^{2}
\label{108}%
\end{equation}
we conclude that they are equal. On the other hand,
\[
\left\vert
\begin{array}
[c]{c}%
\left(  \hat{V}_{1}\right) \\
\left(  \hat{V}_{2}\right) \\
\left(  \hat{V}_{1}\right)
\end{array}
\right\vert =0
\]
and developing the left-hand determinant along the third row, we conclude that
$\hat{V}_{1}$ is perpendicular to $\hat{V}_{1}\times\hat{V}_{2}$. Similarly
$\hat{V}_{2}$ is perpendicular to $\hat{V}_{1}\times\hat{V}_{2}$. Finally, the
absolute value of the determinant of a $3\times3$ matrix%
\[
\left\vert
\begin{array}
[c]{c}%
\left(  \hat{V}_{1}\right) \\
\left(  \hat{V}_{2}\right) \\
\left(  \hat{V}_{1}\times\hat{V}_{2}\right)
\end{array}
\right\vert
\]
is the volume of the parallelepiped determined by the row vectors of the
matrix. But that volume is%
\[
\left(  \left\vert \hat{V}_{1}\right\vert \cdot\left\vert \hat{V}%
_{2}\right\vert \cdot\sin\vartheta\right)  \cdot\left\vert \hat{V}%
_{1}\times\hat{V}_{2}\right\vert
\]
since $\left(  \left\vert \hat{V}_{1}\right\vert \cdot\left\vert \hat{V}%
_{2}\right\vert \cdot\sin\vartheta\right)  $ is the area of the base
of the parallelepiped and $\hat{V}_{1}\times\hat{V}_{2}$ is perpendicular to
both $V_{1}$ and $V_{2}$. So using $\left(  \ref{108}\right)  $%
\[
\left(  \left\vert \hat{V}_{1}\right\vert \cdot\left\vert \hat{V}%
_{2}\right\vert \cdot\sin\vartheta\right)  \cdot\left\vert \hat{V}%
_{1}\times\hat{V}_{2}\right\vert =\left\vert
\begin{array}
[c]{c}%
\left(  \hat{V}_{1}\right) \\
\left(  \hat{V}_{2}\right) \\
\left(  \hat{V}_{1}\times\hat{V}_{2}\right)
\end{array}
\right\vert =\left\vert \hat{V}_{1}\times\hat{V}_{2}\right\vert ^{2}.
\]
So%
\[
\left(  \left\vert \hat{V}_{1}\right\vert \cdot\left\vert \hat{V}%
_{2}\right\vert \cdot\sin\vartheta\right)  =\left\vert \hat{V}%
_{1}\times\hat{V}_{2}\right\vert .
\]
[DS,42-47]
\end{proof}


\subsection*{Curves in Euclidean $3$-space and vectors tangent to them}

\begin{definition}
A \textbf{smooth curve in }$\mathbf{3}$\textbf{-dimensional Euclidean space}
is given by a differentiable mapping%
\begin{gather*}
\hat{X}:\left[  b,e\right]  \rightarrow\mathbb{R}^{3}\\
t\mapsto\left(  \hat{x}\left(  t\right)  ,\hat{y}\left(  t\right)  ,\hat
{z}\left(  t\right)  \right)
\end{gather*}
\hspace{5mm} \hspace{5mm} \hspace{5mm} \hspace{5mm} from an interval $\left[
b,e\right]  $ on the real line. We shall sometimes use the notation%
\[
\left(  \hat{x}\left(  t\right)  ,\hat{y}\left(  t\right)  ,\hat{z}\left(
t\right)  \right)  =\hat{X}\left(  t\right)  .
\]
The mapping $\hat{X}\left(  t\right)  $ must have the additional property that
the tangent vector
\[
\left(  \hat{a}\left(  t\right)  ,\hat{b}\left(  t\right)  ,\hat{c}\left(
t\right)  \right)  =\left(  \frac{d\hat{x}}{dt},\frac{d\hat{y}}{dt}%
,\frac{d\hat{z}}{dt}\right)  =\frac{d\hat{X}}{dt}%
\]
\hspace{5mm} \hspace{5mm} is not the zero vector for any $t$ in $\left[
b,e\right]  $.
\end{definition}

\begin{exercise}\hfil
\begin{enumerate}
\label{1}\item Give two examples of smooth curves,
\begin{align*}
\hat{X}_{1}\left(  s\right)   &  =\left(  \hat{x}_{1}\left(  s\right)
,\hat{y}_{1}\left(  s\right)  ,\hat{z}_{1}\left(  s\right)  \right) \\
\hat{X}_{2}\left(  t\right)   &  =\left(  \hat{x}_{2}\left(  t\right)
,\hat{y}_{2}\left(  t\right)  ,\hat{z}_{2}\left(  t\right)  \right)
\end{align*}
neither of which is a straight line, in $3$-dimensional Euclidean space. Do
this so that the two curves pass through a common point and go in distinct
tangent directions at that point. Please choose curves so that none of the
coordinate functions of $s$ or $t$ is a constant function. [DS,71ff]

\item Compute the tangent vectors of each of the two curves at each of their points.

\item For the two curves you defined in a), what are the coordinates of the point
in Euclidean $3$-space at which the two curves intersect?

\item Use the dot product formula to compute the angle $\vartheta$ between (the
tangent vectors to) your two example curves in a) at the point at which the
curves intersect. [DS,20-21]
\end{enumerate}
\end{exercise}

Sometimes displacement is measured by showing how a given point is displaced,
as in%
\[
\hat{V}=\hat{X}_{2}-\hat{X}_{1}=\left(  \hat{x}_{2}-\hat{x}_{1},\hat{y}%
_{2}-\hat{y}_{1},\hat{z}_{2}-\hat{z}_{1}\right)  ,
\]
and sometimes displacement is expressed as the instantaneous velocity of a
point moving along a curve as in%
\[
\hat{V}=\frac{d\hat{X}\left(  t\right)  }{dt}=\left(  \frac{d\hat{x}\left(
t\right)  }{dt},\frac{d\hat{y}\left(  t\right)  }{dt},\frac{d\hat{z}\left(
t\right)  }{dt}\right)  .
\]
[DS,30ff].

In matrix notation we can think of
\[
\hat{X}_{2}-\hat{X}_{1}=\left(  \hat{x}_{2}-\hat{x}_{1},\hat{y}_{2}-\hat
{y}_{1},\hat{z}_{2}-\hat{z}_{1}\right)
\]
as a $1\times3$ matrix $\left(  \hat{X}_{2}-\hat{X}_{1}\right)  $. Then we can
write the formula for the distance between two points $\hat{X}_{1}$ and
$\hat{X}_{2}$ in Euclidean $3$-space in terms of the dot-product%
\begin{equation}
d\left(  \hat{X}_{1},\hat{X}_{2}\right)  =\sqrt{\left(  \hat{X}_{2}-\hat
{X}_{1}\right)  \bullet\left(  \hat{X}_{2}-\hat{X}_{1}\right)  } \label{13}%
\end{equation}
or in terms of the matrix product%
\[
d\left(  \hat{X}_{1},\hat{X}_{2}\right)  =\sqrt{\left(  \left(  \hat{X}%
_{2}-\hat{X}_{1}\right)  \right)  \cdot\left(  \left(  \hat{X}_{2}-\hat{X}%
_{1}\right)  \right)  ^{t}}.
\]


\subsection*{Length of a smooth curve in Euclidean $3$-space}

\begin{exercise}
Compute the length of the tangent vector
\[
l(t)=\sqrt{\frac{d\hat{X}}{dt}\bullet\frac{d\hat{X}}{dt}}%
\]
to each of your two example curves in Exercise \ref{1} at each of their points.
\end{exercise}

\begin{definition}
The length $L$ of the curve $\hat{X}\left(  t\right)  $, $t\in\left[
b,e\right]  $, in Euclidean $3$-space is obtained by integrating the length of
the tangent vector to the curve, that is,%
\[
L=
{\displaystyle\int\nolimits_{b}^{e}}
l\left(  t\right)  \,dt.
\]
[DS,82] Notice that the length of any curve only depends on the definition of
the dot-product. That is, if we know the formula for the dot-product, we know
(the formula for) the length of any curve.
\end{definition}

Our first example is the path%
\begin{align}
\left(  \hat{x}\left(  t\right)  ,\hat{y}\left(  t\right)  ,\hat{z}\left(
t\right)  \right)   &  =\left(  R\cdot \sin\left(  t\right)  ,0,R\cdot
\cos\left(  t\right)  \right) \label{6}\\
0\,  &  \leq t\leq\pi.
\end{align}


Notice that this path lies on the sphere of radius $R$.

\begin{exercise}
Write the formula for the tangent vector to the path $\left(  \ref{6}\right)
$ at each point using $\left(  \hat{x}\left(  t\right)  ,\hat{y}\left(
t\right)  ,\hat{z}\left(  t\right)  \right)  $-coordinates. Show that the
length of this path is $R\pi$.
\end{exercise}

\begin{exercise}
Compute the length of each of your two example curves in Exercise \ref{1}.
\end{exercise}

\begin{remark}
In this last Exercise, you may easily be confronted with an integral that you
cannot compute. For example, if your curve $\hat{X}_{1}\left(  t\right)  $
happens to describe an ellipse that is not circular, it was proved in the 19
th century that no formula involving only the standard functions from calculus
will give you the length of your path from a fixed beginning point to a
variable ending point on the ellipse. If that kind of thing occurs, go back
and change the definitions of your curves in Exercise \ref{1} until you get
two curves for which you can compute length of your path from a fixed
beginning point to a fixed ending point. [DS,81-82]
\end{remark}

We will want to reserve the notation $\left(  x,y,z\right)  $ for some new
coordinates that we will put on the `same' objects in the next section. These
new coordinates will be chosen to keep the north and south poles from going to
infinity as the radius $R$ of a sphere increases without bound. This change of
viewpoint will eventually let us go non-Euclidean or, in the language of Buzz
Lightyear \textquotedblleft to infinity and beyond." The idea will be like the
change from rectangular to polar coordinates for the plane that you
encountered in calculus, only easier. 

\end{document}
