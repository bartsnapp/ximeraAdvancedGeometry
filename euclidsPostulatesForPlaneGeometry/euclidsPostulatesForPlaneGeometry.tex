\documentclass{ximera}

\usepackage{microtype}
\usepackage{tikz}
\usepackage{tkz-euclide}
\usetkzobj{all}
\tikzstyle geometryDiagrams=[ultra thick,color=blue!50!black]

\renewcommand{\epsilon}{\varepsilon}



\prerequisites{rigidMotions,geometry}
\outcomes{neutralGeometry}

\title{Euclid's postulates for plane geometry} % the title of the activity
\begin{document}
\begin{abstract}
We begin with Euclid's first four assumptions and explore neutral
geometries.
\end{abstract}
\maketitle



\subsection*{Neutral geometry}

%% We first turn our attention to plane (or `flat') two-dimensional
%% geometry.

In Western civilization, the primary source of our understanding of
plane geometry comes from Euclid's \textit{Elements}. The treatise is
of transcendant importance well beyond geometry itself. It is among
the first examples of formal logical deductive reasoning. Certain
fundamentals, that are called \textit{axioms}, are postulated or
`given,' providing the platform on which a `geometry' is built. This
creates a mathematical entity modeling a physical `reality.' Its
properties are arrived at by applying the laws of logic to the given
fundamentals. Euclid gives five axioms for plane geometry, the first
four of which seem to be `obvious' reflections of physical reality. In
paraphrased form, they are:

\begin{axiom}[E1]
Through any point $P$ and any other point $Q$, there lies a unique
line.
\end{axiom}

\begin{axiom}[E2] 
Given any two segments $\bar{AB}$ and $\bar{CD}$, there is a
segment $\bar{AE}$ such that $B$ lies on $\bar{AE}$ and
$\left\vert CD\right\vert =\left\vert BE\right\vert $

Note: We will use both $\left\vert AB\right\vert$ and $d(A,B)$ to
denote the distance between two points $A$ and $B$.
\end{axiom}

\begin{axiom}[E3]
Given and point $P$ and any positive real number $r$, there exists a
(unique) circle of radius $r$ and center $P$. 

Said another way, if you move away from point $P$ along a line in any
direction, you will encounter a unique point at distance $r$ from $P$.
\end{axiom}

\begin{axiom}[E4]
All right angles are congruent. 

Note: A right angle is defined as follows. Let $C$ be the midpoint on
the segment $\bar{AB}$. Let $E$ be any point not equal to
$C$. The angle $\angle ACE$ is called a right angle if $\angle ACE$ is
congruent to $\angle BCE$. [MJG,17-18]
\end{axiom}

\begin{definition}
If we are only given axioms E1--E4, we will call our
geometry \textbf{neutral geometry} (\textbf{NG}).
\end{definition}

\begin{definition}
In \textbf{NG}, two distinct lines are called parallel if and only if they
don't intersect.
\end{definition}

One implicit assumption of two-dimensional neutral geometry is the
existence of (a group of) rigid motions or congruences. That is, it is
assumed that given any point $A$ and any vector
$V$ emanating from $A$ and given any second point
$B$ in the geometry and any vector $W$ emanating
from $B$, then there is a transformation $M$ of the
geometry such that:
\begin{enumerate}
\item $M\cdot A=B$.
\item $M\cdot V$ is a positive scalar multiple of $W$.
\item For all points $A$ and $B$ in the geometry,
  $M$ leaves the distance between them unchanged:
\[
d\left(M\cdot A,  M\cdot A\right) =d\left( A,B\right).
\]
\item For any two vectors $U$ and $V$ emanating from
  $A$, the angle between $M\cdot U$ and
  $M\cdot V$ is the same as the angle between $U$
  and $V$.
\end{enumerate}


%% \begin{exercise}
%% Using a sketch on grid paper \textbf{and} an algebraic formulation in
%% the Euclidean plane, give a concrete example of a rigid motion that
%% takes $\left( 1,2\right) $ to $\left( 3,5\right) $ and the vector
%% $\left( 1,0\right) $ emanating from $\left( 1,2\right) $ to a positive
%% multiple of the vector $\left( 0,2\right) $ emanating from $\left(
%% 3,5\right)$. 
%% \begin{solution}
%% WHAT KIND OF SOLUTION IS THIS?
%% \end{solution}
%% Where is the point $(0,0)$ mapped to by this rigid
%% motion?
%% \begin{solution}
%% The point $(0,0)$ is mapped to 
%% \begin{matrixanswer}[hello]
%% hello
%% \end{matrixanswer}
%% \end{solution}
%% \end{exercise}


\begin{question}
Given a point $(2,3)$, find a matrix that will reflect this point
across the line $y=x$.
\begin{solution}
  \begin{hint}
    Be sure to express $(2,3)$ as a column vector.
  \end{hint}
  \begin{matrixAnswer}[name=M]
    correctMatrix = [['0','1'],['1','0']]
\end{matrixAnswer}
\end{solution}
Given a point $(2,3)$, find a matrix that will rotate this point
counterclockwise $90^\circ$.
\begin{solution}
  \begin{hint}
    Be sure to express $(2,3)$ as a column vector.
  \end{hint}
  \begin{matrixAnswer}[name=M]
    correctMatrix = [['0','-1'],['1','0']]
\end{matrixAnswer}
\end{solution}
Can you find a $2\times 2$ matrix that will translate $(0,0)$ to
$(2,3)$? Explain why or why not.
\begin{solution}
\begin{freeResponse}
If such a matrix exists, it must have the form:
\[
M = 
\begin{bmatrix}
a & b \\
c & d
\end{bmatrix}.
\]
If 
\[
\begin{bmatrix}
a & b \\
c & d
\end{bmatrix}
\begin{bmatrix}
0 \\
0 
\end{bmatrix} = 
\begin{bmatrix}
2 \\
3
\end{bmatrix},
\]
then 
\begin{align*}
a\cdot 0 + b\cdot 0 &= 2\\
c\cdot 0 + d\cdot 0 &= 3
\end{align*}
but this is impossible, therefore there is no such matrix. From this
we can see that translations mapping $\R^2\to \R^2$ \textbf{cannot} be
represented as $2\times 2$ matrices. Instead we should represent them
as column vectors that we add to points. This seeming difference
between the rigid motions of rotations, reflections, and translations
will be alleviated later in this course.
\end{freeResponse}
\end{solution}
\end{question}

\begin{exploration}
Think back to high school days and write the congruence rules SSS,
SAS, and ASA. Be very careful with your wording---it had better be
that triangles can be \textbf{moved onto each other by a rigid motion}
if and only if they satisfy any one (and hence all) of the three
properties (SSS, SAS, ASA).
\begin{solution}
Suppose there is a triangle $\triangle ABC$ and another triangle
$\triangle DEF$. 
\begin{freeResponse}
\paragraph*{SSS}
If there is a single rigid motion that moves
\begin{align*}
\bar{AB} &\mapsto \bar{DE}\\
\bar{BC} &\mapsto \bar{EF}\\
\bar{CA} &\mapsto \bar{FD},
\end{align*}
then this rigid motion moves $\triangle ABC$ to $\triangle DEF$.
\paragraph*{SAS}
If there is a single rigid motion that moves
\begin{align*}
\bar{AB} &\mapsto \bar{DE}\\
\angle{ABC} &\mapsto \angle{DEF}\\
\bar{CA} &\mapsto \bar{FD},
\end{align*}
then this rigid motion moves $\triangle ABC$ to $\triangle DEF$.
\paragraph*{ASA}
If there is a single rigid motion that moves
\begin{align*}
\angle{ABC} &\mapsto \angle{DEF}\\
\bar{AB} &\mapsto \bar{DE}\\
\angle{BAC} &\mapsto \angle{EDF},
\end{align*}
then this rigid motion moves $\triangle ABC$ to $\triangle DEF$.
The upshot is that:
\begin{center}
\textbf{Congruence theorems are theorems about rigid motions.}
\end{center}
\end{freeResponse}
\end{solution}
\end{exploration}

\begin{question}
Give an example to show that there is no universal SSA law.
\begin{solution}
\begin{freeResponse}
Consider:
HERE HERE
\end{freeResponse}
\end{solution}
Can you find a restriction that will allow for an ``SSA-type'' law?
\begin{solution}
Perhaps the simplest restriction is forcing the angle to be a right
angle.
\end{solution}
\end{question}

Although it is a bit tedious to show (and we will not ask you to do it
here), using only E1--E4 you can derive the usual rules for congruent
triangles (SSS, SAS, ASA). Thus these laws hold in any neutral
geometry.

\begin{question}
\label{18} In the diagram below, we see the intersection of $\bar{BC}$ and $\bar{AE}$. Suppose that $\left\vert
BD\right\vert =\left\vert CD\right\vert $ and $\left\vert
AD\right\vert =\left\vert ED\right\vert $.
\begin{image}
\begin{tikzpicture}[geometryDiagrams]
\coordinate (A) at (0,2);
\coordinate (B) at (2,5);
\coordinate (C) at (6.5,.5);
\coordinate (E) at (8,4);
\coordinate (D) at (4,3);
\draw (A)--(B)--(C)--(E)--(D)--cycle;
\tkzMarkSegments[mark=|](B,D D,C)
\tkzMarkSegments[mark=||](A,D D,E)
\tkzLabelPoints[above](B,D,E)
\tkzLabelPoints[below](A,C)
%\draw[step=.5cm] (0,0) grid (10,5);
\end{tikzpicture}
\end{image}
Show that triangle $\triangle BDA$ and triangle $\triangle CDE$ are
congruent. [MJG,138]
\begin{solution}
\begin{hint}
First you should explain why $\angle BDA = \angle CDE$.
\end{hint}
\begin{hint}
Next you should use one of the congruence properties above. 
\end{hint}
\begin{freeResponse}
To start, we claim that $\angle BDA = \angle CDE$. Labeling our
diagram above,
\begin{image}
\begin{tikzpicture}[geometryDiagrams]
\coordinate (A) at (0,2);
\coordinate (B) at (2,5);
\coordinate (C) at (6.5,.5);
\coordinate (E) at (8,4);
\coordinate (D) at (4,3);
\draw (A)--(B)--(C)--(E)--(D)--cycle;

\tkzMarkAngle[size=0.7cm,thin](B,D,A)
\tkzLabelAngle[pos = -0.4](B,D,A){$\alpha$}

\tkzMarkAngle[arc=ll,size=0.5cm,thin](E,D,B)
\tkzLabelAngle[pos = 0.25](E,D,B){$\beta$}

\tkzMarkAngle[size=0.7cm,thin](C,D,E)
\tkzLabelAngle[pos = 0.4](C,D,E){$\gamma$}

\tkzMarkAngle[arc=ll,size=0.5cm,thin](A,D,C)
\tkzLabelAngle[pos = 0.25](A,D,C){$\delta$}

%\draw[step=.5cm] (0,0) grid (10,5);
\end{tikzpicture}
\end{image}
we see that 
\begin{align*}
\alpha+\beta &= 180^\circ\\
\beta + \gamma &= 180^\circ.
\end{align*}
Subtracting the equations above we find that $\alpha=\gamma = 0$.
This means that $\alpha = \gamma$ and hence $\angle BDA = \angle
CDE$. Since we know that $\left\vert BD\right\vert =\left\vert
CD\right\vert $ and $\left\vert AD\right\vert =\left\vert
ED\right\vert $ we may now apply SAS to prove that triangle $\triangle
BDA$ and triangle $\triangle CDE$ are congruent.
\end{freeResponse}
\end{solution}
\end{question}

\begin{question}\hfil
\label{118} 
\begin{enumerate}
\item\label{Exp:ExtAngle} Show in neutral geometry that, for $\triangle ABC$
\begin{image}
\begin{tikzpicture}[geometryDiagrams]
\coordinate (A) at (0,2);
\coordinate (B) at (2,5);
\coordinate (C) at (6.5,2);
\coordinate (F) at (9,2);
\draw (A)--(B)--(C)--cycle;
\draw (C)--(F);
\tkzLabelPoints[above](B)
\tkzLabelPoints[below](A,C)
\tkzMarkAngle[size=0.5cm,thin](F,C,B)
\tkzLabelAngle[pos = 0.25](F,C,B){$\epsilon$}

\tkzMarkAngle[size=0.6cm,thin](A,B,C)
\tkzLabelAngle[pos = 0.35](A,B,C){$\beta$}

\tkzMarkAngle[size=0.6cm,thin](C,A,B)
\tkzLabelAngle[pos = 0.35](C,A,B){$\alpha$}

%\draw[step=.5cm] (0,0) grid (10,5);
\end{tikzpicture}
\end{image}
the exterior angle $\epsilon$ of the triangle at $C$ is greater than
either remote interior angle $\alpha$ or $\beta$. [MJG,119]
\item Use \ref{Exp:ExtAngle} to show that the sum of any two angles of a triangle is less than $180^{\circ}$.
\end{enumerate}
\begin{solution}
\begin{hint}
Add line segments to the diagram above, and attempt to use the
previous problem.
\end{hint}
\begin{freeResponse} 
For part (a), consider the midpoint of $\bar{BC}$, call it
$D$. Considering the segment $\bar{AD}$, extend $\bar{AD}$ so that
it's length is $2\cdot|AD|$ and call the endpoint $E$. After
constructing $\bar{CE}$, we have the following:
\begin{image}
\begin{tikzpicture}[geometryDiagrams]
\coordinate (A) at (0,2);
\coordinate (B) at (2,5);
\coordinate (C) at (6.5,2);
\coordinate (D) at (4.25,3.5);
\coordinate (E) at (8.5,5);
\coordinate (F) at (9,2);
\draw (A)--(B)--(C)--cycle;
\draw (C)--(F);
\draw[thin] (A)--(E);
\draw[thin] (C)--(E);
\tkzMarkSegments[mark=|](B,D D,C)
\tkzMarkSegments[mark=||](A,D D,E)
\tkzLabelPoints[above](B,D,E)
\tkzLabelPoints[below](A,C)
\tkzMarkAngle[size=0.5cm,thin](F,C,B)
\tkzLabelAngle[pos = 0.25](F,C,B){$\epsilon$}

\tkzMarkAngle[size=0.6cm,thin](A,B,C)
\tkzLabelAngle[pos = 0.35](A,B,C){$\beta$}

\tkzMarkAngle[size=0.6cm,thin](C,A,B)
\tkzLabelAngle[pos = 0.35](C,A,B){$\alpha$}

%\draw[step=.5cm] (0,0) grid (10,5);
\end{tikzpicture}
\end{image}
From our previous work, we know that $\beta = \angle ECD$ and from the
diagram we can see that $\beta < \epsilon$.

Now starting with our original diagram again, consider the midpoint of
$\bar{AC}$, call it $D'$. Considering the segment $\bar{BD'}$, extend
$\bar{BD'}$ so that it's length is $2\cdot|BD'|$ and call the endpoint
$E'$. After constructing $\bar{CE'}$, we have the following:
\begin{image}
\begin{tikzpicture}[geometryDiagrams]
\coordinate (A) at (0,2);
\coordinate (B) at (2,5);
\coordinate (C) at (6.5,2);
\coordinate (D') at (3.25,2);
\coordinate (E') at (4.5,-1);
\coordinate (F) at (9,2);
\draw (A)--(B)--(C)--cycle;
\draw (C)--(F);
\draw[thin] (B)--(E');
\draw[thin] (C)--(E');
\tkzMarkSegments[mark=|](A,D' D',C)
\tkzMarkSegments[mark=||](B,D' D',E')
\tkzLabelPoints[above](B)
\tkzLabelPoints[below](A,C,E')
\tkzLabelPoints[below right](D')
\tkzMarkAngle[size=0.5cm,thin](F,C,B)
\tkzLabelAngle[pos = 0.25](F,C,B){$\epsilon$}

\tkzMarkAngle[size=0.6cm,thin](A,B,C)
\tkzLabelAngle[pos = 0.35](A,B,C){$\beta$}

\tkzMarkAngle[size=0.6cm,thin](C,A,B)
\tkzLabelAngle[pos = 0.35](C,A,B){$\alpha$}

%\draw[step=.5cm] (0,0) grid (10,5);
\end{tikzpicture}
\end{image}
From our previous work, we know that $\alpha = \angle
D'CE'$. Finally, extend $\bar{BC}$: 
\begin{image}
\begin{tikzpicture}[geometryDiagrams]
\coordinate (A) at (0,2);
\coordinate (B) at (2,5);
\coordinate (C) at (6.5,2);
\coordinate (D') at (3.25,2);
\coordinate (E') at (4.5,-1);
\coordinate (F) at (9,2);
\coordinate (F') at (8.525,.65);
\draw (A)--(B)--(C)--cycle;
\draw (C)--(F);
\draw (C)--(F');
\draw[thin] (B)--(E');
\draw[thin] (C)--(E');
\tkzMarkSegments[mark=|](A,D' D',C)
\tkzMarkSegments[mark=||](B,D' D',E')
\tkzLabelPoints[above](B)
\tkzLabelPoints[below](A,C,E')
\tkzLabelPoints[below right](D')
\tkzMarkAngle[size=0.5cm,thin](F,C,B)
\tkzLabelAngle[pos = 0.25](F,C,B){$\epsilon$}

\tkzMarkAngle[size=0.6cm,thin](A,B,C)
\tkzLabelAngle[pos = 0.35](A,B,C){$\beta$}

\tkzMarkAngle[size=0.6cm,thin](C,A,B)
\tkzLabelAngle[pos = 0.35](C,A,B){$\alpha$}

%\draw[step=.5cm] (0,0) grid (10,5);
\end{tikzpicture}
\end{image}
Hence, we may use vertical angles to conclude that $\alpha <
\epsilon$.

For part (b), consider:
\begin{image}
\begin{tikzpicture}[geometryDiagrams]
\coordinate (A) at (0,2);
\coordinate (B) at (2,5);
\coordinate (C) at (6.5,2);
\coordinate (F) at (9,2);
\draw (A)--(B)--(C)--cycle;
\draw (C)--(F);
\tkzLabelPoints[above](B)
\tkzLabelPoints[below](A,C)
\tkzMarkAngle[size=0.5cm,thin](F,C,B)
\tkzLabelAngle[pos = 0.25](F,C,B){$\epsilon$}

\tkzMarkAngle[size=0.6cm,thin](A,B,C)
\tkzLabelAngle[pos = 0.35](A,B,C){$\beta$}

\tkzMarkAngle[size=0.6cm,thin](C,A,B)
\tkzLabelAngle[pos = 0.35](C,A,B){$\alpha$}

\tkzMarkAngle[size=0.9cm,thin](B,C,A)
\tkzLabelAngle[pos = 0.7](B,C,A){$\gamma$}

%\draw[step=.5cm] (0,0) grid (10,5);
\end{tikzpicture}
\end{image}
Write
\[
\alpha + \gamma < \epsilon + \gamma = 180^\circ.
\]
In a similar fashion,
\[
\beta + \gamma < \epsilon + \gamma  = 180^\circ.
\]
To consider $\alpha+\beta$, we should repeat the argument from part
(a) again with the exterior angle at $A$. From this we will see
$\alpha + \beta < 180^\circ$.
\end{freeResponse}
\end{solution}
\end{question}

\begin{question}
Show in \textbf{NG} that, if two lines cut by a transversal line have
a pair of congruent alternate interior angles, then they are
parallel. [MJG,117]
\begin{solution}
\begin{hint}
Suppose the assertion is false for some pair of lines. Find a triangle
that violates the conclusion of the previous question.
\end{hint}
Seeking a contradiction, consider the following diagram with
nonparallel lines $\l$ and $m$:
\begin{image}
\begin{tikzpicture}[geometryDiagrams]
\coordinate (m) at (0,2);
\coordinate (l) at (0,5);
\coordinate (E) at (8,2);
\coordinate (F) at (9,2);
\draw (l)--(E);
\draw (m)--(F);
%\draw[step=.5cm] (0,0) grid (10,5);
\end{tikzpicture}
\end{image}
HERE HERE see image

From the diagram we have $\epsilon + \beta = 180^\circ$.  Since we are
assuming that alternate interior angles are congruent, we see that
$\alpha+\beta = 180^\circ$. This is a contradiction.
\end{solution}
\end{question}







\subsection*{Sum of angles in a triangle in NG}


For many centuries, mathematicians attempted to prove that the sum of
the angles in a triangle was $180^{\circ}$ using only E1--E4. However,
the discovery of \textit{hyperbolic geometry} about two centuries ago,
proved this to be a futile pursuit. The thing that separates
hyperbolic geometry from Euclidean (plane) geometry is the sum of
the angles in a triangle. (If you had a good geometry course in high
school, you may remember that you had to use Euclid's fifth axiom in
order to show that the sum of the angles in a triangle was
$180^{\circ}.$ But more on that later.)

There is one important fact about the sum of the angles in a triangle that you
\textit{can} prove in \textbf{NG}, that is, without invoking Euclid's fifth
axiom. We will in fact accomplish that in this section.

\begin{question}\label{20}
Show that the sum of the angles in $\triangle ACE$ is the same as the
sum of the angles in $\triangle ACB$
\begin{image}
\begin{tikzpicture}[geometryDiagrams]
\coordinate (A) at (0,2);
\coordinate (B) at (2,5);
\coordinate (C) at (6.5,.5);
\coordinate (E) at (8,4);
\coordinate (D) at (4,3);
\draw (A)--(B)--(C)--(E)--(D)--cycle;
\draw (A)--(C);
\tkzMarkSegments[mark=|](B,D D,C)
\tkzMarkSegments[mark=||](A,D D,E)
\tkzLabelPoints[above](B,D,E)
\tkzLabelPoints[below](A,C)
%\draw[step=.5cm] (0,0) grid (10,5);
\end{tikzpicture}
\end{image}
\end{question}

\begin{question}\label{22} 
Suppose that there is a triangle $\triangle ABC$ in
\textbf{NG} for which the sum of the angles in a triangle $\triangle ABC$ is
$\left( 180+x\right)^\circ$ with $x>0$. Construct a new triangle,
$\triangle ACE$ such that the sum of the angles in a triangle is still
$\left( 180+x\right)^\circ$, but one of the angles of $\triangle ACE$
is no more than half the size of $\angle BAC$.
\begin{image}
\begin{tikzpicture}[geometryDiagrams]
\coordinate (A) at (0,2);
\coordinate (B) at (2,5);
\coordinate (C) at (6.5,.5);
\coordinate (E) at (8,4);
\coordinate (D) at (4,3);

\draw (A)--(B)--(C)--cycle;
\draw[thin] (A)--(E)--(C);

\tkzLabelPoints[above](B,D,E)
\tkzLabelPoints[below](A,C)

\tkzMarkAngle[size=1cm,thin](D,A,B)
\tkzLabelAngle[pos = 0.6](D,A,B){$\alpha_1$}

\tkzMarkAngle[size=1.2cm,thin](C,A,D)
\tkzLabelAngle[pos = 0.7](C,A,D){$\alpha_2$}
%\draw[step=.5cm] (0,0) grid (10,5);
\end{tikzpicture}
\end{image}
Be sure to carefully explain how points $D$ and $E$ are constructed.
Note: Despite the diagram above, this new `smaller' angle may or may
not have vertex $A$. [MJG,125-127]
\end{question}

\begin{question}
\label{21} Suppose that there is a triangle $\triangle ABC$ in
\textbf{NG} for which the sum of the angles in a triangle $\triangle ABC$ is
$\left( 180+x\right)^\circ$ with $x>0$. Let $\alpha$ denote the
measure of $\angle BAC$. Repeat the construction in Question \ref{22}
over and over again $n$-times to construct a triangle with the sum of
its angles still equal to $\left( 180+x\right)^\circ$ but such that
one of its angles has size less than%
\[
\frac{1}{2^{n}}\alpha.
\]
\end{question}

\begin{question}
\label{121} Suppose that there is a triangle $\triangle ABC$ in
\textbf{NG} for which the sum of the angles in a triangle $\triangle ABC$ is
$\left( 180+x\right)^\circ$ with $x>0$. Show that there is a positive
integer $n$ so that, if you repeat the construction in
Question \ref{22} over and over again $n$-times, the result will be a
triangle with the sum of its angles still equal to $\left(
180+x\right)^\circ$ but with one of its angles having measure less
than $x$. [MJG,125-127]
\end{question}


On the other hand, by Question \ref{118} (b), you cannot have a
triangle with two angles summing to more than $180^\circ$. Hence, from
our work above, we can prove the following theorem:

\begin{theorem}
In \textbf{NG}, the sum of the interior angles in any triangle is no greater
than $180^\circ$.
\end{theorem}

\begin{proof}
We argue by contradiction. Start with $\triangle ABC$ as in Question \ref{22}
for which the sum of the angles in a triangle $\triangle ABC$ is $\left(
180+x\right)^\circ$ with $x>0$. Suppose the measure of the angle at $A$ is denoted by $\alpha$.
By Question \ref{22} there exists a triangle $\triangle A^{\left(  1\right)
}B^{\left(  1\right)  }C^{\left(  1\right)  }$ such that the sum of the angles
in a triangle $\triangle A^{\left(  1\right)  }B^{\left(  1\right)
}C^{\left(  1\right)  }$ is $\left(  180+x\right)^\circ$ and the measure of the angle at the vertex $A^{\left(  1\right)  }$ is less
than or equal to $\frac{\alpha}{2}$. By Question \ref{21} there is a triangle
$\triangle A^{\left(  n\right)  }B^{\left(  n\right)  }C^{\left(  n\right)  }$
such that the sum of the angles in a triangle $\triangle A^{\left(  n\right)
}B^{\left(  n\right)  }C^{\left(  n\right)  }$ is $\left(  180+x\right)^\circ$ and the measure $\alpha_{n}$ of the angle at the vertex $A^{\left(
n\right)  }$ is less than or equal to $\frac{\alpha}{2^{n}}$. If $n$ is
sufficiently big,%
\[
\frac{\alpha}{2^{n}}<x.
\]
So, for that value of $n$, if $\beta_{n}$ is the measure of the angle of
$\triangle A^{\left(  n\right)  }B^{\left(  n\right)  }C^{\left(  n\right)  }$
at the vertex $B^{\left(  n\right)  }$ and $\gamma_{n}$ is the measure of the
angle of $\triangle A^{\left(  n\right)  }B^{\left(  n\right)  }C^{\left(
n\right)  }$ at the vertex $C^{\left(  n\right)  }$, then we have the two
relations%
\begin{gather*}
\alpha_{n}<\frac{\alpha}{2^{n}}<x\\
\alpha_{n}+\beta_{n}+\gamma_{n}=180+x.
\end{gather*}
So%
\[
\beta_{n}+\gamma_{n}=180+\left(  x-\alpha_{n}\right)  >180.
\]
But this is impossible by Question \ref{118} (b).
\end{proof}

\begin{question}
Show the following:
\begin{enumerate}
\item The sum of the interior angles in any quadrilateral is no greater than $360^\circ$.
\item The sum of the interior angles of an $n$-gon is no greater than
$\left( n-2\right)\cdot180^\circ$.
\end{enumerate}
\end{question}
\end{document}
