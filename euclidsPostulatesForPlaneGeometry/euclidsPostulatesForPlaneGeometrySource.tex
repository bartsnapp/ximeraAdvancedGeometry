\headline{In this activity we explore neutral geometries.}
\activitytitle{Euclid's postulates for plane geometry} % the title of the activity
\prerequisites{rigidMotions}
\outcomes{neutralGeometry}



%% We first turn our attention to plane (or `flat') two-dimensional
%% geometry.

In Western civilization, the primary source of our understanding of
this geometry comes from Euclid's \textit{Elements}. The treatise is
of transcendant importance well beyond geometry itself, because it is
among the first, and perhaps the most influential single example of
organized, formal logical deductive reasoning. Certain fundamentals,
that are called \textit{axioms}, are postulated or `given,' providing
the platform on which a `geometry' is built, that is, a mathematical
entity modeling a physical `reality'. Its properties are arrived at by
applying the laws of logic to the given fundamentals. Euclid gives
five axioms for plane geometry, the first four of which seem to be
`obvious' reflections of physical reality. In paraphrased form, they
are:

\begin{axiom}
(E1) Through any point $P$ and any other point $Q$, there lies a
unique line.
\end{axiom}

\begin{axiom}
(E2) Given any two segments $\overline{AB}$ and $\overline{CD}$, there
is a segment $\overline{AE}$ such that $B$ lies on $\overline{AE}$ and
$\left\vert CD\right\vert =\left\vert BE\right\vert $

(NB: In plane geometry we often use the notation $\left\vert
CD\right\vert $ to denote the distance between two points $A$ and $B$
rather than the notation $d\left( A,B\right) $ used previously.
\end{axiom}

\begin{axiom}
(E3) Given and point $P$ and any positive real number $r$, there
exists a (unique) circle of radius $r$ and center $P$. (Said another
way, if you move away from $P$ along a line in any direction, you will
encounter a unique point at distance $r$ from $P$.)
\end{axiom}

\begin{axiom}
(E4) All right angles are congruent. (A right angle is defined as
follows. Let $C$ be the midpoint on the segment $\overline{AB}$. Let
$E$ be any point not equal to $C$. The angle $\angle ACE$ is called a
right angle if $\angle ACE$ is congruent to $\angle BCE$.) [MJG,17-18]
\end{axiom}

\begin{definition}
If we are only given axioms E1--E4, we will call our
geometry \textbf{neutral geometry} (\textbf{NG}).
\end{definition}

\begin{definition}
In \textbf{NG}, two distinct lines are called parallel if and only if they
don't intersect.
\end{definition}

One implicit assumption of two-dimensional neutral (and Euclidean)
geometry is the existence of (a group of) rigid motions or
congruences. That is, it is assumed that given any point $\hat{A}$ and
any tangent vector $\hat{V}$ emanating from $\hat{A}$ and given any
second point $\hat{B}$ in the geometry and any tangent vector
$\hat{W}$ emanating from $\hat{B}$, then there is a transformation
$\hat{M}$ of the geometry such that
\begin{enumerate}
\item $\hat{M}$ takes $\hat{A}$ to $\hat{B}$,
\item $\hat{M}$ takes $\hat{V}$ and to a positive scalar multiple times $\hat{W}$ to $\hat{M}\left( \hat{V}\right) $,
\item for all points $\hat{A}^{\prime},\hat{A}^{\prime\prime}$ in the
geometry, $\hat{M}$ leaves the distance between them unchanged, that
is,
\[
\left\vert \hat{M}\left(  \hat{A}^{\prime}\right)  \hat{M}\left(  \hat
{A}^{\prime\prime}\right) \right\vert =\left\vert \hat{A}^{\prime}\hat
{A}^{\prime\prime}\right\vert ,
\]
\item for any two tangent vectors $\hat{V}^{\prime}$ and
$\hat{V}^{\prime\prime}$ emanating from $\hat{A}$, the angle between
$\hat{M}\left( \hat{V}^{\prime }\right) $ and
$\hat{M}\left( \hat{V}^{\prime\prime}\right) $ is the same as the
angle between $\hat{V}^{\prime}$ and $\hat{V}^{\prime\prime}$.
\end{enumerate}


\begin{exercise}
Using a sketch on grid paper \textbf{and} an algebraic formulation in
the Euclidean plane, give a concrete example of a rigid motion that
takes $\left( 1,2\right) $ to $\left( 3,5\right) $ and the tangent
vector $\left( 1,0\right) $ emanating from $\left( 1,2\right) $ to a
positive multiple of the tangent vector $\left( 0,2\right) $ emanating
from $\left( 3,5\right)$. Where is the point $(0,0)$ mapped to by this
rigid motion?
\end{exercise}

\begin{exploration}
Think back to high school days and write the congruence rules SSS,
SAS, and ASA. Be very careful with your wording---it had better be
that triangles can be \textbf{moved onto each other by a rigid motion}
if and only if they satisfy any one (and hence all) of the three
properties (SSS, SAS, ASA).
\end{exploration}

\begin{question}
Give a counterexample to show that there is no universal SSA law.  Can
you find a restriction that will allow for an ``SSA-type'' law?
\end{question}

Although it is a bit tedious to show (and we will not ask you to do it
here), using only E1--E4 you can derive the usual rules for congruent
triangles (SSS, SAS, ASA). Thus these laws hold in any neutral
geometry, that is, in any geometry satisfying E1--E4.

\begin{question}
\label{18} Suppose, in the diagram below that $\left\vert
BD\right\vert =\left\vert CD\right\vert $ and $\left\vert
AD\right\vert =\left\vert ED\right\vert $.
\[
\begin{tikzpicture}[geometryDiagrams]
\coordinate (A) at (0,2);
\coordinate (B) at (2,5);
\coordinate (C) at (6.5,.5);
\coordinate (E) at (8,4);
\coordinate (D) at (4,3);
\draw (A)--(B)--(C)--(E)--(D)--cycle;
\tkzLabelPoints[above](B,D,E)
\tkzLabelPoints[below](A,C)
%\draw[step=.5cm] (0,0) grid (10,5);
\end{tikzpicture}
\]
Show that triangle $\triangle BDA$ and triangle $\triangle CDE$ are
congruent. [MJG,138]
\begin{solution}
\begin{hint}
First you should explain why $\angle BDA = \angle CDE$.
\end{hint}
\begin{hint}
Next you should use one of the congruence properties above. 
\end{hint}
\answer[free-response]{To start, we claim that $\angle BDA = \angle CDE$. Labeling our diagram above, 
\[
\begin{tikzpicture}[geometryDiagrams]
\coordinate (A) at (0,2);
\coordinate (B) at (2,5);
\coordinate (C) at (6.5,.5);
\coordinate (E) at (8,4);
\coordinate (D) at (4,3);
\draw (A)--(B)--(C)--(E)--(D)--cycle;

\tkzMarkAngle[size=0.7cm,thin](B,D,A)
\tkzLabelAngle[pos = -0.4](B,D,A){$\alpha$}

\tkzMarkAngle[arc=ll,size=0.5cm,thin](E,D,B)
\tkzLabelAngle[pos = 0.25](E,D,B){$\beta$}

\tkzMarkAngle[size=0.7cm,thin](C,D,E)
\tkzLabelAngle[pos = 0.4](C,D,E){$\gamma$}

\tkzMarkAngle[arc=ll,size=0.5cm,thin](A,D,C)
\tkzLabelAngle[pos = 0.25](A,D,C){$\delta$}

%\draw[step=.5cm] (0,0) grid (10,5);
\end{tikzpicture}
\]
we see that 
\begin{align*}
\alpha+\beta &= 180^\circ\\
\beta + \gamma &= 180^\circ.
\end{align*}
Subtracting the equations above we fine that $\alpha=\gamma = 0$.
This means that $\alpha = \gamma$ and hence $\angle BDA = \angle
CDE$. Since we know that $\left\vert BD\right\vert =\left\vert
CD\right\vert $ and $\left\vert AD\right\vert =\left\vert
ED\right\vert $ we may now apply SAS to prove that triangle $\triangle
BDA$ and triangle $\triangle CDE$ are congruent.}
\end{solution}
\end{question}

\begin{question}
\label{118} 
\begin{enumerate}
\item\label{Exp:ExtAngle} Show in neutral geometry that, for $\triangle ABC$
\[
\begin{tikzpicture}[geometryDiagrams]
\coordinate (A) at (0,2);
\coordinate (B) at (2,5);
\coordinate (C) at (6.5,2);
\coordinate (F) at (9,2);
\draw (A)--(B)--(C)--cycle;
\draw (C)--(F);
\tkzLabelPoints[above](B)
\tkzLabelPoints[below](A,C)
\tkzMarkAngle[size=0.5cm,thin](F,C,B)
\tkzLabelAngle[pos = 0.25](F,C,B){$\epsilon$}

\tkzMarkAngle[size=0.6cm,thin](A,B,C)
\tkzLabelAngle[pos = 0.35](A,B,C){$\beta$}

\tkzMarkAngle[size=0.6cm,thin](C,A,B)
\tkzLabelAngle[pos = 0.35](C,A,B){$\alpha$}

%\draw[step=.5cm] (0,0) grid (10,5);
\end{tikzpicture}
\]
the exterior angle $\epsilon$ of the triangle at $C$ is greater than either remote interior angle $\alpha$ or $\beta$. [MJG,119]
\item Use \ref{Exp:ExtAngle} to show that the sum of any two angles of a triangle is less than $180^{\circ}$.
\end{enumerate}
\end{question}

\begin{exploration}
Show in \textbf{NG} that, if two lines cut by a transversal line have
a pair of congruent alternate interior angles, then they are
parallel. [MJG,117]

Hint: Suppose the assertion is false for some pair of lines. Find a
triangle that violates the conclusion of Exercise \ref{118}
part \ref{Exp:ExtAngle}.
\end{exploration}
