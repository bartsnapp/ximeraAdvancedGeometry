\documentclass{ximera}

\title{Hyperbolic geometry is homogeneous}

\begin{document}
\begin{abstract}
Now we will see that hyperbolic geometry looks the same at each point
and in each direction at that point.
\end{abstract}
\maketitle


\subsection*{Rigid motions in $\left(  x,y,z\right)  $-coordinates}

Now \textbf{HG} is a $K$-geometry in $\left( x,y,z\right) $-coordinates, with the equation
\begin{equation}
K\left(  x^{2}+y^{2}\right)  +z^{2}=1 \label{83}%
\end{equation}
with%
\[
K<0
\]
and the $K$-dot product. If we have a curve $X\left( t\right) =\left(
x\left( t\right) ,y\left( t\right) ,z\left( t\right) \right) $ on the
$R$-sphere given in $K$-coordinates as%
\[
1=K\left(  x^{2}+y^{2}\right)  +z^{2},
\]
we have seen that we measure its length $L$ by the formula%
\begin{equation}
L=%
%TCIMACRO{\dint \nolimits_{b}^{e}}%
%BeginExpansion
{\displaystyle\int\nolimits_{b}^{e}}
%EndExpansion
l\left(  t\right)  dt \label{84}%
\end{equation}
where
\begin{equation}
l\left(  t\right)  ^{2}=\frac{dX}{dt}\bullet_{K}\frac{dX}{dt} \label{85}%
\end{equation}
and that we measure angles $\theta$ between tangent vectors
$\hat{V}_{1}$ and $\hat{V}_{2}$ at a point on the $R$-sphere by the
formula%
\[
\theta=\arccos \left(  \frac{V_{1}\bullet_{K}V_{2}}{\left\vert
V_{1}\right\vert _{K}\cdot\left\vert V_{2}\right\vert
_{K}}\right)
\]
where%
\[
\left\vert V\right\vert _{K}^{2}=V\bullet_{K}V.
\]


We again want to explore the condition that a transformation%
\[
\left(  \underline{x},\underline{y},\underline{z}\right)  =\left(
x,y,z\right)  \cdot M
\]
preserve the length of any curve $\left( x\left( t\right) ,y\left(
t\right) ,z\left( t\right) \right) $ lying on the
$R$-sphere. Rewriting the transformation as%
\[
\underline{X}=X\cdot M
\]
all we have to worry about is that%
\[
\frac{d\underline{X}}{dt}\bullet_{K}\frac{d\underline{X}}{dt}=\frac{dX}%
{dt}\bullet_{K}\frac{dX}{dt}.
\]
So, referring to Definition \ref{88} a transformation given by a
matrix $M$ will preserve the length of any path and will preserve the
measure of any angle if $M$ is $K$-orthogonal.

\begin{exercise}
 Show that the matrix%
\[
\left(
\begin{array}
[c]{ccc}%
\cos \theta & \sin \theta & 0\\
-\sin \theta & \cos \theta & 0\\
0 & 0 & 1
\end{array}
\right)
\]
is $K$-orthogonal. Describe geometrically what this transformation is doing to
the $K$-geometry.
\end{exercise}

For our second $K$-rigid motion in \textbf{HG} we will need a pair of
functions
\[
\left(  \cosh \sigma=\frac{e^{\sigma}+e^{-\sigma}}{2},\sinh %
\sigma=\frac{e^{\sigma}-e^{-\sigma}}{2}\right)
\]
that parametrize the unit hyperbola
\[
z^{2}-x^{2}=1
\]
in the same way that $\left(  \cos \sigma,\sin \sigma\right)  $
parametrize the unit circle. That is%
\[
\cosh ^{2}\sigma-\sinh ^{2}\sigma\equiv1.
\]


\begin{exercise}
 Show that the matrix%
\[
\left(
\begin{array}
[c]{ccc}%
\cosh \varphi & 0 & \left\vert K\right\vert ^{1/2}%
\cdot\sinh \varphi\\
0 & 1 & 0\\
\left\vert K\right\vert ^{-1/2}\cdot\sinh \varphi
& 0 & \cosh \varphi
\end{array}
\right)
\]
is $K$-orthogonal. Describe geometrically what this transformation is doing to
the the $K$-geometry.
\end{exercise}

Notice that, when $K>0$, we had the relation%
\[
R^{2}\cdot K=1
\]
where $R$ was the radius of the sphere. The last exercise suggests that when
$K<0$, we should define $R$ by the relation%
\[
R^{2}\cdot\left\vert K\right\vert =1
\]
so that%
\[
R=\left\vert K\right\vert ^{-1/2}.
\]
(Now compare this last Exercise with the corresponding Exercise in Spherical
Geometry.) In what follows, you will find the quantity $\left\vert
K\right\vert ^{-1/2}$ occurring throughout. Feel free to use the (simpler)
notation $R$ for this quantity as you work through the Exercises.

\subsection*{Moving a point to the North Pole by a rigid motion}

So, first of all, the North Pole is the point%
\[
N=\left(  0,0,1\right)  .
\]
Suppose we start with a point%
\[
X_{0}=\left(  x_{0},y_{0},z_{0}\right)
\]
in the geometry, that is, satisfying the equation $\left(  \ref{83}\right)  $.

\begin{exercise}
 Write an explicit $K$-rigid motion%
\[
M_{1}=\left(
\begin{array}
[c]{ccc}%
\cos \theta & \sin \theta & 0\\
-\sin \theta & \cos \theta & 0\\
0 & 0 & 1
\end{array}
\right)
\]
that takes the point $X_{0}$ to a point $X_{1}=\left(  x_{1},0,z_{0}\right)  $.
\end{exercise}

\begin{exercise}
 Write an explicit $K$-rigid motion%
\[
M_{2}=\left(
\begin{array}
[c]{ccc}%
\cosh \varphi & 0 & \left\vert K\right\vert ^{1/2}%
\cdot\sinh \varphi\\
0 & 1 & 0\\
\left\vert K\right\vert ^{-1/2}\cdot\sinh \varphi
& 0 & \cosh \varphi
\end{array}
\right)
\]
that takes the point $X_{1}=\left(  x_{1},0,z_{0}\right)  $ to $N=\left(
0,0,1\right)  $.

Hint: Notice that%
\[
Kx_{1}^{2}+z_{0}^{2}=1=-\left(  -\left\vert K\right\vert ^{1/2}%
\cdot x_{1}\right)  ^{2}+z_{0}^{2}.
\]
So there is a $\varphi$ with
\[
\cosh \varphi=z_{0}%
\]
and%
\[
\sinh \varphi=-\left\vert K\right\vert ^{1/2}\text{\textperiodcentered
}x_{1}.
\]

\end{exercise}

Using these last two Exercises we conclude that the transformation%
\[
\left(  \underline{x},\underline{y},\underline{z}\right)  =\left(
x,y,z\right)  \cdot\left(  M_{1}\cdot M_{2}\right)
\]
is a $K$-rigid motion (why?) and that%
\[
N=\left(  x_{0},y_{0},z_{0}\right)  \cdot\left(  M_{1}\cdot M_{2}\right)
\]
(why?).



\subsection*{Moving a (point, direction) to any other (point, direction) by a
rigid motion}

Let
\[
V_{2}=\left(  a_{2},b_{2},0\right)
\]
be a tangent vector to $K$-geometry at the North Pole $N$.

\begin{exercise}
 Write an explicit $K$-rigid motion%
\[
M_{3}=\left(
\begin{array}
[c]{ccc}%
\cos \theta^{\prime} & \sin \theta^{\prime} & 0\\
-\sin \theta^{\prime} & \cos \theta^{\prime} & 0\\
0 & 0 & 1
\end{array}
\right)
\]
that takes $V_{2}$ to the vector%
\[
\left(  \sqrt{a_{2}^{2}+b_{2}^{2}},0,0\right)  =\left(  \sqrt{V_{2}\bullet
_{K}V_{2}},0,0\right)  .
\]
Why does the transformation given by $M_{3}$ leave the North Pole $N$ fixed?
\end{exercise}

Now suppose we have any point%
\[
X_{0}=\left(  x_{0},y_{0},z_{0}\right)
\]
in $K$-geometry and any $K$-tangent vector%
\[
V_{0}=\left(  a_{0},b_{0},c_{0}\right)
\]
at that point.

\begin{exercise}
Explain why the $K$-rigid motion%
\[
\left(  \underline{x},\underline{y},\underline{z}\right)  =\left(
x,y,z\right)  \cdot\left(  M_{1}\cdot M_{2}\cdot M_{2}\right)
\]
constructed over the last couple of sections takes the point $X_{0}$ to $N$
and the tangent vector $V_{0}$ to $\left(  \sqrt{V_{0}\bullet_{K}V_{0}%
},0,0\right)  $
\end{exercise}

Now suppose that $\left(  X_{0},V_{0}\right)  $ gives a point $X_{0}$ in
$K$-geometry and a tangent direction $V_{0}$ to $K$-geometry at $X_{0}$.
Suppose that $\left(  X_{0}^{\prime},V_{0}^{\prime}\right)  $ gives another
point in $K$-geometry and a tangent direction to $K$-geometry at
$X_{0}^{\prime}$. Finally suppose that%
\[
V_{0}\bullet_{K}V_{0}=V_{0}^{\prime}\bullet_{K}V_{0}^{\prime}.
\]
As above, find a $K$-rigid motion given by%
\[
M=\left(  M_{1}\cdot M_{2}\cdot M_{3}\right)
\]
taking $X_{0}$ to the North Pole and $V_{0}$ to $\left(  \sqrt{V_{0}%
\bullet_{K}V_{0}},0,0\right)  $. Similarly find a $K$-rigid motion given by%
\[
M^{\prime}=\left(  M_{1}^{\prime}\cdot M_{2}^{\prime}\cdot M_{3}^{\prime
}\right)
\]
taking $X_{0}^{\prime}$ to the North Pole and $V_{0}^{\prime}$ to $\left(
\sqrt{V_{0}^{\prime}\bullet_{K}V_{0}^{\prime}},0,0\right)  .$

\begin{exercise}
Explain why the $K$-rigid motion given by%
\[
M\cdot\left(  M^{\prime}\right)  ^{-1}%
\]
takes $\left(  X_{0},V_{0}\right)  $ to $\left(  X_{0}^{\prime},V_{0}^{\prime
}\right)  $ as long as $\sqrt{V_{0}\bullet_{K}V_{0}}=\sqrt{V_{0}^{\prime
}\bullet_{K}V_{0}^{\prime}}$.
\end{exercise}

By completing this Exercise we have shown that \textbf{HG} looks the
same at each point and in each direction at that point. That is, we
have shown that \textbf{HG} is homogeneous.


\end{document}
