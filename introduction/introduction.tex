\documentclass{ximera}
\usepackage{microtype}
\usepackage{tikz}
\usepackage{tkz-euclide}
\usetkzobj{all}
\tikzstyle geometryDiagrams=[ultra thick,color=blue!50!black]

\renewcommand{\epsilon}{\varepsilon}



\title{Introduction}

\begin{document}
\begin{abstract}
Seeing different geometries as a variation on a theme.
\end{abstract}
\maketitle

As a young mathematician I was introduced to the classic \textit{Le\c{c}ons
sur la G\'{e}om\'{e}trie des Espaces de Riemann}, written by the great French
geometer \'{E}lie Cartan. Early in his treatise on geometries in all
dimensions, the author presents the case of two-dimensional geometries, in
particular, those two-dimensional geometries that look the same at all points
and in all directions. (For example, a cylinder looks the same at each of its
points but not in all directions emanating from any one of its points, whereas
a sphere looks the same at all points and in all directions.) It turns out
that there is one and only one such geometry for each real number $K$, called
the \textit{curvature} of the geometry. The case $K=0$ is the (flat) Euclidean
geometry that you learned in high school.

These twenty-five pages of Cartan's book (Chapter VI, \S i-v) so captivated me
that I have returned to them regularly throughout my career and have adapted
and taught them many times at the advanced undergraduate level. They form the
basis for this little book. To me they tell one of the most beautiful and
satisying stories in all of geometry, one which exemplifies a fundamental
principle of all great mathematics, namely that, using the tools at hand but
in a slightly novel way, the clouds part and one sees that objects and
relationships that seemed so different are in fact parts of a single elegant story!

When $K>0$ it turns out that the $K$-geometry is the geometry of the sphere of
radius $R=1/K^{1/2}$ that we can see as a subset of euclidean three-space
$\mathbb{R}^{3}$. But the geometries with $K<0$ are not so easy to
visualize. They are the so-called `hyperbolic' geometries. In fact it took
mathematicians a couple thousand years to realize that the existed at all! It
turns out that the secret to understanding all the two-dimensional geometries,
including the ones with $K<0$, in a unified way is to simply rescale the third
coordinate in $\mathbb{R}^{3}$ and use these `unusual' coordinates to look at each two-dimensional geometry as the solution set
to the equation%
\[
K\left(  x^{2}+y^{2}\right)  +z^{2}=1.
\]
However the idea of changing coordinates without changing the underlying
geometry described by those coordinates is a challenging one that did not come
into mathematics until a couple of centuries ago. It will require that, before
we get into the beautiful uniform study of all two-dimensional geometries, we
practice the coordinate change we are going to use, namely the rescaling of
the third coordinate in euclidean $3$-space. That practice, together with a
review of some concepts from several variable calculus and linear algebra,
will comprise much of this book.

It has often been said that ``mathematics is not a spectator sport.''
This truism is very much in evidence in the writing of this book. It
is written so as to guide you through the entire story, yet permit
you, when possible, to construct the mathematical story for yourself,
that is, to do some mathematics yourself rather than just observe it
done by others. This `doing mathematics oneself' takes the form of
Exercises with enough help (Hints) provided so that the `doing' is not
so onerous as to get in the way of the story itself.

Strong evidence has been provided by students of mathematics over many
centuries that such guided `doing' is indispensible for understanding and
retention. In fact the very form of this book, as a loose-leaf or electronic
notebook, is intended to encourage you to write out (in correctable form)
solutions to the problems that can be inserted at the appropriate places into
the text.

The second half of this book supposes familiarity with several
variable calculus and the linear algebra of matrices. In particular,
it will often be useful to consider a vector, for example
$V=\left(a,b,c\right)$, as a $1\times3$ matrix
\[
V=
\begin{bmatrix}
a & b & c
\end{bmatrix}
\]
with
\[
V^\transpose  =
\begin{bmatrix}
a\\
b\\
c
\end{bmatrix}.
\]
This will allow us to write the scalar product of two vectors%
\begin{align*}
V\bullet W  &  =\left(  a,b,c\right)  \bullet\left(d,e,f\right) \\
&  =ad+be+cf%
\end{align*}
as a product of matrices%
\[
V\cdot W^\transpose=
\begin{bmatrix}
a & b & c
\end{bmatrix} 
\cdot
\begin{bmatrix}
d\\
e\\
f
\end{bmatrix}.
\]
Furthermore we will use the notations $\det(A)$ and $|A|$
interchangeably for the determinant of a square matrix $A$.

You will also need to remember and apply the Chain Rule for differentiable
functions of several variables, written in matrix notation. Here's the
gradual build-up to this general form of the Chain Rule using matrix
notation and matrix multiplication.

\begin{theorem}[Chain Rule] We will present three different versions:
  \begin{enumerate}
  \item Given differentiable functions $y(x)$ and $z(y)$
    and substituting we have%
    \[
    z\left( y\left( x\right) \right) 
    \]
    making $z$ a function of $x$ and%
    \[
      \frac{dz}{dx}=\frac{dz}{dy}\cdot\frac{dy}{dx}. 
    \]
  \item Given differentiable functions
    \begin{align*}
      & \left( y_{1}\left( x\right) ,\ldots,y_{n}\left( x\right) \right) \\
      & z\left( y_{1},\ldots,y_{n}\right)
    \end{align*}
    then substituting we have%
    \[
    z\left( y_{1}\left( x\right) ,\ldots,y_{n}\left( x\right) \right) 
    \]
    making $z$ a function of $x$ and%
    \[
      \frac{dz}{dx}=\begin{bmatrix}
      \frac{\partial z}{\partial y_{1}} & \ldots & \frac{\partial z}{\partial y_{n}
      }
      \end{bmatrix}
      \cdot\begin{bmatrix}
      \frac{dy_{1}}{dx} \\ 
      \vdots \\ 
      \frac{dy_{n}}{dx}%
      \end{bmatrix}
      =\nabla z\cdot
      \begin{bmatrix}
        \frac{dy_{1}}{dx} \\ 
        \vdots \\ 
        \frac{dy_{n}}{dx}%
      \end{bmatrix}
      \]
\item Given differentiable mappings%
  \begin{align*}
    & \left( y_{1}\left( x_{1},\ldots,x_{m}\right) ,\ldots,y_{n}\left(
    x_{1},\ldots,x_{m}\right) \right) \\
    & \left( z_{1}\left( y_{1},\ldots,y_{n}\right) ,\ldots,z_{p}\left(
    y_{1},\ldots,y_{n}\right)\right)
  \end{align*}
  then substituting 
  \[
  z_{k}\left( y_{1}\left( x_{1},\ldots,x_{m}\right) ,\ldots,y_{n}\left(
  x_{1},\ldots,x_{m}\right) \right) 
  \]
  and fixing $x_{1},\ldots,x_{i-1}$ and $x_{i+1},\ldots,x_{m}$ we make
  $z_{k}$ a function of $x_{i}$ and%
  \[
  \frac{\partial z_{k}}{\partial x_{i}}= 
  \begin{bmatrix}
    \frac{\partial z_{k}}{\partial y_{1}} & \ldots & \frac{\partial z_{k}}{%
    \partial y_{n}}%
  \end{bmatrix}
  \cdot
  \begin{bmatrix}
    \frac{\partial y_{1}}{\partial x_{i}} \\ 
    \vdots \\ 
    \frac{\partial y_{n}}{\partial x_{i}}%
  \end{bmatrix}
  =\nabla z_{k}\cdot
  \begin{bmatrix}
    \frac{\partial y_{1}}{\partial x_{i}} \\ 
    \vdots \\ 
    \frac{\partial y_{n}}{\partial x_{i}}%
  \end{bmatrix}. 
  \]
  
\item Putting the previous work together for all indices $k$ and $i$,
  we have the matrix equation%
  \[
  \left[ \frac{\partial z_{k}}{\partial x_{i}}\right] =\left[ \frac{\partial
      z_{k}}{\partial y_{j}}\right] \cdot\left[ \frac{\partial y_{j}}{\partial
      x_{i}}\right]   \label{ChR}
  \]
  where $\left[ \frac{\partial z_{k}}{\partial x_{i}}\right] $ is the $p\times
  m$ matrix whose $(k,i)$-th entry is $\frac{\partial z_{k}}{%
    \partial x_{i}}$, etc.
  \end{enumerate}
\end{theorem}

One of the Chain Rule's important applications is the Substitution Rule
for integrals of functions of several variables.

\begin{theorem}[Substitution Rule] We will present two different versions:
  \begin{enumerate}
  \item If $f\left( y\right) $ is a function of $y$
    and $y\left( x\right) $ is a function of $x$,%
    \[
    \int \nolimits_{\left[ y\left( a\right) ,y\left( b\right) \right] }f\left(
    y\right) \cdot dy=\int \nolimits_{\left[ a,b\right] }f\left( \left( y\left(
    x\right) \right) \right) \cdot\frac{dy}{dx} \d x. 
    \]
    
  \item If $f(\mathbf{y}) $ is a function of $\mathbf{y} = \left(
    y_{1},\ldots ,y_{m}\right) $ and the function
    \[
    \begin{bmatrix}
      y_{1}(x_{1},\ldots ,x_{m}) \\
      \vdots\\
      y_{m}( x_{1},\ldots,x_{m})
    \end{bmatrix}
    \]
    takes the region $R_{x}$ to the region $R_{y},$ then%
    \[
    \int_{R_{y}}f(\mathbf{y})  \d \mathbf{y}
    =\int_{R_{x}}f(( y_{1}(\mathbf{x}) ,\ldots ,y_{m}(\mathbf{x}) )
    ) \cdot \det\left[\frac{\partial y_{j}}{\partial x_{i}}\right]
    \d \mathbf{x}
      \]
      where $\mathbf{x} = (x_{1},\ldots,x_{m})$.
  \end{enumerate}
\end{theorem}

As a help, at some points in the text and in some of the exercises, a more
complete treatment of a particular topic can be found in one of the following
two texts:

[MJG]: Greenberg, Marvin Jay. \textit{Euclidean and Non-Euclidean Geometry:
Development and History.} W.H. Freeman \& Co. 3rd Ed., 1994.

[DS]: Davis, H. and Snider, A.D. \textit{Introduction to Vector Analysis.} Wm.
C. Brown Publishers, 7th Ed. 1994.

The corresponding topics in these texts are referenced. For example, [MJG,311]
refers to page 311 in the Greenberg book and [DS,59ff] refers to page 59 and
those pages just following page 59 in the Davis-Snider book.

Some final remarks about notation in this book. The letters `\textbf{EG}' will
always mean euclidean (usually plane but occasionally $3$-dimensional)
Geometry, the letters `\textbf{SG}' will always mean Spherical Geometry, and
the letters `\textbf{HG}' will always mean Hyperbolic Geometry. One further
kind of geometry, which we call Neutral Geometry, will be explained in the
book and denoted by `\textbf{NG}.'



It is my hope and intention in writing this little book that you engage with
and enjoy this uniform way of understanding all two-dimensional geometries as
much as I did!

\begin{remark}
Special message to current or future teachers of high school geometry:
Many parts of this book are especially relevant to your teaching of
the subject. Look especially closely at the treatment of congruence
(rigid motion), similarity (dilation), circles, expressing geometric
properties with equations, and geometric measurement and dimension,
and compare them with the high school geometry sections of the Common
Core State Standards in Mathematics. The latter can be found at:
\end{remark}

\begin{center}
http://www.corestandards.org/Math/Content/HSG/introduction.
\end{center}

A useful companion course to one based on this book, one that might be
called \textit{Geometry for Teaching}, would explicitly make the
connections between the material covered as in this book and what you
do (or will do) in your high school geometry classroom. The idea is
\textit{not} that the material we will cover will tell you how to
teach that material but rather that the treatment given here will give
you the depth and breadth of geometric understanding that will allow
you to design what you teach and bring it into your classroom in ways
that those who lack that understanding cannot.

\begin{remark}
This book can also be used as a bridge to a first course in Riemannian
geometry. It treats the case of two-dimensional geometries that are
homogeneous, that is, that look the same at all their points. But to treat
these geometries efficiently, we introduce the notion of changing coordinates
for the geometry without changing the geometry itself. It is that notion that
allowed geometers to treat surfaces and higher-dimensional smooth spaces that
look different at different points, ones that can often not be treated at all
their points using a single set of coordinates.
\end{remark}
\end{document}
