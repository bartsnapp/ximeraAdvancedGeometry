\documentclass{ximera}

\usepackage{microtype}
\usepackage{tikz}
\usepackage{tkz-euclide}
\usetkzobj{all}
\tikzstyle geometryDiagrams=[ultra thick,color=blue!50!black]

\renewcommand{\epsilon}{\varepsilon}



\title{Lines, circles, angles, and areas in stereographic projection}
\begin{document}
\begin{abstract}
Here we look at lines, circles, angles, and areas in central
projection coordinates.
\end{abstract}
\maketitle


\section{Stereographic projection preserves circles}

As we know, we cannot make perfect flat maps of 3D-surfaces. In
stereographic projection, shortest paths are sent to either lines or
circles, and perhaps surprisingly, circles are sent to circles!



\begin{problem}
  Show that ``lines'' in $K$-geometry correspond to either lines or
  circles in $(x_{s},y_{s})$-coordinates under stereographic
  projection.

\begin{hint}
  To intersect two surfaces, say $f(x,y,z)=a$ and $g(x,y,z)=b$,
  simply examine
  \[
  f(x,y,z)-g(x,y,z) = a-b.
  \]
\end{hint}

\begin{hint}
  Explain why intersecting the $K$-surface
  \[
  1 = K\left(x^2+y^2\right) + z^2 
  \]
  with the plane
  \[
  ax+by+cz = 0
  \]
  produces a $K$ geometry line.
\end{hint}

\begin{hint}
  Use the projection formulas
  \begin{align*}
      x &= \frac{4x_s}{K\left(x_s^2 + y_s^2\right) + 4},\\
      y &= \frac{4y_s}{K\left(x_s^2 + y_s^2\right) + 4},\\
      z &= \frac{4-K\left(x_s^2 + y_s^2\right)}{4+K\left(x_s^2 + y_s^2\right)}.\\
  \end{align*}
\end{hint}

\begin{hint}
  If $c=0$, then you will find the line
  \[
   ax_s + by_s = 0.
  \]
\end{hint}

\begin{hint}
  If $c\ne 0$, then you will find the circle
  \[
   \left(x_s - \frac{2a}{cK}\right)^2 + \left(y_s -
   \frac{2b}{cK}\right)^2 = \frac{4a^2 + 4b^2 + 4c^2K}{(cK)^2}.
   \]
   You will need to complete the square to find it in this form.
\end{hint}

\begin{freeResponse}
  Let's start by intersecting the surfaces
  \[
  K\left(x^2+y^2\right)+z^2=1\qquad\text{and}\qquad ax+by+cz=0,
  \]
  where the latter is a plane passing through the origin. Write
  \[
  K\left(x^2+y^2\right) + z^2- ax- by-cz=1.
  \]
  Now use the projection formulas to write
  \begin{align*}
    &K\left(\left(\frac{4x_s}{K\left(x_s^2 + y_s^2\right) + 4}\right)^2
    +\left(\frac{4y_s}{K\left(x_s^2 + y_s^2\right) + 4}\right)^2\right)
    + \left(\frac{4-K\left(x_s^2 + y_s^2\right)}{4+K\left(x_s^2 + y_s^2\right)}\right)^2\\
    &-\frac{4ax_s}{K\left(x_s^2 + y_s^2\right) + 4}- \frac{4by_s}{K\left(x_s^2 + y_s^2\right) + 4}-c\frac{4-K\left(x_s^2 + y_s^2\right)}{4+K\left(x_s^2 + y_s^2\right)}=1.
  \end{align*}
Squaring and adding fractions, we find
\[
\frac{\left(4+K\left(x_s^2 + y_s^2\right)\right)^2}{\left(4+K\left(x_s^2 + y_s^2\right)\right)^2}-\frac{4ax_s}{K\left(x_s^2 + y_s^2\right) + 4}- \frac{4by_s}{K\left(x_s^2 + y_s^2\right) + 4}-c\frac{4-K\left(x_s^2 + y_s^2\right)}{4+K\left(x_s^2 + y_s^2\right)}=1.
\]
Hence
\begin{align*}
1-\frac{4ax_s}{K\left(x_s^2 + y_s^2\right) + 4}- \frac{4by_s}{K\left(x_s^2 + y_s^2\right) + 4}-c\frac{4-K\left(x_s^2 + y_s^2\right)}{4+K\left(x_s^2 + y_s^2\right)}&=1\\
 -\frac{4ax_s}{K\left(x_s^2 + y_s^2\right) + 4}- \frac{4by_s}{K\left(x_s^2 + y_s^2\right) + 4}-c\frac{4-K\left(x_s^2 + y_s^2\right)}{4+K\left(x_s^2 + y_s^2\right)}&=0.
\end{align*}
Since $K\left(x_s^2 + y_s^2\right) + 4 \ne 0$, we may clear
denominators to find 
  \begin{align*}
    -4ax_s-4by_s-4c + cK\left(x_s^2+y_s^2\right) &= 0\\
    cKx_s^2-4ax_s + cKy_s^2-4by_s-4c &= 0.
  \end{align*}
  At this point, note if $c=0$, then our equation becomes
  \begin{align*}
    -4ax_s -4by_s &= 0\\
    ax_s + by_s &= 0,
  \end{align*}
  and this is the equation for a line. Now assume that $c\ne 0$ and
  complete the square(s)
  \begin{align*}
  \left(cKx_s^2 - 4ax_s + \frac{4a^2}{cK}\right) + \left(cKy_s^2 - 4by_s + \frac{4b^2}{cK}\right) &=
  \frac{4a^2}{cK} + \frac{4b^2}{cK} + 4c\\
  \left(x_s\sqrt{cK} - \frac{2a}{\sqrt{cK}}\right)^2 + \left(y_s\sqrt{cK} - \frac{2b}{\sqrt{cK}}\right)^2&=\frac{4a^2}{cK} + \frac{4b^2}{cK} + 4c.
  \end{align*}
  Multiply through by $(cK)^{-1}$ to find
  \begin{align*}
    \left(x_s - \frac{2a}{cK}\right)^2 + \left(y_s - \frac{2b}{cK}\right)^2 &= \frac{4a^2}{(cK)^2} + \frac{4b^2}{(cK)^2} + \frac{4c}{cK}\\
    \left(x_s - \frac{2a}{cK}\right)^2 + \left(y_s - \frac{2b}{cK}\right)^2 &= \frac{4a^2 + 4b^2 + 4c^2K}{(cK)^2}.
  \end{align*}
  This is the equation of a circle with center
  \[
  \left(\frac{2a}{cK}, \frac{2b}{cK}\right)
  \]
  and radius
  \[
  \sqrt{\frac{4a^2 + 4b^2 + 4c^2K}{(cK)^2}}.
  \]
\end{freeResponse}
\end{problem}


ADD PICTURES - SEE PAGE 108 HERB's NOTES



\begin{problem}
  If $K<0$ explain why Euclid's fifth axiom:
  \begin{quote}
    Through a point not on a line there passes a unique parallel line.
  \end{quote}
  fails.
  \begin{freeResponse}
    Given a point not on a line, we see that there are many lines
    parallel to the original line through that point.
  \end{freeResponse}
\end{problem}




CIRCLES IN STEREOGRAPHIC PROJECTION.










\subsection*{Area in stereographic projection coordinates}

Suppose you were given a region $G_{s}$ in the $\left(  x_{s},y_{s}\right)
$-coordinate plane. Also suppose that $K>0$. If you trace back everything we
have done with coordinate changes, you can see how $G_{s}$ gives you a region
$\hat{G}$ on the sphere of radius $R=K^{-1/2}$ in euclidean space via the
formulas%
\begin{align*}
\left(  \hat{x},\hat{y},\hat{z}\right)   &  =\left(  x,y,Rz\right) \\
&  =\rho\cdot\left(  x_{s},y_{s},R\left(  2\rho-1\right)
\right) \\
&  =\left(  \frac{x_{s}}{\frac{K}{4}\left(  x_{s}^{2}+y_{s}^{2}\right)
+1},\frac{y_{s}}{\frac{K}{4}\left(  x_{s}^{2}+y_{s}^{2}\right)  +1}%
,\frac{R\left(  1-\frac{K}{4}\left(  x_{s}^{2}+y_{s}^{2}\right)  \right)
}{1+\frac{K}{4}\left(  x_{s}^{2}+y_{s}^{2}\right)  }\right)  .
\end{align*}
Now there is a formula in several variable calculus for computing the area of
the region $\hat{G}$ on the sphere of radius $R$ in euclidean space in
terms of the parameters $\left(  x_{s},y_{s}\right)  $. [DS,49,231]. It is
\[%
%TCIMACRO{\dint \nolimits_{G_{c}}}%
%BeginExpansion
{\displaystyle\int\nolimits_{G_{c}}}
%EndExpansion
\hat{a}\left(  \frac{d\hat{X}}{dx_{s}},\frac{d\hat{X}}{dy_{s}}\right)
dx_{s}dy_{s}%
\]
where $\hat{a}\left(  \frac{d\hat{X}}{dx_{s}},\frac{d\hat{X}}{dy_{s}}\right)
$ is the (euclidean) area of the parallelogram spanned by the two vectors
$\frac{d\hat{X}}{dx_{s}}$ and $\frac{d\hat{X}}{dy_{s}}$ in euclidean
space. That is%
\[
\hat{a}\left(  \frac{d\hat{X}}{dx_{s}},\frac{d\hat{X}}{dy_{s}}\right)
=\left\vert \frac{d\hat{X}}{dx_{s}}\right\vert \cdot \left\vert \frac{d\hat{X}}{dy_{s}}\right\vert \cdot\sin(\theta)
\]
where $\theta$ is the angle between the two vectors $\frac{d\hat{X}}{dx_{s}}$
and $\frac{d\hat{X}}{dy_{s}}$.

\begin{problem}
As in a previous problem show that%
\begin{align*}
\hat{a}\left(  \frac{d\hat{X}}{dx_{s}},\frac{d\hat{X}}{dy_{s}}\right)  ^{2}
&  =\left\vert
\det
\begin{bmatrix}
\frac{d\hat{X}}{dx_{s}}\bullet\frac{d\hat{X}}{dx_{s}} & \frac{d\hat{X}}%
{dy_{s}}\bullet\frac{d\hat{X}}{dx_{s}}\\
\frac{d\hat{X}}{dx_{s}}\bullet\frac{d\hat{X}}{dy_{s}} & \frac{d\hat{X}}%
{dy_{s}}\bullet\frac{d\hat{X}}{dy_{s}}%
\end{bmatrix}
\right\vert \\
&  =\left\vert
\det
\begin{bmatrix}
\frac{dX}{dx_{s}}\bullet_{K}\frac{dX}{dx_{s}} & \frac{dX}{dy_{s}}\bullet
_{K}\frac{dX}{dx_{s}}\\
\frac{dX}{dx_{s}}\bullet_{K}\frac{dX}{dy_{s}} & \frac{dX}{dy_{s}}\bullet
_{K}\frac{dX}{dy_{s}}%
\end{bmatrix}
\right\vert
\end{align*}

\end{problem}

Now notice the matrix $D_{s}$ is simply the $2\times3$ matrix whose
rows are the vectors $\frac{dX}{dx_{s}}$ and $\frac{dX}{dy_{s}}$.

\begin{problem}
Use a previous problem to show that%
\[
\hat{a}\left(  \frac{d\hat{X}}{dx_{s}},\frac{d\hat{X}}{dy_{s}}\right)
^{2}=\rho^{4}=\frac{1}{\left(  \frac{K}{4}\left(  x_{s}^{2}+y_{s}^{2}\right)
+1\right)  ^{4}}.
\]
\end{problem}



\[
{\renewcommand{\arraystretch}{2.7}
  \begin{tabular}{|c||c|c|c|}\hline
    & \begin{minipage}{2cm}\begin{center}Spherical ($K>0$)\end{center}\end{minipage} & \begin{minipage}{2cm}\begin{center}Euclidean ($K=0$)\end{center}\end{minipage} & \begin{minipage}{2cm}\begin{center}Hyperbolic ($K<0$)\end{center}\end{minipage}\\\hline\hline
    \begin{minipage}{2cm}\begin{center}Surface in \\ euclidean space\end{center}\end{minipage} & $\hat{x}^{2}+\hat{y}^{2}+\hat{z}^{2}=R^{2}$ & DNE  & DNE \\\hline
    \begin{minipage}{2cm}\begin{center}Euclidean dot product\end{center}\end{minipage} & $\hat{V}\cdot \hat{W}^\transpose$ & DNE  & DNE\\\hline
     \begin{minipage}{2cm}\begin{center}Surface in $K$-warped space\end{center}\end{minipage} & \multicolumn{3}{c|}{$1=K\left(  x^{2}+y^{2}\right)  +z^{2}$}\\\hline
     \begin{minipage}{2cm}\begin{center}$K$-dot product\end{center}\end{minipage} & $V\left[\begin{smallmatrix}1 & 0 & 0\\ 0 & 1 & 0\\ 0 & 0 & K^{-1}\end{smallmatrix}\right] W^\transpose$ &  DNE & $V\left[\begin{smallmatrix}1 & 0 & 0\\ 0 & 1 & 0\\ 0 & 0 & K^{-1}\end{smallmatrix}\right]W^\transpose$\\\hline
     \begin{minipage}{2cm}\begin{center}Central dot product\end{center}\end{minipage} & \multicolumn{3}{c|}{$V_c\cdot P_c\cdot W_c^\transpose = V_c\left[\begin{smallmatrix}\left(Ky_c^2+1\right)\lambda^4 & -Kx_{c}y_{c}\lambda^4\\
           -Kx_{c}y_{c}\lambda^4 & \left(Kx_c^2+1\right)\lambda^4\end{smallmatrix}\right] W_c^\transpose$}\\\hline
     \begin{minipage}{2cm}\begin{center}Stereographic dot product\end{center}\end{minipage} & \multicolumn{3}{c|}{$V_s\cdot P_s\cdot W_s^\transpose = V_s\left[\begin{smallmatrix}\rho^2 & 0\\
    0 & \rho^2 \end{smallmatrix}\right] W_s^\transpose$}\\\hline
\end{tabular}}
\]





\begin{problem}
Summarize the results from this section. In particular, indicate which
results follow from the others.
\begin{freeResponse}
\end{freeResponse}
\end{problem}


\end{document}
