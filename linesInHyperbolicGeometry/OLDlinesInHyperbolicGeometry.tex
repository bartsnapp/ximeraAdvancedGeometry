\documentclass{ximera}

\title{Lines in hyperbolic geometry}

\begin{document}
\begin{abstract}
Here we examine ``lines'' in hyperbolic geometry.
\end{abstract}
\maketitle

\subsection*{Hyperbolic coordinates, a shortest path from the North Pole}

We next will figure out what is the shortest path you can take between
two points in \textbf{HG}. Again we will do our calculation using only
$\left( x,y,z\right) $-coordinates (since, as we have seen we don't
have $\left( \hat{x},\hat{y},\hat{z}\right) $-coordinates). The
$\left( x,y,z\right) $-coordinates for \textbf{SG}, namely
\begin{align*}
x\left(  \sigma,\tau\right)   &  =R\cdot \sin %
\sigma\cdot \cos \tau\\
y\left(  \sigma,\tau\right)   &  =R\cdot \sin %
\sigma\cdot \sin \tau\\
z\left(  \sigma,\tau\right)   &  =\cos \sigma
\end{align*}
won't work this time because they involve $R$ which has gone off to infinity.
Fortunately there are hyperbolic coordinates%
\[
\left(  \cosh \sigma=\frac{e^{\sigma}+e^{-\sigma}}{2},\sinh %
\sigma=\frac{e^{\sigma}-e^{-\sigma}}{2}\right)
\]
that parametrize the `unit' hyperbola just like $\left(  \cos %
\sigma,\sin \sigma\right)  $ parametrize the unit circle. So we define%
\begin{align*}
x\left(  \sigma,\tau\right)   &  =\left\vert K\right\vert ^{-1/2}%
\cdot \sinh \sigma\cdot\cos \tau\\
y\left(  \sigma,\tau\right)   &  =\left\vert K\right\vert ^{-1/2}%
\cdot \sinh \sigma\cdot\sin \tau\\
z\left(  \sigma,\tau\right)   &  =\cosh \sigma
\end{align*}


\begin{exercise}
Show that these hyperbolic coordinates do actually parametrize
the $K$-geometry, that is, that%
\[
K\left(  x\left(  \sigma,\tau\right)  ^{2}+y\left(  \sigma,\tau\right)
^{2}\right)  +z\left(  \sigma,\tau\right)  ^{2}\equiv1
\]
for all $\left(  \sigma,\tau\right)  $.
\end{exercise}

Again notice that you can write a path on the $R$-sphere by giving a path
$\left(  \sigma\left(  t\right)  ,\tau\left(  t\right)  \right)  $ in the
$\left(  \sigma,\tau\right)  $-plane. In fact, you can use $\sigma$ as the
parameter $t$ and just write
\[
\left(  \sigma,\tau\left(  \sigma\right)  \right)
\]
where $\tau$ is a function of $\sigma$. To write a path that starts at the
North Pole, just write%
\[
\left(  \sigma,\tau\left(  \sigma\right)  \right)  ,\;0\leq\sigma
\leq\varepsilon
\]
and demand that%
\[
\tau\left(  0\right)  =0.
\]
If you want the path to end on the plane $y=0$, demand additionally that%
\[
\tau\left(  \varepsilon\right)  =0.
\]
But if we are going to describe paths on \textbf{HG} by paths in the $\left(
\sigma,\tau\right)  $-plane we are going to need to figure out the $K$-dot
product in $\left(  \sigma,\tau\right)  $-coordinates so that we can compute
the lengths of paths in these coordinates.

\begin{exercise}\hfil
\begin{enumerate}
\item Compute the $2\times3$ matrix $D_{hyp}$ such that%
\[
\left(  \frac{dx}{dt},\frac{dy}{dt},\frac{dz}{dt}\right)  =\left(
\frac{d\sigma}{dt},\frac{d\tau}{dt}\right)  \cdot D_{hyp}%
\]
when a path in $K$-geometry is given by a path in the $\left(\sigma,\tau\right)  $-plane.

Hint: By the Chain Rule from several variable calculus%
\[
D_{hyp}=\left(
\begin{array}
[c]{ccc}%
\frac{dx}{d\sigma} & \frac{dy}{d\sigma} & \frac{dz}{d\sigma}\\
\frac{dx}{d\tau} & \frac{dy}{d\tau} & \frac{dz}{d\tau}%
\end{array}
\right)  .
\]
\item Use a) to compute the $K$-dot product in $\left(  \sigma,\tau\right)
$-coordinates, namely compute the matrix $P_{hyp}$ in the equation%
\begin{align*}
\left(  \frac{d\sigma_{1}}{dt},\frac{d\tau_{1}}{dt}\right)  \bullet
_{hyp}\left(  \frac{d\sigma_{2}}{dt},\frac{d\tau_{2}}{dt}\right)   &  =\left(
\frac{dx_{1}}{dt},\frac{dy_{1}}{dt},\frac{dz_{1}}{dt}\right)  \bullet
_{K}\left(  \frac{dx_{2}}{dt},\frac{dy_{2}}{dt},\frac{dz_{2}}{dt}\right) \\
&  =\left(  \frac{dx_{1}}{dt},\frac{dy_{1}}{dt},\frac{dz_{1}}{dt}\right)
\cdot\left(
\begin{array}
[c]{ccc}%
1 & 0 & 0\\
0 & 1 & 0\\
0 & 0 & K^{-1}%
\end{array}
\right)  \cdot\left(  \frac{dx_{2}}{dt},\frac{dy_{2}}{dt},\frac{dz_{2}}%
{dt}\right)  ^{t}\\
&  =\left(  \frac{d\sigma_{1}}{dt},\frac{d\tau_{1}}{dt}\right)  \cdot
D_{hyp}\cdot\left(
\begin{array}
[c]{ccc}%
1 & 0 & 0\\
0 & 1 & 0\\
0 & 0 & K^{-1}%
\end{array}
\right)  \cdot D_{hyp}^{t}\cdot\left(  \frac{d\sigma_{1}}{dt},\frac{d\tau_{1}%
}{dt}\right)  ^{t}\\
&  =\left(  \frac{d\sigma_{1}}{dt},\frac{d\tau_{1}}{dt}\right)  \cdot
P_{hyp}\cdot\left(  \frac{d\sigma_{1}}{dt},\frac{d\tau_{1}}{dt}\right)  ^{t}.
\end{align*}
\end{enumerate}
\end{exercise}

\begin{exercise}
Show that the length $L$ of any path in our $K$-geometry is
given by%
\[
\left(  \sigma,\tau\left(  \sigma\right)  \right)  ,\;0\leq\sigma
\leq\varepsilon
\]
with%
\[
\tau\left(  0\right)  =0.
\]
and%
\[
\tau\left(  \varepsilon\right)  =0
\]
is given by the formula%
\[
L=\left\vert K\right\vert ^{-1/2}%
%TCIMACRO{\dint \nolimits_{0}^{\varepsilon}}%
%BeginExpansion
{\displaystyle\int\nolimits_{0}^{\varepsilon}}
%EndExpansion
\sqrt{\left(  1,\frac{d\tau}{d\sigma}\right)  \cdot\left(
\begin{array}
[c]{cc}%
1 & 0\\
0 & \sinh ^{2}\sigma
\end{array}
\right)  \cdot\left(  1,\frac{d\tau}{d\sigma}\right)  ^{t}}d\sigma.
\]

\end{exercise}

This last formula for $L$ lets us figure out the shortest path from 
\[
N=\left(\sinh 0\cdot \cos 0,R\cdot \sinh 0\cdot\sin 0,\cosh 0\right)
\]
to
\[
\left( \left\vert K\right\vert ^{-1/2}\cdot \sinh \varepsilon,0,\cosh
\varepsilon\right) =
\left( \left\vert K\right\vert ^{-1/2} \cdot
\sinh\varepsilon\cdot\cos 0,\left\vert K\right\vert
^{-1/2}\cdot\sinh 0\cdot \sin 0,\cosh  \varepsilon\right).
\]
Since%
\[
L=\left\vert K\right\vert ^{-1/2}\cdot %
%TCIMACRO{\dint \nolimits_{0}^{\varepsilon}}%
%BeginExpansion
{\displaystyle\int\nolimits_{0}^{\varepsilon}}
%EndExpansion
\sqrt{1+\sinh ^{2}\sigma\cdot \left(  \frac{d\tau
}{d\sigma}\right)  ^{2}}d\sigma
\]
and $\sinh ^{2}\sigma$ is is positive for almost all $\sigma\in\left[
0,\varepsilon\right]  $, $L$ is minimal only when $\frac{d\tau}{d\sigma}$ is
identically $0$. But this means that $\tau\left(  \sigma\right)  $ is a
constant function. Since $\tau\left(  0\right)  =0$, this means that
$\tau\left(  \sigma\right)  $ is identically $0$. So we have the shown the
following result.

\begin{theorem}
The shortest path in $K$-geometry from the North Pole to a point
$\left( x,y,z\right) =\left( \left\vert K\right\vert ^{-1/2} \cdot
\sinh \varepsilon,0,\cosh  \varepsilon\right) $ is the path lying in
the plane $y=0$. The $K$-length of that shortest path is%
\[
\left\vert K\right\vert ^{-1/2}\cdot \varepsilon.
\]

\end{theorem}

\subsection*{Shortest path between any two points}

\begin{theorem}
Given any two points $X_{1}=\left( x_{1},y_{1},z_{1}\right) $ and
$X_{2}=\left( x_{2},y_{2},z_{2}\right) $ in $K$-geometry, the shortest
path between the two points is the path cut out by the two equations%
\[
K\left(  x^{2}+y^{2}\right)  +z^{2}=1
\]
and the plane%
\begin{equation}
\left\vert \left(
\begin{array}
[c]{ccc}%
x & y & z\\
x_{1} & y_{1} & z_{1}\\
x_{2} & y_{2} & z_{2}%
\end{array}
\right)  \right\vert =0, \label{90}%
\end{equation}
that is, the plane containing $\left(  0,0,0\right)  $ and $X_{1}$ and $X_{2}$.
\end{theorem}

\begin{proof}
Let $V_{1}=\left(  a_{1},b_{1},c_{1}\right)  $ be the tangent vector at
$X_{1}$ of $K$-length $1$ that is tangent to the path cut out by the plane
given by equation $\left(  \ref{90}\right)  $. Then $\left(  x,y,z\right)
=\left(  a_{1},b_{1},c_{1}\right)  $ also satisfies equation $\left(
\ref{90}\right)  $ and so the equation for that plane can also be written%
\begin{equation}
\left\vert \left(
\begin{array}
[c]{ccc}%
x & y & z\\
x_{1} & y_{1} & z_{1}\\
a_{1} & b_{1} & c_{1}%
\end{array}
\right)  \right\vert =0. \label{91}%
\end{equation}
By a previous exercise, there is a $K$-rigid motion $M$ that takes
$X_{1}$ to the North Pole $N$ and $V_{1}$ to $\left( 1,0,0\right)
$. So $M$ takes the plane $\left( \ref{91}\right) $ to the plane given
by the equation%
\[
\left\vert \left(
\begin{array}
[c]{ccc}%
x & y & z\\
0 & 0 & 1\\
1 & 0 & 0
\end{array}
\right)  \right\vert =0,
\]
namely the plane.%
\[
y=0.
\]
So $X_{2}\cdot M$ must also line in this plane since $X_{2}$ lies in the plane
$\left(  \ref{91}\right)  $. So%
\[
X_{2}\cdot M=\left(  \left\vert K\right\vert ^{-1/2}\cdot\sinh \varepsilon,0,\cosh \varepsilon\right)
\]
for some $\varepsilon$ since all points in $K$-geometry with $y=0$ can
be written as $\left( \left\vert K\right\vert
^{-1/2}\cdot\sinh \varepsilon,0,\cosh \varepsilon\right) $ for some
$\varepsilon$. Since $M$ is a $K$-rigid motion it must take the
shortest path from $X_{1}$ to $X_{2}$ to the shortest path from
$X_{1}\cdot M=N$ to $X_{2}\cdot M=\left( \left\vert K\right\vert
^{-1/2}\cdot \sinh \varepsilon,0,\cosh \varepsilon\right) $. But we
already know that the shortest path from $X_{1}\cdot M$ to $X_{2}\cdot
M$ is the one cut out by the plane $y=0$. But that path comes from the
path cut out by the plane given by equation $\left( \ref{91}\right) $,
or, what is the same thing, the plane given by the equation $\left(
\ref{90}\right) $. This path is called the \textit{great hyperbolic
  arc} between $X_{1}$ and $X_{2}$.
\end{proof}

\begin{definition}
A \textbf{line} in \textbf{HG} will be a curve that extends infinitely
in each direction and has the property that, given any two points
$X_{1}$ and $X_{2}$ on the path, the shortest path between $X_{1}$ and
$X_{2}$ lies along that curve. Lines in \textbf{HG} are the
intersections of the $K$-geometry with planes through $\left(
0,0,0\right) $. The length of the shortest path between two points in
$K$-geometry will be called the $K$-distance.
\end{definition}









\end{document}
