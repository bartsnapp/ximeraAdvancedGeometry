\documentclass{article}%
\usepackage{amsfonts}
\usepackage{amsmath}
\usepackage{amssymb}
\usepackage{graphicx}%
\setcounter{MaxMatrixCols}{30}
%TCIDATA{OutputFilter=latex2.dll}
%TCIDATA{Version=5.50.0.2953}
%TCIDATA{CSTFile=40 LaTeX article.cst}
%TCIDATA{Created=Wednesday, August 25, 2010 15:01:18}
%TCIDATA{LastRevised=Wednesday, December 04, 2013 09:33:33}
%TCIDATA{<META NAME="GraphicsSave" CONTENT="32">}
%TCIDATA{<META NAME="SaveForMode" CONTENT="1">}
%TCIDATA{BibliographyScheme=Manual}
%TCIDATA{<META NAME="DocumentShell" CONTENT="Standard LaTeX\Blank - Standard LaTeX Article">}
%BeginMSIPreambleData
\providecommand{\U}[1]{\protect\rule{.1in}{.1in}}
%EndMSIPreambleData
\newtheorem{theorem}{Theorem}
\newtheorem{acknowledgement}[theorem]{Acknowledgement}
\newtheorem{algorithm}[theorem]{Algorithm}
\newtheorem{axiom}[theorem]{Axiom}
\newtheorem{case}[theorem]{Case}
\newtheorem{claim}[theorem]{Claim}
\newtheorem{conclusion}[theorem]{Conclusion}
\newtheorem{condition}[theorem]{Condition}
\newtheorem{conjecture}[theorem]{Conjecture}
\newtheorem{corollary}[theorem]{Corollary}
\newtheorem{criterion}[theorem]{Criterion}
\newtheorem{definition}[theorem]{Definition}
\newtheorem{example}[theorem]{Example}
\newtheorem{exercise}[theorem]{Exercise}
\newtheorem{lemma}[theorem]{Lemma}
\newtheorem{notation}[theorem]{Notation}
\newtheorem{problem}[theorem]{Problem}
\newtheorem{proposition}[theorem]{Proposition}
\newtheorem{remark}[theorem]{Remark}
\newtheorem{solution}[theorem]{Solution}
\newtheorem{summary}[theorem]{Summary}
\newenvironment{proof}[1][Proof]{\noindent\textbf{#1.} }{\ \rule{0.5em}{0.5em}}
\begin{document}

\title{\textbf{Workbook of two-dimensional geometries:}\\\textit{Exploring the complete, simply connected two-dimensional geometries of
constant curvature}}
\author{C. Herbert Clemens\\Mathematics Department\\Ohio State University\\Columbus, OH 43210 }
\date{December, 2013}
\maketitle
\tableofcontents

\pagebreak

\section{Introduction}

As a young mathematician I was introduced to the classic \textit{Le\c{c}ons
sur la G\'{e}om\'{e}trie des Espaces de Riemann}, written by the great French
geometer \'{E}lie Cartan. Early in his treatise on geometries in all
dimensions, the author presents the case of two-dimensional geometries, in
particular, those two-dimensional geometries that look the same at all points
and in all directions. (For example, a cylinder looks the same at each of its
points but not in all directions emanating from any one of its points, whereas
a sphere looks the same at all points and in all directions.) It turns out
that there is one and only one such geometry for each real number $K$, called
the \textit{curvature} of the geometry. The case $K=0$ is the (flat) Euclidean
geometry that you learned in high school .

These twenty-five pages of Cartan's book (Chapter VI, \S i-v) so captivated me
that I have returned to them regularly throughout my career and have adapted
and taught them many times at the advanced undergraduate level. They form the
basis for this little book. To me they tell one of the most beautiful and
satisying stories in all of geometry, one which exemplifies a fundamental
principle of all great mathematics, namely that, using the tools at hand but
in a slightly novel way, the clouds part and one sees that objects and
relationships that seemed so different are in fact parts of a single elegant story!

When $K>0$ it turns out that the $K$-geometry is the geometry of the sphere of
radius $R=1/K^{1/2}$ that we can see as a subset of Euclidean three-space
$\mathbb{R}^{3}$. But the geometries with $K<$ $0$ are not so easy to
visualize. They are the so-called `hyperbolic' geometries. In fact it took
mathematicians a couple thousand years to realize that the existed at all! It
turns out that the secret to understanding all the two-dimensional geometries,
including the ones with $K<0$, in a unified way is to simply rescale the third
coordinate in $\mathbb{R}^{3}$ and, in these `unusual' coordinates $\left(
x,y,z\right)  $, to look at each two-dimensional geometry as the solution set
to the equation%
\[
K\left(  x^{2}+y^{2}\right)  +z^{2}=1.
\]
However the idea of changing coordinates without changing the underlying
geometry described by those coordinates is a challenging one that did not come
into mathematics until a couple of centuries ago. It will require that, before
we get into the beautiful uniform study of all two-dimensional geometries, we
practice the coordinate change we are going to use, namely the rescaling of
the third coordinate in Euclidean $3$-space. That practice, together with a
review of some concepts from several variable calculus and linear algebra,
will comprise Part \ref{I} of this book.

It has often been said that \textquotedblleft mathematics is not a spectator
sport." This truism is very much in evidence in the writing of this book. It
is written so as to guide you through the entire story, yet permit you, when
possible, to construct the mathematical story for yourself, that is, to do
some mathematics yourself rather than just observe it done by others. This
`doing mathematics oneself' takes the form of Exercises with enough help
(Hints) provided so that the `doing' is not so onerous as to get in the way of
the story itself.

Strong evidence has been provided by students of mathematics over many
centuries that such guided `doing' is indispensible for understanding and
retention. In fact the very form of this book, as a loose-leaf or electronic
notebook, is intended to encourage you to write out (in correctable form)
solutions to the Exercises that can be inserted at the appropriate places into
the text.

This book supposes familiarity with several variable calculus and the linear
algebra of matrices. In particular you will need to remember and apply the
chain rule for differentiable functions of several variables, written in
matrix notation. That is:

\begin{theorem}
\textbf{(Chain Rule)} Given differentiable mappings%
\begin{align*}
&  \left(  y_{1}\left(  x_{1},\ldots,x_{m}\right)  ,\ldots,y_{n}\left(
x_{1},\ldots,x_{m}\right)  \right) \\
&  \left(  z_{1}\left(  y_{1},\ldots,y_{n}\right)  ,\ldots,z_{p}\left(
y_{1},\ldots,y_{n}\right)  \right)
\end{align*}
then%
\[
\left(  \frac{\partial z_{k}}{\partial x_{i}}\right)  =\left(  \frac{\partial
z_{k}}{\partial y_{j}}\right)  \cdot\left(  \frac{\partial y_{j}}{\partial
x_{i}}\right)  .
\]

\end{theorem}

As a help, at some points in the text and in some of the Exercises, a more
complete treatment of a particular topic can be found in one of the following
two texts:

[MJG]: Greenberg, Marvin Jay. \textit{Euclidean and Non-Euclidean Geometry:
Development and History.} W.H. Freeman \& Co. 3rd Ed., 1994.

[DS]: Davis, H. and Snider, A.D. \textit{Introduction to Vector Analysis.} Wm.
C. Brown Publishers, 7th Ed. 1994.

The corresponding topics in these texts are referenced. For example, [MJG,311]
refers to page 311 in the Greenberg book and [DS,59ff] refers to page 59 and
those pages just following page 59 in the Davis-Snider book.

Some final remarks about notation in this book. The letters `\textbf{EG}' will
always mean Euclidean (usually plane but occasionally $3$-dimensional)
Geometry, the letters `\textbf{SG}' will always mean Spherical Geometry, and
the letters `\textbf{HG}' will always mean Hyperbolic Geometry. One further
kind of geometry, which we call Neutral Geometry, will be explained in the
book and denoted by `\textbf{NG}.'

Also, it will often be useful to consider a vector, for example, $V=\left(
a,b,c\right)  $, as a $1\times3$ matrix and we will write%
\[
\left(  V\right)  =\left(
\begin{array}
[c]{ccc}%
a & b & c
\end{array}
\right)
\]
or as a $3\times1$ matrix and we will write%
\[
\left(  V\right)  ^{t}=\left(
\begin{array}
[c]{c}%
a\\
b\\
c
\end{array}
\right)  .
\]
This will allow us, for example, to write the scalar product of two vectors%
\begin{align*}
V\bullet V^{\prime}  &  =\left(  a,b,c\right)  \bullet\left(  a^{\prime
},b^{\prime},c^{\prime}\right) \\
&  =aa^{\prime}+bb^{\prime}+cc^{\prime}%
\end{align*}
as a product of matrices%
\[
V\cdot V^{\prime}=\left(
\begin{array}
[c]{ccc}%
a & b & c
\end{array}
\right)  \cdot\left(
\begin{array}
[c]{c}%
a^{\prime}\\
b^{\prime}\\
c^{\prime}%
\end{array}
\right)  .
\]


It is my hope and intention in writing this little book that you engage with
and enjoy this uniform way of understanding all two-dimensional geometries as
much as I did!

\begin{remark}
Special message to current or future teachers of high school geometry: Parts
II and III of this book are especially relevant to your teaching of the
subject. Look especially closely at the treatment of congruence (rigid
motion), similarity (dilation), circles, expressing geometric properties with
equations, and geometric measurement and dimension, and compare them with the
high school geometry sections of the Common Core State Standards in
Mathematics. The latter can be found at:
\end{remark}

\begin{center}
http://www.corestandards.org/Math/Content/HSG/introduction.
\end{center}

A useful companion course to one based on this book, one that might be called
\textit{Geometry for Teaching}, would explicitly make the connections between
the material covered as in this book and what you do (or will do) in your high
school geometry classroom. The idea is \textit{not} that the material we will
cover in Parts II and III will tell you how to teach that material but rather
that the treatment given here will give you the depth and breadth of geometric
understanding that will allow you to design what you teach and bring it into
your classroom in ways that those who lack that understanding cannot.

\begin{remark}
This book can also be used as a bridge to a first course in Riemannian
geometry. It treats the case of two-dimensional geometries that are
homogeneous, that is, that look the same at all their points. But to treat
these geometries efficiently, we introduce the notion of changing coordinates
for the geometry without changing the geometry itself. It is that notion that
allowed geometers to treat surfaces and higher-dimensional smooth spaces that
look different at different points, ones that can often not be treated at all
their points using a single set of coordinates.
\end{remark}

\pagebreak

\part{Neutral geometry\label{II}}

\section{Euclid's postulates for plane geometry}

\subsection{Neutral geometry}

We first turn our attention to plane (or `flat') two-dimensional geometry.

In Western civilization, the primary source of our understanding of this
geometry comes from Euclid's \textit{Elements}. The treatise is of
transcendant importance well beyond geometry itself, because it is among the
first, and perhaps the most influential single example of organized, formal
logical deductive reasoning. Certain fundamentals, that are called
\textit{axioms}, are postulated or `given,' providing the platform on which a
`geometry' is built, that is, a mathematical entity modeling a physical
`reality'. Its properties are arrived at by applying the laws of logic to the
given fundamentals. Euclid gives five axioms for plane geometry, the first
four of which seem to be `obvious' reflections of physical reality. In
paraphrased form, they are:

\begin{axiom}
(E1) Through any point $P$ and any other point $Q$, there lies a unique line.
\end{axiom}

\begin{axiom}
(E2) Given any two segments $\overline{AB}$ and $\overline{CD}$, there is a
segment $\overline{AE}$ such that $B$ lies on $\overline{AE}$ and $\left\vert
CD\right\vert =\left\vert BE\right\vert $

(NB: In plane geometry we often use the notation $\left\vert CD\right\vert $
to denote the distance between two points $A$ and $B$ rather than the notation
$d\left(  A,B\right)  $ used previously.
\end{axiom}

\begin{axiom}
(E3) Given and point $P$ and any positive real number $r$, there exists a
(unique) circle of radius $r$ and center $P$. (Said another way, if you move
away from $P$ along a line in any direction, you will encounter a unique point
at distance $r$ from $P$.)
\end{axiom}

\begin{axiom}
(E4) All right angles are congruent. (A right angle is defined as follows. Let
$C$ be the midpoint on the segment $\overline{AB}$. Let $E$ be any point not
equal to $C$. The angle $\angle ACE$ is called a right angle if $\angle ACE$
is congruent to $\angle BCE$.) [MJG,17-18]
\end{axiom}

\begin{definition}
If we are only given E1-E4, we will call our geometry \textbf{Neutral
Geometry} (\textbf{NG}).
\end{definition}

\begin{definition}
In \textbf{NG}, two distinct lines are called parallel if and only if they
don't intersect.
\end{definition}

One implicit assumption of two-dimensional Neutral (and Euclidean) geometry is
the existence of (a group of) rigid motions or congruences. That is, it is
assumed that given any point $\hat{A}$ and any tangent vector $\hat{V}$
emanating from $\hat{A}$ and given any second point $\hat{B}$ in the geometry
and any tangent vector $\hat{W}$ emanating from $\hat{B}$, then there is a
transformation $\hat{M}$ of the geometry such that

1) $\hat{M}$ takes $\hat{A}$ to $\hat{B}$,

2) $\hat{M}$ takes $\hat{V}$ and to a positive scalar multiple times $\hat{W}$
to $\hat{M}\left(  \hat{V}\right)  $,

3) for all points $\hat{A}^{\prime},\hat{A}^{\prime\prime}$ in the geometry,
$\hat{M}$ leaves the distance between them unchanged, that is,%
\[
\left\vert \hat{M}\left(  \hat{A}^{\prime}\right)  \hat{M}\left(  \hat
{A}^{\prime\prime}\right)  \right\vert =\left\vert \hat{A}^{\prime}\hat
{A}^{\prime\prime}\right\vert ,
\]


4) for any two tangent vectors $\hat{V}^{\prime}$ and $\hat{V}^{\prime\prime}$
emanating from $\hat{A}$, the angle between $\hat{M}\left(  \hat{V}^{\prime
}\right)  $ and $\hat{M}\left(  \hat{V}^{\prime\prime}\right)  $ is the same
as the angle between $\hat{V}^{\prime}$ and $\hat{V}^{\prime\prime}$.

\begin{exercise}
Using a sketch on grid paper or an algebraic formultation in the Euclidean
plane, give a concrete example of a rigid motion that takes $\left(
1,2\right)  $ to $\left(  3,5\right)  $ and the tangent vector $\left(
1,0\right)  $ emanating from $\left(  1,2\right)  $ to a positive multiple of
the tangent vector $\left(  0,2\right)  $ emanating from $\left(  3,5\right)
$.
\end{exercise}

\begin{exercise}
(\textbf{NG}) Think back to high school days and write the congruence rules
SSS, SAS, and ASA. Be very careful with your wording--it had better be that
triangles can be moved onto each other by a rigid motion if and only if they
satisfy any one (and hence all) of the three properties(SSS, SAS, ASA).
\end{exercise}

\begin{exercise}
Give a counterexample to show that there is no universal SSA law.
\end{exercise}

Although it is a bit tedious to show (and we will not ask you to do it here),
using only E1-E4 you can derive the usual rules for congruent triangles (SSS,
SAS, ASA). Thus these laws hold in any neutral geometry, that is, in any
geometry satisfying E1-E4.

\begin{exercise}
\label{18} (\textbf{NG}) Suppose, in the diagram below that $\left\vert
BD\right\vert =\left\vert CD\right\vert $ and $\left\vert AD\right\vert
=\left\vert ED\right\vert $.%
\[%
%TCIMACRO{\FRAME{itbpF}{4.0828in}{2.0937in}{0in}{}{}{Figure}%
%{\special{ language "Scientific Word";  type "GRAPHIC";
%maintain-aspect-ratio TRUE;  display "USEDEF";  valid_file "T";
%width 4.0828in;  height 2.0937in;  depth 0in;  original-width 4.0326in;
%original-height 2.0548in;  cropleft "0";  croptop "1";  cropright "1";
%cropbottom "0";  tempfilename 'MXAJBX00.png';tempfile-properties "XPR";}}}%
%BeginExpansion
{\includegraphics[
natheight=2.054800in,
natwidth=4.032600in,
height=2.0937in,
width=4.0828in
]%
{MXAJBX00.png}%
}%
%EndExpansion
\]


Show that triangle $\triangle BDA$ and triangle $\triangle CDE$ are congruent. [MJG,138]
\end{exercise}

\begin{exercise}
\label{118} a) Show in Neutral Geometry that, for $\triangle ABC$ as in
Exercise \ref{18}, the exterior angle of the triangle at $C$ is greater than
either remote interior angle. [MJG,119]

b) Use a) to show that the sum of any two angles of a triangle is less than
$180^{\circ}$.
\end{exercise}

\begin{exercise}
(\textbf{NG}) Show in \textbf{NG} that, if two lines cut by a transversal line
have a pair of congruent alternate interior angles, then they are parallel. [MJG,117]

Hint: Suppose the assertion is false for some pair of lines. Find a triangle
that violates the conclusion of Exercise \ref{118}a).
\end{exercise}

\pagebreak

\subsection{Sum of angles in a triangle in \textbf{NG}}

We will now reason to one of the fundamental results about Neutral Geometry,
one that puzzled mathematicians for many centuries, in fact, until the
discovery of hyperbolic geometry about two centuries ago. That discovery
showed that there was more than one geometry that satisfied the axioms of
Neutral Geometry, and that attempts to show that Euclid's fifth axiom (below)
was a consequence of the four axioms of Neutral Geometry were futile. There
was another geometry, namely Hyperbolic Geometry, that satisfied E1-E4. The
thing that separates Hyperbolic Geometry from Euclidean (plane) Geometry is
the sum of the angles in a triangle. (If you had a good geometry course in
high school, you may remember that you had to use Euclid's fifth axiom in
order to show that the sum of the angles in a triagle was $180^{\circ}.$ But
more on that later.)

There is one important fact about the sum of the angles in a triangle that you
\textit{can} prove in \textbf{NG}, that is, without invoking Euclid's fifth
Axiom. We will in fact accomplish that in this section.

\begin{exercise}
\label{20}(\textbf{NG}) For the diagram in Exercise \ref{18}, show that the
sum of the angles in $\triangle ACE$ is the same as the sum of the angles in
$\triangle ACB$
\end{exercise}

\begin{exercise}
\label{22}(\textbf{NG}) Suppose that there is a triangle $\triangle ABC$ in
\textbf{NG} for which the sum of the angles in a triangle $\triangle ABC$ is
$\left(  180+x\right)
%TCIMACRO{\U{ba}}%
%BeginExpansion
{{}^o}%
%EndExpansion
$ with $x>0$. For the $\triangle ABC$ in Exercise \ref{18}, show that one of
the angles of $\triangle ACE$ is no more than half the size of $\angle BAC$.
Yet by Exercise \ref{20} the sum of the angles in a triangle $\triangle ABC$
is still $\left(  180+x\right)
%TCIMACRO{\U{ba}}%
%BeginExpansion
{{}^o}%
%EndExpansion
$.

Hint: In the diagram in Exercise \ref{18}, this new `smaller' angle may or may
not have vertex $A$. [MJG,125-127]
\end{exercise}

\begin{exercise}
\label{21}(\textbf{NG}) Suppose that there is a triangle $\triangle ABC$ in
\textbf{NG} for which the sum of the angles in a triangle $\triangle ABC$ is
$\left(  180+x\right)
%TCIMACRO{\U{ba}}%
%BeginExpansion
{{}^o}%
%EndExpansion
$ with $x>0$. Let $\alpha$ denote the measure of $\angle BAC$. Repeat the
construction in Exercise \ref{22} over and over again $n$-times to construct a
triangle with the sum of its angles still equal to $\left(  180+x\right)
%TCIMACRO{\U{ba}}%
%BeginExpansion
{{}^o}%
%EndExpansion
$ but such that one of its angles has size less than%
\[
\frac{1}{2^{n}}\alpha.
\]

\end{exercise}

\begin{exercise}
(\textbf{NG})\label{121} Suppose that there is a triangle $\triangle ABC$ in
\textbf{NG} for which the sum of the angles in a triangle $\triangle ABC$ is
$\left(  180+x\right)
%TCIMACRO{\U{ba}}%
%BeginExpansion
{{}^o}%
%EndExpansion
$ with $x>0$. Show that there is a positive integer $n$ so that, if you repeat
the construction in Exercise \ref{22} over and over again $n$-times, the
result will be a triangle with the sum of its angles still equal to $\left(
180+x\right)
%TCIMACRO{\U{ba}}%
%BeginExpansion
{{}^o}%
%EndExpansion
$ but with one of its angles having measure less than $x$. [MJG,125-127]
\end{exercise}

\begin{exercise}
(\textbf{NG})\label{122} On the other hand, by Exercise \ref{118}b), you
cannot have a triangle with two angles summing to more than $180%
%TCIMACRO{\U{ba}}%
%BeginExpansion
{{}^o}%
%EndExpansion
$. [MJG,124] Use this fact to conclude the following theorem:
\end{exercise}

\begin{theorem}
In \textbf{NG}, the sum of the interior angles in any triangle is no greater
than $180%
%TCIMACRO{\U{ba}}%
%BeginExpansion
{{}^o}%
%EndExpansion
$.
\end{theorem}

\begin{proof}
We argue by contradiction. Start with $\triangle ABC$ as in Exercise \ref{22}
for which the sum of the angles in a triangle $\triangle ABC$ is $\left(
180+x\right)
%TCIMACRO{\U{ba}}%
%BeginExpansion
{{}^o}%
%EndExpansion
$ with $x>0$. Suppose the measure of the angle at $A$ is denoted by $\alpha$.
By Exercise \ref{22} there exists a triangle $\triangle A^{\left(  1\right)
}B^{\left(  1\right)  }C^{\left(  1\right)  }$ such that the sum of the angles
in a triangle $\triangle A^{\left(  1\right)  }B^{\left(  1\right)
}C^{\left(  1\right)  }$ is $\left(  180+x\right)
%TCIMACRO{\U{ba}}%
%BeginExpansion
{{}^o}%
%EndExpansion
$ and the measure of the angle at the vertex $A^{\left(  1\right)  }$ is less
than or equal to $\frac{\alpha}{2}$. By Exercise \ref{21} there is a triangle
$\triangle A^{\left(  n\right)  }B^{\left(  n\right)  }C^{\left(  n\right)  }$
such that the sum of the angles in a triangle $\triangle A^{\left(  n\right)
}B^{\left(  n\right)  }C^{\left(  n\right)  }$ is $\left(  180+x\right)
%TCIMACRO{\U{ba}}%
%BeginExpansion
{{}^o}%
%EndExpansion
$ and the measure $\alpha_{n}$ of the angle at the vertex $A^{\left(
n\right)  }$ is less than or equal to $\frac{\alpha}{2^{n}}$. If $n$ is
sufficiently big,%
\[
\frac{\alpha}{2^{n}}<x.
\]
So, for that value of $n$, if $\beta_{n}$ is the measure of the angle of
$\triangle A^{\left(  n\right)  }B^{\left(  n\right)  }C^{\left(  n\right)  }$
at the vertex $B^{\left(  n\right)  }$ and $\gamma_{n}$ is the measure of the
angle of $\triangle A^{\left(  n\right)  }B^{\left(  n\right)  }C^{\left(
n\right)  }$ at the vertex $C^{\left(  n\right)  }$, then we have the two
relations%
\begin{gather*}
\alpha_{n}<\frac{\alpha}{2^{n}}<x\\
\alpha_{n}+\beta_{n}+\gamma_{n}=180+x.
\end{gather*}
So%
\[
\beta_{n}+\gamma_{n}=180+\left(  x-\alpha_{n}\right)  >180.
\]
But this is impossible by Exercise \ref{118}b).
\end{proof}

\begin{exercise}
(\textbf{NG}) Show the following:

a) The sum of the interior angles in any quadrilateral is no greater than $360%
%TCIMACRO{\U{b0}}%
%BeginExpansion
{{}^\circ}%
%EndExpansion
$.

b) The sum of the interior angles of an $n$-gon is no greater than $\left(
n-2\right)  $\textperiodcentered$180%
%TCIMACRO{\U{ba}}%
%BeginExpansion
{{}^o}%
%EndExpansion
$. \pagebreak
\end{exercise}

\part{Euclidean (plane) geometry\label{IV}}

\section{Rectangles and cartesian coordinates}

\subsection{Euclid's Axiom 5, the Parallel Postulate}

We are finally ready to introduce Euclid's fifth and final axiom, the
so-called \textit{Parallel Postulate}.

\begin{axiom}
(E5) Through a point not on a line there passes a unique parallel line.
\end{axiom}

\textbf{NG} together with E5 is called Euclidean geometry (\textbf{EG}). As
mentioned above, we will see later that there is another geometry
(\textbf{HG}) that satisfies all the postulates of \textbf{NG} but not E5. In
it, the sum of the interior angles of a triangle will \textit{always} be less
than $180%
%TCIMACRO{\U{b0}}%
%BeginExpansion
{{}^\circ}%
%EndExpansion
$! [MJG,134]

\begin{exercise}
\label{24}(\textbf{EG}) a) Show that, if two parallel lines are cut by a
transversal line, opposite interior angles are equal.
\end{exercise}

We will call the set of points on a line which lie on one side of a given
point a \textit{ray}. We call the given point the \textit{origin} of the ray.

We call two rays in the plane parallel if they lie on parallel lines and they
both lie on the same side of the transversal line passing through their origins.

Strictly speaking, an angle in the plane is the union of two ordered rays with
common origin and choice of one of the two connected regions into which the
union of the rays divides the plane. We often denote angles by $\angle BAC$
where $A$ is the common origin and $B$ a point along one of the rays, called
the initial ray, and $C$ is a point along the other ray, called the final ray.
The choice of the region is either clear from the context or explicitly given.

\begin{exercise}
a) Given an angle $\angle BAC$ show by drawings the two regions into which it
divides the plane. Show how the (signed) measure of the angle depends on which
region you pick and on which is the initial ray and which is the final ray of
the angle.

b) Show that angles $\angle BAC$ and $\angle B^{\prime}A^{\prime}C^{\prime}$
in the Euclidean plane are either equal (i.e. have the same measure) if
corresponding rays are parallel.

c) Show that $\angle BAC$ and $\angle B^{\prime}A^{\prime}C^{\prime}$ are
equal (i.e. have the same measure) if $\angle B^{\prime}A^{\prime}C^{\prime}$
can be rotated around $A^{\prime}$ to obtain an angle $\angle B^{\prime\prime
}A^{\prime}C^{\prime\prime}$ with corresponding rays parallel to those of
$\angle BAC$.
\end{exercise}

\begin{exercise}
(\textbf{EG}) Use the `uniqueness' assertion in E5 together with what we have
established about Neutral Geometry to show that in \textbf{EG} the sum of the
interior angles of any triangle is $180%
%TCIMACRO{\U{ba}}%
%BeginExpansion
{{}^o}%
%EndExpansion
$.
\end{exercise}

\begin{exercise}
(\textbf{EG}) Show that in \textbf{EG} the sum of the interior angles of a
quadrilateral is $360%
%TCIMACRO{\U{ba}}%
%BeginExpansion
{{}^o}%
%EndExpansion
$.
\end{exercise}

\begin{exercise}
\label{112}(\textbf{EG}) Show in \textbf{EG} that, given any positive real
numbers $a$ and $b$, there exist rectangles with adjacent side of lengths $a$
and $b$.

Hint: To show that opposite sides are of equal length, suppose not. For
example, suppose the top of the rectangle has length $a^{\prime}$ and the
bottom has length $a$ and, for example, $a^{\prime}>a$. Mark the point at
length $a$ along the top, starting at the left-hand vertex. Connect tht point
to the right-hand bottom vertex. Find a triangle that contradicts Exercise
\ref{118}a).
\end{exercise}

\begin{exercise}
(\textbf{EG}) Show that there is a cartesian coordinate system on \textbf{EG},
that is, the set of points of \textbf{EG} are in $1-1$ correspondence with the
set of pairs of real numbers.
\end{exercise}

\pagebreak

\subsection{The distance formula in \textbf{EG}}

It is the existence of a cartesian coordinate system in \textbf{EG} that
allows us to define distance between points%
\begin{equation}
d\left(  \left(  a_{1},b_{1}\right)  ,\left(  a_{2},b_{2}\right)  \right)
=\sqrt{\left(  a_{2}-a_{1}\right)  ^{2}+\left(  b_{2}-b_{1}\right)  ^{2}}
\label{113}%
\end{equation}
and so gives rigorous mathematical meaning to a concept that the ancient
Greeks were never able to describe precisely, namely the similarity of figures
in \textbf{EG}. For that we will require the notion of a dilation or
magnification in \textbf{EG}. And we need a cartesian coordinate system to
describe dilation precisely, a reality backed up by the fact that similarities
do not exist in \textbf{HG} or \textbf{SG}. (Try drawing two triangles that
are similar but not congruent on a perfectly spherical balloon!)

\begin{exercise}
(\textbf{EG}) State the Pythagorean theorem in \textbf{EG} and use Exercise
\ref{112} to prove it.

Hint: In the cartesian plane, construct a square with vertices $\left(
0,0\right)  ,\left(  a+b,0\right)  ,\left(  0,a+b\right)  $ and $\left(
a+b,a+b\right)  $. Inside that square, construct the square with vertices
$\left(  a,0\right)  ,\left(  a+b,a\right)  ,\left(  b,a+b\right)  $ and
$\left(  0,b\right)  $. Show that the area of the big square is the area of
the little square plus the area of $4$ right triangles, each of area
$\frac{ab}{2}$.
\end{exercise}

\begin{exercise}
Use the Pythagorean theorem to justify the definition $\left(  \ref{113}%
\right)  $.
\end{exercise}

\pagebreak

\subsection{Dilations in \textbf{EG}}

\begin{definition}
(\textbf{EG}) A \textbf{dilation} is a one-to-one onto transformation of the
cartesian plane to itself that

a) fixes one point called the center of the dilation,

b) takes each line through the fixpoint to itself,

c) multiples all distances by a fixed positive real number called the
magnification factor of the dilation.
\end{definition}

\begin{definition}
Given a point $\left(  x_{0},y_{0}\right)  $ in the (cartesian) plane and a
positive real number $r$, we define a mapping $D$ with center $\left(
x_{0},y_{0}\right)  $ and magnification factor $r$ by the formula%
\begin{equation}
D\left(  x,y\right)  =\left(  x_{0},y_{0}\right)  +r\left(  x-x_{0}%
,y-y_{0}\right)  . \label{23}%
\end{equation}
We will also denote the output $D\left(  x,y\right)  $ of the dilation as
$\left(  \underline{x},\underline{y}\right)  $.
\end{definition}

\begin{exercise}
(\textbf{EG}) Using cartesian coordinates for the plane, show that the mapping
$D$ defined in $\left(  \ref{23}\right)  $ is a dilation with magnification
factor $r$ and center $\left(  x_{0},y_{0}\right)  $.
\end{exercise}

\begin{exercise}
(\textbf{EG}) Show (using several-variable calculus if you wish) that a
dilation with magnification factor $r$ multiplies all areas by a factor of
$r^{2}$.
\end{exercise}

\begin{exercise}
(\textbf{EG}) a) Show that the inverse mapping of a dilation is again a
dilation with the same center but with magnification factor $r^{-1}$.

b) Show that a dilation takes lines to lines.

Hint: For a)\ solve for $\left(  x,y\right)  $ in terms of $\left(
\underline{x},\underline{y}\right)  $. For b) write the equation%
\[
ax+by=c
\]
for the given line. Then substitute for $\left(  x,y\right)  $ its expression
in terms of $\left(  \underline{x},\underline{y}\right)  $.
\end{exercise}

\begin{exercise}
(\textbf{EG}) Show that a dilation takes any line to a line parallel (or
equal) to itself.

Hint: Compute slopes.
\end{exercise}

\begin{exercise}
(\textbf{EG}) Show that a dilation by a factor of $r$ takes any vector to $r$
times itself.

Hint: Realize the vector as the difference of two points.
\end{exercise}

\begin{exercise}
(\textbf{EG}) Show that a dilation of the plane preserves angles.

Hint: Use the dot product of vectors emanating from the same point to measure
angles \pagebreak
\end{exercise}

\subsection{Similarity in \textbf{EG}}

\begin{definition}
(\textbf{EG}) Two triangles are \textbf{similar} if there is a dilation of the
plane that takes one to a triangle which is congruent to the other. We write%
\[
\triangle ABC\sim\triangle A^{\prime}B^{\prime}C^{\prime}%
\]
to denote that these two triangles are similar (where the order of the
vertices tells us which vertices correspond).
\end{definition}

\begin{exercise}
(\textbf{EG}) a) Show that, if two triangles are similar, then corresponding
sides are proportional.

Hint: You have to start from the supposition that the two triangles satisfy
the definition of similar triangles.

b) Show that, if corresponding sides of two triangles are proportional, then
the two triangles are similar.

Hint: You have to start from the supposition that corresponding sides of the
two triangles are proportional and use SSS to show that there is a dilation of
$\triangle ABC$ is congruent to $\triangle A^{\prime}B^{\prime}C^{\prime}$.
\end{exercise}

\begin{exercise}
(\textbf{EG}) a) Show that, if two triangles are similar, then corresponding
angles are equal.

Hint: You have to start from the supposition that the two triangles satisfy
the definition of similar triangles.

b) Show that, if corresponding angles of two triangles are equal, then the two
triangles are similar.

Hint: You have to start from the supposition that corresponding angles of the
two triangles are equal, then use a dilation with $r=|A^{\prime}B^{\prime
}|/|AB|$ and ASA to show that the dilation of one triangle that is congruent
to the other.
\end{exercise}

\begin{exercise}
\label{39}(\textbf{EG}) Show that two triangles are similar if corresponding
sides are parallel.

Hint: Use Exercise \ref{24}b).
\end{exercise}

\begin{exercise}
Show that two triangles are similar if corresponding sides are perpendicular.

Hint: Extend one of the rays of the first angle until it crosses the
corresponding ray of the second angle.
\end{exercise}

\pagebreak

\subsection{Concurrence theorems in \textbf{EG}, Ceva's theorem}

Before leaving (plane) Euclidean Geometry, we will visit two more of its many
sets of memorable properties, one's that you may or may not have seen in high
school. The first of these comes under the name of concurrence theorems--these
theorems relate the measures of the three sides (or angles) of a triangle to
the measure of quantities constructed from those sides by some uniform rule.

\begin{exercise}
\label{25}(\textbf{EG}) Denote the measure or area of a triangle $\triangle
ABC$ as $\left\vert \triangle AFC\right\vert $. Show that, in the diagram
below,%
\[
\frac{\left\vert AF\right\vert }{\left\vert FB\right\vert }=\frac{\left\vert
\triangle AFC\right\vert }{\left\vert \triangle CFB\right\vert }%
=\frac{\left\vert \triangle AFX\right\vert }{\left\vert \triangle
XFB\right\vert }.
\]%
\[%
%TCIMACRO{\FRAME{itbpF}{2.9438in}{1.4122in}{0in}{}{}{Figure}%
%{\special{ language "Scientific Word";  type "GRAPHIC";
%maintain-aspect-ratio TRUE;  display "USEDEF";  valid_file "T";
%width 2.9438in;  height 1.4122in;  depth 0in;  original-width 10.8569in;
%original-height 5.1863in;  cropleft "0";  croptop "1";  cropright "1";
%cropbottom "0";  tempfilename 'MXAJBX01.png';tempfile-properties "XPR";}}}%
%BeginExpansion
{\includegraphics[
natheight=5.186300in,
natwidth=10.856900in,
height=1.4122in,
width=2.9438in
]%
{MXAJBX01.png}%
}%
%EndExpansion
\]

\end{exercise}

\begin{exercise}
\label{26}(\textbf{EG}) Use Exercise \ref{25} to show by pure algebra that%
\begin{equation}
\frac{\left\vert AF\right\vert }{\left\vert FB\right\vert }=\frac{\left\vert
\triangle AXC\right\vert }{\left\vert \triangle CXB\right\vert }. \label{27}%
\end{equation}

\end{exercise}

\begin{exercise}
\label{28}(\textbf{EG}) For three concurrent segments $\overline{AD}$,
$\overline{BE}$ and $\overline{CF}$ as given in Exercise \ref{25}, use
Exercise \ref{26} to show that%
\[
\frac{\left\vert AF\right\vert }{\left\vert FB\right\vert }%
\text{\textperiodcentered}\frac{\left\vert BD\right\vert }{\left\vert
DC\right\vert }\text{\textperiodcentered}\frac{\left\vert CE\right\vert
}{\left\vert EA\right\vert }=1.
\]


Hint: Use $\left(  \ref{27}\right)  $, $\left(  \ref{27}\right)  $ with side
$\overline{BC}$ replacing $\overline{AB}$ $,$and $\left(  \ref{27}\right)  $
with side $\overline{CA}$ replacing $\overline{AB}$.
\end{exercise}

This last result with its converse, which you will show in the next Exercise,
is called \textit{Ceva's Theorem}. [MJG,287-288]

\begin{exercise}
(\textbf{EG}) Show the converse of the result in Exercise \ref{28}, namely
that, if%
\[
\frac{\left\vert AF\right\vert }{\left\vert FB\right\vert }%
\text{\textperiodcentered}\frac{\left\vert BD\right\vert }{\left\vert
DC\right\vert }\text{\textperiodcentered}\frac{\left\vert CE\right\vert
}{\left\vert EA\right\vert }=1
\]
then the lines $AD$, $BE$, and $CF$ pass through a common point.

Hint: Suppose that they do not pass through a common point.
\[%
%TCIMACRO{\FRAME{itbpF}{3.2301in}{1.5567in}{0in}{}{}{Figure}%
%{\special{ language "Scientific Word";  type "GRAPHIC";
%maintain-aspect-ratio TRUE;  display "USEDEF";  valid_file "T";
%width 3.2301in;  height 1.5567in;  depth 0in;  original-width 11.0774in;
%original-height 5.3186in;  cropleft "0";  croptop "1";  cropright "1";
%cropbottom "0";  tempfilename 'MXAJBX02.png';tempfile-properties "XPR";}}}%
%BeginExpansion
{\includegraphics[
natheight=5.318600in,
natwidth=11.077400in,
height=1.5567in,
width=3.2301in
]%
{MXAJBX02.png}%
}%
%EndExpansion
\]
Notice that if, for example, $F$ moves along the segment $\overline{AB}$, then
$\frac{\left\vert AF\right\vert }{\left\vert FB\right\vert }$ is a strictly
increasing function of $\left\vert AF\right\vert $. Now use Exercise \ref{28}
to determine a position $F^{\prime}$ for $F$ along the segment $\overline{AB}$
at which
\[
\frac{\left\vert AF^{\prime}\right\vert }{\left\vert F^{\prime}B\right\vert
}\text{\textperiodcentered}\frac{\left\vert BD\right\vert }{\left\vert
DC\right\vert }\text{\textperiodcentered}\frac{\left\vert CE\right\vert
}{\left\vert EA\right\vert }=1.
\]

\end{exercise}

\begin{exercise}
(\textbf{EG}) A \textbf{median} of a triangle is a line segment from a vertex
to the midpoint of the opposite side. Show that the medians of any triangle
meet in a common point.

Hint: Use Ceva's Theorem.
\end{exercise}

\begin{exercise}
(\textbf{EG}) Use Ceva's theorem to show that the three altitudes of a
triangle are concurrent.%
\[%
%TCIMACRO{\FRAME{itbpF}{3.3261in}{1.6086in}{0in}{}{}{Figure}%
%{\special{ language "Scientific Word";  type "GRAPHIC";
%maintain-aspect-ratio TRUE;  display "USEDEF";  valid_file "T";
%width 3.3261in;  height 1.6086in;  depth 0in;  original-width 10.8569in;
%original-height 5.2312in;  cropleft "0";  croptop "1";  cropright "1";
%cropbottom "0";  tempfilename 'MXAJBX03.png';tempfile-properties "XPR";}}}%
%BeginExpansion
{\includegraphics[
natheight=5.231200in,
natwidth=10.856900in,
height=1.6086in,
width=3.3261in
]%
{MXAJBX03.png}%
}%
%EndExpansion
\]


Hint: Use all three similarities of the form $\triangle CEB\sim\triangle CDA$
and then apply Ceva's theorem.\pagebreak
\end{exercise}

\section{Properties of circles in \textbf{EG}}

\subsection{Basics}

Our final topic before leaving Euclidean Geometry is circles. We include this
partly for its own interest, and partly because the properites we visit here
will be useful later on. Again we explore the topic through a sequence of
Exercises (with Hints to their solutions to ease the way). We begin with
perhaps the most basic fact of all about circles in \textbf{EG}.

\begin{exercise}
(\textbf{EG}) The circle of radius $1$ has (interior) area $\pi$. Use this to
reason to the fact that the circle of radius $1$ has circumference $2\pi$.%
\[%
%TCIMACRO{\FRAME{itbpF}{1.3413in}{1.3206in}{0in}{}{}{Figure}%
%{\special{ language "Scientific Word";  type "GRAPHIC";
%maintain-aspect-ratio TRUE;  display "USEDEF";  valid_file "T";
%width 1.3413in;  height 1.3206in;  depth 0in;  original-width 5.5383in;
%original-height 5.4509in;  cropleft "0";  croptop "1";  cropright "1";
%cropbottom "0";  tempfilename 'MXAJBX04.png';tempfile-properties "XP";}}}%
%BeginExpansion
{\includegraphics[
natheight=5.450900in,
natwidth=5.538300in,
height=1.3206in,
width=1.3413in
]%
{MXAJBX04.png}%
}%
%EndExpansion
\]


Hint: Approximate a rectangle by rearranging the slices in the picture.
Compute the area of the "rectangle."
\end{exercise}

Next we turn to some facts about chords in circles and angels inscribed in circles.

\begin{exercise}
(\textbf{EG}) On the circle with center O below,
\[%
%TCIMACRO{\FRAME{itbpF}{1.8127in}{1.4814in}{0in}{}{}{Figure}%
%{\special{ language "Scientific Word";  type "GRAPHIC";
%maintain-aspect-ratio TRUE;  display "USEDEF";  valid_file "T";
%width 1.8127in;  height 1.4814in;  depth 0in;  original-width 6.9453in;
%original-height 5.6697in;  cropleft "0";  croptop "1";  cropright "1";
%cropbottom "0";  tempfilename 'MXAJBY05.png';tempfile-properties "XPR";}}}%
%BeginExpansion
{\includegraphics[
natheight=5.669700in,
natwidth=6.945300in,
height=1.4814in,
width=1.8127in
]%
{MXAJBY05.png}%
}%
%EndExpansion
\]
show that%
\[
\angle AXB=(1/2)(\angle AOB).
\]


Hint: $\triangle OAX$ is isosceles.
\end{exercise}

\begin{exercise}
(\textbf{EG}) On the circle with center $O$ below,
\[%
%TCIMACRO{\FRAME{itbpF}{1.6509in}{1.5878in}{0in}{}{}{Figure}%
%{\special{ language "Scientific Word";  type "GRAPHIC";  display "USEDEF";
%valid_file "T";  width 1.6509in;  height 1.5878in;  depth 0in;
%original-width 6.3296in;  original-height 6.2855in;  cropleft "0";
%croptop "1";  cropright "1";  cropbottom "0";
%tempfilename 'MXAJBY06.png';tempfile-properties "XPR";}}}%
%BeginExpansion
{\includegraphics[
natheight=6.285500in,
natwidth=6.329600in,
height=1.5878in,
width=1.6509in
]%
{MXAJBY06.png}%
}%
%EndExpansion
\]
show that%
\[
\angle AXB=(1/2)(\angle AOB).
\]


Hint: Draw the diameter through $O$ and $X$ and add.
\end{exercise}

\begin{exercise}
(\textbf{EG}) On the circle with center $O$ below,
\[%
%TCIMACRO{\FRAME{itbpF}{1.7218in}{1.5031in}{0in}{}{}{Figure}%
%{\special{ language "Scientific Word";  type "GRAPHIC";
%maintain-aspect-ratio TRUE;  display "USEDEF";  valid_file "T";
%width 1.7218in;  height 1.5031in;  depth 0in;  original-width 6.5492in;
%original-height 5.7147in;  cropleft "0";  croptop "1";  cropright "1";
%cropbottom "0";  tempfilename 'MXAJBY07.png';tempfile-properties "XPR";}}}%
%BeginExpansion
{\includegraphics[
natheight=5.714700in,
natwidth=6.549200in,
height=1.5031in,
width=1.7218in
]%
{MXAJBY07.png}%
}%
%EndExpansion
\]
show that{}%
\[
\angle AXB=(1/2)(\angle AOB).
\]


Hint: Draw the diameter through $O$ and $X$ and subtract.
\end{exercise}

We can summarize the results of the last three exercises into the following Theorem.

\begin{theorem}
\label{43}The measure of any angle inscribed in a circle is one-half of the
measure of the corresponding central angle.
\end{theorem}

\begin{exercise}
(\textbf{EG}) Use similar triangles and the previous Exercises to show that
$\left\vert AX\right\vert $\textperiodcentered$\left\vert XB\right\vert
=\left\vert A^{\prime}X\right\vert $\textperiodcentered$\left\vert XB^{\prime
}\right\vert $ in the figure below.%
\[%
%TCIMACRO{\FRAME{itbpF}{1.6319in}{1.3206in}{0pt}{}{}{Figure}%
%{\special{ language "Scientific Word";  type "GRAPHIC";
%maintain-aspect-ratio TRUE;  display "USEDEF";  valid_file "T";
%width 1.6319in;  height 1.3206in;  depth 0pt;  original-width 7.0776in;
%original-height 5.7147in;  cropleft "0";  croptop "1";  cropright "1";
%cropbottom "0";  tempfilename 'MXAJBY08.png';tempfile-properties "XPR";}}}%
%BeginExpansion
\raisebox{-0pt}{\includegraphics[
natheight=5.714700in,
natwidth=7.077600in,
height=1.3206in,
width=1.6319in
]%
{MXAJBY08.png}%
}%
%EndExpansion
\]


Hint: Draw $\overline{AB^{\prime}}$ and $\overline{A^{\prime}B}$.
\end{exercise}

\begin{exercise}
(\textbf{EG}) Use similar triangles and the previous Exercises to show that
$\left\vert AX\right\vert $\textperiodcentered$\left\vert XB\right\vert
=\left\vert A^{\prime}X\right\vert $\textperiodcentered$\left\vert XB^{\prime
}\right\vert $ in the figure below.%
\[%
%TCIMACRO{\FRAME{itbpF}{2.9032in}{1.3284in}{0in}{}{}{Figure}%
%{\special{ language "Scientific Word";  type "GRAPHIC";
%maintain-aspect-ratio TRUE;  display "USEDEF";  valid_file "T";
%width 2.9032in;  height 1.3284in;  depth 0in;  original-width 12.4395in;
%original-height 5.6697in;  cropleft "0";  croptop "1";  cropright "1";
%cropbottom "0";  tempfilename 'MXAJBY09.png';tempfile-properties "XPR";}}}%
%BeginExpansion
{\includegraphics[
natheight=5.669700in,
natwidth=12.439500in,
height=1.3284in,
width=2.9032in
]%
{MXAJBY09.png}%
}%
%EndExpansion
\]


Hint: Draw $\overline{AB^{\prime}}$ and $\overline{A^{\prime}B}$.
\end{exercise}

\begin{exercise}
Show that, given any three non-collinear points in the Euclidean plane, there
is a unique circle passing through the three points.

Hint: Show that the center of the circle must be the intersection of the
perpendicular bisectors of any two of the sides of the triangle whose vertices
are the three given points.
\end{exercise}

But how about four points in the plane, no three of which are collinear?

\begin{exercise}
a) Draw four points in the Euclidean plane, no $3$ of which are collinear,
that cannot lie on a single circle.

b) Draw four points in the Euclidean plane that do lie on a single circle.
\end{exercise}

The issue we will explore in the next two sections is the question of finding
a numerical condition about the four points that tells us exactly when they
all lie on a single circle. For that, we will need a very famous mathematical
relationship, one very closely related to the notion of perspective in
painting. That is, how do you faithfully render depth on a flat canvas? This
relationship is called the \textit{cross-ratio} of the four points. \pagebreak

\subsection{Cross-ratio of points on the number line}

We begin by studying the cross-ratio of four points on a line. Start with the
set of points on the real number line with coordinate $t$ and add one extra
point called $t=\infty$. Call the resulting set $\overline{\mathbb{R}}$. You
could think of the resulting set as the set of all lines through the origin in
$\mathbb{R}^{2}$ by assigning to each line the real number that is its slope
and to the $y$-axis the slope $\infty$.

\begin{exercise}
a) Show that the transformation%
\[
\left(  x,y\right)  \mapsto\left(  \underline{x},\underline{y}\right)
=\left(  x,y\right)  \cdot\left(
\begin{array}
[c]{cc}%
d & b\\
c & a
\end{array}
\right)
\]
is a $1-1$, onto (linear) transformation of $\mathbb{R}^{2}$ as long as%
\begin{equation}
\left\vert
\begin{array}
[c]{cc}%
d & b\\
c & a
\end{array}
\right\vert \neq0. \label{52}%
\end{equation}


b) For the transformation in the previous Exercise, show that every line
through the origin in $\left(  x,y\right)  $-space is sent to a line through
the origin in $\left(  \underline{x},\underline{y}\right)  $-space. The slope
$t$ of the line through $\left(  0,0\right)  $ and $\left(  x,y\right)  $ is
of course $t=\frac{y}{x}$. What is the slope \underline{$t$} of the line
through $\left(  \underline{0},\underline{0}\right)  $ and $\left(
\underline{x},\underline{y}\right)  $? Show that%
\begin{equation}
\underline{t}=\frac{at+b}{ct+d} \label{41}%
\end{equation}

\end{exercise}

\begin{definition}
Functions $\left(  \ref{41}\right)  $ for which the condition $\left(
\ref{52}\right)  $ holds are called \textbf{linear fractional transformations}.
\end{definition}

\begin{exercise}
Show that a linear fractional transformation%
\begin{gather*}
\overline{\mathbb{R}}\rightarrow\overline{\mathbb{R}}\\
t\mapsto\underline{t}=\frac{at+b}{ct+d}%
\end{gather*}
is $1-1$ and onto. What is its inverse function? (Your answer should show that
the inverse function is also a linear fractional transformation.)

Hint: By algebra solve for $t$ in terms of \underline{$t$}. Then graph%
\[
\underline{t}=\frac{at+b}{ct+d}%
\]
in the $\left(  t,\underline{t}\right)  $-plane. If $c=0$ show that the graph
is a straight line with non-zero slope and%
\[
\infty\mapsto\infty.
\]
If $c\neq0$, show that the graph has exactly one horizontal asymptote where
$t\mapsto\infty$ and one vertical asymptote where $\underline{t}\mapsto\infty$.
\end{exercise}

\begin{exercise}
Show that the set of linear fractional transformations form a group under the
operation of composition of functions. That is, check associativity, identity
element and existence of inverses.
\end{exercise}

\begin{exercise}
\label{59}Show that, for any three distinct points $t_{2},t_{3}$ and $t_{4}$,
the function of $t$ given by the formula%
\[
\underline{t}=\frac{t_{3}-t_{4}}{t_{3}-t_{2}}\frac{t-t_{2}}{t-t_{4}}%
=\frac{t-t_{2}}{t_{3}-t_{2}}\div\frac{t-t_{4}}{t_{3}-t_{4}}%
\]
takes $t_{2}$ to $0$, takes $t_{3}$ to $1$ and takes $t_{4}$ to $\infty$. Show
that this function is a linear fractional transformation, that is, a function
of the form $\left(  \ref{41}\right)  $ for which the condition $\left(
\ref{52}\right)  $ holds.
\end{exercise}

\begin{exercise}
\label{57}Show that any linear fractional transformation $\left(
\ref{41}\right)  $ that leaves $0$, $1$, and $\infty$ fixed is the identity map.
\end{exercise}

\begin{exercise}
\label{42}Suppose that you are given a function $\left(  \ref{41}\right)  $
and four points $t_{1},t_{2},t_{3}$ and $t_{4}$. Let
\[
\underline{t_{i}}=\frac{at_{i}+b}{ct_{i}+d}%
\]
for $i=1,2,3,4$. Show that%
\[
\frac{\underline{t_{1}}-\underline{t_{2}}}{\underline{t_{3}}-\underline{t_{2}%
}}\div\frac{\underline{t_{1}}-\underline{t_{4}}}{\underline{t_{3}}%
-\underline{t_{4}}}=\frac{t_{1}-t_{2}}{t_{3}-t_{2}}\div\frac{t_{1}-t_{4}%
}{t_{3}-t_{4}}.
\]
[MJG,288]

Hint: Just write out the formula for each side and do the high school algebra.
There is a fancier way that uses that the set of linear fractional
transformations form a group whose operation is composition. It goes like
this. Use Exercise \ref{59} to show that the inverse of the linear fractional
transformation
\[
t\mapsto\frac{t-t_{2}}{t_{3}-t_{2}}\div\frac{t-t_{4}}{t_{3}-t_{4}}%
\]
followed by%
\[
t\mapsto\underline{t}%
\]
and then followed by
\[
t\mapsto\frac{t-\underline{t_{2}}}{\underline{t_{3}}-\underline{t_{2}}}%
\div\frac{t-\underline{t_{4}}}{\underline{t_{3}}-\underline{t_{4}}}%
\]
fixes $0$, $1$, and $\infty$ and so is the identity transformation by Exercise
\ref{57}. So%
\[
t\mapsto\frac{t-t_{2}}{t_{3}-t_{2}}\div\frac{t-t_{4}}{t_{3}-t_{4}}%
\]
is the same transformation as%
\[
t\mapsto\frac{\underline{t}-\underline{t_{2}}}{\underline{t_{3}}%
-\underline{t_{2}}}\div\frac{\underline{t}-\underline{t_{4}}}{\underline
{t_{3}}-\underline{t_{4}}}.
\]

\end{exercise}

\begin{definition}
\label{44}The cross-ratio $\left(  t_{1}:t_{2}:t_{3}:t_{4}\right)  $ of four
(ordered) points $t_{1},t_{2},t_{3}$ and $t_{4}$ is defined by%
\[
\left(  t_{1}:t_{2}:t_{3}:t_{4}\right)  =\frac{t_{1}-t_{2}}{t_{3}-t_{2}}%
\div\frac{t_{1}-t_{4}}{t_{3}-t_{4}}.
\]

\end{definition}

Exercise \ref{42} shows that if four points are moved by any function $\left(
\ref{41}\right)  $ the cross-ratio $\left(  \underline{t_{1}}:\underline
{t_{2}}:\underline{t_{3}}:\underline{t_{4}}\right)  $ of the output four
points is the same as the cross-ratio $\left(  t_{1}:t_{2}:t_{3}:t_{4}\right)
$ of the original four points.\pagebreak

\subsection{Cross-ratio of points on a circle}

\begin{exercise}
\label{46}(\textbf{EG}) a) In the diagram
\[%
%TCIMACRO{\FRAME{itbpF}{1.1087in}{1.0032in}{0in}{}{}{Figure}%
%{\special{ language "Scientific Word";  type "GRAPHIC";
%maintain-aspect-ratio TRUE;  display "USEDEF";  valid_file "T";
%width 1.1087in;  height 1.0032in;  depth 0in;  original-width 6.0217in;
%original-height 5.4509in;  cropleft "0";  croptop "1";  cropright "1";
%cropbottom "0";  tempfilename 'MXAJBY0A.png';tempfile-properties "XPR";}}}%
%BeginExpansion
{\includegraphics[
natheight=5.450900in,
natwidth=6.021700in,
height=1.0032in,
width=1.1087in
]%
{MXAJBY0A.png}%
}%
%EndExpansion
\]
show that%
\[
\frac{\left\vert AB\right\vert }{\left\vert CB\right\vert }=\frac
{\mathrm{sin}\alpha}{\mathrm{sin}\beta}=\frac{\mathrm{sin}\left(  \angle
AOB\right)  }{\mathrm{sin}\left(  \angle COB\right)  }.
\]


Hint: Notice that by Theorem \ref{43}
\[
m\left(  \angle BAO\right)  +m\left(  OCB\right)  =180^{\circ}%
\]
so that%
\[
\mathrm{sin}\left(  \angle BAO\right)  =\mathrm{sin}\left(  OCB\right)  .
\]
Now use the Law of Sines.
\end{exercise}

\begin{exercise}
\label{47}(\textbf{EG}) Show that if, in the above figure, $B$ moves along the
circle to the other side of $C$, it is still true that%
\[
\frac{\left\vert AB\right\vert }{\left\vert CB\right\vert }=\frac
{\mathrm{sin}\left(  \angle AOB\right)  }{\mathrm{sin}\left(  \angle
COB\right)  }%
\]

\end{exercise}

\begin{exercise}
\label{48}(\textbf{EG})\ In the diagram%
\begin{equation}%
%TCIMACRO{\FRAME{itbpF}{1.9666in}{1.1312in}{0in}{}{}{Figure}%
%{\special{ language "Scientific Word";  type "GRAPHIC";
%maintain-aspect-ratio TRUE;  display "USEDEF";  valid_file "T";
%width 1.9666in;  height 1.1312in;  depth 0in;  original-width 11.7364in;
%original-height 6.7256in;  cropleft "0";  croptop "1";  cropright "1";
%cropbottom "0";  tempfilename 'MXAJBZ0B.png';tempfile-properties "XPR";}}}%
%BeginExpansion
{\includegraphics[
natheight=6.725600in,
natwidth=11.736400in,
height=1.1312in,
width=1.9666in
]%
{MXAJBZ0B.png}%
}%
%EndExpansion
\label{45}%
\end{equation}

\end{exercise}

show that%
\[
\frac{\left\vert A^{\prime}B^{\prime}\right\vert }{\left\vert C^{\prime
}B^{\prime}\right\vert }=\frac{\mathrm{sin}\alpha}{\mathrm{sin}\beta}\div
\frac{\mathrm{sin}\gamma}{\mathrm{sin}\delta}=\frac{\mathrm{sin}\left(  \angle
A^{\prime}OB^{\prime}\right)  }{\mathrm{sin}\left(  \angle C^{\prime
}OB^{\prime}\right)  }\div\frac{\mathrm{sin}\left(  \angle B^{\prime}%
A^{\prime}O\right)  }{\mathrm{sin}\left(  \angle B^{\prime}C^{\prime}O\right)
}.
\]
[MJG,266-267]

\begin{exercise}
\label{49}(\textbf{EG}) Show that if, in the above figure, $B^{\prime}$ moves
along the line to the other side of $C^{\prime}$, it is still true that%
\[
\frac{\left\vert A^{\prime}B^{\prime}\right\vert }{\left\vert C^{\prime
}B^{\prime}\right\vert }=\frac{\mathrm{sin}\left(  \angle A^{\prime}%
OB^{\prime}\right)  }{\mathrm{sin}\left(  \angle C^{\prime}OB^{\prime}\right)
}\div\frac{\mathrm{sin}\left(  \angle B^{\prime}A^{\prime}O\right)
}{\mathrm{sin}\left(  \angle B^{\prime}C^{\prime}O\right)  }.
\]

\end{exercise}

These last two Exercises allow us to define the cross-ratio of four points on
a circle.

\begin{definition}
(\textbf{EG}) For a sequence of four (ordered) points $A,B,C,$ and $D$ on a
circle, we define%
\[
\left(  A:B:C:D\right)  =\frac{\left\vert AB\right\vert }{\left\vert
CB\right\vert }\div\frac{\left\vert AD\right\vert }{\left\vert CD\right\vert }%
\]
which we call the cross-ratio of the ordered sequence of the four points.
Similarly for a sequence of four (ordered) points $A^{\prime},B^{\prime
},C^{\prime},$ and $D^{\prime}$ on a line, we define%
\[
\left(  A^{\prime}:B^{\prime}:C^{\prime}:D^{\prime}\right)  =\frac{\left\vert
A^{\prime}B^{\prime}\right\vert }{\left\vert C^{\prime}B^{\prime}\right\vert
}\div\frac{\left\vert A^{\prime}D^{\prime}\right\vert }{\left\vert C^{\prime
}D^{\prime}\right\vert }%
\]
which we call the cross-ratio of the ordered sequence of the four points.
\end{definition}

Notice that Definition \ref{44} is just a refinement of the definition of
$\left(  A^{\prime}:B^{\prime}:C^{\prime}:D^{\prime}\right)  $ just above. In
Definition \ref{44} we are keeping track of the signs of the terms in the
quotients whereas $\left(  A^{\prime}:B^{\prime}:C^{\prime}:D^{\prime}\right)
$ is always non-negative.

\begin{exercise}
\label{50}a) Show that, in the figure%
\[%
%TCIMACRO{\FRAME{itbpF}{2.0522in}{0.9842in}{0in}{}{}{Figure}%
%{\special{ language "Scientific Word";  type "GRAPHIC";
%maintain-aspect-ratio TRUE;  display "USEDEF";  valid_file "T";
%width 2.0522in;  height 0.9842in;  depth 0in;  original-width 14.0662in;
%original-height 6.7256in;  cropleft "0";  croptop "1";  cropright "1";
%cropbottom "0";  tempfilename 'MXAJBZ0C.png';tempfile-properties "XPR";}}}%
%BeginExpansion
{\includegraphics[
natheight=6.725600in,
natwidth=14.066200in,
height=0.9842in,
width=2.0522in
]%
{MXAJBZ0C.png}%
}%
%EndExpansion
\]
we have the equality%
\[
\left(  A:B:C:D\right)  =\left(  A^{\prime}:B^{\prime}:C^{\prime}:D^{\prime
}\right)  .
\]


Hint: Use Exercises \ref{46}-\ref{49}.

b) What happens in a) if we move $B$ to the other side of $C$?
\end{exercise}

We say that \textquotedblleft Cross-ratio is invariant under stereographic
projection."\pagebreak

\subsection{Ptolemy's Theorem}

Given any three non-collinear points in the Euclidean plane, there is one and
only one circle that passes through the three points. (How do you construct
it?) You can easily convince yourself with a few examples that, given four
non-collinear points $A,B,C$ and $D$ in the plane, it is not always true that
there is a circle that passes through all four. A famous theorem of classical
Euclidean geometry gives the condition that there is a circle that passes
through all four.

\begin{theorem}
(Ptolemy) If the ordered sequence of points $A,B,C$ and $D$ lies on a circle,
\[%
%TCIMACRO{\FRAME{itbpF}{1.4278in}{1.2263in}{0in}{}{}{Figure}%
%{\special{ language "Scientific Word";  type "GRAPHIC";
%maintain-aspect-ratio TRUE;  display "USEDEF";  valid_file "T";
%width 1.4278in;  height 1.2263in;  depth 0in;  original-width 6.1981in;
%original-height 5.3186in;  cropleft "0";  croptop "1";  cropright "1";
%cropbottom "0";  tempfilename 'MXAJBZ0D.png';tempfile-properties "XPR";}}}%
%BeginExpansion
{\includegraphics[
natheight=5.318600in,
natwidth=6.198100in,
height=1.2263in,
width=1.4278in
]%
{MXAJBZ0D.png}%
}%
%EndExpansion
\]
then%
\[
\left\vert AC\right\vert \text{\textperiodcentered}\left\vert BD\right\vert
=\left\vert AD\right\vert \text{\textperiodcentered}\left\vert BC\right\vert
+\left\vert AB\right\vert \text{\textperiodcentered}\left\vert CD\right\vert
.
\]
That is, the product of the diagonals of the quadrilateral $ABCD$ is the sum
of the products of pairs of opposite sides.
\end{theorem}

\begin{proof}
We need to check that%
\[
\left\vert AC\right\vert \text{\textperiodcentered}\left\vert BD\right\vert
=\left\vert AD\right\vert \text{\textperiodcentered}\left\vert BC\right\vert
+\left\vert AB\right\vert \text{\textperiodcentered}\left\vert CD\right\vert
\]
or, what is the same, we need to check that%
\[
\frac{\left\vert AC\right\vert \text{\textperiodcentered}\left\vert
BD\right\vert }{\left\vert AD\right\vert \text{\textperiodcentered}\left\vert
BC\right\vert }=1+\frac{\left\vert AB\right\vert \text{\textperiodcentered
}\left\vert CD\right\vert }{\left\vert AD\right\vert \text{\textperiodcentered
}\left\vert BC\right\vert }.
\]
That is, we need to check that
\[
\left(  A:C:B:D\right)  =1+\left(  A:B:C:D\right)  .
\]
But by Exercise \ref{50} this is the same as checking that%
\[
\left(  A^{\prime}:C^{\prime}:B^{\prime}:D^{\prime}\right)  =1+\left(
A^{\prime}:B^{\prime}:C^{\prime}:D^{\prime}\right)
\]
for the projection of the four points onto a line from a point $O$ on the
circle. But that is the same thing as showing that
\[
\frac{\left\vert A^{\prime}C^{\prime}\right\vert \text{\textperiodcentered
}\left\vert B^{\prime}D^{\prime}\right\vert }{\left\vert A^{\prime}D^{\prime
}\right\vert \text{\textperiodcentered}\left\vert B^{\prime}C^{\prime
}\right\vert }=1+\frac{\left\vert A^{\prime}B^{\prime}\right\vert
\text{\textperiodcentered}\left\vert C^{\prime}D^{\prime}\right\vert
}{\left\vert A^{\prime}D^{\prime}\right\vert \text{\textperiodcentered
}\left\vert B^{\prime}C^{\prime}\right\vert }%
\]
which is the same thing as showing that%
\[
\left\vert A^{\prime}C^{\prime}\right\vert \text{\textperiodcentered
}\left\vert B^{\prime}D^{\prime}\right\vert =\left\vert A^{\prime}D^{\prime
}\right\vert \text{\textperiodcentered}\left\vert B^{\prime}C^{\prime
}\right\vert +\left\vert A^{\prime}B^{\prime}\right\vert
\text{\textperiodcentered}\left\vert C^{\prime}D^{\prime}\right\vert .
\]
Now check this last equality by high-school algebra.
\end{proof}

\pagebreak

\part{Spherical Geometry I}

\section{Surface area and volume of the $R$-sphere in Euclidean $3$-space}

\subsection{Volumes of pyramids}

We are now going to study the geometry of the $R$-sphere in Euclidean
$3$-space. We will first study this geometry in the usual way, namely using
$\left(  \hat{x},\hat{y},\hat{z}\right)  $-coordinates. We start with a
relationship that shows why there is a factor of $1/3$ in many formulas for
volumes in $3$-dimensional Euclidean geometry.just like there is a factor of
$1/2$ in many formulas for areas in $2$-dimensional Euclidean geometry.

\begin{exercise}
(\textbf{EG}) Show that an $r\times r\times r$ cube can be constructed from
three equal pyramids with an $r\times r$ square base. Conclude that the volume
of each pyramid is $1/3$ the volume of the cube, namely
\[
\frac{r^{3}}{3}.
\]


Hint: Suppose the cube had a hollow interior and infinitely thin faces. Put
your (infinitely tiny) eye at one vertex of the cube and look inside. How many
faces of the cube can you see?
\end{exercise}

We next want to show why any pyramid with $r\times r$ square base and vertical
altitude $r$ has the same volume. That is, if we put the vertex of the pyramid
anywhere in a plane parallel to the base and at distance $r$, the volume is unchanged.

This fact is an example of \textit{Cavalieri's Principle}: Shearing a figure
parallel to a fixed direction does not change the $n$-dimensional measure of
an object in Euclidean $n$-space. (Think of a stack of (very thin) books.)

\begin{exercise}
Show that Cavalieri's Principle is true for the pyramid using several variable calculus.

Hint: Put the base of the pyramid $P$ so that its vertices are $\left(
0,0\right)  $, $\left(  r,0\right)  $, $\left(  0,r\right)  $ and $\left(
r,r\right)  $ in $3$-dimensional Euclidean space. Consider the transformation%
\[
\left(  \underline{\hat{x}},\underline{\hat{y}},\underline{\hat{z}}\right)
=\left(  \hat{x},\hat{y},\hat{z}\right)  \left(
\begin{array}
[c]{ccc}%
1 & 0 & 0\\
0 & 1 & 0\\
a & b & 1
\end{array}
\right)
\]
and notice that%
\[%
%TCIMACRO{\dint \nolimits_{\underline{P}}}%
%BeginExpansion
{\displaystyle\int\nolimits_{\underline{P}}}
%EndExpansion
d\underline{\hat{x}}d\underline{\hat{y}}d\underline{\hat{z}}=\left\vert
\left(
\begin{array}
[c]{ccc}%
1 & 0 & 0\\
0 & 1 & 0\\
a & b & 1
\end{array}
\right)  \right\vert
%TCIMACRO{\dint \nolimits_{P}}%
%BeginExpansion
{\displaystyle\int\nolimits_{P}}
%EndExpansion
d\hat{x}d\hat{y}d\hat{z}.
\]
\pagebreak
\end{exercise}

\subsection{Magnification principle}

\textit{Magnification principle:} If an object in Euclidean $n$-space is
magnified by factors of $r_{1}$,\ldots$,r_{n}$, its $n$-dimensional measure is
multiplied by $r_{1}$\textperiodcentered\ldots$.r_{n}$.

\begin{exercise}
(\textbf{EG}) Use this magnification principle to justify the volume formula%
\[
(1/3)B\text{\textperiodcentered}h
\]
for any pyramid with rectangular base of area $B$ and vertical altitude $h$.
\end{exercise}

\begin{exercise}
Prove the magnification principle using several variable calculus.

Hint: Consider the transformation%
\[
\left(  \underline{\hat{x}_{1}},\ldots,\underline{\hat{x}_{n}}\right)
=\left(  \hat{x}_{1},\ldots,\hat{x}_{n}\right)  \left(
\begin{array}
[c]{ccc}%
r_{1} & \ldots & 0\\
\ldots & \ldots & \ldots\\
0 & \ldots & r_{n}%
\end{array}
\right)
\]
and notice that%
\[%
%TCIMACRO{\dint \nolimits_{\underline{P}}}%
%BeginExpansion
{\displaystyle\int\nolimits_{\underline{P}}}
%EndExpansion
d\underline{\hat{x}_{1}}\ldots d\underline{\hat{x}_{n}}=r_{1}%
\text{\textperiodcentered}\ldots\text{\textperiodcentered}r_{n}%
%TCIMACRO{\dint \nolimits_{P}}%
%BeginExpansion
{\displaystyle\int\nolimits_{P}}
%EndExpansion
d\hat{x}_{1}\ldots d\hat{x}_{n}.
\]

\end{exercise}

Now suppose we have any pyramid%
%TCIMACRO{\FRAME{ftbpF}{1.1736in}{1.0084in}{0pt}{}{}{Figure}%
%{\special{ language "Scientific Word";  type "GRAPHIC";
%maintain-aspect-ratio TRUE;  display "USEDEF";  valid_file "T";
%width 1.1736in;  height 1.0084in;  depth 0pt;  original-width 7.165in;
%original-height 6.154in;  cropleft "0";  croptop "1";  cropright "1";
%cropbottom "0";  tempfilename 'MXAJBZ0E.png';tempfile-properties "XPR";}}}%
%BeginExpansion
\begin{figure}
[ptb]
\begin{center}
\includegraphics[
natheight=6.154000in,
natwidth=7.165000in,
height=1.0084in,
width=1.1736in
]%
{MXAJBZ0E.png}%
\end{center}
\end{figure}
%EndExpansion
with any shaped base of area $B$ and any verticle altitude $h$. Approximate
the base as close as you want (i.e $\varepsilon$-close) by tiling its interior
with rectangles. Let $t$ denote the sum of the areas of these rectangles.
Approximate the base as close as you want (i.e $\varepsilon$-close) by
covering it entirely with rectangles. Let $T$ denote the sum of the areas of
these rectangles. Why is the area $B$ of the base of the pyramid caught
between $B-\varepsilon$ and $B+\varepsilon$?

\begin{exercise}
(\textbf{EG}) Show that the volume $V$ of the pyramid is caught between
$\left(  1/3\right)  $\textperiodcentered$t$\textperiodcentered$h$ and
$\left(  1/3\right)  $\textperiodcentered$T$\textperiodcentered$h$.
\end{exercise}

\begin{exercise}
(\textbf{EG}) Argue that, given any positive real number $\varepsilon$,
however small, the volume $V$ of the pyramid is caught between $\left(
1/3\right)  $\textperiodcentered$\left(  B-\varepsilon\right)  $%
\textperiodcentered$h$ and $\left(  1/3\right)  $\textperiodcentered$\left(
B+\varepsilon\right)  $\textperiodcentered$h$.
\end{exercise}

\begin{exercise}
(\textbf{EG}) Show that
\[
V=\left(  1/3\right)  \text{\textperiodcentered}B\text{\textperiodcentered}h.
\]
\pagebreak
\end{exercise}

\subsection{Relation between volume and surface area of a sphere}

Think of a disco-ball. Its surface is approximately a sphere, but that surface
is made up of tiny flat mirrors.

\begin{exercise}
(\textbf{SG}) a)Why can you think of the disco-ball as being made up of
pyramids, with each pyramid having base one of the tiny mirrors and vertex at
the interior point $O$ at the center of the disco-ball.

b) Argue that the volume of the disco-ball is $\left(  1/3\right)  $ times the
distance $h$ from a mirror to $O$ times the sum of the areas of all the mirrors.
\end{exercise}

\begin{exercise}
(\textbf{SG}) Argue that, as the mirrors are made to be smaller and smaller,

1) the sum of the areas of the mirrors approaches the surface area of a sphere,

2) the distance $h$ approaches the radius $R$ of that sphere,

3) the volume of the disco-ball approaches the volume of the sphere.

Conclude that, for a sphere of radius $R$ in Euclidean $3$-space, the relation
betwee the volume $V$ of the sphere and the surface area $S$ of the sphere is
given by the formula%
\[
V=\frac{R\text{\textperiodcentered}S}{3}.
\]

\end{exercise}

Our goal in the next Subsection is to compute the surface area of the sphere
of radius $R$ in Euclidean $3$-space. The formula just above will then let us
compute the volume of the same sphere.\pagebreak

\subsection{Surface area}

To compute the surface area of the sphere of radius $R$ in $3$-dimensional
Euclidean space, we will show that its surface area is equal to the surface
area of something we can lay out flat. The argument for this goes way back to
the great physicist and mathematician, Archimedes of Alexandria, in the $2$nd
century B.C. To follow his argument, we have to begin by computing the area of
a `lamp shade' or `collar.' We think of a circular collar as in the figure
below%
\[%
%TCIMACRO{\FRAME{itbpF}{0.9772in}{0.6218in}{0in}{}{}{Figure}%
%{\special{ language "Scientific Word";  type "GRAPHIC";
%maintain-aspect-ratio TRUE;  display "USEDEF";  valid_file "T";
%width 0.9772in;  height 0.6218in;  depth 0in;  original-width 4.9225in;
%original-height 3.1202in;  cropleft "0";  croptop "1";  cropright "1";
%cropbottom "0";  tempfilename 'MXAJBZ0F.png';tempfile-properties "XPR";}}}%
%BeginExpansion
{\includegraphics[
natheight=3.120200in,
natwidth=4.922500in,
height=0.6218in,
width=0.9772in
]%
{MXAJBZ0F.png}%
}%
%EndExpansion
\]
as approximated by an arrangement of trapezoids. To achieve this, we
approximate the bottom circle of the collar by an inscribed regular $n$-gon
whose vertices are the points of intersection with the slant lines in the
figure. Similarly approximate the top circle by an inscribed regular $n$-gon
positioned directly above the bottom one, again with vertices given by the
points of intersection with the slant lines. Complete a side of the bottom
$n$-gon and the side of the top $n$-gon directly above it to a trapezoid by
adjoining the two slant lines in the figure that connect endpoints. Let
$b_{n}$ denote the length of a side of the bottom regular $n$-gon and let
$t_{n}$ denote the length of a side of the top $n$-gon. Then the trapezoid has
area%
\[
\left(  \frac{b_{n}+t_{n}}{2}\right)  \text{\textperiodcentered}h_{n}%
\]
where $h_{n}$ is the vertical height of the trapeziod. The collar is
approximated by the union of these $n$ trapezoids, so the area of the collar
is approximated by the sum of the areas of the $n$ congruent trapezoids,
namely%
\[
n\text{\textperiodcentered}\left(  \frac{b_{n}+t_{n}}{2}\right)
\text{\textperiodcentered}h_{n}=\left(  \frac{n\text{\textperiodcentered}%
b_{n}+n\text{\textperiodcentered}t_{n}}{2}\right)  \text{\textperiodcentered
}h_{n}.
\]
As $n$ goes to infinity, the area of the approximation approaches the area of
the collar. But
\begin{align*}
\underset{n\rightarrow\infty}{\mathrm{lim}}b_{n}  &  =c_{b}\\
\underset{n\rightarrow\infty}{\mathrm{lim}}t_{n}  &  =c_{t}\\
\underset{n\rightarrow\infty}{\mathrm{lim}}h_{n}  &  =s
\end{align*}
where $c_{b}$ is the circumference of the bottom circle and $c_{t}$ is the
circumference of the top circle and $s$ is the slant height of the collar as
shown in the above figure. We conclude that the area of the collar is%
\begin{align}
\frac{c_{b}+c_{t}}{2}\text{\textperiodcentered}s  &  =\pi
\text{\textperiodcentered}\left(  r_{b}+r_{t}\right)
\text{\textperiodcentered}s\label{38}\\
&  =2\pi\text{\textperiodcentered}r_{a}\text{\textperiodcentered}s\nonumber
\end{align}
where $r_{b}$ and $r_{t}$ are the radii of the respective circles and $r_{a}$
is the average of the two.

\begin{theorem}
(\textbf{SG}) The surface area of the sphere of radius $R$ is the same as the
surface area of the label of the smallest can into which the sphere will fit.%
\[%
%TCIMACRO{\FRAME{itbpF}{1.0188in}{1.107in}{0in}{}{}{Figure}%
%{\special{ language "Scientific Word";  type "GRAPHIC";
%maintain-aspect-ratio TRUE;  display "USEDEF";  valid_file "T";
%width 1.0188in;  height 1.107in;  depth 0in;  original-width 5.5383in;
%original-height 6.0217in;  cropleft "0";  croptop "1";  cropright "1";
%cropbottom "0";  tempfilename 'MXAJBZ0G.png';tempfile-properties "XPR";}}}%
%BeginExpansion
{\includegraphics[
natheight=6.021700in,
natwidth=5.538300in,
height=1.107in,
width=1.0188in
]%
{MXAJBZ0G.png}%
}%
%EndExpansion
\]
Namely the surface area of the sphere of radius $R$ is
\[
2\pi R\text{\textperiodcentered}2R=4\pi R^{2}.
\]

\end{theorem}

\begin{exercise}
(\textbf{SG}) Show why the above Theorem is true.

Hint: Slice the picture above into $n$ horizontal slices. Approximate the
piece of the surface of the sphere between the $i$-th pair of successive
slices by a collar $C_{i}$. Let $a\left(  C_{i}\right)  $ denote the area of
$C_{i}$, let $r_{i}$ denote its average radius and let $s_{i}$ denote its
slant height. Then the surface area of the sphere is approximately%
\[%
%TCIMACRO{\dsum \nolimits_{i=1}^{n}}%
%BeginExpansion
{\displaystyle\sum\nolimits_{i=1}^{n}}
%EndExpansion
2\pi\text{\textperiodcentered}r_{i}\text{\textperiodcentered}s_{i},
\]
at least if the slices are pretty thin. (Why?) Also the approximate area $%
%TCIMACRO{\dsum \nolimits_{i=1}^{n}}%
%BeginExpansion
{\displaystyle\sum\nolimits_{i=1}^{n}}
%EndExpansion
a\left(  C_{i}\right)  $ approaches the exact surface area of the sphere as
the slices get thinner and thinner.

Next let $h_{i}$ denote the vertical height of the label on the can between
the $i$-th pair of successive slices. The area of the label is exactly%
\[%
%TCIMACRO{\dsum \nolimits_{i=1}^{n}}%
%BeginExpansion
{\displaystyle\sum\nolimits_{i=1}^{n}}
%EndExpansion
2\pi\text{\textperiodcentered}h_{i}.
\]
(Why?) Explain why the relationship between each $r_{i}$, $s_{i}$ and $h_{i}$
is given by the picture below.
\[%
%TCIMACRO{\FRAME{itbpF}{1.5999in}{0.9954in}{0in}{}{}{Figure}%
%{\special{ language "Scientific Word";  type "GRAPHIC";
%maintain-aspect-ratio TRUE;  display "USEDEF";  valid_file "T";
%width 1.5999in;  height 0.9954in;  depth 0in;  original-width 1.8023in;
%original-height 1.1104in;  cropleft "0";  croptop "1";  cropright "1";
%cropbottom "0";  tempfilename 'MXAJBZ0H.png';tempfile-properties "XPR";}}}%
%BeginExpansion
{\includegraphics[
natheight=1.110400in,
natwidth=1.802300in,
height=0.9954in,
width=1.5999in
]%
{MXAJBZ0H.png}%
}%
%EndExpansion
\]
Now use Exercise \ref{39}b) to explain why
\[
r_{i}\text{\textperiodcentered}s_{i}=h_{i}\text{\textperiodcentered}R.
\]
Finally explain why this completes the proof of the Theorem.
\end{exercise}

\pagebreak

\section{Areas on spheres in Euclidean $3$-space}

\subsection{Lunes}

In the picture we have shaded in an ` $\alpha$-lune' on the $R$-sphere in
Euclidean $3$-space.%
\[%
%TCIMACRO{\FRAME{itbpF}{1.2505in}{1.19in}{0in}{}{}{Figure}%
%{\special{ language "Scientific Word";  type "GRAPHIC";
%maintain-aspect-ratio TRUE;  display "USEDEF";  valid_file "T";
%width 1.2505in;  height 1.19in;  depth 0in;  original-width 5.4509in;
%original-height 5.1863in;  cropleft "0";  croptop "1";  cropright "1";
%cropbottom "0";  tempfilename 'MXAJBZ0I.png';tempfile-properties "XPR";}}}%
%BeginExpansion
{\includegraphics[
natheight=5.186300in,
natwidth=5.450900in,
height=1.19in,
width=1.2505in
]%
{MXAJBZ0I.png}%
}%
%EndExpansion
\]


The lune has two vertices. They are at opposite (antipodal) points on the
$R$-sphere, that is, the line in Eucludean $3$-space that joins the two
vertices runs through the center of the sphere. The angle at a vertex of the
lune is $\alpha$ radians.

\begin{exercise}
\label{67}(\textbf{SG}) Explain why the area of the $\alpha$-lune is $2\alpha
$\textperiodcentered$R^{2}$.
\end{exercise}

\pagebreak

\subsection{Spherical triangles}

If a triangle on the sphere of radius $R$ has interior angles with radian
measures $\alpha$, $\beta$, and $\gamma$, it can be covered three times by
lunes as shown in the figure below.%
\[%
%TCIMACRO{\FRAME{itbpF}{1.1865in}{1.4425in}{0pt}{}{}{Figure}%
%{\special{ language "Scientific Word";  type "GRAPHIC";
%maintain-aspect-ratio TRUE;  display "USEDEF";  valid_file "T";
%width 1.1865in;  height 1.4425in;  depth 0pt;  original-width 5.0548in;
%original-height 6.154in;  cropleft "0";  croptop "1";  cropright "1";
%cropbottom "0";  tempfilename 'MXAJBZ0J.png';tempfile-properties "XPR";}}}%
%BeginExpansion
\raisebox{-0pt}{\includegraphics[
natheight=6.154000in,
natwidth=5.054800in,
height=1.4425in,
width=1.1865in
]%
{MXAJBZ0J.png}%
}%
%EndExpansion
\]
Notice that each lune has one vertex at a vertex of the triangle and angle
equal to that interior angle of the triangle. The other vertices of each lune
are vertices of an `opposite' triangle that has the same area as the given one
since it is just the image of the given one under the rigid motion%
\[
\left(  \underline{\hat{x}},\underline{\hat{y}},\underline{\hat{z}}\right)
=\left(  \hat{x},\hat{y},\hat{z}\right)  \cdot\left(
\begin{array}
[c]{ccc}%
-1 & 0 & 0\\
0 & -1 & 0\\
0 & 0 & -1
\end{array}
\right)  .
\]
(See, for example, the formula $\left(  \ref{68}\right)  $.) The three lunes
cover the triangle three times. The three opposite lunes cover the opposite
triangle three times. If you take all six lunes together, they cover each of
the two triangles three times and everything else exactly once.

\begin{exercise}
(\textbf{SG}) a) Show that the area of the spherical triangle is given by the
formula%
\[
R^{2}\left(  \left(  \alpha+\beta+\gamma\right)  -\pi\right)  ,
\]
that is,%
\[
\left\vert K\right\vert ^{-1}\left(  \left(  \alpha+\beta+\gamma\right)
-\pi\right)  .
\]


Hint: Use Exercise \ref{67} to turn the sentence just preceding the Exercise
into an equation.

b) Give a formula for the area of any spherical $n$-gon.

Hint: Divide the spherical $n$-gon into spherical triangles.\pagebreak
\end{exercise}

\part{Usual and unusual coordinates for three-dimensional Euclidean
space\label{I}}

\section{Euclidean three-space as a metric space}

\subsection{Points and vectors in Euclidean 3-space}

In this book, we will study all the \textit{two}-dimensional geometries
(spheres, the plane and hyperbolic spaces). Each one of these geometries looks
the same at each of its points and it also looks the same in every direction
emanating from any of its points. But to study them all at the same time and
in a uniform way we will need to visualize them all as different surfaces
lying in some common \textit{three}-dimensional space. We start with the most
familiar cases, namely the spherical geometries.

For those we begin with three-dimensional Euclidean space%
\[
\mathbb{R}^{3}=\left\{  \left(  \hat{x},\hat{y},\hat{z}\right)  :\hat{x}%
,\hat{y},\hat{z}\in\mathbb{R}\right\}  ,
\]
where there is a standard way to measure distance between two points%
\begin{align*}
\hat{X}_{1}  &  =\left(  \hat{x}_{1},\hat{y}_{1},\hat{z}_{1}\right) \\
\hat{X}_{2}  &  =\left(  \hat{x}_{2},\hat{y}_{2},\hat{z}_{2}\right)  ,
\end{align*}
namely%
\begin{equation}
d\left(  \hat{X}_{1},\hat{X}_{2}\right)  =\sqrt{\left(  \hat{x}_{2}-\hat
{x}_{1}\right)  ^{2}+\left(  \hat{y}_{2}-\hat{y}_{1}\right)  ^{2}+\left(
\hat{z}_{2}-\hat{z}_{1}\right)  ^{2}}. \label{0}%
\end{equation}


As we will see, the formula $\left(  \ref{0}\right)  $ is compatible with
distances on such objects as spheres%
\[
\left\{  \left(  \hat{x},\hat{y},\hat{z}\right)  \in\mathbb{R}^{3}:\hat{x}%
^{2}+\hat{y}^{2}+\hat{z}^{2}=R^{2}\right\}
\]
of a fixed radius $R$, since these can be faithfully represented in ordinary
Euclidean three-space. However, there is one disconcerting fact about studying
the geometry of spheres in this way. Namely, as $R$ approaches infinity, the
geometry of the $R$-sphere at any point looks more and more like plane
geometry, but on the other hand, that `limit' plane geometry is `out at
infinity.' So in order to study all the two-dimensional geometries, including
plane geometry and the hyperbolic geometries, in a uniform way we will have to
\textit{change} the coordinate system we use, or, what will turn out to be the
same thing, we will have to change the distance formula slightly for each
geometry. We will do that in later sections, but first we want to review some
of the basic properties of ordinary Euclidean three-space you learned about it
in calculus.

We write $\left(  \hat{x},\hat{y},\hat{z}\right)  $ for our ordinary Euclidean
coordinates. When you see $\left(  \hat{x},\hat{y},\hat{z}\right)  $ in what
follows, that means that distance between points is measured by the formula
$\left(  \ref{0}\right)  $. One more thing--in Euclidean three-space it will
be important thoughout to make the distinction between \textbf{points} and
\textbf{vectors}: Although each will be represented by a triple of real
numbers we will use%
\[
\hat{X}=\left(  \hat{x},\hat{y},\hat{z}\right)
\]
to denote \textbf{points}, that is, \textbf{position} in Euclidean $3$-space,
and%
\[
\hat{V}=\left(  \hat{a},\hat{b},\hat{c}\right)
\]
to denote \textbf{vectors}, that is, \textbf{displacement} by which we mean
the amount and direction a given point is being moved. So vectors always
indicate \textit{motion} from an explicit (or implicit) \textit{point} of
reference. \newpage

\subsection{Dot product of vectors emanating from a point}

There are various operations we can perform on one or more vectors when we
think of them as emanating from the same point in Euclidean $3$-space. The
first is the dot product of two vectors.

\begin{definition}
The dot product of two vectors%
\begin{align*}
\hat{V}_{1}  &  =\left(  \hat{a}_{1},\hat{b}_{1},\hat{c}_{1}\right) \\
\hat{V}_{2}  &  =\left(  \hat{a}_{2},\hat{b}_{2},\hat{c}_{2}\right)
\end{align*}
\newline emanating from the same point in 3-dimensional Euclidean space is
defined as the real number given by the formula%
\[
\hat{a}_{1}\hat{a}_{2}+\hat{b}_{1}\hat{b}_{2}+\hat{c}_{1}\hat{c}_{2}%
\]
or in matrix notation as%
\[
\left(
\begin{array}
[c]{ccc}%
\hat{a}_{1} & \hat{b}_{1} & \hat{c}_{1}%
\end{array}
\right)  \cdot\left(
\begin{array}
[c]{c}%
\hat{a}_{2}\\
\hat{b}_{2}\\
\hat{c}_{2}%
\end{array}
\right)  .
\]
It is also denoted as%
\[
\hat{V}_{1}\bullet\hat{V}_{2}%
\]
or in matrix notation as%
\[
\hat{V}_{1}\cdot\left(  \hat{V}_{2}\right)  ^{t}.
\]

\end{definition}

\begin{exercise}
Give the formula for the length $\left\vert \hat{V}\right\vert $ of a vector
$\hat{V}=\left(  \hat{a},\hat{b},\hat{c}\right)  $ in 3-dimensional Euclidean
space in terms of dot product.
\end{exercise}

\begin{exercise}
As you work through the proof of the Law of Cosines in the following Lemma,
construct a diagram or picture for each step.
\end{exercise}

\begin{lemma}
\label{110}(Law of Cosines) The (smaller) angle $\vartheta$ between two
vectors $\hat{V}_{1}$ and $\hat{V}_{2}$ emanating from $O=\left(
0,0,0\right)  $ satisfies the relation%
\[
\left\vert \hat{V}_{2}-\hat{V}_{1}\right\vert ^{2}=\left\vert \hat{V}%
_{1}\right\vert ^{2}+\left\vert \hat{V}_{2}\right\vert ^{2}-2\left\vert
\hat{V}_{1}\right\vert \cdot\left\vert \hat{V}_{2}\right\vert \cdot
\mathrm{cos}\vartheta.
\]

\end{lemma}

\begin{proof}
Without loss of generality we can assume that $\left\vert \hat{V}%
_{1}\right\vert \leq\left\vert \hat{V}_{2}\right\vert $. Consider the triangle
with one side given by the interval from $O=\left(  0,0,0\right)  $ to the
endpoint $P_{1}$ of $\hat{V}_{1}$, with a second side $S_{2}$ given by the
interval from $O$ to the endpoint $P_{2}$ of $\hat{V}_{2}$ and with the third
side given by the interval joining $P_{1}$ and $P_{2}$. Let $P$ be the point
on $S_{2}$ so that the interval between $P_{1}$ and $P$ is perpendicular to
$S_{2}$. By the Pythagorean theorem%
\begin{align*}
\left\vert P_{1}P_{2}\right\vert ^{2}-\left\vert P_{2}P\right\vert ^{2}  &
=\left\vert PP_{1}\right\vert ^{2}\\
&  =\left\vert OP_{1}\right\vert ^{2}-\left\vert OP\right\vert ^{2}\\
\left\vert P_{1}P_{2}\right\vert ^{2}  &  =\left\vert OP_{1}\right\vert
^{2}+\left(  \left\vert P_{2}P\right\vert ^{2}-\left\vert OP\right\vert
^{2}\right) \\
&  =\left\vert OP_{1}\right\vert ^{2}+\left(  \left\vert P_{2}P\right\vert
+\left\vert OP\right\vert \right)  \left(  \left\vert P_{2}P\right\vert
-\left\vert OP\right\vert \right) \\
&  =\left\vert OP_{1}\right\vert ^{2}+\left\vert OP_{2}\right\vert \left(
\left\vert P_{2}P\right\vert -\left\vert OP\right\vert \right) \\
&  =\left\vert OP_{1}\right\vert ^{2}+\left\vert OP_{2}\right\vert \left(
\left\vert OP_{2}\right\vert -2\left\vert OP\right\vert \right)
\end{align*}
But%
\[
\left\vert OP\right\vert =\left\vert OP_{1}\right\vert \cdot\mathrm{cos}%
\vartheta.
\]

\end{proof}

\begin{exercise}
What can you say about the cosine of the larger of the two angles between two
vectors $\hat{V}_{1}$ and $\hat{V}_{2}$, that is about $\left(  360^{\circ
}-\vartheta\right)  $?
\end{exercise}

\begin{lemma}
\label{111}The angle $\vartheta$ between two vectors $\hat{V}_{1}$ and
$\hat{V}_{2}$ emanating from the same point in Euclidean $3$-space satisfies
the relation
\begin{equation}
\hat{V}_{1}\bullet\hat{V}_{2}=\left\vert \hat{V}_{1}\right\vert \cdot
\left\vert \hat{V}_{2}\right\vert \cdot\mathrm{cos}\vartheta. \label{2}%
\end{equation}
[DS,30ff]
\end{lemma}

\begin{proof}
Multipying out using the definition and algebraic properties of dot product,%
\begin{align*}
\left\vert \hat{V}_{2}-\hat{V}_{1}\right\vert ^{2}  &  =\left(  \hat{V}%
_{2}-\hat{V}_{1}\right)  \bullet\left(  \hat{V}_{2}-\hat{V}_{1}\right) \\
&  =\left\vert \hat{V}_{1}\right\vert ^{2}+\left\vert \hat{V}_{2}\right\vert
^{2}-2\left(  \hat{V}_{1}\bullet\hat{V}_{2}\right)  .
\end{align*}
Now apply Lemma \ref{110}.
\end{proof}

The significance of Lemma \ref{111} is that the measure of angles between
vectors depends only on the definition of the dot product.

\begin{corollary}
The formula for the angle $\vartheta$ between two vectors $\hat{V}_{1}=\left(
\hat{a}_{1},\hat{b}_{1},\hat{c}_{1}\right)  $ and $\hat{V}_{2}=\left(  \hat
{a}_{2},\hat{b}_{2},\hat{c}_{2}\right)  $ in $3$-dimensional Euclidean space
depends only on the dot products of the two vectors with themselves and with
each other. Namely%
\[
\vartheta=\mathrm{arccos}\left(  \frac{\hat{V}_{1}\bullet\hat{V}_{2}%
}{\left\vert \hat{V}_{1}\right\vert \cdot\left\vert \hat{V}_{2}\right\vert
}\right)  .
\]

\end{corollary}

In fact it is also true that the formula for the area of the parallelogram
determined by two vectors $\hat{V}_{1}$ and $\hat{V}_{2}$ depends only on the
dot products of the two vectors with themselves and with each other. You will
see this by answering the following Exercises.

\begin{exercise}
Show that the area of the parallelogram determined by $\hat{V}_{1}$ and
$\hat{V}_{2}$ emanating from the same point in Euclidean $3$-space is given by%
\begin{equation}
\left\vert \hat{V}_{1}\right\vert \cdot\left\vert \hat{V}_{2}\right\vert
\cdot\mathrm{sin}\vartheta. \label{3}%
\end{equation}

\end{exercise}

\begin{exercise}
\label{9}Show that the area of the parallelogram determined by $\hat{V}_{1}$
and $\hat{V}_{2}$ emanating from the same point in Euclidean $3$-space is also
given by%
\[
\sqrt{\left\vert
\begin{array}
[c]{cc}%
\hat{V}_{1}\bullet\hat{V}_{1} & \hat{V}_{2}\bullet\hat{V}_{1}\\
\hat{V}_{1}\bullet\hat{V}_{2} & \hat{V}_{2}\bullet\hat{V}_{2}%
\end{array}
\right\vert }.
\]


Hint: Start from the square of $\left(  \ref{3}\right)  $, substitute $\left(
1-\mathrm{cos}^{2}\vartheta\right)  $ for $\mathrm{sin}^{2}\vartheta$, and use
$\left(  \ref{2}\right)  $. Alternatively start from $\left(  \ref{3}\right)
$ and show that%
\[
\mathrm{sin}\left(  \mathrm{arccos}\left(  \frac{\hat{V}_{1}\bullet\hat{V}%
_{2}}{\left\vert \hat{V}_{1}\right\vert \cdot\left\vert \hat{V}_{2}\right\vert
}\right)  \right)  =\frac{\sqrt{\left\vert
\begin{array}
[c]{cc}%
\hat{V}_{1}\bullet\hat{V}_{1} & \hat{V}_{2}\bullet\hat{V}_{1}\\
\hat{V}_{1}\bullet\hat{V}_{2} & \hat{V}_{2}\bullet\hat{V}_{2}%
\end{array}
\right\vert }}{\left\vert \hat{V}_{1}\right\vert \cdot\left\vert \hat{V}%
_{2}\right\vert }.
\]

\end{exercise}

\begin{exercise}
Show that we have the following equality of matrices%
\[
\left(
\begin{array}
[c]{cc}%
\hat{V}_{1}\bullet\hat{V}_{1} & \hat{V}_{2}\bullet\hat{V}_{1}\\
\hat{V}_{1}\bullet\hat{V}_{2} & \hat{V}_{2}\bullet\hat{V}_{2}%
\end{array}
\right)  =\left(
\begin{array}
[c]{c}%
\left(  \hat{V}_{1}\right) \\
\left(  \hat{V}_{2}\right)
\end{array}
\right)  \cdot\left(
\begin{array}
[c]{cc}%
\left(  \hat{V}_{1}\right)  ^{t} & \left(  \hat{V}_{2}\right)  ^{t}%
\end{array}
\right)  .
\]

\end{exercise}

Again, the significance of Exercise \ref{9} is that, to compute areas, we only
need to know how to compute dot-products--the definition of the dot-product of
the vectors completely determines the calculation of the area of the
parallelogram they generate.

We finish this section with one other related fact.

\begin{lemma}
The area of the parallelogram determined by two vectors $\hat{V}_{1}=\left(
\hat{a}_{1},\hat{b}_{1},\hat{c}_{1}\right)  $ and $\hat{V}_{2}=\left(  \hat
{a}_{2},\hat{b}_{2},\hat{c}_{2}\right)  $ emanating from the same point in
Euclidean $3$-space is given by the length of the cross-product%
\[
\hat{V}_{1}\times\hat{V}_{2}=\left(  \left\vert
\begin{array}
[c]{cc}%
\hat{b}_{1} & \hat{c}_{1}\\
\hat{b}_{2} & \hat{c}_{2}%
\end{array}
\right\vert ,\left\vert
\begin{array}
[c]{cc}%
\hat{c}_{1} & \hat{a}_{1}\\
\hat{c}_{2} & \hat{a}_{2}%
\end{array}
\right\vert ,%
\begin{array}
[c]{cc}%
\hat{a}_{1} & \hat{b}_{1}\\
\hat{a}_{2} & \hat{b}_{2}%
\end{array}
\right)  .
\]

\end{lemma}

\begin{proof}
We use some facts from linear algebra. First of all, develop the determinant%
\[
\left\vert
\begin{array}
[c]{c}%
\left(  \hat{V}_{1}\right) \\
\left(  \hat{V}_{2}\right) \\
\left(  \hat{V}_{1}\times\hat{V}_{2}\right)
\end{array}
\right\vert
\]
along the third row. Then writing out both sides of the equation%
\begin{equation}
\left\vert
\begin{array}
[c]{c}%
\left(  \hat{V}_{1}\right) \\
\left(  \hat{V}_{2}\right) \\
\left(  \hat{V}_{1}\times\hat{V}_{2}\right)
\end{array}
\right\vert =\left\vert \hat{V}_{1}\times\hat{V}_{2}\right\vert ^{2}
\label{108}%
\end{equation}
we conclude that they are equal. On the other hand,
\[
\left\vert
\begin{array}
[c]{c}%
\left(  \hat{V}_{1}\right) \\
\left(  \hat{V}_{2}\right) \\
\left(  \hat{V}_{1}\right)
\end{array}
\right\vert =0
\]
and developing the left-hand determinant along the third row, we conclude that
$\hat{V}_{1}$ is perpendicular to $\hat{V}_{1}\times\hat{V}_{2}$. Similarly
$\hat{V}_{2}$ is perpendicular to $\hat{V}_{1}\times\hat{V}_{2}$. Finally, the
absolute value of the determinant of a $3\times3$ matrix%
\[
\left\vert
\begin{array}
[c]{c}%
\left(  \hat{V}_{1}\right) \\
\left(  \hat{V}_{2}\right) \\
\left(  \hat{V}_{1}\times\hat{V}_{2}\right)
\end{array}
\right\vert
\]
is the volume of the parallelepiped determined by the row vectors of the
matrix. But that volume is%
\[
\left(  \left\vert \hat{V}_{1}\right\vert \cdot\left\vert \hat{V}%
_{2}\right\vert \cdot\mathrm{sin}\vartheta\right)  \cdot\left\vert \hat{V}%
_{1}\times\hat{V}_{2}\right\vert
\]
since $\left(  \left\vert \hat{V}_{1}\right\vert \cdot\left\vert \hat{V}%
_{2}\right\vert \cdot\mathrm{sin}\vartheta\right)  $ is the area of the base
of the parallelepiped and $\hat{V}_{1}\times\hat{V}_{2}$ is perpendicular to
both $V_{1}$ and $V_{2}$. So using $\left(  \ref{108}\right)  $%
\[
\left(  \left\vert \hat{V}_{1}\right\vert \cdot\left\vert \hat{V}%
_{2}\right\vert \cdot\mathrm{sin}\vartheta\right)  \cdot\left\vert \hat{V}%
_{1}\times\hat{V}_{2}\right\vert =\left\vert
\begin{array}
[c]{c}%
\left(  \hat{V}_{1}\right) \\
\left(  \hat{V}_{2}\right) \\
\left(  \hat{V}_{1}\times\hat{V}_{2}\right)
\end{array}
\right\vert =\left\vert \hat{V}_{1}\times\hat{V}_{2}\right\vert ^{2}.
\]
So%
\[
\left(  \left\vert \hat{V}_{1}\right\vert \cdot\left\vert \hat{V}%
_{2}\right\vert \cdot\mathrm{sin}\vartheta\right)  =\left\vert \hat{V}%
_{1}\times\hat{V}_{2}\right\vert .
\]
[DS,42-47]
\end{proof}

\newpage

\subsection{Curves in Euclidean $3$-space and vectors tangent to them}

\begin{definition}
A \textbf{smooth curve in }$\mathbf{3}$\textbf{-dimensional Euclidean space}
is given by a differentiable mapping%
\begin{gather*}
\hat{X}:\left[  b,e\right]  \rightarrow\mathbb{R}^{3}\\
t\mapsto\left(  \hat{x}\left(  t\right)  ,\hat{y}\left(  t\right)  ,\hat
{z}\left(  t\right)  \right)
\end{gather*}
\hspace{5mm} \hspace{5mm} \hspace{5mm} \hspace{5mm} from an interval $\left[
b,e\right]  $ on the real line. We shall sometimes use the notation%
\[
\left(  \hat{x}\left(  t\right)  ,\hat{y}\left(  t\right)  ,\hat{z}\left(
t\right)  \right)  =\hat{X}\left(  t\right)  .
\]
The mapping $\hat{X}\left(  t\right)  $ must have the additional property that
the tangent vector
\[
\left(  \hat{a}\left(  t\right)  ,\hat{b}\left(  t\right)  ,\hat{c}\left(
t\right)  \right)  =\left(  \frac{d\hat{x}}{dt},\frac{d\hat{y}}{dt}%
,\frac{d\hat{z}}{dt}\right)  =\frac{d\hat{X}}{dt}%
\]
\hspace{5mm} \hspace{5mm} is not the zero vector for any $t$ in $\left[
b,e\right]  $.
\end{definition}

\begin{exercise}
\label{1}a) Give two examples of smooth curves,
\begin{align*}
\hat{X}_{1}\left(  s\right)   &  =\left(  \hat{x}_{1}\left(  s\right)
,\hat{y}_{1}\left(  s\right)  ,\hat{z}_{1}\left(  s\right)  \right) \\
\hat{X}_{2}\left(  t\right)   &  =\left(  \hat{x}_{2}\left(  t\right)
,\hat{y}_{2}\left(  t\right)  ,\hat{z}_{2}\left(  t\right)  \right)
\end{align*}
neither of which is a straight line, in $3$-dimensional Euclidean space. Do
this so that the two curves pass through a common point and go in distinct
tangent directions at that point. Please choose curves so that none of the
coordinate functions of $s$ or $t$ is a constant function. [DS,71ff]

b) Compute the tangent vectors of each of the two curves at each of their points.

c) For the two curves you defined in a), what are the coordinates of the point
in Euclidean $3$-space at which the two curves intersect?

d) Use the dot product formula to compute the angle $\vartheta$ between (the
tangent vectors to) your two example curves in a) at the point at which the
curves intersect. [DS,20-21]
\end{exercise}

Sometimes displacement is measured by showing how a given point is displaced,
as in%
\[
\hat{V}=\hat{X}_{2}-\hat{X}_{1}=\left(  \hat{x}_{2}-\hat{x}_{1},\hat{y}%
_{2}-\hat{y}_{1},\hat{z}_{2}-\hat{z}_{1}\right)  ,
\]
and sometimes displacement is expressed as the instantaneous velocity of a
point moving along a curve as in%
\[
\hat{V}=\frac{d\hat{X}\left(  t\right)  }{dt}=\left(  \frac{d\hat{x}\left(
t\right)  }{dt},\frac{d\hat{y}\left(  t\right)  }{dt},\frac{d\hat{z}\left(
t\right)  }{dt}\right)  .
\]
[DS,30ff].

In matrix notation we can think of
\[
\hat{X}_{2}-\hat{X}_{1}=\left(  \hat{x}_{2}-\hat{x}_{1},\hat{y}_{2}-\hat
{y}_{1},\hat{z}_{2}-\hat{z}_{1}\right)
\]
as a $1\times3$ matrix $\left(  \hat{X}_{2}-\hat{X}_{1}\right)  $. Then we can
write the formula for the distance between two points $\hat{X}_{1}$ and
$\hat{X}_{2}$ in Euclidean $3$-space in terms of the dot-product%
\begin{equation}
d\left(  \hat{X}_{1},\hat{X}_{2}\right)  =\sqrt{\left(  \hat{X}_{2}-\hat
{X}_{1}\right)  \bullet\left(  \hat{X}_{2}-\hat{X}_{1}\right)  } \label{13}%
\end{equation}
or in terms of the matrix product%
\[
d\left(  \hat{X}_{1},\hat{X}_{2}\right)  =\sqrt{\left(  \left(  \hat{X}%
_{2}-\hat{X}_{1}\right)  \right)  \cdot\left(  \left(  \hat{X}_{2}-\hat{X}%
_{1}\right)  \right)  ^{t}}.
\]
\newpage

\subsection{Length of a smooth curve in Euclidean $3$-space}

\begin{exercise}
Compute the length of the tangent vector
\[
l(t)=\sqrt{\frac{d\hat{X}}{dt}\bullet\frac{d\hat{X}}{dt}}%
\]
to each of your two example curves in Exercise \ref{1} at each of their points.
\end{exercise}

\begin{definition}
The length $L$ of the curve $\hat{X}\left(  t\right)  $, $t\in\left[
b,e\right]  $, in Euclidean $3$-space is obtained by integrating the length of
the tangent vector to the curve, that is,%
\[
L=%
%TCIMACRO{\dint \nolimits_{b}^{e}}%
%BeginExpansion
{\displaystyle\int\nolimits_{b}^{e}}
%EndExpansion
l\left(  t\right)  dt.
\]
[DS,82] Notice that the length of any curve only depends on the definition of
the dot-product. That is, if we know the formula for the dot-product, we know
(the formula for) the length of any curve.
\end{definition}

Our first example is the path%
\begin{align}
\left(  \hat{x}\left(  t\right)  ,\hat{y}\left(  t\right)  ,\hat{z}\left(
t\right)  \right)   &  =\left(  R\cdot sin\left(  t\right)  ,0,R\cdot
cos\left(  t\right)  \right) \label{6}\\
0\,  &  \leq t\leq\pi.\nonumber
\end{align}


Notice that this path lies on the sphere of radius $R$.

\begin{exercise}
Write the formula for the tangent vector to the path $\left(  \ref{6}\right)
$ at each point using $\left(  \hat{x}\left(  t\right)  ,\hat{y}\left(
t\right)  ,\hat{z}\left(  t\right)  \right)  $-coordinates. Show that the
length of this path is $R\pi$.
\end{exercise}

\begin{exercise}
Compute the length of each of your two example curves in Exercise \ref{1}.
\end{exercise}

\begin{remark}
In this last Exercise, you may easily be confronted with an integral that you
cannot compute. For example, if your curve $\hat{X}_{1}\left(  t\right)  $
happens to describe an ellipse that is not circular, it was proved in the 19
th century that no formula involving only the standard functions from calculus
will give you the length of your path from a fixed beginning point to a
variable ending point on the ellipse. If that kind of thing occurs, go back
and change the definitions of your curves in Exercise \ref{1} until you get
two curves for which you can compute length of your path from a fixed
beginning point to a fixed ending point. [DS,81-82]
\end{remark}

We will want to reserve the notation $\left(  x,y,z\right)  $ for some new
coordinates that we will put on the `same' objects in the next section. These
new coordinates will be chosen to keep the north and south poles from going to
infinity as the radius $R$ of a sphere increases without bound. This change of
viewpoint will eventually let us go non-Euclidean or, in the language of Buzz
Lightyear \textquotedblleft to infinity and beyond." The idea will be like the
change from rectangular to polar coordinates for the plane that you
encountered in calculus, only easier. \newpage

\section{Changing coordinates}

\subsection{Bringing the North Pole of the $R$-sphere to $\left(
0,0,1\right)  $}

We are now ready to introduce a slightly different set of coordinates for
$\mathbb{R}^{3}$, three-dimensional Euclidean space. To see why we do this,
suppose we are standing at the North Pole%
\[
N=\left(  0,0,R\right)
\]
of the sphere%
\begin{equation}
\hat{x}^{2}+\hat{y}^{2}+\hat{z}^{2}=R^{2} \label{4}%
\end{equation}
of radius $R$. As $R$ increases (but we stay our same size, the sphere around
us becomes more and more like a flat, plane surface. However it can never get
completely flat because we are zooming out the positive $\hat{z}$-axis and we
would have to be `at infinity' for our surface to become exactly flat. We
remedy that unfortunate situation by considering another copy of
$\mathbb{R}^{3}$, whose coordinates we denote as $\left(  x,y,z\right)  $ and
make the following rule in order to pass between the two $\mathbb{R}^{3}$'s:%
\begin{align}
\hat{x}  &  =x\label{105}\\
\hat{y}  &  =y\nonumber\\
\hat{z}  &  =Rz.\nonumber
\end{align}
We think of the $\left(  x,y,z\right)  $-coordinates as simply being a
different set of addresses for the points in Euclidean $3$-space, for example,%
\[
\left(  x,y,z\right)  =\left(  0,0,1\right)
\]
tells me that the point in Euclidean $3$-space that I'm referring to is%
\[
\left(  \hat{x},\hat{y},\hat{z}\right)  =\left(  0,0,R\right)  =N,
\]
and the sphere of radius $R$ in Euclidean $3$-space is given by%
\begin{align*}
R^{2}  &  =\hat{x}^{2}+\hat{y}^{2}+\hat{z}^{2}\\
&  =x^{2}+y^{2}+R^{2}z^{2}%
\end{align*}
that is, by the equation%
\begin{equation}
1=\frac{1}{R^{2}}\left(  x^{2}+y^{2}\right)  +z^{2}. \label{5}%
\end{equation}
The quantity%
\[
K=\frac{1}{R^{2}}%
\]
is called the curvature of the $R$-sphere. So in $\left(  x,y,z\right)
$-coordinates, as $R$ goes to infinity, $K$ goes to $0$. The formula $\left(
\ref{5}\right)  $ is rewritten as%
\begin{equation}
1=K\left(  x^{2}+y^{2}\right)  +z^{2}, \label{7}%
\end{equation}
and so goes to%
\[
1=z^{2}%
\]
as $R$ goes to infinity. So, in the $\left(  x,y,z\right)  $-coordinates, our
`$R$-geometry' does indeed go to something finite and flat as $R$ goes to
infinity, namely the set given by the formula%
\[
z=\pm1
\]
which is in fact (two copies of) a plane!

\begin{exercise}
a) Sketch the solution set in $\left(  x,y,z\right)  $-coordinates
representing the sphere%
\[
R^{2}=\hat{x}^{2}+\hat{y}^{2}+\hat{z}^{2}=2^{2}%
\]
of radius $2$ in Euclidean three-space.

b) Sketch the solution set in the same $\left(  x,y,z\right)  $-coordinates
representing the sphere%
\[
R^{2}=\hat{x}^{2}+\hat{y}^{2}+\hat{z}^{2}=10^{2}%
\]
of radius $10$ in Euclidean three-space.

c) Sketch the solution set in the same $\left(  x,y,z\right)  $-coordinates
representing the sphere%
\[
R^{2}=\hat{x}^{2}+\hat{y}^{2}+\hat{z}^{2}=10^{-2}%
\]
of radius $10^{-1}$ in Euclidean three-space.\pagebreak
\end{exercise}

\subsection{$K$-geometry: Formulas for Euclidean lengths and angles in terms
of $\left(  x,y,z\right)  $-coordinates}

To prepare ourselves to do hyperbolic geometry, which has no satisfactory
model in Euclidean three-space, we will `practice' by doing spherical geometry
(which \textit{does} have a completely satisfactory model in Euclidean
three-space) using these `slightly strange' $\left(  x,y,z\right)
$-coordinates. Gradually throughout this course we will discover that the same
rules that govern spherical geometry, expressed in $\left(  x,y,z\right)
$-coordinates, also govern flat and hyperbolic geometry! In all three cases,
the space in $\left(  x,y,z\right)  $-coordinates that we will study is%
\begin{equation}
1=K\left(  x^{2}+y^{2}\right)  +z^{2}. \label{11}%
\end{equation}
If $K>0$, the geometry we will be studying is the geometry of the the
Euclidean sphere of radius%
\[
R=\frac{1}{\sqrt{K}}.
\]
If $K=0$ we will be studying flat (plane) geometry. If $K<0$, we will be
studying hyperbolic geometry. The number $K$, in all cases, is called the
\textit{curvature} of the geometry.

In short, we want to use $\left(  x,y,z\right)  $-coordinates to compute with,
but we want lengths and angles to be the usual Euclidean ones in $\left(
\hat{x},\hat{y},\hat{z}\right)  $-coordinates.

\begin{exercise}
a) Suppose we have functions%
\[
\left(  \hat{x}\left(  x,y,z\right)  ,\hat{y}\left(  x,y,z\right)  ,\hat
{z}\left(  x,y,z\right)  \right)
\]
where%
\begin{align*}
x  &  =f\left(  t\right) \\
y  &  =g\left(  t\right) \\
z  &  =h\left(  t\right)  .
\end{align*}
State the Chain Rule for%
\begin{align*}
\frac{d\hat{x}}{dt}  &  =\\
\frac{d\hat{y}}{dt}  &  =\\
\frac{d\hat{z}}{dt}  &  =.
\end{align*}


b) Rewrite the Chain Rule in matrix notation%
\[
\left(
\begin{array}
[c]{ccc}%
\frac{d\hat{x}}{dt} & \frac{d\hat{y}}{dt} & \frac{d\hat{z}}{dt}%
\end{array}
\right)  =\left(
\begin{array}
[c]{ccc}%
\frac{dx}{dt} & \frac{dy}{dt} & \frac{dz}{dt}%
\end{array}
\right)  \cdot\left(
\begin{array}
[c]{ccc}
&  & \\
&  & \\
&  &
\end{array}
\right)  .
\]

\end{exercise}

\begin{exercise}
\label{8} Recalling that $R$ is a positive constant, use $\left(
\ref{105}\right)  $ and the Chain Rule to show that, for any path $\hat
{X}\left(  t\right)  =\left(  \hat{x}\left(  t\right)  ,\hat{y}\left(
t\right)  ,\hat{z}\left(  t\right)  \right)  $ in Euclidean $3$-space,%
\begin{align*}
\frac{d\hat{x}}{dt}  &  =\frac{dx}{dt}\\
\frac{d\hat{y}}{dt}  &  =\frac{dy}{dt}\\
\frac{d\hat{z}}{dt}  &  =R\frac{dz}{dt}.
\end{align*}

\end{exercise}

\begin{exercise}
Use matrix multiplication [DS,307] and Exercise \ref{8} to show that%
\begin{align*}
\frac{d\hat{X}\left(  t\right)  }{dt}  &  =\left(  \frac{dX\left(  t\right)
}{dt}\right)  \left(
\begin{array}
[c]{ccc}%
1 & 0 & 0\\
0 & 1 & 0\\
0 & 0 & R
\end{array}
\right) \\
\frac{dX\left(  t\right)  }{dt}  &  =\left(  \frac{d\hat{X}\left(  t\right)
}{dt}\right)  \left(
\begin{array}
[c]{ccc}%
1 & 0 & 0\\
0 & 1 & 0\\
0 & 0 & R^{-1}%
\end{array}
\right)  .
\end{align*}

\end{exercise}

\textit{NB}: This last computation shows that, if
\begin{align*}
\hat{V}_{1}  &  =\left(  \hat{a}_{1},\hat{b}_{1},\hat{c}_{1}\right) \\
\hat{V}_{2}  &  =\left(  \hat{a}_{2},\hat{b}_{2},\hat{c}_{2}\right)
\end{align*}
are tangent vectors in $\left(  \hat{x},\hat{y},\hat{z}\right)  $-coordinates
and%
\begin{align*}
V_{1}  &  =\left(  a_{1},b_{1},c_{1}\right) \\
V_{2}  &  =\left(  a_{2},b_{2},c_{2}\right)
\end{align*}
are their transformations into $\left(  x,y,z\right)  $-coordinates, then%
\begin{align*}
\hat{V}_{1}  &  =\left(  V_{1}\right)  \left(
\begin{array}
[c]{ccc}%
1 & 0 & 0\\
0 & 1 & 0\\
0 & 0 & R
\end{array}
\right) \\
\hat{V}_{2}  &  =\left(  V_{2}\right)  \left(
\begin{array}
[c]{ccc}%
1 & 0 & 0\\
0 & 1 & 0\\
0 & 0 & R
\end{array}
\right)
\end{align*}
and%
\begin{align*}
\hat{V}_{1}\bullet\hat{V}_{2}  &  =\left(  \hat{V}_{1}\right)  \cdot\left(
\hat{V}_{2}\right)  ^{t}\\
&  =\left(  V_{1}\right)  \left(
\begin{array}
[c]{ccc}%
1 & 0 & 0\\
0 & 1 & 0\\
0 & 0 & R
\end{array}
\right)  \left(  \left(  V_{2}\right)  \left(
\begin{array}
[c]{ccc}%
1 & 0 & 0\\
0 & 1 & 0\\
0 & 0 & R
\end{array}
\right)  \right)  ^{t}\\
&  =\left(  V_{1}\right)  \left(
\begin{array}
[c]{ccc}%
1 & 0 & 0\\
0 & 1 & 0\\
0 & 0 & R
\end{array}
\right)  \left(
\begin{array}
[c]{ccc}%
1 & 0 & 0\\
0 & 1 & 0\\
0 & 0 & R
\end{array}
\right)  \left(  V_{2}\right)  ^{t}\\
&  =\left(  V_{1}\right)  \left(
\begin{array}
[c]{ccc}%
1 & 0 & 0\\
0 & 1 & 0\\
0 & 0 & K^{-1}%
\end{array}
\right)  \left(  V_{2}\right)  ^{t}.
\end{align*}
This last computation says that we can compute the Euclidean dot $\hat{V}%
_{1}\bullet\hat{V}_{2}$ without ever referring to Euclidean coordinates. We
incorporate that fact into the following definition.

\begin{definition}
\textquotedblleft$K$-dot-product\textquotedblright\ of vectors:%
\begin{align}
V_{1}\bullet_{K}V_{2}  &  =\left(  V_{1}\right)  \left(
\begin{array}
[c]{ccc}%
1 & 0 & 0\\
0 & 1 & 0\\
0 & 0 & K^{-1}%
\end{array}
\right)  \left(  V_{2}\right)  ^{t}\label{10}\\
&  =\left(
\begin{array}
[c]{ccc}%
a_{1} & b_{1} & c_{1}%
\end{array}
\right)  \left(
\begin{array}
[c]{ccc}%
1 & 0 & 0\\
0 & 1 & 0\\
0 & 0 & K^{-1}%
\end{array}
\right)  \left(
\begin{array}
[c]{c}%
a_{2}\\
b_{2}\\
c_{2}%
\end{array}
\right)  .\nonumber
\end{align}

\end{definition}

So, suppose we have a curve on the $R$-sphere in Euclidean $3$-space but it is
given to us in $X\left(  t\right)  =\left(  x\left(  t\right)  ,y\left(
t\right)  ,z\left(  t\right)  \right)  $-coordinates. Then the length of that
curve in Euclidean $3$-space is%
\[%
%TCIMACRO{\dint \nolimits_{b}^{e}}%
%BeginExpansion
{\displaystyle\int\nolimits_{b}^{e}}
%EndExpansion
\sqrt{\frac{dX}{dt}\bullet_{K}\frac{dX}{dt}}dt.
\]


\begin{exercise}
\label{222}Use Exercise \ref{9} to show that, if we have any two vectors in
Euclidean three-space that are tangent to the $R$-sphere at some point on it,
but the two vectors are given to us in $\left(  x,y,z\right)  $-coordinates as%
\begin{align*}
V_{1}  &  =\left(  a_{1},b_{1},c_{1}\right) \\
V_{2}  &  =\left(  a_{2},b_{2},c_{2}\right)  ,
\end{align*}
then the area of the parallelogram spanned by those two vectors in Euclidean
$3$-space is%
\[
\sqrt{\left\vert
\begin{array}
[c]{cc}%
V_{1}\bullet_{K}V_{1} & V_{2}\bullet_{K}V_{1}\\
V_{1}\bullet_{K}V_{2} & V_{2}\bullet_{K}V_{2}%
\end{array}
\right\vert }=\sqrt{\left\vert \left(
\begin{array}
[c]{c}%
\left(  V_{1}\right) \\
\left(  V_{2}\right)
\end{array}
\right)  \cdot\left(
\begin{array}
[c]{ccc}%
1 & 0 & 0\\
0 & 1 & 0\\
0 & 0 & K^{-1}%
\end{array}
\right)  \cdot\left(
\begin{array}
[c]{cc}%
\left(  V_{1}\right)  ^{t} & \left(  V_{2}\right)  ^{t}%
\end{array}
\right)  \right\vert }.
\]

\end{exercise}

\textit{Moral of the story:} The dot-product rules! That is, if you know the
dot-product you know everything there is to know about a geometry, lengths,
areas, angles, everything. And the set $\left(  \ref{11}\right)  $ continues
to make sense even when $K$ is negative. And as we will see later on, the
definition of the $K$-dot product also makes sense for tangent vectors to that
set when $K$ is negative.The geometry we get, when the constant $K$ is chosen
to be negative is called a hyperbolic geometry. The geometry we get, when the
constant $K$ is just chosen to be non-zero is called a non-euclidean geometry.
In fact all the non-euclidean $2$-dimensional geometries are either spherical
or hyperbolic.

\textit{Coming attractions:} A big idea is that in hyperbolic geometry
$K^{-1}$ in $\left(  \ref{10}\right)  $ becomes negative, so that the third
coordinate of velocity, that is, the $c$-direction, actually
\textit{contracts} lengths. It was the understanding of this mysterious fact
that allowed Einstein to discover (special) relativity.\pagebreak

\section{Congruences, that is, rigid motions}

\subsection{Transformations of Euclidean $3$-space}

Consider the following mapping of Euclidean $3$-space to itself:%
\begin{equation}
\left(
\begin{array}
[c]{ccc}%
\underline{\hat{x}} & \underline{\hat{y}} & \underline{\hat{z}}%
\end{array}
\right)  =\left(
\begin{array}
[c]{ccc}%
\hat{x} & \hat{y} & \hat{z}%
\end{array}
\right)  \cdot\hat{M} \label{12}%
\end{equation}
where $\hat{M}$ is an invertible $3\times3$ matrix. Then by matrix
multiplication%
\[
\left(
\begin{array}
[c]{ccc}%
\hat{x} & \hat{y} & \hat{z}%
\end{array}
\right)  =\left(
\begin{array}
[c]{ccc}%
\underline{\hat{x}} & \underline{\hat{y}} & \underline{\hat{z}}%
\end{array}
\right)  \cdot\hat{M}^{-1}%
\]
so that this mapping is $1-1$ and onto.

\begin{definition}
The mapping of Euclidean $3$-space to itself given by the rule $\left(
\ref{12}\right)  $ is called a rigid motion if the distance between any two
points in Euclidean $3$-space is left unchanged by the mapping, that is, for
any two points, $\hat{X}_{1}$ and $\hat{X}_{2}$ in Euclidean $3$-space%
\[
d\left(  \left(  \hat{X}_{1}\right)  \cdot\hat{M},\left(  \hat{X}_{2}\right)
\cdot\hat{M}\right)  =d\left(  \hat{X}_{1},\hat{X}_{2}\right)  .
\]

\end{definition}

We saw in formula $\left(  \ref{13}\right)  $ that the square of the distance
between $\hat{X}_{1}$ and $\hat{X}_{1}$ is just the dot-product of the vector%
\[
\hat{V}=\hat{X}_{2}-\hat{X}_{1}%
\]
with itself. So the transformation given by the matrix $\hat{M}$ will leave
distances unchanged if and only if, for all vectors $\hat{V}$,%
\[
\left(  \left(  \hat{V}\right)  \cdot\hat{M}\right)  \bullet\left(  \left(
\hat{V}\right)  \cdot\hat{M}\right)  =\hat{V}\bullet\hat{V}%
\]
that is%
\[
\left(  \left(  \hat{V}\right)  \cdot\hat{M}\right)  \cdot\left(  \left(
\hat{V}\right)  \cdot\hat{M}\right)  ^{t}=\left(  \hat{V}\right)  \cdot\left(
\hat{V}\right)  ^{t}.
\]
We can rewrite this requirement as%
\begin{equation}
\left(  \hat{V}\right)  \cdot\hat{M}\cdot\hat{M}^{t}\cdot\left(  \hat
{V}\right)  ^{t}=\left(  \hat{V}\right)  \cdot\left(  \hat{V}\right)  ^{t}
\label{116}%
\end{equation}
for all vectors $\hat{V}$. Condition $\left(  \ref{116}\right)  $ is certainly
satisfied for all vectors $\hat{V}$ if
\begin{equation}
\hat{M}\cdot\hat{M}^{t}=I=\left(
\begin{array}
[c]{ccc}%
1 & 0 & 0\\
0 & 1 & 0\\
0 & 0 & 1
\end{array}
\right)  . \label{16}%
\end{equation}


\begin{exercise}
Suppose $\hat{M}$ is such that%
\[
\hat{M}\cdot\hat{M}^{t}=\left(
\begin{array}
[c]{ccc}%
1 & 2 & 0\\
2 & 1 & 0\\
0 & 0 & 1
\end{array}
\right)  .
\]
Find a vector $\hat{V}=\left(  \hat{a},\hat{b},\hat{c}\right)  $ such that%
\[
\left(  \hat{V}\right)  \cdot\hat{M}\cdot\hat{M}^{t}\cdot\left(  \hat
{V}\right)  ^{t}\neq\left(  \hat{V}\right)  \cdot\left(  \hat{V}\right)
^{t}.
\]

\end{exercise}

In fact, reasoning as in this last Exercise, one can show that, if a matrix
$\hat{M}$ satisfies the condition $\left(  \ref{116}\right)  $ for
\textit{all} vectors $\hat{V}$, then the matrix $\hat{M}$ also satisfies
$\left(  \ref{16}\right)  $. A matrix $\hat{M}$ satisfying $\left(
\ref{16}\right)  $ is called an \textit{orthogonal matrix}. [DS,316-321]

\begin{exercise}
\label{14} Show that the matrix%
\[
\hat{M}=\left(
\begin{array}
[c]{ccc}%
\mathrm{cos}\theta & \mathrm{sin}\theta & 0\\
\mathrm{-sin}\theta & \mathrm{cos}\theta & 0\\
0 & 0 & 1
\end{array}
\right)
\]
is orthogonal. Can you describe geometrically what this rigid motion is doing
to the points in Euclidean $3$-space?
\end{exercise}

\begin{exercise}
Show that the matrix%
\[
\hat{M}=\left(
\begin{array}
[c]{ccc}%
\mathrm{cos}\psi & \mathrm{0} & \mathrm{sin}\psi\\
\mathrm{0} & \mathrm{1} & 0\\
\mathrm{-sin}\psi & 0 & \mathrm{cos}\psi
\end{array}
\right)
\]
is orthogonal. Can you describe geometrically what this rigid motion is doing
to the points in Euclidean $3$-space?
\end{exercise}

For any curve%
\[
\hat{X}\left(  t\right)  =\left(  \hat{x}\left(  t\right)  ,\hat{y}\left(
t\right)  ,\hat{z}\left(  t\right)  \right)  ,\;b\leq t\leq e,
\]
its length is%
\[%
%TCIMACRO{\dint \nolimits_{b}^{e}}%
%BeginExpansion
{\displaystyle\int\nolimits_{b}^{e}}
%EndExpansion
\sqrt{\left(  \frac{d\hat{X}}{dt}\right)  \cdot\left(  \frac{d\hat{X}}%
{dt}\right)  ^{t}}dt.
\]
Suppose now that the curve is moved by a transformation given by an orthogonal
matrix $\hat{M}$. After it is moved, its length is given by%
\[%
%TCIMACRO{\dint \nolimits_{b}^{e}}%
%BeginExpansion
{\displaystyle\int\nolimits_{b}^{e}}
%EndExpansion
\sqrt{\left(  \frac{d\left(  \hat{X}\cdot\hat{M}\right)  }{dt}\right)
\cdot\left(  \frac{d\left(  \hat{X}\cdot\hat{M}\right)  }{dt}\right)  ^{t}%
}dt.
\]
But%
\begin{align*}%
%TCIMACRO{\dint \nolimits_{b}^{e}}%
%BeginExpansion
{\displaystyle\int\nolimits_{b}^{e}}
%EndExpansion
\sqrt{\left(  \frac{d\left(  \hat{X}\cdot\hat{M}\right)  }{dt}\right)
\cdot\left(  \frac{d\left(  \hat{X}\cdot\hat{M}\right)  }{dt}\right)  ^{t}}dt
&  =%
%TCIMACRO{\dint \nolimits_{b}^{e}}%
%BeginExpansion
{\displaystyle\int\nolimits_{b}^{e}}
%EndExpansion
\sqrt{\left(  \frac{d\hat{X}}{dt}\cdot\hat{M}\right)  \cdot\left(  \frac
{d\hat{X}}{dt}\cdot\hat{M}\right)  ^{t}}dt\\
&  =%
%TCIMACRO{\dint \nolimits_{b}^{e}}%
%BeginExpansion
{\displaystyle\int\nolimits_{b}^{e}}
%EndExpansion
\sqrt{\left(  \frac{d\hat{X}}{dt}\cdot\hat{M}\right)  \cdot\left(  \hat{M}%
^{t}\cdot\frac{d\hat{X}}{dt}^{t}\right)  }dt\\
&  =%
%TCIMACRO{\dint \nolimits_{b}^{e}}%
%BeginExpansion
{\displaystyle\int\nolimits_{b}^{e}}
%EndExpansion
\sqrt{\left(  \frac{d\hat{X}}{dt}\right)  \cdot\left(  \frac{d\hat{X}}%
{dt}\right)  ^{t}}dt.
\end{align*}


\begin{corollary}
If a curve
\[
\hat{X}\left(  t\right)  =\left(  \hat{x}\left(  t\right)  ,\hat{y}\left(
t\right)  ,\hat{z}\left(  t\right)  \right)  ,\;b\leq t\leq e
\]
is moved by a transformation given by an orthogonal matrix $\hat{M}$, its
length is unchanged.\pagebreak
\end{corollary}

\subsection{Formula in $\left(  x,y,z\right)  $-coordinates for rigid motions
of Euclidean $3$-space}

We now wish to figure out how to write the transformation $\left(
\ref{12}\right)  $ in $\left(  x,y,z\right)  $-coordinates. This is a simple
substitution problem:%
\begin{equation}
\left(
\begin{array}
[c]{ccc}%
\underline{\hat{x}} & \underline{\hat{y}} & \underline{\hat{z}}%
\end{array}
\right)  =\left(
\begin{array}
[c]{ccc}%
\hat{x} & \hat{y} & \hat{z}%
\end{array}
\right)  \cdot\hat{M} \label{100}%
\end{equation}%
\begin{gather*}
\left(
\begin{array}
[c]{ccc}%
\hat{x} & \hat{y} & \hat{z}%
\end{array}
\right)  =\left(
\begin{array}
[c]{ccc}%
x & y & Rz
\end{array}
\right)  =\left(
\begin{array}
[c]{ccc}%
x & y & z
\end{array}
\right)  \cdot\left(
\begin{array}
[c]{ccc}%
1 & 0 & 0\\
0 & 1 & 0\\
0 & 0 & R
\end{array}
\right) \\
\left(
\begin{array}
[c]{ccc}%
\underline{\hat{x}} & \underline{\hat{y}} & \underline{\hat{z}}%
\end{array}
\right)  =\left(
\begin{array}
[c]{ccc}%
\underline{x} & \underline{y} & R\underline{z}%
\end{array}
\right)  =\left(
\begin{array}
[c]{ccc}%
\underline{x} & \underline{y} & \underline{z}%
\end{array}
\right)  \cdot\left(
\begin{array}
[c]{ccc}%
1 & 0 & 0\\
0 & 1 & 0\\
0 & 0 & R
\end{array}
\right)
\end{gather*}
So we have the diagram%
\[%
\begin{array}
[c]{ccc}%
\left(
\begin{array}
[c]{ccc}%
\hat{x} & \hat{y} & \hat{z}%
\end{array}
\right)  \in\mathbb{R}^{3} & \overset{\cdot\left(
\begin{array}
[c]{ccc}%
1 & 0 & 0\\
0 & 1 & 0\\
0 & 0 & R
\end{array}
\right)  }{\longleftarrow} & \left(
\begin{array}
[c]{ccc}%
x & y & z
\end{array}
\right)  \in\mathbb{R}^{3}\\
\downarrow\cdot\hat{M} &  & \downarrow\cdot M=?\\
\left(
\begin{array}
[c]{ccc}%
\underline{\hat{x}} & \underline{\hat{y}} & \underline{\hat{z}}%
\end{array}
\right)  & \overset{\cdot\left(
\begin{array}
[c]{ccc}%
1 & 0 & 0\\
0 & 1 & 0\\
0 & 0 & R^{-1}%
\end{array}
\right)  }{\longrightarrow} & \left(
\begin{array}
[c]{ccc}%
\underline{x} & \underline{y} & \underline{z}%
\end{array}
\right)
\end{array}
\]


\begin{exercise}
Starting from the equality $\left(  \ref{100}\right)  $ describing the
transformation in Euclidean coordinates, explain why%
\[
\left(
\begin{array}
[c]{ccc}%
\underline{x} & \underline{y} & \underline{z}%
\end{array}
\right)  =\left(
\begin{array}
[c]{ccc}%
x & y & z
\end{array}
\right)  \cdot\left(
\begin{array}
[c]{ccc}%
1 & 0 & 0\\
0 & 1 & 0\\
0 & 0 & R
\end{array}
\right)  \cdot\hat{M}\cdot\left(
\begin{array}
[c]{ccc}%
1 & 0 & 0\\
0 & 1 & 0\\
0 & 0 & R^{-1}%
\end{array}
\right)  .
\]

\end{exercise}

So, if we let%
\[
M=\left(
\begin{array}
[c]{ccc}%
1 & 0 & 0\\
0 & 1 & 0\\
0 & 0 & R
\end{array}
\right)  \cdot\hat{M}\cdot\left(
\begin{array}
[c]{ccc}%
1 & 0 & 0\\
0 & 1 & 0\\
0 & 0 & R^{-1}%
\end{array}
\right)  ,
\]
then%
\begin{equation}
\left(
\begin{array}
[c]{ccc}%
\underline{x} & \underline{y} & \underline{z}%
\end{array}
\right)  =\left(
\begin{array}
[c]{ccc}%
x & y & z
\end{array}
\right)  \cdot M, \label{15}%
\end{equation}
that is $M$ is the matrix that gives the transformation $\left(
\ref{12}\right)  $ in $\left(  x,y,z\right)  $-coordinates.

So how would we check whether a transformation given in $\left(  x,y,z\right)
$-coordinates by a matrix $M$ preserves distances in Euclidean $3$-space?
Again, starting from $\left(  \ref{16}\right)  $ this is just a substitution
problem:%
\begin{gather*}
\hat{M}\cdot\hat{M}^{t}=I\\
M=\left(
\begin{array}
[c]{ccc}%
1 & 0 & 0\\
0 & 1 & 0\\
0 & 0 & R
\end{array}
\right)  \cdot\hat{M}\cdot\left(
\begin{array}
[c]{ccc}%
1 & 0 & 0\\
0 & 1 & 0\\
0 & 0 & R^{-1}%
\end{array}
\right) \\
\left(
\begin{array}
[c]{ccc}%
1 & 0 & 0\\
0 & 1 & 0\\
0 & 0 & R^{-1}%
\end{array}
\right)  \cdot M\cdot\left(
\begin{array}
[c]{ccc}%
1 & 0 & 0\\
0 & 1 & 0\\
0 & 0 & R
\end{array}
\right)  =\hat{M}%
\end{gather*}


\begin{exercise}
Finish the matrix algebra computations just above to show that the condition
that a transformation $M$ in $\left(  x,y,z\right)  $-coordinates preserves
distances in Euclidean $3$-space is the condition that%
\begin{equation}
M\cdot\left(
\begin{array}
[c]{ccc}%
1 & 0 & 0\\
0 & 1 & 0\\
0 & 0 & K^{-1}%
\end{array}
\right)  \cdot M^{t}=\left(
\begin{array}
[c]{ccc}%
1 & 0 & 0\\
0 & 1 & 0\\
0 & 0 & K^{-1}%
\end{array}
\right)  . \label{17}%
\end{equation}

\end{exercise}

This is the condition (in $\left(  x,y,z\right)  $-coordinates) which affirms
that the transformation which takes the path $\left(  x(t),y(t),z(t)\right)  $
to the path $\left(  x(t),y(t),z(t)\right)  \cdot M$ preserves lengths of
tangent vectors at corresponding points. Therefore, by integrating, the
(total) length of the curve $\left\{  \left(  x(t),y(t),z(t)\right)  \cdot
M:b\leq t\leq e\right\}  $ is the same as the total length of the curve
$\left\{  \left(  x(t),y(t),z(t)\right)  :b\leq t\leq e\right\}  $.

\begin{exercise}
Check that $\left(  \ref{17}\right)  $ is the correct condition by showing
that any $3\times3$ matrix $M$ that satisfies $\left(  \ref{17}\right)  $ also
satisfies%
\[
\left(  \left(  V\right)  \cdot M\right)  \bullet_{K}\left(  \left(  V\right)
\cdot M\right)  =V\bullet_{K}V
\]
where%
\[
V=X_{2}-X_{1}.
\]
That is, the transformation given in $\left(  x,y,z\right)  $-coordinates by a
matrix $M$ that satisfies $\left(  \ref{17}\right)  $ preserves the $K$-dot product.
\end{exercise}

\pagebreak

\part{$K$-geometry\label{III}}

\section{Uniform coordinates for the two-dimensional geometries}

\subsection{The two-dimensional geometries in $\left(  x,y,z\right)
$-coordinates}

At the beginning of this book we changed the coordinates on Euclidean
three-space so that the equations for the sphere of radius $R$ became%
\begin{equation}
1=K\left(  x^{2}+y^{2}\right)  +z^{2} \label{29}%
\end{equation}
where%
\[
K=\frac{1}{R^{2}}.
\]
In these new $\left(  x,y,z\right)  $-coordinates, the set of points $\left(
x,y,z\right)  $ satisfying $\left(  \ref{29}\right)  $ when $K>0$ matched up
in $1-1$ fashion with the $R$-sphere%
\[
\left\{  \left(  \hat{x},\hat{y},\hat{z}\right)  \in\mathbb{R}^{3}:\hat{x}%
^{2}+\hat{y}^{2}+\hat{z}^{2}=R^{2}\right\}
\]
in the usual coordinates $\left(  \hat{x},\hat{y},\hat{z}\right)  $ of
$3$-dimensional Euclidean space.

If we have a curve $\left(  \hat{x}\left(  t\right)  ,\hat{y}\left(  t\right)
,\hat{z}\left(  t\right)  \right)  $ lying in the $R$-sphere in Euclidean
space, then for all $t\in\left[  b,e\right]  $,%
\[
\hat{x}\left(  t\right)  ^{2}+\hat{y}\left(  t\right)  ^{2}+\hat{z}\left(
t\right)  ^{2}=R^{2}.
\]
Differentiating both sides with respect to $t$ we obtain%
\[
2\hat{x}\left(  t\right)  \frac{d\hat{x}}{dt}+2\hat{y}\left(  t\right)
\frac{d\hat{y}}{dt}+2\hat{z}\left(  t\right)  \frac{d\hat{z}}{dt}=0
\]
which we can rewrite as%
\[
\left(  \hat{x}\left(  t\right)  ,\hat{y}\left(  t\right)  ,\hat{z}\left(
t\right)  \right)  \bullet\frac{d\hat{X}\left(  t\right)  }{dt}=0.
\]
[DS,105ff] Said another way, vectors $\hat{V}$ are tangent to the $R$-sphere
at $\hat{X}\left(  t\right)  $ if and only if%
\[
\hat{X}\left(  t\right)  \bullet\hat{V}=0.
\]
[DS,106,109]

Repeating the same calculation in $\left(  x,y,z\right)  $-coordinates, the
corresponding curve $\left(  x\left(  t\right)  ,y\left(  t\right)  ,z\left(
t\right)  \right)  $ lies in the set $\left(  \ref{29}\right)  $ so that%
\begin{align*}
1  &  =K\left(  x\left(  t\right)  ^{2}+y\left(  t\right)  ^{2}\right)
+z\left(  t\right)  ^{2}\\
0  &  =K\left(  2x\left(  t\right)  \frac{dx}{dt}+2y\left(  t\right)
\frac{dy}{dt}\right)  +2z\left(  t\right)  \frac{dz}{dt}.
\end{align*}
That is, a vector $V=\left(  a,b,c\right)  $ is tangent to the set $\left(
\ref{29}\right)  $ if and only if%
\begin{equation}
\left(  x\left(  t\right)  ,y\left(  t\right)  ,z\left(  t\right)  \right)
\cdot\left(
\begin{array}
[c]{ccc}%
2K & 0 & 0\\
0 & 2K & 0\\
0 & 0 & 2
\end{array}
\right)  \cdot V^{t}=0. \label{56}%
\end{equation}


\begin{exercise}
\label{82}For $K\neq0$, show that the condition $\left(  \ref{56}\right)  $ on
$V$ is exactly the same condition as%
\[
\left(  x\left(  t\right)  ,y\left(  t\right)  ,z\left(  t\right)  \right)
\bullet_{K}V=0.
\]

\end{exercise}

We will call the set of $\left(  x,y,z\right)  $ satisfying $\left(
\ref{29}\right)  $ $K$-geometry. Its tangent vectors at a point $\left(
x,y,z\right)  $ in the set are the vectors $V=\left(  a,b,c\right)  $ such
that%
\[
\left(  x,y,z\right)  \bullet_{K}V=0.
\]


If you get nervous using these weird coordinates to compute things that are
clearer in $\left(  \hat{x},\hat{y},\hat{z}\right)  $-coordinates, just go
through each construction in Part \ref{III} in the special case $K=1$ first.
In that special case
\[
\left(  x,y,z\right)  =\left(  \hat{x},\hat{y},\hat{z}\right)
\]
and your calculations reduce to the usual ones on the unit sphere in ordinary
Euclidean $3$-space.\pagebreak

\subsection{Rigid motions in $\left(  x,y,z\right)  $-coordinates}

We are now going to study $K$-geometry using only $\left(  x,y,z\right)
$-coordinates. If we have a curve $X\left(  t\right)  =\left(  x\left(
t\right)  ,y\left(  t\right)  ,z\left(  t\right)  \right)  $ on the surface
given in $K$-coordinates as%
\begin{equation}
1=K\left(  x^{2}+y^{2}\right)  +z^{2}, \label{86}%
\end{equation}
we have seen that we measure its length $L$ by the formula%
\begin{equation}
L=%
%TCIMACRO{\dint \nolimits_{b}^{e}}%
%BeginExpansion
{\displaystyle\int\nolimits_{b}^{e}}
%EndExpansion
l\left(  t\right)  dt \label{60}%
\end{equation}
where
\begin{equation}
l\left(  t\right)  ^{2}=\frac{dX}{dt}\bullet_{K}\frac{dX}{dt} \label{61}%
\end{equation}
and that we measure angles $\theta$ between tangent vectors $V_{1}$ and
$V_{2}$ at a point on the surface by the formula%
\[
\theta=\mathrm{arccos}\left(  \frac{V_{1}\bullet_{K}V_{2}}{\left\vert
V_{1}\right\vert _{K}\text{\textperiodcentered}\left\vert V_{2}\right\vert
_{K}}\right)
\]
where%
\[
\left\vert V\right\vert _{K}^{2}=V\bullet_{K}V.
\]


We now want to explore the condition that a transformation%
\[
\left(  \underline{x},\underline{y},\underline{z}\right)  =\left(
x,y,z\right)  \cdot M
\]
take the surface $\left(  \ref{86}\right)  $ to itself andpreserve the length
of any curve $\left(  x\left(  t\right)  ,y\left(  t\right)  ,z\left(
t\right)  \right)  $ lying on the surface. Rewriting the transformation as%
\[
\left(  \underline{X}\right)  =\left(  X\right)  \cdot M
\]
the formulas $\left(  \ref{60}\right)  $ and $\left(  \ref{61}\right)  $ show
that all we have to worry about is that%
\[
\frac{d\underline{X}}{dt}\bullet_{K}\frac{d\underline{X}}{dt}=\frac{dX}%
{dt}\bullet_{K}\frac{dX}{dt}%
\]
for all values $t$ of the parameter of the curve. But%
\[
\left(  \frac{d\underline{X}}{dt}\right)  =\left(  \frac{dX}{dt}\right)  \cdot
M
\]
by the product rule since $M$ is a constant matrix. So the transformation
given by the matrix $\hat{M}$ will preserve the length of any path and will
preserve the measure of any angle if%
\begin{equation}
\left(  \frac{dX}{dt}\right)  \cdot M\cdot\left(
\begin{array}
[c]{ccc}%
1 & 0 & 0\\
0 & 1 & 0\\
0 & 0 & K^{-1}%
\end{array}
\right)  \cdot\left(  \left(  \frac{dX}{dt}\right)  \cdot M\right)
^{t}=\left(  \frac{dX}{dt}\right)  \cdot\left(
\begin{array}
[c]{ccc}%
1 & 0 & 0\\
0 & 1 & 0\\
0 & 0 & K^{-1}%
\end{array}
\right)  \cdot\left(  \frac{dX}{dt}\right)  ^{t}. \label{89}%
\end{equation}


\begin{exercise}
\label{87}a) Show that this last equality is always true if%
\begin{equation}
M\cdot\left(
\begin{array}
[c]{ccc}%
1 & 0 & 0\\
0 & 1 & 0\\
0 & 0 & K^{-1}%
\end{array}
\right)  \cdot M^{t}=\left(
\begin{array}
[c]{ccc}%
1 & 0 & 0\\
0 & 1 & 0\\
0 & 0 & K^{-1}%
\end{array}
\right)  . \label{new89}%
\end{equation}


b) Show that, if $M$ satisfies the identity \ref{new89},then the
transformation $\left(  \underline{x},\underline{y},\underline{z}\right)
=\left(  x,y,z\right)  \cdot M$ takes the set of points $\left(  x,y,z\right)
$ such that%
\[
1=K\left(  x^{2}+y^{2}\right)  +z^{2},
\]
to the set of points $\left(  \underline{x},\underline{y},\underline
{z}\right)  $ such that%
\[
1=K\left(  \underline{x}^{2}+\underline{y}^{2}\right)  +\underline{z}^{2}.
\]
That is, $M$ gives a $1-1$, onto mapping of $K$-geometry to itself.

Hint: For $K\neq0$, write the equation $1=K\left(  \underline{x}%
^{2}+\underline{y}^{2}\right)  +\underline{z}^{2}$ in matrix notation as%
\[
\left(
\begin{array}
[c]{ccc}%
\underline{x} & \underline{y} & \underline{z}%
\end{array}
\right)  \cdot\left(
\begin{array}
[c]{ccc}%
1 & 0 & 0\\
0 & 1 & 0\\
0 & 0 & K^{-1}%
\end{array}
\right)  \cdot\left(
\begin{array}
[c]{c}%
\underline{x}\\
\underline{y}\\
\underline{z}%
\end{array}
\right)  =\frac{1}{K}.
\]

\end{exercise}

\begin{definition}
\label{88}A $3\times3$ matrix $M$ is called $K$-orthogonal if
\[
M\cdot\left(
\begin{array}
[c]{ccc}%
1 & 0 & 0\\
0 & 1 & 0\\
0 & 0 & K^{-1}%
\end{array}
\right)  \cdot M^{t}=\left(
\begin{array}
[c]{ccc}%
1 & 0 & 0\\
0 & 1 & 0\\
0 & 0 & K^{-1}%
\end{array}
\right)  .
\]

\end{definition}

\begin{definition}
A $K$-distance-preserving transformation of $K$-geometry is called a
$K$\textbf{-rigid motion} or a $K$\textbf{-congruence}.
\end{definition}

So $K$-orthogonal matrices give $K$-rigid motions.

\begin{exercise}
For $K\neq0$, show that the set of $K$-orthogonal matrices $M$ form a group.
That is, show that

a) the product of two $K$-orthogonal matrices is $K$-orthogonal,

b) the identity matrix is $K$-orthogonal,

c) the inverse matrix $M^{-1}$ of a $K$-orthogonal matrix $M$ is $K$-orthogonal.

Hint: Write%
\[
M\cdot M^{-1}=I=M\cdot\left(
\begin{array}
[c]{ccc}%
1 & 0 & 0\\
0 & 1 & 0\\
0 & 0 & K^{-1}%
\end{array}
\right)  \cdot M^{t}\cdot\left(
\begin{array}
[c]{ccc}%
1 & 0 & 0\\
0 & 1 & 0\\
0 & 0 & K
\end{array}
\right)
\]
and use matrix multiplication to reduce to showing that
\[
\left(
\begin{array}
[c]{ccc}%
1 & 0 & 0\\
0 & 1 & 0\\
0 & 0 & K^{-1}%
\end{array}
\right)  \cdot M^{t}\cdot\left(
\begin{array}
[c]{ccc}%
1 & 0 & 0\\
0 & 1 & 0\\
0 & 0 & K
\end{array}
\right)
\]
is $K$-orthogonal.\pagebreak
\end{exercise}

\subsection{Why use $K$-coordinates?}

We saw that we could measure the usual Euclidean lengths of curves $\hat
{X}\left(  t\right)  $ on the usual Euclidean $R$-sphere just in terms of the
formulas $X\left(  t\right)  $ for their paths in $\left(  x,y,z\right)
$-coordinates using the $K$-dot product, since lengths depended only on
lengths of tangent vectors and%
\[
\frac{d\hat{X}\left(  t\right)  }{dt}\bullet\frac{d\hat{X}\left(  t\right)
}{dt}=\frac{dX\left(  t\right)  }{dt}\bullet_{K}\frac{dX\left(  t\right)
}{dt}%
\]
where%
\[
\frac{dX\left(  t\right)  }{dt}\bullet_{K}\frac{dX\left(  t\right)  }%
{dt}=\left(  \frac{dX\left(  t\right)  }{dt}\right)  \cdot\left(
\begin{array}
[c]{ccc}%
1 & 0 & 0\\
0 & 1 & 0\\
0 & 0 & K^{-1}%
\end{array}
\right)  \cdot\left(  \frac{dX\left(  t\right)  }{dt}\right)  ^{t}.
\]
In other words, the usual geometry of the sphere of radius $R$ is simply the
geometry of the set $\left(  \ref{29}\right)  $ with $K=1/R^{2}$ and with
lengths (and areas) given by the $K$-dot product. Said another way, we can do
all of spherical geometry in $\left(  x,y,z\right)  $-coordinates. All we need
is the set $\left(  \ref{29}\right)  $ and the $K$-dot product. But the set
$\left(  \ref{29}\right)  $ continues to exist even if $K=0$ or $K<0$, and the
$K$-dot product formula continues to make sense even if $K<0$. In short we
have the following table:
\begin{equation}%
\begin{tabular}
[c]{ccc}%
Spherical ($K>0$) & Euclidean ($K=0$) & Hyperbolic ($K<0$)\\
$\hat{x}^{2}+\hat{y}^{2}+\hat{z}^{2}=R^{2}$ &  & \\
$\hat{V}\bullet\hat{V}$ &  & \\
$1=K\left(  x^{2}+y^{2}\right)  +z^{2}$ & $1=K\left(  x^{2}+y^{2}\right)
+z^{2}$ & $1=K\left(  x^{2}+y^{2}\right)  +z^{2}$\\
$V_{1}\bullet_{K}V_{2}$ &  & $V_{1}\bullet_{K}V_{2}$%
\end{tabular}
\ \ \ \ \label{66}%
\end{equation}
This table tells us that `there is something else out there,' that is, some
other type of two-dimensional geometry beyond plane geometry and spherical
geometry. But the gap in the bottom row of the table is a bit disturbing. If
we can't express the usual dot-product in plane geometry as the $K$-dot
product for $K=0$, we can't pass smoothly from spherical through plane
geometry to hyperbolic geometry using $\left(  x,y,z\right)  $-coordinates. We
now examine two ways to produce coordinates uniformly for spherical, plane and
hyperbolic geometry that overcome this difficulty.

\pagebreak

\section{Central projection}

\subsection{Central projection coordinates}

Let's project $K$-geometry, that is, the set
\begin{equation}
1=K\left(  x^{2}+y^{2}\right)  +z^{2} \label{30}%
\end{equation}
onto the set%
\[
z=1
\]
using the origin%
\[
O=\left(  0,0,0\right)
\]
as the center of projection:%

\begin{tabular}
[c]{cc}%
$%
%TCIMACRO{\FRAME{itbpF}{2.6852in}{1.3543in}{0in}{}{}{Figure}%
%{\special{ language "Scientific Word";  type "GRAPHIC";
%maintain-aspect-ratio TRUE;  display "USEDEF";  valid_file "T";
%width 2.6852in;  height 1.3543in;  depth 0in;  original-width 5.2745in;
%original-height 2.6481in;  cropleft "0";  croptop "1";  cropright "1";
%cropbottom "0";  tempfilename 'MXAJBZ0K.png';tempfile-properties "XPR";}}}%
%BeginExpansion
{\includegraphics[
natheight=2.648100in,
natwidth=5.274500in,
height=1.3543in,
width=2.6852in
]%
{MXAJBZ0K.png}%
}%
%EndExpansion
$ &
%TCIMACRO{\FRAME{itbpF}{2.6913in}{1.6345in}{0in}{}{}{Figure}%
%{\special{ language "Scientific Word";  type "GRAPHIC";
%maintain-aspect-ratio TRUE;  display "USEDEF";  valid_file "T";
%width 2.6913in;  height 1.6345in;  depth 0in;  original-width 5.7363in;
%original-height 3.4722in;  cropleft "0";  croptop "1";  cropright "1";
%cropbottom "0";  tempfilename 'MXAJBZ0L.png';tempfile-properties "XPR";}}}%
%BeginExpansion
{\includegraphics[
natheight=3.472200in,
natwidth=5.736300in,
height=1.6345in,
width=2.6913in
]%
{MXAJBZ0L.png}%
}%
%EndExpansion
\end{tabular}


That is,%
\begin{equation}
r\text{\textperiodcentered}\left(  x_{c},y_{c},1\right)  =\left(
x,y,z\right)  . \label{131}%
\end{equation}
So%
\[
r=z
\]
and, from the equation $\left(  \ref{30}\right)  $%
\begin{gather*}
K\left(  \left(  rx_{c}\right)  ^{2}+\left(  ry_{c}\right)  ^{2}\right)
+r^{2}=1\\
r^{2}=\frac{1}{K\left(  x_{c}^{2}+y_{c}^{2}\right)  +1}.
\end{gather*}
Notice that, when $K<0$ this last formula only makes sense when%
\begin{gather}
K\left(  x_{c}^{2}+y_{c}^{2}\right)  >-1\label{75}\\
x_{c}^{2}+y_{c}^{2}<\frac{-1}{K}.\nonumber
\end{gather}


\begin{exercise}
\label{31}a) For the projection of the set $\left(  \ref{30}\right)  $ onto
the $z=1$ plane with center of projection $O$, write $\left(  x_{c}%
,y_{c}\right)  $ as a function of $\left(  x,y,z\right)  $.

b) For the projection of the set $\left(  \ref{30}\right)  $ onto the $z=1$
plane with center of projection $O$, write $\left(  x,y,z\right)  $ as a
function of $\left(  x_{c},y_{c}\right)  $.\pagebreak
\end{exercise}

\subsection{Rigid motion in central projection coordinates}

Suppose now we have a $K$-rigid motion%
\[
\left(  \underline{x},\underline{y},\underline{z}\right)  =\left(
x,y,z\right)  \cdot M
\]
of $K$-geometry, given by a $K$-orthogonal matrix%
\[
M=\left(
\begin{array}
[c]{ccc}%
m_{11} & m_{12} & m_{13}\\
m_{21} & m_{22} & m_{23}\\
m_{31} & m_{32} & m_{33}%
\end{array}
\right)  .
\]
To see what this $K$-rigid motion looks like in central projection coordinates
we simply do the matrix multiplication%
\begin{gather*}
\left(  \underline{x},\underline{y},\underline{z}\right)  =\left(
x,y,z\right)  \cdot\left(
\begin{array}
[c]{ccc}%
m_{11} & m_{12} & m_{13}\\
m_{21} & m_{22} & m_{23}\\
m_{31} & m_{32} & m_{33}%
\end{array}
\right) \\
=\left(  \left(  m_{11}x+m_{21}y+m_{31}z\right)  ,\left(  m_{12}%
x+m_{22}y+m_{32}z\right)  ,\left(  m_{13}x+m_{23}y+m_{33}z\right)  \right)  .
\end{gather*}
Then%
\begin{align}
\underline{x_{c}}  &  =\frac{\underline{x}}{\underline{z}}\label{77}\\
&  =\frac{m_{11}x+m_{21}y+m_{31}z}{m_{13}x+m_{23}y+m_{33}z}\nonumber\\
&  =\frac{m_{11}\left(  x/z\right)  +m_{21}\left(  y/z\right)  +m_{31}}%
{m_{13}\left(  x/z\right)  +m_{23}\left(  y/z\right)  +m_{33}}\nonumber\\
&  =\frac{m_{11}x_{c}+m_{21}y_{c}+m_{31}}{m_{13}x_{c}+m_{23}y_{c}+m_{33}%
}\nonumber
\end{align}
and similarly%
\[
\underline{y_{c}}=\frac{m_{12}x_{c}+m_{22}y_{c}+m_{32}}{m_{13}x_{c}%
+m_{23}y_{c}+m_{33}}.
\]
So we write%
\begin{gather*}
\left(  \underline{x_{c}},\underline{y_{c}}\right)  =M_{c}\left(  x_{c}%
,y_{c}\right) \\
=\left(  \frac{m_{11}x_{c}+m_{21}y_{c}+m_{31}}{m_{13}x_{c}+m_{23}y_{c}+m_{33}%
},\frac{m_{12}x_{c}+m_{22}y_{c}+m_{32}}{m_{13}x_{c}+m_{23}y_{c}+m_{33}%
}\right)  .
\end{gather*}


\pagebreak

\subsection{Length and angle in central projection coordinates}

\begin{exercise}
\label{33}For the $K$-geometry coordinates%
\[
X=\left(  x,y,z\right)
\]
use the formulas you derived in Exercise \ref{31}b) to calculate%
\[
dX=\left(  \frac{\partial X}{\partial x_{c}}\right)  dx_{c}+\left(
\frac{\partial X}{\partial y_{c}}\right)  dy_{c}%
\]
That is, calculate the $2\times3$ matrix%
\[
D_{c}=\left(
\begin{array}
[c]{ccc}%
\frac{\partial x}{\partial x_{c}} & \frac{\partial y}{\partial x_{c}} &
\frac{\partial z}{\partial x_{c}}\\
\frac{\partial x}{\partial y_{c}} & \frac{\partial y}{\partial y_{c}} &
\frac{\partial z}{\partial y_{c}}%
\end{array}
\right)  =\left(
\begin{array}
[c]{c}%
\left(  \frac{\partial X}{\partial x_{c}}\right) \\
\left(  \frac{\partial X}{\partial y_{c}}\right)
\end{array}
\right)  .
\]
Hint: Use logarithmic differentiation:%
\begin{align*}
dx  &  =d\left(  rx_{c}\right)  =x_{c}dr+rdx_{c}\\
r^{-1}dx  &  =x_{c}d\mathrm{ln}\left(  r\right)  +dx_{c}%
\end{align*}
and similarly for $y$ and $z$ since it is easier to compute $r^{-1}\left(
\frac{dx}{dt},\frac{dy}{dt},\frac{dz}{dt}\right)  $ than $\left(  \frac
{dx}{dt},\frac{dy}{dt},\frac{dz}{dt}\right)  $. Next use that%
\begin{align*}
2d\mathrm{ln}\left(  r\right)   &  =d\mathrm{ln}\left(  r^{2}\right)
=-d\mathrm{ln}\left(  K\left(  x_{c}^{2}+y_{c}^{2}\right)  +1\right) \\
&  =-\frac{1}{K\left(  x_{c}^{2}+y_{c}^{2}\right)  +1}d\left(  K\left(
x_{c}^{2}+y_{c}^{2}\right)  +1\right) \\
&  =-r^{2}K\left(  2x_{c}dx_{c}+2y_{c}dy_{c}\right)  .
\end{align*}

\end{exercise}

\begin{exercise}
\label{prev}Now suppose we have a path,%
\[
\left(  x_{c}\left(  t\right)  ,y_{c}\left(  t\right)  \right)  ,\;a\leq t\leq
b
\]
in the $\left(  x_{c},y_{c}\right)  $-plane, that is, in the central
projection plane%
\[
\left(  x_{c},y_{c},1\right)  .
\]
Use the formula you derived in Exercise \ref{31}b) to write the corresponding
path%
\[
x\left(  x_{c}\left(  t\right)  ,y_{c}\left(  t\right)  \right)  ,y\left(
x_{c}\left(  t\right)  ,y_{c}\left(  t\right)  \right)  ,z\left(  x_{c}\left(
t\right)  ,y_{c}\left(  t\right)  \right)
\]
in the $K$-geometry space of $\left(  x,y,z\right)  $ such that $K\left(
x^{2}+y^{2}\right)  +z^{2}=1$.
\end{exercise}

\begin{exercise}
For the path $\left(  x\left(  t\right)  ,y\left(  t\right)  ,z\left(
t\right)  \right)  $ in Exercise \ref{prev} lying on the set $\left(
\ref{30}\right)  $, use the Chain Rule from calculus of several variables to
compute%
\[
\left(  \frac{dx}{dt},\frac{dy}{dt},\frac{dz}{dt}\right)  =\left(
\frac{dx_{c}\left(  t\right)  }{dt},\frac{dy_{c}\left(  t\right)  }%
{dt}\right)  \cdot D_{c}.
\]

\end{exercise}

This last Exercise allows us to do something very nice. Namely now, not only
can we use the coordinates $\left(  x_{c},y_{c}\right)  $ for our geometry but
we can also compute the $K$-dot product in terms of these coordinates. By the
Chain Rule from calculus of several variables%
\[
\left(  \frac{dx}{dt},\frac{dy}{dt},\frac{dz}{dt}\right)  =\left(
\frac{dx_{c}}{dt},\frac{dy_{c}}{dt}\right)  \cdot D_{c}.
\]
So%
\begin{align*}
\left(  \frac{dx}{dt},\frac{dy}{dt},\frac{dz}{dt}\right)  \bullet_{K}\left(
\frac{dx}{dt},\frac{dy}{dt},\frac{dz}{dt}\right)   &  =\left(
\begin{array}
[c]{ccc}%
\frac{dx}{dt} & \frac{dy}{dt} & \frac{dz}{dt}%
\end{array}
\right)  \left(
\begin{array}
[c]{ccc}%
1 & 0 & 0\\
0 & 1 & 0\\
0 & 0 & K^{-1}%
\end{array}
\right)  \left(
\begin{array}
[c]{c}%
\frac{dx}{dt}\\
\frac{dy}{dt}\\
\frac{dz}{dt}%
\end{array}
\right) \\
&  =\left(
\begin{array}
[c]{cc}%
\frac{dx_{c}}{dt} & \frac{dy_{c}}{dt}%
\end{array}
\right)  \cdot D_{c}\cdot\left(
\begin{array}
[c]{ccc}%
1 & 0 & 0\\
0 & 1 & 0\\
0 & 0 & K^{-1}%
\end{array}
\right)  \cdot D_{c}^{t}\cdot\left(
\begin{array}
[c]{c}%
\frac{dx_{c}}{dt}\\
\frac{dy_{c}}{dt}%
\end{array}
\right)  ,
\end{align*}


\begin{exercise}
\label{32}Compute the $2\times2$ matrix%
\[
P_{c}=D_{c}\cdot\left(
\begin{array}
[c]{ccc}%
1 & 0 & 0\\
0 & 1 & 0\\
0 & 0 & K^{-1}%
\end{array}
\right)  \cdot D_{c}^{t},
\]
that that gives the $K$-dot product in $\left(  x_{c},y_{c}\right)
$-coordinates. That is, use matrix multiplication to show that%
\[
P_{c}=\left(
\begin{array}
[c]{cc}%
r^{2}\left(  1-r^{2}Kx_{c}^{2}\right)  & -r^{4}Kx_{c}y_{c}\\
-r^{4}Kx_{c}y_{c} & r^{2}\left(  1-r^{2}Ky_{c}^{2}\right)
\end{array}
\right)  .
\]


Hint: For example%
\begin{align*}
\frac{\partial x}{\partial x_{c}}  &  =r\left(  x_{c}\frac{\partial ln\left(
r\right)  }{\partial x_{c}}+1\right)  =-r^{3}Kx_{c}^{2}+r\\
\frac{\partial y}{\partial x_{c}}  &  =r\left(  y_{c}\frac{\partial ln\left(
r\right)  }{\partial x_{c}}\right)  =-r^{3}Kx_{c}y_{c}\\
\frac{\partial z}{\partial x_{c}}  &  =r\left(  \frac{\partial ln\left(
r\right)  }{\partial x_{c}}\right)  =-r^{3}Kx_{c}%
\end{align*}
so that%
\begin{align*}
&  \left(  \frac{\partial x}{\partial x_{c}},\frac{\partial y}{\partial x_{c}%
},\frac{\partial z}{\partial x_{c}}\right)  \bullet_{K}\left(  \frac{\partial
x}{\partial x_{c}},\frac{\partial y}{\partial x_{c}},\frac{\partial
z}{\partial x_{c}}\right) \\
&  =r^{6}K^{2}x_{c}^{4}-2r^{4}Kx_{c}^{2}+r^{2}+r^{6}K^{2}x_{c}^{2}y_{c}%
^{2}+r^{6}Kx_{c}^{2}\\
&  =\left(  r^{6}K^{2}x_{c}^{4}+r^{6}K^{2}x_{c}^{2}y_{c}^{2}+r^{6}Kx_{c}%
^{2}\right)  -2r^{4}Kx_{c}^{2}+r^{2}\\
&  =r^{4}Kx_{c}^{2}-2r^{4}Kx_{c}^{2}+r^{2}=r^{2}\left(  1-r^{2}Kx_{c}%
^{2}\right)  .
\end{align*}

\end{exercise}

So, if, if $K>0$ and you have a path on the sphere of radius $R=K^{-1/2}$ in
Euclidean $3$-space given in $\left(  x_{c},y_{c}\right)  $-coordinates as
$\left(  x_{c}\left(  t\right)  ,y_{c}\left(  t\right)  \right)  $ for
$t\in\left[  b,e\right]  $, you can trace back everything we have done with
coordinate changes to see that the length of the path on the sphere of radius
$R=K^{-1/2}$ in Euclidean $3$-space is given by%
\[%
%TCIMACRO{\dint \nolimits_{b}^{e}}%
%BeginExpansion
{\displaystyle\int\nolimits_{b}^{e}}
%EndExpansion
l\left(  t\right)  dt
\]
where%
\begin{align*}
l\left(  t\right)  ^{2}  &  =\left(  \frac{dx_{c}}{dt},\frac{dy_{c}}%
{dt}\right)  \bullet_{c}\left(  \frac{dx_{c}}{dt},\frac{dy_{c}}{dt}\right) \\
&  =\left(
\begin{array}
[c]{cc}%
\frac{dx_{c}}{dt} & \frac{dy_{c}}{dt}%
\end{array}
\right)  \cdot P_{c}\cdot\left(
\begin{array}
[c]{c}%
\frac{dx_{c}}{dt}\\
\frac{dy_{c}}{dt}%
\end{array}
\right)  .
\end{align*}
Notice that the matrix $P_{c}$ still makes sense when $K=0$ and when $K$
becomes negative. So we do have%
\[
\frame{%
\begin{tabular}
[c]{ccc}%
\textit{Spherical} ($K>0$) & \textit{Euclidean} ($K=0$) & \textit{Hyperbolic}
($K<0$)\\
$\hat{x}^{2}+\hat{y}^{2}+\hat{z}^{2}=R^{2}$ &  & \\
$\hat{V}\bullet\hat{V}$ &  & \\
$1=K\left(  x^{2}+y^{2}\right)  +z^{2}$ & $1=K\left(  x^{2}+y^{2}\right)
+z^{2}$ & $1=K\left(  x^{2}+y^{2}\right)  +z^{2}$\\
$V_{1}\bullet_{K}V_{2}$ &  & $V_{1}\bullet_{K}V_{2}$\\
$V_{1}^{c}\bullet_{c}V_{2}^{c}$ & $V_{1}^{c}\bullet_{c}V_{2}^{c}$ & $V_{1}%
^{c}\bullet_{c}V_{2}^{c}$%
\end{tabular}
}%
\]
where%
\[
V_{1}^{c}\bullet_{c}V_{2}^{c}=\left(  V_{1}^{c}\right)  \cdot P_{c}%
\cdot\left(  V_{2}^{c}\right)  ^{t}.
\]
Of course if $K>0$, we again have Euclidean angles $\theta$ between vectors
$\hat{V}_{1}$ and $\hat{V}_{2}$ tangent to the $R$-sphere at some point
computed by%
\begin{align*}
\hat{V}_{1}\bullet\hat{V}_{2}  &  =\left\vert \hat{V}_{1}\right\vert
\text{\textperiodcentered}\left\vert \hat{V}_{2}\right\vert
\text{\textperiodcentered\textrm{cos}}\left(  \theta\right) \\
&  =V_{1}^{c}\bullet_{c}V_{2}^{c}.
\end{align*}
\pagebreak

\subsection{Area in central projection coordinates}

Suppose you were given a region $G_{c}$ in the $\left(  x_{c},y_{c}\right)
$-coordinate plane. Also suppose that $K>0$. If you trace back everything we
have done with coordinate changes, you can see how $G_{c}$ gives you a region
$\hat{G}$ on the sphere of radius $R=K^{-1/2}$ in Euclidean $3$-space via the
formulas%
\begin{align*}
\left(  \hat{x},\hat{y},\hat{z}\right)   &  =\left(  x,y,Rz\right) \\
&  =r\text{\textperiodcentered}\left(  x_{c},y_{c},R\right) \\
&  =\left(  \frac{x_{c}}{\sqrt{K\left(  x_{c}^{2}+y_{c}^{2}\right)  +1}}%
,\frac{y_{c}}{\sqrt{K\left(  x_{c}^{2}+y_{c}^{2}\right)  +1}},\frac{R}%
{\sqrt{K\left(  x_{c}^{2}+y_{c}^{2}\right)  +1}}\right)  .
\end{align*}
Now there is a formula in several variable calculus for computing the area of
the region $\hat{G}$ on the sphere of radius $R$ in Euclidean $3$-space in
terms of the parameters $\left(  x_{c},y_{c}\right)  $. [DS,49,231]. It is
\begin{equation}%
%TCIMACRO{\dint \nolimits_{G_{c}}}%
%BeginExpansion
{\displaystyle\int\nolimits_{G_{c}}}
%EndExpansion
\hat{a}\left(  \frac{d\hat{X}}{dx_{c}},\frac{d\hat{X}}{dy_{c}}\right)
dx_{c}dy_{c} \label{68}%
\end{equation}
where $\hat{a}\left(  \frac{d\hat{X}}{dx_{c}},\frac{d\hat{X}}{dy_{c}}\right)
$ is the (Euclidean) area of the parallelogram spanned by the two vectors
$\frac{d\hat{X}}{dx_{c}}$ and $\frac{d\hat{X}}{dy_{c}}$ in Euclidean
$3$-space. Thus%
\[
\hat{a}\left(  \frac{d\hat{X}}{dx_{c}},\frac{d\hat{X}}{dy_{c}}\right)
=\left\vert \frac{d\hat{X}}{dx_{c}}\right\vert \text{\textperiodcentered
}\left\vert \frac{d\hat{X}}{dy_{c}}\right\vert \text{\textperiodcentered
}\mathrm{sin}\left(  \theta\right)
\]
where $\theta$ is the angle between the two vectors.

\begin{exercise}
Using Exercise \ref{9} and Exercise \ref{222} show that%
\begin{align*}
\hat{a}\left(  \frac{d\hat{X}}{dx_{c}},\frac{d\hat{X}}{dy_{c}}\right)  ^{2}
&  =\left\vert
\begin{array}
[c]{cc}%
\frac{d\hat{X}}{dx_{c}}\bullet\frac{d\hat{X}}{dx_{c}} & \frac{d\hat{X}}%
{dy_{c}}\bullet\frac{d\hat{X}}{dx_{c}}\\
\frac{d\hat{X}}{dx_{c}}\bullet\frac{d\hat{X}}{dy_{c}} & \frac{d\hat{X}}%
{dy_{c}}\bullet\frac{d\hat{X}}{dy_{c}}%
\end{array}
\right\vert \\
&  =\left\vert
\begin{array}
[c]{cc}%
\frac{dX}{dx_{c}}\bullet_{K}\frac{dX}{dx_{c}} & \frac{dX}{dy_{c}}\bullet
_{K}\frac{dX}{dx_{c}}\\
\frac{dX}{dx_{c}}\bullet_{K}\frac{dX}{dy_{c}} & \frac{dX}{dy_{c}}\bullet
_{K}\frac{dX}{dy_{c}}%
\end{array}
\right\vert \\
&  =\left\vert \left(
\begin{array}
[c]{c}%
\left(  \frac{dX}{dx_{c}}\right) \\
\left(  \frac{dX}{dy_{c}}\right)
\end{array}
\right)  \left(
\begin{array}
[c]{ccc}%
1 & 0 & 0\\
0 & 1 & 0\\
0 & 0 & K^{-1}%
\end{array}
\right)  \left(
\begin{array}
[c]{cc}%
\left(  \frac{dX}{dx_{c}}\right)  ^{t} & \left(  \frac{dX}{dy_{c}}\right)
^{t}%
\end{array}
\right)  \right\vert \\
&  =\left\vert P_{c}\right\vert .
\end{align*}

\end{exercise}

\begin{exercise}
\label{79}Use Exercise \ref{32} to show that%
\[
\hat{a}\left(  \frac{d\hat{X}}{dx_{c}},\frac{d\hat{X}}{dy_{c}}\right)
^{2}=r^{6}=\frac{1}{\left(  K\left(  x_{c}^{2}+y_{c}^{2}\right)  +1\right)
^{3}}%
\]
as a function of $\left(  x_{c},y_{c}\right)  $.

Hint: Notice that the matrix $D_{c}$ in Exercise \ref{33} is simply the
$2\times3$ matix whose rows are the vectors $\frac{dX}{dx_{c}}$ and $\frac
{dX}{dy_{c}}$. So referring to Exercise \ref{32}, we know that%
\[
\left(
\begin{array}
[c]{cc}%
\frac{d\hat{X}}{dx_{c}}\bullet\frac{d\hat{X}}{dx_{c}} & \frac{d\hat{X}}%
{dy_{c}}\bullet\frac{d\hat{X}}{dx_{c}}\\
\frac{d\hat{X}}{dx_{c}}\bullet\frac{d\hat{X}}{dy_{c}} & \frac{d\hat{X}}%
{dy_{c}}\bullet\frac{d\hat{X}}{dy_{c}}%
\end{array}
\right)  =\left(
\begin{array}
[c]{cc}%
r^{2}\left(  1-r^{2}Kx_{c}^{2}\right)  & -r^{4}Kx_{c}y_{c}\\
-r^{4}Kx_{c}y_{c} & r^{2}\left(  1-r^{2}Ky_{c}^{2}\right)
\end{array}
\right)  .
\]

\end{exercise}

Since all these computations can be extended to $K$-geometry for all $K$, we
define the $K$-area of a region $G_{c}$ in the $\left(  x_{c},y_{c}\right)
$-coordinate plane by first computing the $K$-area of the parallelogram
spanned by $\frac{dX}{dx_{c}}$ and $\frac{dX}{dy_{c}}$ at each point of
$G_{c}$ as%
\begin{align*}
a_{K}\left(  \frac{dX}{dx_{c}},\frac{dY}{dy_{c}}\right)   &  =\left\vert
\frac{dX}{dx_{c}}\right\vert _{K}\cdot\left\vert \frac{dX}{dy_{c}}\right\vert
_{K}\cdot\mathrm{sin}\left(  \theta_{K}\right) \\
&  =\sqrt{\left\vert
\begin{array}
[c]{cc}%
\frac{dX}{dx_{c}}\bullet_{K}\frac{dX}{dx_{c}} & \frac{dX}{dy_{c}}\bullet
_{K}\frac{dX}{dx_{c}}\\
\frac{dX}{dx_{c}}\bullet_{K}\frac{dX}{dy_{c}} & \frac{dX}{dy_{c}}\bullet
_{K}\frac{dX}{dy_{c}}%
\end{array}
\right\vert }%
\end{align*}
and then integrating this area over $G_{c}$ to get%
\begin{align*}
A_{K}\left(  G_{c}\right)   &  =%
%TCIMACRO{\dint \nolimits_{G_{c}}}%
%BeginExpansion
{\displaystyle\int\nolimits_{G_{c}}}
%EndExpansion
a_{K}\left(  \frac{dX}{dx_{c}},\frac{dY}{dy_{c}}\right)  dx_{c}dy_{c}\\
&  =%
%TCIMACRO{\dint \nolimits_{G_{c}}}%
%BeginExpansion
{\displaystyle\int\nolimits_{G_{c}}}
%EndExpansion
\sqrt{\left\vert
\begin{array}
[c]{cc}%
\frac{dX}{dx_{c}}\bullet_{K}\frac{dX}{dx_{c}} & \frac{dX}{dy_{c}}\bullet
_{K}\frac{dX}{dx_{c}}\\
\frac{dX}{dx_{c}}\bullet_{K}\frac{dX}{dy_{c}} & \frac{dX}{dy_{c}}\bullet
_{K}\frac{dX}{dy_{c}}%
\end{array}
\right\vert }dx_{c}dy_{c}.
\end{align*}
\pagebreak

\section{Stereographic projection}

\subsection{Stereographic projection coordinates}

On the other hand we can project the set
\[
1=K\left(  x^{2}+y^{2}\right)  +z^{2}%
\]
onto the set%
\[
z=1
\]
using the `South Pole'%
\[
S=\left(  0,0,-1\right)
\]
as the center of projection:%

\begin{tabular}
[c]{cc}%
%TCIMACRO{\FRAME{itbpF}{2.789in}{1.9389in}{0in}{}{}{Figure}%
%{\special{ language "Scientific Word";  type "GRAPHIC";
%maintain-aspect-ratio TRUE;  display "USEDEF";  valid_file "T";
%width 2.789in;  height 1.9389in;  depth 0in;  original-width 4.9121in;
%original-height 3.4065in;  cropleft "0";  croptop "1";  cropright "1";
%cropbottom "0";  tempfilename 'MXAJBZ0M.png';tempfile-properties "XPR";}}}%
%BeginExpansion
{\includegraphics[
natheight=3.406500in,
natwidth=4.912100in,
height=1.9389in,
width=2.789in
]%
{MXAJBZ0M.png}%
}%
%EndExpansion
&
%TCIMACRO{\FRAME{itbpF}{2.226in}{1.9415in}{0in}{}{}{Figure}%
%{\special{ language "Scientific Word";  type "GRAPHIC";
%maintain-aspect-ratio TRUE;  display "USEDEF";  valid_file "T";
%width 2.226in;  height 1.9415in;  depth 0in;  original-width 4.8896in;
%original-height 4.2635in;  cropleft "0";  croptop "1";  cropright "1";
%cropbottom "0";  tempfilename 'MXAJBZ0N.png';tempfile-properties "XPR";}}}%
%BeginExpansion
{\includegraphics[
natheight=4.263500in,
natwidth=4.889600in,
height=1.9415in,
width=2.226in
]%
{MXAJBZ0N.png}%
}%
%EndExpansion
\end{tabular}


That is,%
\[
\rho\text{\textperiodcentered}\left(  x_{s},y_{s},1-\left(  -1\right)
\right)  =\left(  x,y,z-\left(  -1\right)  \right)  .
\]
So%
\begin{align*}
\rho &  =\frac{z+1}{2}\\
z  &  =2\rho-1
\end{align*}
and, from the equation $\left(  \ref{30}\right)  $%
\begin{gather*}
K\left(  \left(  \rho x_{s}\right)  ^{2}+\left(  \rho y_{s}\right)
^{2}\right)  +\left(  2\rho-1\right)  ^{2}=1\\
K\left(  \left(  \rho x_{s}\right)  ^{2}+\left(  \rho y_{s}\right)
^{2}\right)  +4\rho^{2}-4\rho=0\\
\rho K\left(  x_{s}^{2}+y_{s}^{2}\right)  +4\rho=4\\
\rho=\frac{1}{\frac{K}{4}\left(  x_{s}^{2}+y_{s}^{2}\right)  +1}.
\end{gather*}
Notice that, when $K<0$ this last formula only makes sense when%
\begin{gather*}
\frac{K}{4}\left(  x_{s}^{2}+y_{s}^{2}\right)  >-1\\
x_{s}^{2}+y_{s}^{2}<\frac{-4}{K}.
\end{gather*}


\begin{exercise}
\label{35}a) For the projection of the set $\left(  \ref{30}\right)  $ onto
the $z=1$ plane with center of projection $S$, write $\left(  x_{s}%
,y_{s}\right)  $ as a function of $\left(  x,y,z\right)  $.

b) For the projection of the set $\left(  \ref{30}\right)  $ onto the $z=1$
plane with center of projection $S$, write $\left(  x,y,z\right)  $ as a
function of $\left(  x_{s},y_{s}\right)  $.\pagebreak
\end{exercise}

\subsection{Length and angle in stereographic projection coordinates}

\begin{exercise}
\label{37}Suppose we have a path%
\[
X\left(  x_{s}\left(  t\right)  ,y_{s}\left(  t\right)  \right)  =\left(
x\left(  x_{s}\left(  t\right)  ,y_{s}\left(  t\right)  \right)  ,y\left(
x_{s}\left(  t\right)  ,y_{s}\left(  t\right)  \right)  ,z\left(  x_{s}\left(
t\right)  ,y_{s}\left(  t\right)  \right)  \right)
\]
lying on the set $\left(  \ref{30}\right)  $ given in terms of its projection
$\left(  x_{s}\left(  t\right)  ,y_{s}\left(  t\right)  \right)  $ in the
plane $z=1$. Use the formula you derived in Exercise \ref{35}b) and the Chain
Rule from calculus of several variables to find the $2\times3$ matrix%
\[
D_{s}=\left(
\begin{array}
[c]{c}%
\left(  \frac{\partial X}{\partial x_{s}}\right) \\
\left(  \frac{\partial X}{\partial y_{s}}\right)
\end{array}
\right)
\]
such that%
\[
\left(  \frac{dx}{dt},\frac{dy}{dt},\frac{dz}{dt}\right)  =\left(
\frac{dx_{s}\left(  t\right)  }{dt},\frac{dy_{s}\left(  t\right)  }%
{dt}\right)  \cdot D_{s}.
\]
Hint: Use logarithmic differentiation:%
\begin{gather*}
dx=d\left(  \rho x_{s}\right)  =x_{s}d\rho+\rho dx_{s}\\
\rho^{-1}dx=x_{s}d\mathrm{ln}\left(  \rho\right)  +dx_{s}%
\end{gather*}
and similarly for $y$. Also%
\begin{align*}
d\mathrm{ln}\left(  \rho\right)   &  =-d\mathrm{ln}\left(  \frac{K}{4}\left(
x_{s}^{2}+y_{s}^{2}\right)  +1\right) \\
&  =-\frac{1}{\frac{K}{4}\left(  x_{s}^{2}+y_{s}^{2}\right)  +1}d\left(
\frac{K}{4}\left(  x_{s}^{2}+y_{s}^{2}\right)  +1\right) \\
&  =-\rho\frac{K}{4}\left(  2x_{s}dx_{s}+2y_{s}dy_{s}\right)  .
\end{align*}

\end{exercise}

This last Exercise allows us to do something very nice. Namely now, not only
can we use the coordinates $\left(  x_{s},y_{s}\right)  $ for our geometry but
we can also compute the $K$-dot product in terms of these coordinates.:%
\[
\left(  \frac{dx}{dt},\frac{dy}{dt},\frac{dz}{dt}\right)  =\left(
\frac{dx_{s}}{dt},\frac{dy_{s}}{dt}\right)  \cdot D_{s}%
\]
so that%
\begin{align*}
\left(  \frac{dx}{dt},\frac{dy}{dt},\frac{dz}{dt}\right)  \bullet_{K}\left(
\frac{dx}{dt},\frac{dy}{dt},\frac{dz}{dt}\right)   &  =\left(
\begin{array}
[c]{ccc}%
\frac{dx}{dt} & \frac{dy}{dt} & \frac{dz}{dt}%
\end{array}
\right)  \left(
\begin{array}
[c]{ccc}%
1 & 0 & 0\\
0 & 1 & 0\\
0 & 0 & K^{-1}%
\end{array}
\right)  \left(
\begin{array}
[c]{c}%
\frac{dx}{dt}\\
\frac{dy}{dt}\\
\frac{dz}{dt}%
\end{array}
\right) \\
&  =\left(
\begin{array}
[c]{cc}%
\frac{dx_{s}}{dt} & \frac{dy_{s}}{dt}%
\end{array}
\right)  \cdot D_{s}\cdot\left(
\begin{array}
[c]{ccc}%
1 & 0 & 0\\
0 & 1 & 0\\
0 & 0 & K^{-1}%
\end{array}
\right)  \cdot D_{s}^{t}\cdot\left(
\begin{array}
[c]{c}%
\frac{dx_{s}}{dt}\\
\frac{dy_{s}}{dt}%
\end{array}
\right)  ,
\end{align*}


\begin{exercise}
\label{36}Use matrix multiplication to compute the $2\times2$ matrix%
\[
P_{s}=D_{s}\cdot\left(
\begin{array}
[c]{ccc}%
1 & 0 & 0\\
0 & 1 & 0\\
0 & 0 & K^{-1}%
\end{array}
\right)  \cdot D_{s}^{t},
\]
that is, to compute the $K$-dot product in $\left(  x_{s},y_{s}\right)
$-coordinates. (You may be surprised at the answer! It is quite simple and
only involves the quantity $\rho$.)
\end{exercise}

So, if, if $K>0$ and you have a path on the sphere of radius $R=K^{-1/2}$ in
Euclidean $3$-space given in $\left(  x_{s},y_{s}\right)  $-coordinates as
$\left(  x_{s}\left(  t\right)  ,y_{s}\left(  t\right)  \right)  $ for
$t\in\left[  b,e\right]  $, you can trace back everything we have done with
coordinate changes to see that the length of the path on the sphere of radius
$R=K^{-1/2}$ in Euclidean $3$-space is given by%
\[%
%TCIMACRO{\dint \nolimits_{b}^{e}}%
%BeginExpansion
{\displaystyle\int\nolimits_{b}^{e}}
%EndExpansion
l\left(  t\right)  dt
\]
where%
\begin{align*}
l\left(  t\right)  ^{2}  &  =\left(  \frac{dx_{s}}{dt},\frac{dy_{s}}%
{dt}\right)  \bullet_{s}\left(  \frac{dx_{s}}{dt},\frac{dy_{s}}{dt}\right) \\
&  =\left(
\begin{array}
[c]{cc}%
\frac{dx_{s}}{dt} & \frac{dy_{s}}{dt}%
\end{array}
\right)  \cdot P_{s}\cdot\left(
\begin{array}
[c]{c}%
\frac{dx_{s}}{dt}\\
\frac{dy_{s}}{dt}%
\end{array}
\right)
\end{align*}
and that the measure $\theta$ of an angle between vectors $\hat{V}_{1}$ and
$\hat{V}_{2}$ on the $R$-sphere is computed by%
\[
\mathrm{arccos}\left(  \frac{V_{1}^{s}\cdot P_{s}\cdot\left(  V_{2}%
^{s}\right)  ^{t}}{\left\vert V_{1}^{s}\right\vert _{s}\left\vert V_{2}%
^{s}\right\vert _{s}}\right)  .
\]


Notice that the matrix $P_{s}$ still makes sense when $K=0$ and when $K$
becomes negative.

\begin{exercise}
Write the formula for the $K$-dot product $\left(  x_{s},y_{s}\right)
$-coordinates when $K=0$. Does it look familiar?
\end{exercise}

So we do have%
\[%
\begin{tabular}
[c]{ccc}%
Spherical ($K>0$) & Euclidean ($K=0$) & Hyperbolic ($K<0$)\\
$\hat{x}^{2}+\hat{y}^{2}+\hat{z}^{2}=R^{2}$ &  & \\
$\hat{V}\bullet\hat{V}$ &  & \\
$1=K\left(  x^{2}+y^{2}\right)  +z^{2}$ & $1=K\left(  x^{2}+y^{2}\right)
+z^{2}$ & $1=K\left(  x^{2}+y^{2}\right)  +z^{2}$\\
$V_{1}\bullet_{K}V_{2}$ &  & $V_{1}\bullet_{K}V_{2}$\\
$V_{1}^{c}\bullet_{c}V_{2}^{c}$ & $V_{1}^{c}\bullet_{c}V_{2}^{c}$ & $V_{1}%
^{c}\bullet_{c}V_{2}^{c}$\\
$V_{1}^{s}\bullet_{s}V_{2}^{s}$ & $V_{1}^{s}\bullet_{s}V_{2}^{s}$ & $V_{1}%
^{s}\bullet_{s}V_{2}^{s}$%
\end{tabular}
\ \
\]
where%
\[
V_{1}^{s}\bullet_{s}V_{2}^{s}=\left(  V_{1}^{s}\right)  \cdot P_{s}%
\cdot\left(  V_{2}^{s}\right)  ^{t}.
\]
Of course if $K>0$, we again have Euclidean angles $\theta$ between vectors
$\hat{V}_{1}$ and $\hat{V}_{2}$ tangent to the $R$-sphere at some point
computed by%
\begin{align*}
\hat{V}_{1}\bullet\hat{V}_{2}  &  =\left\vert \hat{V}_{1}\right\vert
\text{\textperiodcentered}\left\vert \hat{V}_{2}\right\vert
\text{\textperiodcentered\textrm{cos}}\left(  \theta\right) \\
&  =V_{1}^{s}\bullet_{s}V_{2}^{s}.
\end{align*}
\pagebreak

\subsection{Area in stereographic projection coordinates}

Suppose you were given a region $G_{s}$ in the $\left(  x_{s},y_{s}\right)
$-coordinate plane. Also suppose that $K>0$. If you trace back everything we
have done with coordinate changes, you can see how $G_{s}$ gives you a region
$\hat{G}$ on the sphere of radius $R=K^{-1/2}$ in Euclidean $3$-space via the
formulas%
\begin{align*}
\left(  \hat{x},\hat{y},\hat{z}\right)   &  =\left(  x,y,Rz\right) \\
&  =\rho\text{\textperiodcentered}\left(  x_{s},y_{s},R\left(  2\rho-1\right)
\right) \\
&  =\left(  \frac{x_{s}}{\frac{K}{4}\left(  x_{s}^{2}+y_{s}^{2}\right)
+1},\frac{y_{s}}{\frac{K}{4}\left(  x_{s}^{2}+y_{s}^{2}\right)  +1}%
,\frac{R\left(  1-\frac{K}{4}\left(  x_{s}^{2}+y_{s}^{2}\right)  \right)
}{1+\frac{K}{4}\left(  x_{s}^{2}+y_{s}^{2}\right)  }\right)  .
\end{align*}
Now there is a formula in several variable calculus for computing the area of
the region $\hat{G}$ on the sphere of radius $R$ in Euclidean $3$-space in
terms of the parameters $\left(  x_{s},y_{s}\right)  $. [DS,49,231]. It is
\[%
%TCIMACRO{\dint \nolimits_{G_{c}}}%
%BeginExpansion
{\displaystyle\int\nolimits_{G_{c}}}
%EndExpansion
\hat{a}\left(  \frac{d\hat{X}}{dx_{s}},\frac{d\hat{X}}{dy_{s}}\right)
dx_{s}dy_{s}%
\]
where $\hat{a}\left(  \frac{d\hat{X}}{dx_{s}},\frac{d\hat{X}}{dy_{s}}\right)
$ is the (Euclidean) area of the parallelogram spanned by the two vectors
$\frac{d\hat{X}}{dx_{s}}$ and $\frac{d\hat{X}}{dy_{s}}$ in Euclidean
$3$-space. That is%
\[
\hat{a}\left(  \frac{d\hat{X}}{dx_{s}},\frac{d\hat{X}}{dy_{s}}\right)
=\left\vert \frac{d\hat{X}}{dx_{s}}\right\vert \text{\textperiodcentered
}\left\vert \frac{d\hat{X}}{dy_{s}}\right\vert \text{\textperiodcentered
}\mathrm{sin}\left(  \theta\right)
\]
where $\theta$ is the angle between the two vectors $\frac{d\hat{X}}{dx_{s}}$
and $\frac{d\hat{X}}{dy_{s}}$.

\begin{exercise}
As in Exercise \ref{9} show that%
\begin{align*}
\hat{a}\left(  \frac{d\hat{X}}{dx_{s}},\frac{d\hat{X}}{dy_{s}}\right)  ^{2}
&  =\left\vert
\begin{array}
[c]{cc}%
\frac{d\hat{X}}{dx_{s}}\bullet\frac{d\hat{X}}{dx_{s}} & \frac{d\hat{X}}%
{dy_{s}}\bullet\frac{d\hat{X}}{dx_{s}}\\
\frac{d\hat{X}}{dx_{s}}\bullet\frac{d\hat{X}}{dy_{s}} & \frac{d\hat{X}}%
{dy_{s}}\bullet\frac{d\hat{X}}{dy_{s}}%
\end{array}
\right\vert \\
&  =\left\vert
\begin{array}
[c]{cc}%
\frac{dX}{dx_{s}}\bullet_{K}\frac{dX}{dx_{s}} & \frac{dX}{dy_{s}}\bullet
_{K}\frac{dX}{dx_{s}}\\
\frac{dX}{dx_{s}}\bullet_{K}\frac{dX}{dy_{s}} & \frac{dX}{dy_{s}}\bullet
_{K}\frac{dX}{dy_{s}}%
\end{array}
\right\vert
\end{align*}

\end{exercise}

Now notice the matrix $D_{s}$ in Exercise \ref{37} is simply the $2\times3$
matrix whose rows are the vectors $\frac{dX}{dx_{s}}$ and $\frac{dX}{dy_{s}}$.

\begin{exercise}
Use Exercise \ref{36} to show that%
\[
\hat{a}\left(  \frac{d\hat{X}}{dx_{s}},\frac{d\hat{X}}{dy_{s}}\right)
^{2}=\rho^{4}=\frac{1}{\left(  \frac{K}{4}\left(  x_{s}^{2}+y_{s}^{2}\right)
+1\right)  ^{4}}.
\]

\end{exercise}

\pagebreak\pagebreak

\section{Relationship between central and stereographic projection
coordinates}

\subsection{Two addresses for the same point in $K$-geometry}

Finally we should compare the relationship between the two kinds of
coordinates for the set%
\begin{equation}
\left\{  \left(  x,y,z\right)  \in\mathbb{R}^{3}:1=K\left(  x^{2}%
+y^{2}\right)  +z^{2}\right\}  \label{40}%
\end{equation}
that we have been exploring, namely central projection coordinates%
\begin{align}
x_{c}  &  =\frac{x}{z}\label{70}\\
y_{c}  &  =\frac{y}{z}\nonumber
\end{align}
and stereographic projection coordinates%
\begin{align}
x_{s}  &  =\frac{2x}{z+1}\label{71}\\
y_{s}  &  =\frac{2y}{z+1}.\nonumber
\end{align}
To do this we use the Exercises in which we wrote $\left(  x,y,z\right)  $ in
the set $\left(  \ref{40}\right)  $ as functions of $\left(  x_{c}%
,y_{c}\right)  $ and $\left(  x_{s},y_{s}\right)  $ respectively. Namely%
\begin{align*}
x  &  =\frac{x_{c}}{\sqrt{K\left(  x_{c}^{2}+y_{c}^{2}\right)  +1}}\\
y  &  =\frac{y_{c}}{\sqrt{K\left(  x_{c}^{2}+y_{c}^{2}\right)  +1}}\\
z  &  =\frac{1}{\sqrt{K\left(  x_{c}^{2}+y_{c}^{2}\right)  +1}}%
\end{align*}
and%
\begin{align*}
x  &  =\frac{x_{s}}{1+\frac{K}{4}\left(  x_{s}^{2}+y_{s}^{2}\right)  }\\
y  &  =\frac{y_{s}}{1+\frac{K}{4}\left(  x_{s}^{2}+y_{s}^{2}\right)  }\\
z  &  =\frac{1-\frac{K}{4}\left(  x_{s}^{2}+y_{s}^{2}\right)  }{1+\frac{K}%
{4}\left(  x_{s}^{2}+y_{s}^{2}\right)  }.
\end{align*}
The rest is simple algebra. [MJG,258]

\begin{exercise}
Do the algebra to write the explicit formulas for%
\[
\left(  x_{s}\left(  x_{c},y_{c}\right)  ,y_{s}\left(  x_{c},y_{c}\right)
\right)
\]
and%
\[
\left(  x_{c}\left(  x_{s},y_{s}\right)  ,y_{c}\left(  x_{s},y_{s}\right)
\right)  .
\]
\pagebreak
\end{exercise}

\subsection{Plane sections of $K$-geometry}

\begin{exercise}
\label{73}Suppose we intersect $K$-geometry $\left(  \ref{40}\right)  $ with a
plane%
\[
ax+by+z=0.
\]


a) Find the equation for the resulting path in central projection coordinates.

b) Show that the equation for the resulting path in stereographic projection
coordinates is%
\[
\left(  x_{s}-\frac{2a}{K}\right)  ^{2}+\left(  y_{s}-\frac{2b}{K}\right)
^{2}=\frac{4\left(  K+a^{2}+b^{2}\right)  }{K^{2}}.
\]


c) What is the equation for the resulting path in stereographic projection
coordinates if we intersect the $K$-geometry with a plane given by%
\[
ax+by=0,
\]
that is, a plane containing the $z$-axis?\pagebreak
\end{exercise}

\subsection{When $K$ is negative, there are asymptotic cones}

Notice that, if $K<0$, the equation of $K$-geometry becomes%
\[
z^{2}-\left\vert K\right\vert \left(  x^{2}+y^{2}\right)  =1.
\]
Thus $K$-geometry forms a $2$-sheeted hyperboloid with the $z$-axis as major
axis. The hyerboloid is obtained by rotating the hyperbola%
\begin{equation}
z^{2}-\left\vert K\right\vert x^{2}=1 \label{hyp}%
\end{equation}
in the $\left(  x,z\right)  $-plane around the $z$-axis. We will only consider
the sheet on which $z$ is positive as forming the $K$-geometry.

Now the asymptotes of this last hyperbola are the pair of lines%
\[
\left(  z+\left\vert K\right\vert ^{1/2}x\right)  \left(  z+\left\vert
K\right\vert ^{1/2}x\right)  =z^{2}-\left\vert K\right\vert x^{2}=0.
\]
Rotating the asymptotes around the $z$-axis we obtain the \textit{asymptotic
cone}%
\[
z^{2}-\left\vert K\right\vert \left(  x^{2}+y^{2}\right)  =0
\]
for $K$-geometry.

Since the entire $K$-geometry lies inside the asymptotic cone, the central
projection coordinates $\left(  x_{c},y_{c}\right)  $ only correspond to
points in $K$-geometry when $\left(  x_{c},y_{c},1\right)  $ lies
\textit{inside} the asymptotic cone, that is, when%
\[
1-\left\vert K\right\vert \left(  x^{2}+y^{2}\right)  >0.
\]
Said otherwise, the disk of radius $\left\vert K\right\vert ^{-1/2}$ around
$\left(  0,0\right)  $ in the $\left(  x_{c},y_{c}\right)  $-plane captures
all the points $\left(  x,y,z\right)  $ of $K$-geometry under central projection.

\begin{exercise}
a) Show that the slopes of the asymptotes to the hyperbola $\left(
\ref{hyp}\right)  $ are the limits as $x$ goes to $\pm\infty$ of the slopes of
the lines through $\left(  0,0\right)  $ and the point $\left(  x,z\right)  $
on the hyperbola.

b) Show that the slopes of the asymptotes to the hyperbola $\left(
\ref{hyp}\right)  $ are the limits as $x$ goes to $\pm\infty$ of the slopes of
the lines through $\left(  0,-1\right)  $ and the point $\left(  x,z\right)  $
on the hyperbola.

c) Use b) to compute the radius of the disk around $\left(  0,0\right)  $ in
the $\left(  x_{s},y_{s}\right)  $-plane that captures all the points $\left(
x,y,z\right)  $ of $K$-geometry under stereographic projection.\pagebreak\ 
\end{exercise}

\part{Spherical geometry II\label{V}}

\section{Rigid motions in spherical geometry}

\subsection{Rigid motions in Euclidean coordinates}

If we have a curve $\hat{X}\left(  t\right)  =\left(  \hat{x}\left(  t\right)
,\hat{y}\left(  t\right)  ,\hat{z}\left(  t\right)  \right)  $ on the
$R$-sphere, we have seen that we measure its length $L$ by the formula%
\begin{equation}
L=%
%TCIMACRO{\dint \nolimits_{b}^{e}}%
%BeginExpansion
{\displaystyle\int\nolimits_{b}^{e}}
%EndExpansion
l\left(  t\right)  dt \label{54}%
\end{equation}
where
\begin{equation}
l\left(  t\right)  ^{2}=\frac{d\hat{X}}{dt}\bullet\frac{d\hat{X}}{dt}%
=\frac{dX}{dt}\bullet_{K}\frac{dX}{dt} \label{53}%
\end{equation}
and that we measure angles $\theta$ between tangent vectors $\hat{V}_{1}$ and
$\hat{V}_{2}$ at a point on the $R$-sphere by the formula%
\[
\theta=\mathrm{arccos}\left(  \frac{\hat{V}_{1}\bullet\hat{V}_{2}}{\left\vert
\hat{V}_{1}\right\vert \text{\textperiodcentered}\left\vert \hat{V}%
_{2}\right\vert }\right)  =\mathrm{arccos}\left(  \frac{V_{1}\bullet_{K}V_{2}%
}{\left\vert V_{1}\right\vert _{K}\text{\textperiodcentered}\left\vert
V_{2}\right\vert _{K}}\right)
\]
where%
\[
\left\vert V\right\vert _{K}^{2}=V\bullet_{K}V.
\]


Let $\hat{M}$ denote an invertible $3\times3$ matrix. We begin by noting the
condition that a transformation%
\[
\left(  \underline{\hat{x}},\underline{\hat{y}},\underline{\hat{z}}\right)
=\left(  \hat{x},\hat{y},\hat{z}\right)  \cdot\hat{M}%
\]
preserve the length of any curve lying on the $R$-sphere. Rewriting the
transformation as%
\begin{equation}
\underline{\hat{X}}=\hat{X}\cdot\hat{M} \label{trans1}%
\end{equation}
the formulas $\left(  \ref{54}\right)  $ and $\left(  \ref{53}\right)  $ show
that all we have to worry about is that%
\[
\frac{d\underline{\hat{X}}}{dt}\bullet\frac{d\underline{\hat{X}}}{dt}%
=\frac{d\hat{X}}{dt}\bullet\frac{d\hat{X}}{dt}%
\]
for all values $t$ of the parameter of the curve. But%
\begin{equation}
\frac{d\underline{\hat{X}}}{dt}=\frac{d\hat{X}}{dt}\cdot\hat{M} \label{trans2}%
\end{equation}
by the product rule since $\hat{M}$ is a constant matrix. So the
transformation given by the matrix $\hat{M}$ will preserve the length of any
path and will preserve the measure of any angle if%
\begin{equation}
\frac{d\hat{X}}{dt}\cdot\hat{M}\cdot\left(  \frac{d\hat{X}}{dt}\cdot\hat
{M}\right)  ^{t}=\frac{d\hat{X}}{dt}\cdot\left(  \frac{d\hat{X}}{dt}\right)
^{t}. \label{trans3}%
\end{equation}


\begin{exercise}
\label{trans}(\textbf{SG}) a) Show that the transformation $\left(
\ref{trans1}\right)  $ takes the $R$-sphere to itself if%
\[
\hat{M}\cdot\hat{M}^{t}=I
\]
where $I$ is the $3\times3$ identity matrix. (A matrix $\hat{M}$ satisfying
this condition is called an orthogonal matrix.)

b) Show that $\left(  \ref{trans3}\right)  $ also holds if $\hat{M}$ is orthogonal.
\end{exercise}

\begin{exercise}
(\textbf{SG}) Show that the matrix%
\[
\left(
\begin{array}
[c]{ccc}%
\mathrm{cos}\theta & \mathrm{sin}\theta & 0\\
-\mathrm{sin}\theta & \mathrm{cos}\theta & 0\\
0 & 0 & 1
\end{array}
\right)
\]
is orthogonal. Describe geometrically what this transformation is doing to the
$R$-sphere. [DS,316ff]
\end{exercise}

\begin{exercise}
(\textbf{SG}) Show that the matrix%
\[
\left(
\begin{array}
[c]{ccc}%
\mathrm{cos}\varphi & 0 & \mathrm{sin}\varphi\\
0 & 1 & 0\\
-\mathrm{sin}\varphi & 0 & \mathrm{cos}\varphi
\end{array}
\right)
\]
is orthogonal. Describe geometrically what this transformation is doing to the
$R$-sphere.
\end{exercise}

\begin{definition}
A distance-preserving transformation of a space or geometry is called a
\textbf{rigid motion} or a \textbf{congruence}.
\end{definition}

So, by Exercise \ref{trans}, every orthogonal matrix $\hat{M}$ corresponds to
a rigid motion of the $R$-sphere. (It can be shown that every rigid motion of
the $R$-sphere is given by an orthogonal matrix, but we will not treat that
subtlety in this book.)

\begin{exercise}
(\textbf{SG}) Show that the set of orthogonal matrices $\hat{M}$ form a group.
That is, show that

a) the product of two orthogonal matrices is orthogonal,

b) the identity matrix is orthogonal,

c) the inverse matrix $\hat{M}^{-1}$ of an orthogonal matrix $\hat{M}$ is orthogonal.

Hint: Write%
\[
\hat{M}\cdot\hat{M}^{-1}=I=\hat{M}\cdot\hat{M}^{t}%
\]
and use matrix multiplication to reduce to showing that the transpose of an
orthogonal matrix is orthogonal.) [MJG,311]
\end{exercise}

Our conclusion from this exercise is that the set of rigid motions of the
Euclidean $R$-sphere form a group.\pagebreak

\subsection{Orthogonal and $K$-orthogonal matrices}

Recalling again the fact that the Euclidean $R$-sphere is a $K$-geometry with
$K=1/R^{2}$ we should compare the $K$-orthogonal transformations $M$ in
Definition \ref{88} with the orthogonal ones just above.

\begin{exercise}
(\textbf{SG}) Referring to Definition \ref{88}, show that the transformations%
\[
\left(  \underline{\hat{x}},\underline{\hat{y}},\underline{\hat{z}}\right)
=\left(  \hat{x},\hat{y},\hat{z}\right)  \cdot\hat{M}%
\]
and%
\[
\left(  \underline{x},\underline{y},\underline{z}\right)  =\left(
x,y,z\right)  \cdot M
\]
give the same rigid motion of the $R$-sphere if
\[
M=\left(
\begin{array}
[c]{ccc}%
1 & 0 & 0\\
0 & 1 & 0\\
0 & 0 & R
\end{array}
\right)  \cdot\hat{M}\cdot\left(
\begin{array}
[c]{ccc}%
1 & 0 & 0\\
0 & 1 & 0\\
0 & 0 & R^{-1}%
\end{array}
\right)  .
\]

\end{exercise}

\begin{exercise}
(\textbf{SG}) Show that the matrix%
\[
\left(
\begin{array}
[c]{ccc}%
\mathrm{cos}\theta & \mathrm{sin}\theta & 0\\
-\mathrm{sin}\theta & \mathrm{cos}\theta & 0\\
0 & 0 & 1
\end{array}
\right)
\]
is both orthogonal and $K$-orthogonal and gives the same transformation of the
Euclidean $R$-sphere. Describe geometrically what this transformation is doing
to the $R$-sphere.
\end{exercise}

\begin{exercise}
(\textbf{SG}) Show that the matrix%
\[
\hat{M}=\left(
\begin{array}
[c]{ccc}%
\mathrm{cos}\varphi & 0 & \mathrm{sin}\varphi\\
0 & 1 & 0\\
-\mathrm{sin}\varphi & 0 & \mathrm{cos}\varphi
\end{array}
\right)
\]
is orthogonal and that the matrix%
\[
M=\left(
\begin{array}
[c]{ccc}%
\mathrm{cos}\varphi & 0 & R^{-1}\text{\textperiodcentered}\mathrm{sin}%
\varphi\\
0 & 1 & 0\\
-R\text{\textperiodcentered}\mathrm{sin}\varphi & 0 & \mathrm{cos}\varphi
\end{array}
\right)
\]
is the $K$-orthogonal matrix describing the same transformation of the
$R$-sphere. Describe geometrically what this transformation is doing to the
$R$-sphere.
\end{exercise}

\pagebreak

\section{Spherical geometry is homogeneous}

\subsection{Moving a point to the North Pole by a rigid motion}

As the heading suggests, we are next going to move any point on the $R$-sphere
to the North Pole by a rigid motion. However, we are going to describe the
entire process in $\left(  x,y,z\right)  $-coordinates, that is, in
$K$-geometry. This will allow us to use all the computations we did in Part
\ref{III} since \textbf{SG} is a $K$-geometry in the sense of Part \ref{III}.
Recll that, in $\left(  x,y,z\right)  $-coordinates, the equation for the
$R$-sphere becomes%
\begin{equation}
K\left(  x^{2}+y^{2}\right)  +z^{2}=1 \label{72}%
\end{equation}
with%
\[
K=\frac{1}{R^{2}}%
\]
and the Euclidean dot product is given by the $K$-dot product. Again, if you
get nervous using these weird coordinates to compute things that are clearer
in $\left(  \hat{x},\hat{y},\hat{z}\right)  $-coordinates, just go through the
constructions in the case $K=1$ first. In that special case
\[
\left(  x,y,z\right)  =\left(  \hat{x},\hat{y},\hat{z}\right)
\]
and your calculations (as well as all those in Part \ref{III} above reduce to
the usual ones on the unit sphere in ordinary Euclidean $3$-space.

So, first of all, in $\left(  x,y,z\right)  $-coordinates the North Pole is
the point%
\[
N=\left(  0,0,1\right)  .
\]
Suppose we start with a point%
\[
X_{0}=\left(  x_{0},y_{0},z_{0}\right)
\]
in the geometry, that is, satisfying the equation $\left(  \ref{72}\right)  $.

\begin{exercise}
(\textbf{SG}) Write an explicit $K$-rigid motion%
\[
M_{1}=\left(
\begin{array}
[c]{ccc}%
\mathrm{cos}\theta & \mathrm{sin}\theta & 0\\
-\mathrm{sin}\theta & \mathrm{cos}\theta & 0\\
0 & 0 & 1
\end{array}
\right)
\]
that takes the point $X_{0}$ to a point $X_{1}=\left(  x_{1},0,z_{0}\right)  $.

Hint: Start from the identity%
\[
\frac{-y_{0}}{\sqrt{x_{0}^{2}+y_{0}^{2}}}\text{\textperiodcentered}x_{0}%
+\frac{x_{0}}{\sqrt{x_{0}^{2}+y_{0}^{2}}}\text{\textperiodcentered}y_{0}=0
\]%
\[
\mathrm{sin}\theta\text{\textperiodcentered}x_{0}+\mathrm{cos}\theta
\text{\textperiodcentered}y_{0}=0
\]
and then show that there is a $\theta$ so that%
\begin{align*}
\mathrm{cos}\theta &  =\frac{x_{0}}{\sqrt{x_{0}^{2}+y_{0}^{2}}}\\
\mathrm{sin}\theta &  =\frac{-y_{0}}{\sqrt{x_{0}^{2}+y_{0}^{2}}}.
\end{align*}

\end{exercise}

\begin{exercise}
(\textbf{SG}) Write an explicit $K$-rigid motion%
\[
M_{2}=\left(
\begin{array}
[c]{ccc}%
\mathrm{cos}\varphi & 0 & R^{-1}\text{\textperiodcentered}\mathrm{sin}%
\varphi\\
0 & 1 & 0\\
-R\text{\textperiodcentered}\mathrm{sin}\varphi & 0 & \mathrm{cos}\varphi
\end{array}
\right)
\]
that takes the point $X_{1}=\left(  x_{1},0,z_{0}\right)  $ to $N=\left(
0,0,1\right)  $.
\end{exercise}

Using these last two Exercises we conclude that the transformation%
\[
\left(  \underline{x},\underline{y},\underline{z}\right)  =\left(
x,y,z\right)  \cdot\left(  M_{1}\cdot M_{2}\right)
\]
is a $K$-rigid motion (why?) and that%
\[
N=\left(  x_{0},y_{0},z_{0}\right)  \cdot\left(  M_{1}\cdot M_{2}\right)
\]
(why?).

\pagebreak

\subsection{Moving a (point, direction) to any other (point, direction) by a
rigid motion}

Let
\[
V_{2}=\left(  a_{2},b_{2},0\right)
\]
be a tangent vector to $K$-geometry at the North Pole $N$.

\begin{exercise}
(\textbf{SG}) Write an explicit $K$-rigid motion%
\[
M_{3}=\left(
\begin{array}
[c]{ccc}%
\mathrm{cos}\theta^{\prime} & \mathrm{sin}\theta^{\prime} & 0\\
-\mathrm{sin}\theta^{\prime} & \mathrm{cos}\theta^{\prime} & 0\\
0 & 0 & 1
\end{array}
\right)
\]
that takes $V_{2}$ to the vector%
\[
\left(  \sqrt{a_{2}^{2}+b_{2}^{2}},0,0\right)  =\left(  \sqrt{V_{2}\bullet
_{K}V_{2}},0,0\right)  .
\]
Why does the transformation given by $M_{3}$ leave the North Pole $N$ fixed?
\end{exercise}

Now suppose we have any point%
\[
X_{0}=\left(  x_{0},y_{0},z_{0}\right)
\]
in $K$-geometry and any $K$-tangent vector%
\[
V_{0}=\left(  a_{0},b_{0},c_{0}\right)
\]
at that point.

\begin{exercise}
(\textbf{SG}) Explain why the $K$-rigid motion%
\[
\left(  \underline{x},\underline{y},\underline{z}\right)  =\left(
x,y,z\right)  \cdot\left(  M_{1}\cdot M_{2}\cdot M_{3}\right)
\]
constructed over the last couple of sections takes the point $X_{0}$ to $N$
and the tangent vector $V_{0}$ to $\left(  \sqrt{V_{0}\bullet_{K}V_{0}%
},0,0\right)  $
\end{exercise}

Now suppose that $\left(  X_{0},V_{0}\right)  $ gives a point $X_{0}$ in
$K$-geometry and a tangent direction $V_{0}$ to $K$-geometry at $X_{0}$.
Suppose that $\left(  X_{0}^{\prime},V_{0}^{\prime}\right)  $ gives another
point in $K$-geometry and a tangent direction to $K$-geometry at
$X_{0}^{\prime}$. Finally suppose that%
\[
V_{0}\bullet_{K}V_{0}=V_{0}^{\prime}\bullet_{K}V_{0}^{\prime}.
\]
As above, find a $K$-rigid motion given by%
\[
M=\left(  M_{1}\cdot M_{2}\cdot M_{3}\right)
\]
taking $X_{0}$ to the North Pole and $V_{0}$ to $\left(  \sqrt{V_{0}%
\bullet_{K}V_{0}},0,0\right)  $. Similarly find a $K$-rigid motion given by%
\[
M^{\prime}=\left(  M_{1}^{\prime}\cdot M_{2}^{\prime}\cdot M_{3}^{\prime
}\right)
\]
taking $X_{0}^{\prime}$ to the North Pole and $V_{0}^{\prime}$ to $\left(
\sqrt{V_{0}^{\prime}\bullet_{K}V_{0}^{\prime}},0,0\right)  .$

\begin{exercise}
\label{64}(\textbf{SG}) Explain why the $K$-rigid motion given by%
\[
M\cdot\left(  M^{\prime}\right)  ^{-1}%
\]
takes $\left(  X_{0},V_{0}\right)  $ to $\left(  X_{0}^{\prime},\left(
\frac{\left\vert V_{0}\right\vert _{K}}{\left\vert V_{0}^{\prime}\right\vert
_{K}}\right)  \cdot V_{0}^{\prime}\right)  $.
\end{exercise}

By completing this Exercise we have shown that $K$-geometry looks the same at
each point and in each direction at that point. That is, we have shown that
each $K$-geometry is homogeneous.\pagebreak

\section{Lines in Spherical Geometry}

\subsection{Spherical coordinates, a shortest path from the North Pole}

We next will figure out what is the shortest path you can take between two
points on the Euclidean $R$-sphere. Again we will do our calculation using
only $\left(  x,y,z\right)  $-coordinates (since, as we have seen in $\left(
\ref{66}\right)  $ we won't have $\left(  \hat{x},\hat{y},\hat{z}\right)
$-coordinates when we get to Hyperbolic Geometry. For our purposes, it will be
convenient to use yet another set of coordinates for $K$-geometry, namely what
are commonly known as spherical coordinates:%
\begin{align}
x\left(  \sigma,\tau\right)   &  =R\text{\textperiodcentered}\mathrm{sin}%
\sigma\text{\textperiodcentered}\mathrm{cos}\tau\nonumber\\
y\left(  \sigma,\tau\right)   &  =R\text{\textperiodcentered}\mathrm{sin}%
\sigma\text{\textperiodcentered}\mathrm{sin}\tau\label{133}\\
z\left(  \sigma,\tau\right)   &  =\mathrm{cos}\sigma\nonumber
\end{align}


\begin{exercise}
(\textbf{SG}) Show that these spherical coordinates do actually parametrize
the $R$-sphere, that is, that%
\[
K\left(  x\left(  \sigma,\tau\right)  ^{2}+y\left(  \sigma,\tau\right)
^{2}\right)  +z\left(  \sigma,\tau\right)  ^{2}\equiv1
\]
for all $\left(  \sigma,\tau\right)  $.
\end{exercise}

Notice that you can write a path on the $R$-sphere by giving a path $\left(
\sigma\left(  t\right)  ,\tau\left(  t\right)  \right)  $ in the $\left(
\sigma,\tau\right)  $-plane. In fact, you can use $\sigma$ as the parameter
$t$ and just write
\begin{equation}
\left(  \sigma,\tau\left(  \sigma\right)  \right)  \label{62}%
\end{equation}
where $\tau$ is a function of $\sigma$. To write a path that starts at the
North Pole, just write%
\[
\left(  \sigma,\tau\left(  \sigma\right)  \right)  ,\;0\leq\sigma
\leq\varepsilon
\]
and demand that%
\[
\tau\left(  0\right)  =0.
\]
If you want the path to end on the plane $y=\hat{y}=0$, demand additionally
that%
\[
\tau\left(  \varepsilon\right)  =0.
\]
But if we are going to describe paths on the $R$-sphere by paths in the
$\left(  \sigma,\tau\right)  $-plane we are going to need to figure out the
$K$-dot product in $\left(  \sigma,\tau\right)  $-coordiates so that we can
compute the lengths of paths in these coordinates.

\begin{exercise}
\label{3333}(\textbf{SG}) a) Refering to $\left(  \ref{133}\right)  $compute
the $2\times3$ matrix
\[
D_{sph}=\left(
\begin{array}
[c]{ccc}%
\frac{dx}{d\sigma} & \frac{dy}{d\sigma} & \frac{dz}{d\sigma}\\
\frac{dx}{d\tau} & \frac{dy}{d\tau} & \frac{dz}{d\tau}%
\end{array}
\right)  .
\]


b) Show that, if a path in $K$-geometry is given by a path $\left(
\sigma\left(  t\right)  ,\tau\left(  t\right)  \right)  $ in the $\left(
\sigma,\tau\right)  $-plane,%
\[
\left(  \frac{dx}{dt},\frac{dy}{dt},\frac{dz}{dt}\right)  =\left(
\frac{d\sigma}{dt},\frac{d\tau}{dt}\right)  \cdot D_{sph}.
\]


c) For two paths in $K$-geometry given by paths $\left(  \sigma_{1}\left(
t\right)  ,\tau_{1}\left(  t\right)  \right)  $ and $\left(  \sigma_{2}\left(
t\right)  ,\tau_{2}\left(  t\right)  \right)  $ in the $\left(  \sigma
,\tau\right)  $-plane, use a) and b) to show that
\begin{gather*}
\left(  \frac{d\hat{x}_{1}}{dt},\frac{d\hat{y}_{1}}{dt},\frac{d\hat{z}_{1}%
}{dt}\right)  \cdot\left(  \frac{dx_{2}}{dt},\frac{d\hat{y}_{2}}{dt}%
,\frac{d\hat{z}_{2}}{dt}\right)  ^{t}=\\
\left(  \frac{dx_{1}}{dt},\frac{dy_{1}}{dt},\frac{dz_{1}}{dt}\right)
\cdot\left(
\begin{array}
[c]{ccc}%
1 & 0 & 0\\
0 & 1 & 0\\
0 & 0 & K^{-1}%
\end{array}
\right)  \cdot\left(  \frac{dx_{2}}{dt},\frac{dy_{2}}{dt},\frac{dz_{2}}%
{dt}\right)  ^{t}=\\
\left(  \frac{d\sigma_{1}}{dt},\frac{d\tau_{1}}{dt}\right)  \cdot\left(
\begin{array}
[c]{cc}%
K^{-1} & 0\\
0 & K^{-1}\mathrm{sin}^{2}\sigma
\end{array}
\right)  \cdot\left(  \frac{d\sigma_{2}}{dt},\frac{d\tau_{2}}{dt}\right)  ^{t}%
\end{gather*}


d) Explain why the definition%
\[
\left(  \frac{d\sigma_{1}}{dt},\frac{d\tau_{1}}{dt}\right)  \bullet
_{sph}\left(  \frac{d\sigma_{2}}{dt},\frac{d\tau_{2}}{dt}\right)  =\left(
\frac{d\sigma_{1}}{dt},\frac{d\tau_{1}}{dt}\right)  \cdot\left(
\begin{array}
[c]{cc}%
K^{-1} & 0\\
0 & K^{-1}\mathrm{sin}^{2}\sigma
\end{array}
\right)  \cdot\left(  \frac{d\sigma_{2}}{dt},\frac{d\tau_{2}}{dt}\right)  ^{t}%
\]
allows us to compute the dot product of two tangent vectors to the $R$-sphere
in Euclidean space if we just know the values of the two corresponding vectors
in the $\left(  \sigma,\tau\right)  $-plane.
\end{exercise}

\begin{exercise}
(\textbf{SG}) Show that the length $L$ of any path on the $R$-sphere given by%
\[
\left(  \sigma,\tau\left(  \sigma\right)  \right)  ,\;0\leq\sigma
\leq\varepsilon
\]
with%
\[
\tau\left(  0\right)  =0.
\]
and%
\[
\tau\left(  \varepsilon\right)  =0
\]
is given by the formula%
\[
L=R%
%TCIMACRO{\dint \nolimits_{0}^{\varepsilon}}%
%BeginExpansion
{\displaystyle\int\nolimits_{0}^{\varepsilon}}
%EndExpansion
\sqrt{\left(  1,\frac{d\tau}{d\sigma}\right)  \cdot\left(
\begin{array}
[c]{cc}%
1 & 0\\
0 & \mathrm{sin}^{2}\sigma
\end{array}
\right)  \cdot\left(  1,\frac{d\tau}{d\sigma}\right)  ^{t}}d\sigma.
\]


Hint: Use Exercise \ref{3333} with $t=\sigma$.
\end{exercise}

This last formula for $L$ lets us figure out the shortest path from $N=\left(
R\text{\textperiodcentered}\mathrm{sin}0\text{\textperiodcentered}%
\mathrm{cos}0,R\text{\textperiodcentered}\mathrm{sin}%
0\text{\textperiodcentered}\mathrm{sin}0,\mathrm{cos}0\right)  $ to $\left(
R\text{\textperiodcentered}\mathrm{sin}\varepsilon,0,\mathrm{cos}%
\varepsilon\right)  =\left(  R\text{\textperiodcentered}\mathrm{sin\varepsilon
}\text{\textperiodcentered}\mathrm{cos}0,R\text{\textperiodcentered
}\mathrm{sin}0\text{\textperiodcentered}\mathrm{sin}0,\mathrm{cos}%
\varepsilon\right)  $. Since%
\[
L=R%
%TCIMACRO{\dint \nolimits_{0}^{\varepsilon}}%
%BeginExpansion
{\displaystyle\int\nolimits_{0}^{\varepsilon}}
%EndExpansion
\sqrt{1+\mathrm{sin}^{2}\sigma\text{\textperiodcentered}\left(  \frac{d\tau
}{d\sigma}\right)  ^{2}}d\sigma
\]
and $\mathrm{sin}^{2}\sigma$ is is positive for almost all $\sigma\in\left[
0,\varepsilon\right]  $, $L$ is minimal only when $\frac{d\tau}{d\sigma}$ is
identically $0$. But this means that $\tau\left(  \sigma\right)  $ is a
constant function. Since $\tau\left(  0\right)  =0$, this means that
$\tau\left(  \sigma\right)  $ is identically $0$. So we have the shown the
following result.

\begin{theorem}
(\textbf{SG}) The shortest path on the $R$-sphere from the North Pole to a
point $\left(  x,y,z\right)  =\left(  R\text{\textperiodcentered}%
\mathrm{sin}\varepsilon,0,\mathrm{cos}\varepsilon\right)  $ is the path lying
in the plane $y=0$.
\end{theorem}

\pagebreak

\subsection{Shortest path between any two points}

We next prove the theorem that shows that shortest path on the surface of the
earth from Rio de Janeiro to Los Angeles is the one cut on the surface of the
earth by the plane that passes through the center of the earth and through Rio
and through Los Angeles. That is usually the route an airplane would take when
flying between the two cities.

\begin{theorem}
(\textbf{SG}) Given any two points $X_{1}=\left(  x_{1},y_{1},z_{1}\right)  $
and $X_{2}=\left(  x_{2},y_{2},z_{_{2}}\right)  $ in $K$-geometry, the
shortest path between the two points is the path cut out by the two equations%
\[
K\left(  x^{2}+y^{2}\right)  +z^{2}=1
\]%
\begin{equation}
\left\vert \left(
\begin{array}
[c]{ccc}%
x & y & z\\
x_{1} & y_{1} & z_{1}\\
x_{2} & y_{2} & z_{2}%
\end{array}
\right)  \right\vert =0, \label{63}%
\end{equation}
that is, the plane containing $\left(  0,0,0\right)  $ and $X_{1}$ and $X_{2}$.
\end{theorem}

\begin{proof}
By Exercise \ref{64} there is a $K$-rigid motion $M$ that takes $X_{1}$ to the
North Pole $N$ and $X_{2}$ to $\left(  K^{-1/2}sin\varepsilon,0,cos\varepsilon
\right)  $ for some $\varepsilon$. That is%
\[
X_{2}\cdot M=\left(  R\text{\textperiodcentered}\mathrm{sin}\varepsilon
,0,\mathrm{cos}\varepsilon\right)
\]
for some $\varepsilon$ since all points in $K$-geometry with \underline{$y$%
}$=0$ can be written as $\left(  R\text{\textperiodcentered}\mathrm{sin}%
\varepsilon,0,\mathrm{cos}\varepsilon\right)  $ for some $\varepsilon$. But%
\begin{align*}
\left\vert \left(
\begin{array}
[c]{ccc}%
\underline{x} & \underline{y} & \underline{z}\\
0 & 0 & 1\\
K^{-1/2}sin\varepsilon & 0 & cos\varepsilon
\end{array}
\right)  \right\vert  &  =\left\vert \left(
\begin{array}
[c]{ccc}%
x & y & z\\
x_{1} & y_{1} & z_{1}\\
x_{2} & y_{2} & z_{2}%
\end{array}
\right)  \cdot M\right\vert \\
&  =\left\vert \left(
\begin{array}
[c]{ccc}%
x & y & z\\
x_{1} & y_{1} & z_{1}\\
x_{2} & y_{2} & z_{2}%
\end{array}
\right)  \right\vert \cdot\left\vert M\right\vert =0,
\end{align*}
Since $\left\vert M\right\vert \neq0$ and $K^{-1/2}sin\varepsilon\neq0$ if
$\varepsilon<\pi$, $\left(  x,y,z\right)  $ lies in the plane $\left(
\ref{63}\right)  $ if and only if
\[
\underline{y}=0.
\]
Since $M$ is a $K$-rigid motion it must take the shortest path from $X_{1}$ to
$X_{2}$ to the shortest path from $X_{1}\cdot M=N$ to $X_{2}\cdot M=\left(
R\text{\textperiodcentered}\mathrm{sin}\varepsilon,0,\mathrm{cos}%
\varepsilon\right)  $. But we already know that the shortest path from
$X_{1}\cdot M$ to $X_{2}\cdot M$ is the one cut out by the plane $y=0$. But
that path comes from the path cut out by the plane given by equation $\left(
\ref{63}\right)  $. This path is called the \textit{great circular arc}
between $X_{1}$ and $X_{2}$.
\end{proof}

\begin{definition}
A \textbf{line} in \textbf{SG} will be a curve that extends infinitely in each
direction and has the property that, given any two points $X_{1}$ and $X_{2}$
on the path, the shortest path between $X_{1}$ and $X_{2}$ lies along that
curve. Lines in \textbf{SG} are usually called great circles on the
$R$-sphere. They are the intersections of the $R$-sphere with planes through
$\left(  0,0,0\right)  $.
\end{definition}

Letting $X_{2}$ approach $X_{1}$ along the great circular arc joining $X_{1}$
and $X_{2}$ we see that the solution set to equation $\left(  \ref{63}\right)
$ does not change. Taking a limit as $X_{2}$ approaches $X_{1}$, this set can
also be expressed as the solution set of the equation%
\[
\left\vert \left(
\begin{array}
[c]{ccc}%
x & y & z\\
x_{1} & y_{1} & z_{1}\\
a_{1} & b_{1} & c_{1}%
\end{array}
\right)  \right\vert =0
\]
where $\left(  a_{1},b_{1},c_{1}\right)  $ is a tangent vector at the point
$X_{1}$ pointing in the direction of $X_{2}.$\pagebreak

\section{Central projection in \textbf{SG}}

\subsection{Central projection preserves lines}

We all probably realize that you can't make a perfect map of the world; that
is, you can't make a map so that angles on the map are equal to the
corresponding angles on the sphere and straight lines on the map correspond to
great circular arcs on the sphere. We do the next best thing--we make two maps
of the sphere, one that has the property that angles are faithfully
represented and the other for which straight lines on the map correspond to
shortest paths on the sphere. We start with a simple way to make a map for
which straight lines on the map correspond to shortest paths on the sphere.
The map coordinates we use to do this are the \textit{central projection
coordinates} we learrned about in Part \ref{III}.

Now \textbf{SG} is a $K$-geometry in the sense of Part \ref{III} since, in
$\left(  x,y,z\right)  $-coordinates, the equation for the $R$-sphere becomes%
\[
K\left(  x^{2}+y^{2}\right)  +z^{2}=1
\]
with%
\[
K=\frac{1}{R^{2}}%
\]
and the Euclidean dot product is given by the $K$-dot product. So all the
computations of Part \ref{III} hold for Spherical Geometry as long as we
understand that we are computing it in $\left(  x,y,z\right)  $-coordinates.

\begin{exercise}
Show that central projection of a point on the $R$-sphere in $\left(  \hat
{x},\hat{y},\hat{z}\right)  $-space to the plane $\hat{z}=R$ is the same as
central projection of the corresponding point in $\left(  x,y,z\right)
$-coordinates to the plane $z=1$.

Hint: Recall $\left(  \ref{131}\right)  $ and write the corresponding relation
$\hat{r}\left(  \hat{x}_{c},\hat{y}_{c},R\right)  =\left(  \hat{x},\hat
{y},\hat{z}\right)  $ in $\left(  \hat{x},\hat{y},\hat{z}\right)
$-coordinates. Conclude that $\hat{r}=r$. (Why?)
\end{exercise}

\begin{exercise}
(\textbf{SG}) Show that lines (i.e. shortest paths in \textbf{SG}) correspond
under central projection to straight lines in the $\left(  x_{c},y_{c}\right)
$-coordinates.

Hint: See Exercise \ref{73}a). Or just write the equation for a line in
$\left(  x_{c},y_{c}\right)  $-coordinates and substitute $\left(
\ref{70}\right)  $. Then reverse the process.\pagebreak
\end{exercise}

\subsection{Spherical area computed in central projection coordinates}

Recall that in Part \ref{III} we learned how to compute the $K$-area in
$K$-geometry of a region $G$ given by a region $G_{c}$ in the $\left(
x_{c},y_{c}\right)  $-plane. That is, we learned how to compute the area of a
region $\hat{G}$ on the $R$-sphere given by a region $G_{c}$ in the $\left(
x_{c},y_{c}\right)  $-plane

\begin{exercise}
(\textbf{SG}) Show that, if a region $\hat{G}$ on the $R$-sphere is
parametrized by a region $G_{c}$ in $\left(  x_{c},y_{c}\right)
$-coordinates, then the area $\hat{A}$ of $\hat{G}$ is given by the formula%
\[
\hat{A}=%
%TCIMACRO{\dint \nolimits_{G_{c}}}%
%BeginExpansion
{\displaystyle\int\nolimits_{G_{c}}}
%EndExpansion
\left(  K\left(  x_{c}^{2}+y_{c}^{2}\right)  +1\right)  ^{-3/2}dx_{c}dy_{c}.
\]

\end{exercise}

\pagebreak

\section{Stereographic projection in \textbf{SG}}

\subsection{Stereographic projection preserves angles}

We now turn to a simple way to make a map of the $R$-sphere in such a way that
the measure of any angle on the map is exactly the same as the measure of the
corresponding angle on the $R$-sphere. The map coordinates that do the job are
the \textit{stereographic projection coordinates} that again we learned about
in Part \ref{III}.

\begin{exercise}
(\textbf{SG}) a) Compute the stereographic projection of a point on the
$R$-sphere in $\left(  \hat{x},\hat{y},\hat{z}\right)  $-space to the plane
$\hat{z}=R$.

b) Show that the coordinates $\left(  \hat{x}_{s},\hat{y}_{s}\right)  $ of the
stereographic projection of a point on the $R$-sphere in $\left(  \hat{x}%
,\hat{y},\hat{z}\right)  $-space to the plane $\hat{z}=R$ are the same as the
coordinates $\left(  x_{s},y_{s}\right)  $ of the stereographic projection of
the corresponding point in $\left(  x,y,z\right)  $-coordinates to the plane
$z=1$.

Hint: Reduce to showing that%
\[
R\left(  \frac{2\hat{x}}{\hat{z}+R},\frac{2\hat{y}}{\hat{z}+R}\right)
=\left(  \frac{2x}{z+1},\frac{2y}{z+1}\right)  .
\]

\end{exercise}

\begin{exercise}
(\textbf{SG}) a) Show that stereographic projection is conformal, that is,
that the angle between two paths through a point on the $R$-sphere in $\left(
\hat{x},\hat{y},\hat{z}\right)  $-space is the same as the usual (Euclidean)
angle between the corresponding two paths through the corresponding point in
the $\left(  x_{s},y_{s}\right)  $-plane.

Hint: From Exercise \ref{36} we know that, for tangent vectors $\hat{V}_{1}$
and $\hat{V}_{2}$ emanating from the same point on the $R$-sphere
\begin{align*}
\hat{V}_{1}\bullet\hat{V}_{2}  &  =V_{1}\bullet_{K}V_{2}\\
&  =V_{1}^{s}\bullet_{s}V_{2}^{s}\\
&  =\left(  V_{1}^{s}\right)  \cdot\left(
\begin{array}
[c]{cc}%
\rho^{2} & 0\\
0 & \rho^{2}%
\end{array}
\right)  \cdot\left(  V_{2}^{s}\right)  ^{t}.
\end{align*}


b) Draw a picture of an angle between two paths through a point on the
Euclidean $R$-sphere and the stereographic projection of that angle onto the
plane $\hat{z}=R$. Try to give an intuitive geometric explanation for why it
should have the same measure as the original angle.\pagebreak
\end{exercise}

\subsection{Areas of spherical triangles in stereographic projection
coordinates}

Using Exercise \ref{73}b) to show that lines in \textbf{SG} become circles
under stereographic projection unless the line in \textbf{SG} passes through
the North Pole (in which case it corresponds to a line through $\left(
x_{s},y_{s}\right)  =\left(  0,0\right)  $ in the $\left(  x_{s},y_{s}\right)
$-plane). Suppose a spherical triangle $T$ corresponds to a region $T_{s}$ in
$\left(  x_{s},y_{s}\right)  $-coordinates and the vertices of $T$ correspond
to $\left(  x_{s},y_{s}\right)  =\left(  -2,0\right)  $, $\left(  x_{s}%
,y_{s}\right)  =\left(  2,0\right)  $, and $\left(  x_{s},y_{s}\right)
=\left(  0,2\right)  $. So one side of $T_{s}$ lies on the line $y_{s}=0$.

\begin{exercise}
\label{st}a) Use Exercise \ref{73}b) to compute the equations for the other
two sides of $T_{s}$.

b) In the $\left(  x_{s},y_{s}\right)  $-plane, draw $T_{s}$ as accurately as
you can when $K=4$, then then when $K=\frac{1}{4}$.

c) Compute the area of $T$ in both cases in b).

Hint: You will need the radian measure of the angle at each of the vertices of
$T_{s}$. Why? To calculate these angles, calculate $\frac{dy_{s}}{dx_{s}}$ by
implicit differentiation of the equations in a), then take $\mathrm{arctan}%
\left(  \frac{dy_{s}}{dx_{s}}\right)  $ in radians. Your job will be easier if
you notice that the $y_{s}$-axis divides $T_{s}$ into two congruent isosceles triangles.
\end{exercise}

\begin{exercise}
Explain why we know from an Exercise in Part \ref{III} that in all of the
cases in Exercise \ref{st} the area of the spherical triangle $T$ is also
given by the formula%
\[%
%TCIMACRO{\dint \nolimits_{T_{s}}}%
%BeginExpansion
{\displaystyle\int\nolimits_{T_{s}}}
%EndExpansion
\frac{1}{\left(  1+\frac{K}{4}\left(  x_{s}^{2}+y_{s}^{2}\right)  \right)
^{2}}dx_{s}dy_{s}.
\]

\end{exercise}

\pagebreak

\part{Hyperbolic geometry\label{VI}}

\section{The curvature $K$ becomes negative}

\subsection{The world sheet and the light cone}

We now turn to the case in which the radius $R$ of the Euclidean $R$-sphere
goes to infinity and beyond! Of course that doesn't make any sense in $\left(
\hat{x},\hat{y},\hat{z}\right)  $-space but if we look at the $R$-sphere in
$\left(  x,y,z\right)  $-coordinates, it makes perfect sense because there the
equation of the $R$-sphere is%
\begin{equation}
K\left(  x^{2}+y^{2}\right)  +z^{2}=1 \label{81}%
\end{equation}
for $K=\frac{1}{R^{2}}$ so that $R$ going to infinity means that $K$ goes to
$0$ and `beyond' simply means that $K$ becomes negative. We have seen that all
we need to have a geometry with lengths, angles, areas and congruences is to
have a smooth set and a dot-product between vectors tangent to that set. Now
if $K$ becomes negative, our geometry becomes a hyperboloid of two sheets
(obtained by rotating a hyperbola in the $\left(  x,z\right)  $-plane with
major axis the $z$-axis around that axis). So that we have a connected
universe, we will only consider the `top' sheet (where $z>0$) as our
$K$-geometry. (In special relativity, this sheet might be called something
like the `world sheet.') If, instead of rotating a hyperbola around the
$z$-axis we rotate the asymptotes of the hyperbola around the $z$-axis, we
obtain a cone given by the equation%
\[
K\left(  x^{2}+y^{2}\right)  +z^{2}=0.
\]
(Again this might be called something like the `light cone.')

There is one potential problem we need to worry about when $K<0$, and it is
regarding the length of tangent vectors. Namely, our formulas for lengths
invove taking square roots of dot products of tangent vectors with themselves,
so those dot-products had better be positive (and only zero if the tangent
vector itself is the zero-vector.) Our $K$-dot product is given by the formula%
\[
\left(
\begin{array}
[c]{ccc}%
a & b & c
\end{array}
\right)  \cdot\left(
\begin{array}
[c]{ccc}%
1 & 0 & 0\\
0 & 1 & 0\\
0 & 0 & K^{-1}%
\end{array}
\right)  \cdot\left(
\begin{array}
[c]{c}%
a\\
b\\
c
\end{array}
\right)  .
\]
So, when $K<0$, it seems entirely possible that some tangent vector $V$ has
the property that $V\bullet_{K}V<0$. (Indeed that will always happen if $c$ is
sufficiently big and $x$ and $y$ are sufficiently small.
[MJG,241-242]\pagebreak

\subsection{Non-zero tangent vectors in \textbf{HG} have positive length}

\begin{exercise}
Suppose that $V$ emanates from $\left(  0,0,0\right)  $ in $\left(
x,y,z\right)  $-space.

a) Show that $V\bullet_{K}V=0$ if and only if $V$ points in a direction of the
light cone.

b) Show that $V\bullet_{K}V<0$ if and only if $V$ points in a direction inside
the light cone.

Hint: Use that the (Euclidean) angle $\theta$ that the light cone makes with
the plane $z=0$ is given by taking any point $\left(  x,y,z\right)  $ on the
light cone with $z>0$ and computing%
\[
\mathrm{tan}\left(  \theta\right)  =\frac{z}{\sqrt{x^{2}+y^{2}}}=\left\vert
K\right\vert ^{1/2}.
\]
Compute the angle that $V$ makes with the plane $z=0$ in a similar way.
\end{exercise}

Now our world sheet lies \textit{inside} the light cone but tangent vectors to
it point \textit{outside} the light cone. That is what saves our $K$-dot
product, as we see in the next Lemma.

\begin{lemma}
(\textbf{HG}) Let $V=\left(  a,b,c\right)  $ denote a vector that is tangent
to our $K$-geometry, that is, to the set $\left(  \ref{81}\right)  $. Then%
\[
V\bullet_{K}V\geq0
\]
and $V\bullet_{K}V=0$ if and only if $V=0$.
\end{lemma}

\begin{proof}
If $c=0$, then the assertion of the Lemma is obviously true. So we can assume
$c\neq0$. Notice, since $V$ is assumed to be a tangent vector at $\left(
x,y,z\right)  $, this means that $\left(  x,y,z\right)  $ is not the North
Pole so that
\[
x^{2}+y^{2}>0.
\]


Next replacing $V$ with $\frac{1}{c}\left(  V\right)  $ just multiplies
$V\bullet_{K}V$ by $\frac{1}{c^{2}}$ so it suffices to consider the case in
which%
\[
V=\left(  a,b,1\right)
\]
and we must show that%
\[
\left(  a^{2}+b^{2}\right)  +K^{-1}>0.
\]
Since $V$ is tangent to our $K$-geometry at some point $\left(  x,y,z\right)
$, we know by Exercise \ref{82} that $\left(  x,y,z\right)  \bullet_{K}V=0$,
that is,%
\[
ax+by+\frac{z}{K}=0.
\]
On the other hand%
\[
K\left(  x^{2}+y^{2}\right)  +z^{2}=1.
\]
Substituting this becomes%
\[
K\left(  x^{2}+y^{2}\right)  +K^{2}\left(  ax+by\right)  ^{2}=1.
\]
On the other hand%
\begin{align*}
\left(  ay-bx\right)  ^{2}  &  \geq0\\
\left(  ay\right)  ^{2}+\left(  bx\right)  ^{2}  &  \geq2abxy
\end{align*}
so that%
\begin{gather*}
K\left(  x^{2}+y^{2}\right)  +K^{2}\left(  \left(  ax\right)  ^{2}+\left(
by\right)  ^{2}+\left(  ay\right)  ^{2}+\left(  bx\right)  ^{2}\right)
\geq1\\
K\left(  x^{2}+y^{2}\right)  +K^{2}\left(  a^{2}+b^{2}\right)  \left(
x^{2}+y^{2}\right)  \geq1\\
K^{-1}+\left(  a^{2}+b^{2}\right)  \geq\frac{1}{K^{2}\left(  x^{2}%
+y^{2}\right)  }.
\end{gather*}
\pagebreak
\end{proof}

\section{Hyperbolic geometry is homogeneous}

\subsection{Rigid motions in $\left(  x,y,z\right)  $-coordinates}

Now \textbf{HG} is a $K$-geometry in the sense of Part \ref{III} since, in
$\left(  x,y,z\right)  $-coordinates, the equation for the $K$-geometry
becomes%
\begin{equation}
K\left(  x^{2}+y^{2}\right)  +z^{2}=1 \label{83}%
\end{equation}
with%
\[
K<0
\]
and the $K$-dot product. If we have a curve $X\left(  t\right)  =\left(
x\left(  t\right)  ,y\left(  t\right)  ,z\left(  t\right)  \right)  $ on the
$R$-sphere given in $K$-coordinates as%
\[
1=K\left(  x^{2}+y^{2}\right)  +z^{2},
\]
we have seen that we measure its length $L$ by the formula%
\begin{equation}
L=%
%TCIMACRO{\dint \nolimits_{b}^{e}}%
%BeginExpansion
{\displaystyle\int\nolimits_{b}^{e}}
%EndExpansion
l\left(  t\right)  dt \label{84}%
\end{equation}
where
\begin{equation}
l\left(  t\right)  ^{2}=\frac{dX}{dt}\bullet_{K}\frac{dX}{dt} \label{85}%
\end{equation}
and that we measure angles $\theta$ between tangent vectors $\hat{V}_{1}$ and
$\hat{V}_{2}$ at a point on the $R$-sphere by the formula%
\[
\theta=\mathrm{arccos}\left(  \frac{V_{1}\bullet_{K}V_{2}}{\left\vert
V_{1}\right\vert _{K}\text{\textperiodcentered}\left\vert V_{2}\right\vert
_{K}}\right)
\]
where%
\[
\left\vert V\right\vert _{K}^{2}=V\bullet_{K}V.
\]


We again want to explore the condition that a transformation%
\[
\left(  \underline{x},\underline{y},\underline{z}\right)  =\left(
x,y,z\right)  \cdot M
\]
preserve the length of any curve $\left(  x\left(  t\right)  ,y\left(
t\right)  ,z\left(  t\right)  \right)  $ lying on the $R$-sphere. Rewriting
the transformation as%
\[
\underline{X}=X\cdot M
\]
all we have to worry about is that%
\[
\frac{d\underline{X}}{dt}\bullet_{K}\frac{d\underline{X}}{dt}=\frac{dX}%
{dt}\bullet_{K}\frac{dX}{dt}.
\]
So, referring to Definition \ref{88} a transformation given by a matrix $M$
will preserve the length of any path and will preserve the measure of any
angle if $M$ is $K$-orthogonal.

\begin{exercise}
(\textbf{HG}) Show that the matrix%
\[
\left(
\begin{array}
[c]{ccc}%
\mathrm{cos}\theta & \mathrm{sin}\theta & 0\\
-\mathrm{sin}\theta & \mathrm{cos}\theta & 0\\
0 & 0 & 1
\end{array}
\right)
\]
is $K$-orthogonal. Describe geometrically what this transformation is doing to
the $K$-geometry.
\end{exercise}

For our second $K$-rigid motion in \textbf{HG} we will need a pair of
functions
\[
\left(  \mathrm{cosh}\sigma=\frac{e^{\sigma}+e^{-\sigma}}{2},\mathrm{sinh}%
\sigma=\frac{e^{\sigma}-e^{-\sigma}}{2}\right)
\]
that parametrize the unit hyperbola
\[
z^{2}-x^{2}=1
\]
in the same way that $\left(  \mathrm{cos}\sigma,\mathrm{sin}\sigma\right)  $
parametrize the unit circle. That is%
\[
\mathrm{cosh}^{2}\sigma-\mathrm{sinh}^{2}\sigma\equiv1.
\]


\begin{exercise}
(\textbf{HG}) Show that the matrix%
\[
\left(
\begin{array}
[c]{ccc}%
\mathrm{cosh}\varphi & 0 & \left\vert K\right\vert ^{1/2}%
\text{\textperiodcentered}\mathrm{sinh}\varphi\\
0 & 1 & 0\\
\left\vert K\right\vert ^{-1/2}\text{\textperiodcentered}\mathrm{sinh}\varphi
& 0 & \mathrm{cosh}\varphi
\end{array}
\right)
\]
is $K$-orthogonal. Describe geometrically what this transformation is doing to
the the $K$-geometry.
\end{exercise}

Notice that, when $K>0$, we had the relation%
\[
R^{2}\cdot K=1
\]
where $R$ was the radius of the sphere. The last exercise suggests that when
$K<0$, we should define $R$ by the relation%
\[
R^{2}\cdot\left\vert K\right\vert =1
\]
so that%
\[
R=\left\vert K\right\vert ^{-1/2}.
\]
(Now compare this last Exercise with the corresponding Exercise in Spherical
Geometry.) In what follows, you will find the quantity $\left\vert
K\right\vert ^{-1/2}$ occurring throughout. Feel free to use the (simpler)
notation $R$ for this quantity as you work through the Exercises.\pagebreak

\subsection{Moving a point to the North Pole by a rigid motion}

So, first of all, the North Pole is the point%
\[
N=\left(  0,0,1\right)  .
\]
Suppose we start with a point%
\[
X_{0}=\left(  x_{0},y_{0},z_{0}\right)
\]
in the geometry, that is, satisfying the equation $\left(  \ref{83}\right)  $.

\begin{exercise}
(\textbf{HG}) Write an explicit $K$-rigid motion%
\[
M_{1}=\left(
\begin{array}
[c]{ccc}%
\mathrm{cos}\theta & \mathrm{sin}\theta & 0\\
-\mathrm{sin}\theta & \mathrm{cos}\theta & 0\\
0 & 0 & 1
\end{array}
\right)
\]
that takes the point $X_{0}$ to a point $X_{1}=\left(  x_{1},0,z_{0}\right)  $.
\end{exercise}

\begin{exercise}
(\textbf{HG}) Write an explicit $K$-rigid motion%
\[
M_{2}=\left(
\begin{array}
[c]{ccc}%
\mathrm{cosh}\varphi & 0 & \left\vert K\right\vert ^{1/2}%
\text{\textperiodcentered}\mathrm{sinh}\varphi\\
0 & 1 & 0\\
\left\vert K\right\vert ^{-1/2}\text{\textperiodcentered}\mathrm{sinh}\varphi
& 0 & \mathrm{cosh}\varphi
\end{array}
\right)
\]
that takes the point $X_{1}=\left(  x_{1},0,z_{0}\right)  $ to $N=\left(
0,0,1\right)  $.

Hint: Notice that%
\[
Kx_{1}^{2}+z_{0}^{2}=1=-\left(  -\left\vert K\right\vert ^{1/2}%
\text{\textperiodcentered}x_{1}\right)  ^{2}+z_{0}^{2}.
\]
So there is a $\varphi$ with
\[
\mathrm{cosh}\varphi=z_{0}%
\]
and%
\[
\mathrm{sinh}\varphi=-\left\vert K\right\vert ^{1/2}\text{\textperiodcentered
}x_{1}.
\]

\end{exercise}

Using these last two Exercises we conclude that the transformation%
\[
\left(  \underline{x},\underline{y},\underline{z}\right)  =\left(
x,y,z\right)  \cdot\left(  M_{1}\cdot M_{2}\right)
\]
is a $K$-rigid motion (why?) and that%
\[
N=\left(  x_{0},y_{0},z_{0}\right)  \cdot\left(  M_{1}\cdot M_{2}\right)
\]
(why?).

\pagebreak

\subsection{Moving a (point, direction) to any other (point, direction) by a
rigid motion}

Let
\[
V_{2}=\left(  a_{2},b_{2},0\right)
\]
be a tangent vector to $K$-geometry at the North Pole $N$.

\begin{exercise}
(\textbf{HG}) Write an explicit $K$-rigid motion%
\[
M_{3}=\left(
\begin{array}
[c]{ccc}%
\mathrm{cos}\theta^{\prime} & \mathrm{sin}\theta^{\prime} & 0\\
-\mathrm{sin}\theta^{\prime} & \mathrm{cos}\theta^{\prime} & 0\\
0 & 0 & 1
\end{array}
\right)
\]
that takes $V_{2}$ to the vector%
\[
\left(  \sqrt{a_{2}^{2}+b_{2}^{2}},0,0\right)  =\left(  \sqrt{V_{2}\bullet
_{K}V_{2}},0,0\right)  .
\]
Why does the transformation given by $M_{3}$ leave the North Pole $N$ fixed?
\end{exercise}

Now suppose we have any point%
\[
X_{0}=\left(  x_{0},y_{0},z_{0}\right)
\]
in $K$-geometry and any $K$-tangent vector%
\[
V_{0}=\left(  a_{0},b_{0},c_{0}\right)
\]
at that point.

\begin{exercise}
(\textbf{HG}) Explain why the $K$-rigid motion%
\[
\left(  \underline{x},\underline{y},\underline{z}\right)  =\left(
x,y,z\right)  \cdot\left(  M_{1}\cdot M_{2}\cdot M_{2}\right)
\]
constructed over the last couple of sections takes the point $X_{0}$ to $N$
and the tangent vector $V_{0}$ to $\left(  \sqrt{V_{0}\bullet_{K}V_{0}%
},0,0\right)  $
\end{exercise}

Now suppose that $\left(  X_{0},V_{0}\right)  $ gives a point $X_{0}$ in
$K$-geometry and a tangent direction $V_{0}$ to $K$-geometry at $X_{0}$.
Suppose that $\left(  X_{0}^{\prime},V_{0}^{\prime}\right)  $ gives another
point in $K$-geometry and a tangent direction to $K$-geometry at
$X_{0}^{\prime}$. Finally suppose that%
\[
V_{0}\bullet_{K}V_{0}=V_{0}^{\prime}\bullet_{K}V_{0}^{\prime}.
\]
As above, find a $K$-rigid motion given by%
\[
M=\left(  M_{1}\cdot M_{2}\cdot M_{3}\right)
\]
taking $X_{0}$ to the North Pole and $V_{0}$ to $\left(  \sqrt{V_{0}%
\bullet_{K}V_{0}},0,0\right)  $. Similarly find a $K$-rigid motion given by%
\[
M^{\prime}=\left(  M_{1}^{\prime}\cdot M_{2}^{\prime}\cdot M_{3}^{\prime
}\right)
\]
taking $X_{0}^{\prime}$ to the North Pole and $V_{0}^{\prime}$ to $\left(
\sqrt{V_{0}^{\prime}\bullet_{K}V_{0}^{\prime}},0,0\right)  .$

\begin{exercise}
(\textbf{HG}) Explain why the $K$-rigid motion given by%
\[
M\cdot\left(  M^{\prime}\right)  ^{-1}%
\]
takes $\left(  X_{0},V_{0}\right)  $ to $\left(  X_{0}^{\prime},V_{0}^{\prime
}\right)  $ as long as $\sqrt{V_{0}\bullet_{K}V_{0}}=\sqrt{V_{0}^{\prime
}\bullet_{K}V_{0}^{\prime}}$.
\end{exercise}

By completing this Exercise we have shown that \textbf{HG} looks the same at
each point and in each direction at that point. That is, we have shown that
\textbf{HG} is homogeneous.\pagebreak

\section{Lines in Hyperbolic Geometry}

\subsection{Hyperbolic coordinates, a shortest path from the North Pole}

We next will figure out what is the shortest path you can take between two
points in \textbf{HG}. Again we will do our calculation using only $\left(
x,y,z\right)  $-coordinates (since, as we have seen in $\left(  \ref{66}%
\right)  $ we don't have $\left(  \hat{x},\hat{y},\hat{z}\right)
$-coordinates). The $\left(  x,y,z\right)  $-coordinates for \textbf{SG},
namely
\begin{align*}
x\left(  \sigma,\tau\right)   &  =R\text{\textperiodcentered}\mathrm{sin}%
\sigma\text{\textperiodcentered}\mathrm{cos}\tau\\
y\left(  \sigma,\tau\right)   &  =R\text{\textperiodcentered}\mathrm{sin}%
\sigma\text{\textperiodcentered}\mathrm{sin}\tau\\
z\left(  \sigma,\tau\right)   &  =\mathrm{cos}\sigma
\end{align*}
won't work this time because they involve $R$ which has gone off to infinity.
Fortunately there are hyperbolic coordinates%
\[
\left(  \mathrm{cosh}\sigma=\frac{e^{\sigma}+e^{-\sigma}}{2},\mathrm{sinh}%
\sigma=\frac{e^{\sigma}-e^{-\sigma}}{2}\right)
\]
that parametrize the `unit' hyperbola just like $\left(  \mathrm{cos}%
\sigma,\mathrm{sin}\sigma\right)  $ parametrize the unit circle. So we define%
\begin{align*}
x\left(  \sigma,\tau\right)   &  =\left\vert K\right\vert ^{-1/2}%
\text{\textperiodcentered}\mathrm{sinh}\sigma\text{\textperiodcentered
}\mathrm{cos}\tau\\
y\left(  \sigma,\tau\right)   &  =\left\vert K\right\vert ^{-1/2}%
\text{\textperiodcentered}\mathrm{sinh}\sigma\text{\textperiodcentered
}\mathrm{sin}\tau\\
z\left(  \sigma,\tau\right)   &  =\mathrm{cosh}\sigma
\end{align*}


\begin{exercise}
(\textbf{HG}) Show that these hyperbolic coordinates do actually parametrize
the $K$-geometry, that is, that%
\[
K\left(  x\left(  \sigma,\tau\right)  ^{2}+y\left(  \sigma,\tau\right)
^{2}\right)  +z\left(  \sigma,\tau\right)  ^{2}\equiv1
\]
for all $\left(  \sigma,\tau\right)  $.
\end{exercise}

Again notice that you can write a path on the $R$-sphere by giving a path
$\left(  \sigma\left(  t\right)  ,\tau\left(  t\right)  \right)  $ in the
$\left(  \sigma,\tau\right)  $-plane. In fact, you can use $\sigma$ as the
parameter $t$ and just write
\[
\left(  \sigma,\tau\left(  \sigma\right)  \right)
\]
where $\tau$ is a function of $\sigma$. To write a path that starts at the
North Pole, just write%
\[
\left(  \sigma,\tau\left(  \sigma\right)  \right)  ,\;0\leq\sigma
\leq\varepsilon
\]
and demand that%
\[
\tau\left(  0\right)  =0.
\]
If you want the path to end on the plane $y=0$, demand additionally that%
\[
\tau\left(  \varepsilon\right)  =0.
\]
But if we are going to describe paths on \textbf{HG} by paths in the $\left(
\sigma,\tau\right)  $-plane we are going to need to figure out the $K$-dot
product in $\left(  \sigma,\tau\right)  $-coordiates so that we can compute
the lengths of paths in these coordinates.

\begin{exercise}
(\textbf{HG}) a) Compute the $2\times3$ matrix $D_{hyp}$ such that%
\[
\left(  \frac{dx}{dt},\frac{dy}{dt},\frac{dz}{dt}\right)  =\left(
\frac{d\sigma}{dt},\frac{d\tau}{dt}\right)  \cdot D_{hyp}%
\]
when a path in $K$-geometry is given by a path in the $\left(  \sigma
,\tau\right)  $-plane.

Hint: By the Chain Rule from several variable calculus%
\[
D_{hyp}=\left(
\begin{array}
[c]{ccc}%
\frac{dx}{d\sigma} & \frac{dy}{d\sigma} & \frac{dz}{d\sigma}\\
\frac{dx}{d\tau} & \frac{dy}{d\tau} & \frac{dz}{d\tau}%
\end{array}
\right)  .
\]


b) Use a) to compute the $K$-dot product in $\left(  \sigma,\tau\right)
$-coordinates, namely compute the matrix $P_{hyp}$ in the equation%
\begin{align*}
\left(  \frac{d\sigma_{1}}{dt},\frac{d\tau_{1}}{dt}\right)  \bullet
_{hyp}\left(  \frac{d\sigma_{2}}{dt},\frac{d\tau_{2}}{dt}\right)   &  =\left(
\frac{dx_{1}}{dt},\frac{dy_{1}}{dt},\frac{dz_{1}}{dt}\right)  \bullet
_{K}\left(  \frac{dx_{2}}{dt},\frac{dy_{2}}{dt},\frac{dz_{2}}{dt}\right) \\
&  =\left(  \frac{dx_{1}}{dt},\frac{dy_{1}}{dt},\frac{dz_{1}}{dt}\right)
\cdot\left(
\begin{array}
[c]{ccc}%
1 & 0 & 0\\
0 & 1 & 0\\
0 & 0 & K^{-1}%
\end{array}
\right)  \cdot\left(  \frac{dx_{2}}{dt},\frac{dy_{2}}{dt},\frac{dz_{2}}%
{dt}\right)  ^{t}\\
&  =\left(  \frac{d\sigma_{1}}{dt},\frac{d\tau_{1}}{dt}\right)  \cdot
D_{hyp}\cdot\left(
\begin{array}
[c]{ccc}%
1 & 0 & 0\\
0 & 1 & 0\\
0 & 0 & K^{-1}%
\end{array}
\right)  \cdot D_{hyp}^{t}\cdot\left(  \frac{d\sigma_{1}}{dt},\frac{d\tau_{1}%
}{dt}\right)  ^{t}\\
&  =\left(  \frac{d\sigma_{1}}{dt},\frac{d\tau_{1}}{dt}\right)  \cdot
P_{hyp}\cdot\left(  \frac{d\sigma_{1}}{dt},\frac{d\tau_{1}}{dt}\right)  ^{t}.
\end{align*}

\end{exercise}

\begin{exercise}
(\textbf{HG}) Show that the length $L$ of any path in our $K$-geometry is
given by%
\[
\left(  \sigma,\tau\left(  \sigma\right)  \right)  ,\;0\leq\sigma
\leq\varepsilon
\]
with%
\[
\tau\left(  0\right)  =0.
\]
and%
\[
\tau\left(  \varepsilon\right)  =0
\]
is given by the formula%
\[
L=\left\vert K\right\vert ^{-1/2}%
%TCIMACRO{\dint \nolimits_{0}^{\varepsilon}}%
%BeginExpansion
{\displaystyle\int\nolimits_{0}^{\varepsilon}}
%EndExpansion
\sqrt{\left(  1,\frac{d\tau}{d\sigma}\right)  \cdot\left(
\begin{array}
[c]{cc}%
1 & 0\\
0 & \mathrm{sinh}^{2}\sigma
\end{array}
\right)  \cdot\left(  1,\frac{d\tau}{d\sigma}\right)  ^{t}}d\sigma.
\]

\end{exercise}

This last formula for $L$ lets us figure out the shortest path from $N=\left(
\mathrm{sinh}0\text{\textperiodcentered}\mathrm{cos}%
0,R\text{\textperiodcentered}\mathrm{sinh}0\text{\textperiodcentered
}\mathrm{sin}0,\mathrm{cosh}0\right)  $ to $\left(  \left\vert K\right\vert
^{-1/2}\text{\textperiodcentered}\mathrm{sinh}\varepsilon,0,\mathrm{cosh}%
\varepsilon\right)  =\left(  \left\vert K\right\vert ^{-1/2}%
\text{\textperiodcentered}\mathrm{sin\varepsilon}\text{\textperiodcentered
}\mathrm{cos}0,\left\vert K\right\vert ^{-1/2}\text{\textperiodcentered
}\mathrm{sin}0\text{\textperiodcentered}\mathrm{sin}0,\mathrm{cosh}%
\varepsilon\right)  $. Since%
\[
L=\left\vert K\right\vert ^{-1/2}\text{\textperiodcentered}%
%TCIMACRO{\dint \nolimits_{0}^{\varepsilon}}%
%BeginExpansion
{\displaystyle\int\nolimits_{0}^{\varepsilon}}
%EndExpansion
\sqrt{1+\mathrm{sinh}^{2}\sigma\text{\textperiodcentered}\left(  \frac{d\tau
}{d\sigma}\right)  ^{2}}d\sigma
\]
and $\mathrm{sinh}^{2}\sigma$ is is positive for almost all $\sigma\in\left[
0,\varepsilon\right]  $, $L$ is minimal only when $\frac{d\tau}{d\sigma}$ is
identically $0$. But this means that $\tau\left(  \sigma\right)  $ is a
constant function. Since $\tau\left(  0\right)  =0$, this means that
$\tau\left(  \sigma\right)  $ is identically $0$. So we have the shown the
following result.

\begin{theorem}
(\textbf{HG}) The shortest path in $K$-geometry from the North Pole to a point
$\left(  x,y,z\right)  =\left(  \left\vert K\right\vert ^{-1/2}%
\text{\textperiodcentered}\mathrm{sinh}\varepsilon,0,\mathrm{cosh}%
\varepsilon\right)  $ is the path lying in the plane $y=0$. The $K$-length of
that shortest path is%
\[
\left\vert K\right\vert ^{-1/2}\text{\textperiodcentered}\varepsilon.
\]
\pagebreak
\end{theorem}

\subsection{Shortest path between any two points}

\begin{theorem}
(\textbf{HG}) Given any two points $X_{1}=\left(  x_{1},y_{1},z_{1}\right)  $
and $X_{2}=\left(  x_{2},y_{2},z_{2}\right)  $ in $K$-geometry, the shortest
path between the two points is the path cut out by the two equations%
\[
K\left(  x^{2}+y^{2}\right)  +z^{2}=1
\]
and the plane%
\begin{equation}
\left\vert \left(
\begin{array}
[c]{ccc}%
x & y & z\\
x_{1} & y_{1} & z_{1}\\
x_{2} & y_{2} & z_{2}%
\end{array}
\right)  \right\vert =0, \label{90}%
\end{equation}
that is, the plane containing $\left(  0,0,0\right)  $ and $X_{1}$ and $X_{2}$.
\end{theorem}

\begin{proof}
Let $V_{1}=\left(  a_{1},b_{1},c_{1}\right)  $ be the tangent vector at
$X_{1}$ of $K$-length $1$ that is tangent to the path cut out by the plane
given by equation $\left(  \ref{90}\right)  $. Then $\left(  x,y,z\right)
=\left(  a_{1},b_{1},c_{1}\right)  $ also satisfies equation $\left(
\ref{90}\right)  $ and so the equation for that plane can also be written%
\begin{equation}
\left\vert \left(
\begin{array}
[c]{ccc}%
x & y & z\\
x_{1} & y_{1} & z_{1}\\
a_{1} & b_{1} & c_{1}%
\end{array}
\right)  \right\vert =0. \label{91}%
\end{equation}
By Exercise \ref{64} there is a $K$-rigid motion $M$ that takes $X_{1}$ to the
North Pole $N$ and $V_{1}$ to $\left(  1,0,0\right)  $. So $M$ takes the plane
$\left(  \ref{91}\right)  $ to the plane given by the equation%
\[
\left\vert \left(
\begin{array}
[c]{ccc}%
x & y & z\\
0 & 0 & 1\\
1 & 0 & 0
\end{array}
\right)  \right\vert =0,
\]
namely the plane.%
\[
y=0.
\]
So $X_{2}\cdot M$ must also line in this plane since $X_{2}$ lies in the plane
$\left(  \ref{91}\right)  $. So%
\[
X_{2}\cdot M=\left(  \left\vert K\right\vert ^{-1/2}\text{\textperiodcentered
}\mathrm{sinh}\varepsilon,0,\mathrm{cosh}\varepsilon\right)
\]
for some $\varepsilon$ since all points in $K$-geometry with $y=0$ can be
written as $\left(  \left\vert K\right\vert ^{-1/2}\text{\textperiodcentered
}\mathrm{sinh}\varepsilon,0,\mathrm{cosh}\varepsilon\right)  $ for some
$\varepsilon$. Since $M$ is a $K$-rigid motion it must take the shortest path
from $X_{1}$ to $X_{2}$ to the shortest path from $X_{1}\cdot M=N$ to
$X_{2}\cdot M=\left(  \left\vert K\right\vert ^{-1/2}\text{\textperiodcentered
}\mathrm{sinh}\varepsilon,0,\mathrm{cosh}\varepsilon\right)  $. But we already
know that the shortest path from $X_{1}\cdot M$ to $X_{2}\cdot M$ is the one
cut out by the plane $y=0$. But that path comes from the path cut out by the
plane given by equation $\left(  \ref{91}\right)  $, or, what is the same
thing, the plane given by the equation $\left(  \ref{90}\right)  $. This path
is called the \textit{great hyperbolic arc} between $X_{1}$ and $X_{2}$.
\end{proof}

\begin{definition}
A \textbf{line} in \textbf{HG} will be a curve that extends infinitely in each
direction and has the property that, given any two points $X_{1}$ and $X_{2}$
on the path, the shortest path between $X_{1}$ and $X_{2}$ lies along that
curve. Lines in \textbf{HG} are the intersections of the $K$-geometry with
planes through $\left(  0,0,0\right)  $. The length of the shortest path
between two points in $K$-geometry will be called the $K$-distance.\pagebreak
\end{definition}

\section{Central projection in \textbf{HG}}

\subsection{The edge of the universe}

Again \textbf{HG} is a $K$-geometry in the sense of Part \ref{III} since, in
$\left(  x,y,z\right)  $-coordinates, the equation for the $K$-geometry
\begin{equation}
K\left(  x^{2}+y^{2}\right)  +z^{2}=1 \label{74}%
\end{equation}
with%
\[
K<0
\]
and the $K$-dot product. So all the calculations in Part \ref{III} hold, in
particular $\left(  \ref{75}\right)  $. So the $\left(  x_{c},y_{c}\right)
$-coordinates that parametrize the entire $K$-geometry are the $\left(
x_{c},y_{c}\right)  $ such that%
\[
x_{c}^{2}+y_{c}^{2}<\frac{1}{\left\vert K\right\vert }.
\]
So we will call the circle%
\[
x_{c}^{2}+y_{c}^{2}=\frac{1}{\left\vert K\right\vert }%
\]
the \textit{edge of the universe}. (The $\left(  x_{c},y_{c}\right)
$-coordinates are called Klein coordinates for hyperbolic geometry and the
disk of radius $\left\vert K\right\vert ^{-1/2}$ is called the \textit{Klein
model} for \textbf{HG} in honor of the famous German geometer, Felix
Klein.)\pagebreak

\subsection{Lines go to chords}

Again all the calculations in Part \ref{III} hold, in particular Exercise
\ref{73}a). We conclude that lines in \textbf{HG} correspond to chords on the
Klein $\left(  x_{c},y_{c}\right)  $-disk that connect two points on the edge
of the universe.lines in the $\left(  x_{c},y_{c}\right)  $-disk.

\begin{exercise}
(\textbf{HG}) a) Explain why the $K$-line $y=0$ is given by the $x_{c}$-axis
and the North Pole $N$ is given by $\left(  x_{c},y_{c}\right)  =\left(
0,0\right)  $.

b) Explain why the point $\left(  \left\vert K\right\vert ^{-1/2}%
\text{\textperiodcentered}\mathrm{sinh}\varepsilon,0,\mathrm{cosh}%
\varepsilon\right)  $ in $K$-geometry is given by the point%
\[
\left(  x_{c},y_{c}\right)  =\left(  \left\vert K\right\vert ^{-1/2}%
\text{\textperiodcentered}\frac{e^{\varepsilon}-e^{-\varepsilon}%
}{e^{\varepsilon}+e^{-\varepsilon}},0\right)  .
\]


c) Explain why the $K$-distance between $\left(  x_{c},y_{c}\right)  =\left(
0,0\right)  $ and $\left(  x_{c},y_{c}\right)  =\left(  \left\vert
K\right\vert ^{-1/2}\text{\textperiodcentered}\frac{e^{\varepsilon
}-e^{-\varepsilon}}{e^{\varepsilon}+e^{-\varepsilon}},0\right)  $ is
$\left\vert K\right\vert ^{-1/2}$\textperiodcentered$\varepsilon$.
\end{exercise}

\begin{exercise}
(\textbf{HG}) Explain why lines in HG extend infinitely in each direction.

Hint: There is a $K$-rigid motion that takes any two points to $\left(
0,0\right)  $ and $\left(  \left\vert K\right\vert ^{-1/2}%
\text{\textperiodcentered}\frac{e^{\varepsilon}-e^{-\varepsilon}%
}{e^{\varepsilon}+e^{-\varepsilon}},0\right)  $ for some $\varepsilon
>0$.\pagebreak
\end{exercise}

\subsection{$K$-perpendicularity in the Klein model for \textbf{HG}}

Suppose we are given any three distinct points $P^{\prime}$, $R^{\prime}$ and
$Q^{\prime}$ on the edge of the universe of the Klein $K$-disk. We construct
the line $L^{\prime}$ through $P^{\prime}$ and $Q^{\prime}$ and mark a point
$A^{\prime}$ on it as shown in the figure below.
\[%
%TCIMACRO{\FRAME{itbpF}{1.1225in}{0.9954in}{0pt}{}{}{Figure}%
%{\special{ language "Scientific Word";  type "GRAPHIC";
%maintain-aspect-ratio TRUE;  display "USEDEF";  valid_file "T";
%width 1.1225in;  height 0.9954in;  depth 0pt;  original-width 8.1759in;
%original-height 7.2523in;  cropleft "0";  croptop "1";  cropright "1";
%cropbottom "0";  tempfilename 'MXAJC00O.png';tempfile-properties "XPR";}}}%
%BeginExpansion
\raisebox{-0pt}{\includegraphics[
natheight=7.252300in,
natwidth=8.175900in,
height=0.9954in,
width=1.1225in
]%
{MXAJC00O.png}%
}%
%EndExpansion
\]


We know that there is a $K$-rigid motion $M_{c}$ that takes $A^{\prime}$ to
$\left(  0,0\right)  $ and $L^{\prime}$ to the $x_{c}$-axis. (Why?) Viewed as
a transformation
\begin{gather*}
\left(  \underline{x_{c}},\underline{y_{c}}\right)  =M_{c}\left(  x_{c}%
,y_{c}\right) \\
=\left(  \frac{m_{11}x_{c}+m_{21}y_{c}+m_{31}}{m_{13}x_{c}+m_{23}y_{c}+m_{33}%
},\frac{m_{12}x_{c}+m_{22}y_{c}+m_{32}}{m_{13}x_{c}+m_{23}y_{c}+m_{33}%
}\right)  .
\end{gather*}
of the entire $\left(  x_{c},y_{c}\right)  $-plane, this transformation takes
the tangent line to the edge of the universe at $P^{\prime}$ to the tangent
line $T_{-}$ to the edge of the universe at $\left(  -\left\vert K\right\vert
^{-1/2},0\right)  $ and the tangent line to the edge of the universe at
$Q^{\prime}$ to the tangent line $T_{+}$ to the edge of the universe at
$\left(  \left\vert K\right\vert ^{-1/2},0\right)  $. Since the tangent lines
at $\left(  -\left\vert K\right\vert ^{-1/2},0\right)  $ and $\left(
\left\vert K\right\vert ^{-1/2},0\right)  $ are vertical, the point
$S^{\prime}$must have gone to infinity under the $K$-rigid motion. So the line
through $A^{\prime}$ and $R^{\prime}$ must go to a line that goes through
$\left(  0,0\right)  $ and that lies between $T_{-}$ and $T_{+}$. But here is
only one such line, namely the $y_{c}$-axis.

\begin{exercise}
\label{99}Explain why there is a $K$-rigid motion $M_{c}$ that takes any three
points $P^{\prime}$, $R^{\prime}$ and $Q^{\prime}$ in order along the edge of
the universe to any other three points $P^{\prime\prime}$, $R^{\prime\prime}$
and $Q^{\prime\prime}$ in order along the edge of the universe.

Hint: Use that the set of $K$-rigid motions form a group under the composition operation.
\end{exercise}

\begin{exercise}
(\textbf{HG}) Explain why the above discussion implies that the angles $\angle
P^{\prime}A^{\prime}R^{\prime}$ and $\angle Q^{\prime}A^{\prime}R^{\prime}$
must both be $K$-right angles, that is, their $K$-measures must each be
$90^{\circ}$. So the line segments $\overline{P^{\prime}Q^{\prime}}$ and
$\overline{A^{\prime}R^{\prime}}$ are $K$-perpendicular. [MJG,238-239]

Hint: You may need to use the fact that, since there is a $K$-rigid motion
that interchanges $\left(  -\left\vert K\right\vert ^{-1/2},0\right)  $ and
$\left(  \left\vert K\right\vert ^{-1/2},0\right)  $ and leaves $\left(
0,0\right)  $ fixed, the $x_{c}$-axis and the $y_{c}$-axis are $K$-perpendicular.
\end{exercise}

\begin{exercise}
(\textbf{HG}) Use the previous Exercise and the fact that $A^{\prime}$ can be
any point along the chord $\overline{P^{\prime}Q^{\prime}}$ in the figure
above to explain why the Klein model is not conformal, that is, it does not
faithfully represent the measure of angles in \textbf{HG}.\pagebreak
\end{exercise}

\subsection{Quadrilaterals in HG, in fact, in any NG}

\begin{exercise}
(\textbf{HG}) Use $\left(  x_{c},y_{c}\right)  $-coordinates to show that
\textbf{HG} satisfies the four Euclidean postulates E1, E2, E3, and E4.Thus
hyperbolic geometry is a Neutral Geometry (\textbf{NG}).
\end{exercise}

The next Exercise will help us appreciate some important properties of
\textbf{HG }that are properties of any Neutral Geometry. So do the Exercise
assuming that the context is any Neutral Geometry, that is, any
two-dimensional geometry satisfying E1-E4.

\begin{exercise}
(\textbf{NG}): Let $ABCD$ be a quadrilateral with $\angle ABC=\angle BCD$
right angles. (We denote polygons by naming their vertices in counterclockwise
order.) [MJG,164-165]

a) Show in \textbf{NG} that

$\left\vert AB\right\vert =\left\vert CD\right\vert $ implies that $\angle
BAD=\angle ADC$,

$\left\vert AB\right\vert >\left\vert CD\right\vert $ implies that $\angle
BAD<\angle ADC$,

$\left\vert AB\right\vert <\left\vert CD\right\vert $ implies that $\angle
BAD>\angle ADC$.

b) Use a) and pure logic to show that

$\angle BAD<\angle ADC$ implies that $\left\vert AB\right\vert >\left\vert
CD\right\vert $,

$\angle BAD=\angle ADC$ implies that $\left\vert AB\right\vert =\left\vert
CD\right\vert $,

$\angle BAD>\angle ADC$ implies that $\left\vert AB\right\vert <\left\vert
CD\right\vert $.

Hint: For the first implication in a) show that quadrilateral $ABCD$ is
(self-)congruent to the quadrilateral $DCBA$. Now suppose that the second
implication in a) is false for some quadrilateral $ABCD$. Construct
$A^{\prime}$ on $\overline{AB}$ so that $\left\vert A^{\prime}B\right\vert
=\left\vert CD\right\vert $. $B$y Exercise \ref{18}%
\[
\angle BAD<\angle BA^{\prime}D.
\]
By the (already proved) first implication%
\[
\angle BA^{\prime}D=\angle CDA^{\prime}.
\]
Finally%
\[
\angle A^{\prime}DC<\angle ADC
\]
since the segment $DA^{\prime}$ lies between the segment $DA$ and the segment
$DC$. The proof of the third implication in a) fis the same as the proof of
the second implication--just interchange $A$ and $D$ and interchange $B$ and
$C$.

For b), just use pure logic: If the first implication is false, then $\angle
BAD<\angle ADC$ and either $\left\vert AB\right\vert <\left\vert CD\right\vert
$ or $\left\vert AB\right\vert =\left\vert CD\right\vert $. If $\left\vert
AB\right\vert <\left\vert CD\right\vert $, then by a)%
\[
\angle BAD>\angle ADC.
\]
Contradiction! Etc., etc.
\end{exercise}

\begin{exercise}
Use $\left(  x_{c},y_{c}\right)  $-coordinates to show that \textbf{HG} does
not satisfy Euclid's postulate E5.That is, through a point not on a line, it
is not true that there passes a unique parallel (i.e. non-intersecting) line.
\end{exercise}

\pagebreak

\subsection{Computing $K$-distances in Klein coordinates}

In fact the tool that will let us compute all $K$-distances in $\left(
x_{c},y_{c}\right)  $-coordinates is the cross-ratio from Definition \ref{44}.
Let $d_{K}\left(  A_{c},B_{c}\right)  $ denote the $K$-distance between two
points $A_{c}$ and $B_{c}$ in the Klein $K$-disk. Now we know that
\[
d_{K}\left(  \left(  0,0\right)  ,\left(  \left\vert K\right\vert
^{-1/2}\text{\textperiodcentered}\frac{e^{\varepsilon}-e^{-\varepsilon}%
}{e^{\varepsilon}+e^{-\varepsilon}},0\right)  \right)  =\left\vert
K\right\vert ^{-1/2}\text{\textperiodcentered}\varepsilon.
\]
To see what this has to do with cross-ratio, we begin by computing the cross
ratio%
\[
\left(  0:-\left\vert K\right\vert ^{-1/2}:\left\vert K\right\vert
^{-1/2}\text{\textperiodcentered}\frac{e^{\varepsilon}-e^{-\varepsilon}%
}{e^{\varepsilon}+e^{-\varepsilon}}:\left\vert K\right\vert ^{-1/2}\right)
\]
given by the two points $\left(  0,0\right)  $, $\left(  \left\vert
K\right\vert ^{-1/2}\text{\textperiodcentered}\frac{e^{\varepsilon
}-e^{-\varepsilon}}{e^{\varepsilon}+e^{-\varepsilon}},0\right)  $ and the two
points $\left(  -\left\vert K\right\vert ^{-1/2},0\right)  $ and $\left(
\left\vert K\right\vert ^{-1/2},0\right)  $ where the $x_{c}$-axis intersects
the edge of the universe.

\begin{exercise}
(\textbf{HG}) a) Draw a picture of the Klein $K$-disk, the edge of the
universe, and the four points on the $x_{c}$-axis.

b) Show that%
\[
\left(  0:-\left\vert K\right\vert ^{-1/2}:\left\vert K\right\vert
^{-1/2}\text{\textperiodcentered}\frac{e^{\varepsilon}-e^{-\varepsilon}%
}{e^{\varepsilon}+e^{-\varepsilon}}:\left\vert K\right\vert ^{-1/2}\right)
=\left(  0:-1:\text{\textperiodcentered}\frac{e^{\varepsilon}-e^{-\varepsilon
}}{e^{\varepsilon}+e^{-\varepsilon}}:1\right)  .
\]
In particular, notice that the computation doesn't depend on $K$.

c) Show that%
\[
\left(  0:-1:\text{\textperiodcentered}\frac{e^{\varepsilon}-e^{-\varepsilon}%
}{e^{\varepsilon}+e^{-\varepsilon}}:1\right)  =e^{-2\varepsilon}.
\]

\end{exercise}

From these two Exercises we conclude that%
\begin{equation}
d_{K}\left(  \left(  0,0\right)  ,\left(  \left\vert K\right\vert
^{-1/2}\text{\textperiodcentered}\frac{e^{\varepsilon}-e^{-\varepsilon}%
}{e^{\varepsilon}+e^{-\varepsilon}},0\right)  \right)  =\frac{\left\vert
K\right\vert ^{-1/2}}{2}\text{\textperiodcentered}\left\vert \mathrm{ln}%
\left(  0:-\left\vert K\right\vert ^{-1/2}:\left\vert K\right\vert
^{-1/2}\text{\textperiodcentered}\frac{e^{\varepsilon}-e^{-\varepsilon}%
}{e^{\varepsilon}+e^{-\varepsilon}}:\left\vert K\right\vert ^{-1/2}\right)
\right\vert . \label{78}%
\end{equation}


Now suppose we are given any two $A_{c}$ and $B_{c}$ in the Klein $K$-disk.
They determine a line
\begin{equation}
\alpha x_{c}+\beta y_{c}+\gamma=0 \label{76}%
\end{equation}
and so points $P_{c}$ and $Q_{c}$ where that line meets the edge of the
universe as shown in the figure below.
\[%
%TCIMACRO{\FRAME{itbpF}{1.6942in}{1.5748in}{0in}{}{}{Figure}%
%{\special{ language "Scientific Word";  type "GRAPHIC";
%maintain-aspect-ratio TRUE;  display "USEDEF";  valid_file "T";
%width 1.6942in;  height 1.5748in;  depth 0in;  original-width 9.89in;
%original-height 9.1869in;  cropleft "0";  croptop "1";  cropright "1";
%cropbottom "0";  tempfilename 'MXAJC00P.png';tempfile-properties "XPR";}}}%
%BeginExpansion
{\includegraphics[
natheight=9.186900in,
natwidth=9.890000in,
height=1.5748in,
width=1.6942in
]%
{MXAJC00P.png}%
}%
%EndExpansion
\]
We are now ready to prove the following Theorem.

\begin{theorem}
(\textbf{HG}) For any two points $A_{c}$ and $B_{c}$ on the Klein $K$-disk,
the $K$-distance between them $d_{K}\left(  A_{c},B_{c}\right)  $ is given by
the formula%
\[
d_{K}\left(  A_{c},B_{c}\right)  =\frac{\left\vert K\right\vert ^{-1/2}}%
{2}\text{\textperiodcentered}\left\vert \mathrm{ln}\left(  x_{c}\left(
A_{c}\right)  :x_{c}\left(  P_{c}\right)  :x_{c}\left(  B_{c}\right)
:x_{c}\left(  Q_{c}\right)  \right)  \right\vert
\]
where $P_{c}$ and $Q_{c}$ are the endpoints of the chord through $A_{c}$ and
$B_{c}$. (Compare with [MJG,268].)
\end{theorem}

\begin{proof}
We know that there is a $K$-rigid motion $\left(  \underline{x_{c}}%
,\underline{y_{c}}\right)  =M_{c}\left(  x_{c},y_{c}\right)  $ of the Klein
disk that takes $A_{c}$ to $\left(  0,0\right)  $ and $B_{c}$ to some point
$\left(  \left\vert K\right\vert ^{-1/2}\text{\textperiodcentered}%
\frac{e^{\varepsilon}-e^{-\varepsilon}}{e^{\varepsilon}+e^{-\varepsilon}%
},0\right)  $ on the positive $x_{c}$-axis. From $\left(  \ref{77}\right)  $
we know that%
\[
\underline{x_{c}}=\frac{m_{11}x_{c}+m_{21}y_{c}+m_{31}}{m_{13}x_{c}%
+m_{23}y_{c}+m_{33}}.
\]
But from $\left(  \ref{76}\right)  $ we know that for our four points $A_{c}$,
$B_{c}$, $P_{c}$, and $Q_{c}$
\begin{gather*}
\alpha x_{c}+\beta y_{c}+\gamma=0\\
y_{c}=\frac{-\alpha x_{c}-\gamma}{\beta}.
\end{gather*}
So if we calculate $M_{c}$ only for these four points we have%
\begin{align*}
\underline{x_{c}}  &  =\frac{m_{11}x_{c}+m_{21}\left(  \frac{-\alpha
x_{c}-\gamma}{\beta}\right)  +m_{31}}{m_{13}x_{c}+m_{23}\left(  \frac{-\alpha
x_{c}-\gamma}{\beta}\right)  +m_{33}}\\
&  =\frac{\left(  m_{11}-\frac{m_{21}\alpha}{\beta}\right)  x_{c}+\left(
m_{31}-\frac{m_{21}\gamma}{\beta}\right)  }{\left(  m_{13}-\frac{m_{23}\alpha
}{\beta}\right)  x_{c}+\left(  m_{33}-\frac{m_{23}\gamma}{\beta}\right)  .}%
\end{align*}
That is, the function $x_{c}\mapsto\underline{x_{c}}$ is a linear fractional
transformation! So by Exercise \ref{42}%
\begin{align*}
\left(  x_{c}\left(  A_{c}\right)  :x_{c}\left(  P_{c}\right)  :x_{c}\left(
B_{c}\right)  :x_{c}\left(  Q_{c}\right)  \right)   &  =\left(  \underline
{x_{c}}\left(  A_{c}\right)  :\underline{x_{c}}\left(  P_{c}\right)
:\underline{x_{c}}\left(  B_{c}\right)  :\underline{x_{c}}\left(
Q_{c}\right)  \right) \\
&  =\left(  0:-\left\vert K\right\vert ^{-1/2}:\left\vert K\right\vert
^{-1/2}\text{\textperiodcentered}\frac{e^{\varepsilon}-e^{-\varepsilon}%
}{e^{\varepsilon}+e^{-\varepsilon}}:\left\vert K\right\vert ^{-1/2}\right) \\
&  =e^{-2\varepsilon}.
\end{align*}
Therefore%
\begin{align*}
d_{K}\left(  A_{c},B_{c}\right)   &  =d_{K}\left(  \left(  0,0\right)
,\left(  \left\vert K\right\vert ^{-1/2}\text{\textperiodcentered}%
\frac{e^{\varepsilon}-e^{-\varepsilon}}{e^{\varepsilon}+e^{-\varepsilon}%
},0\right)  \right) \\
&  =\frac{\left\vert K\right\vert ^{-1/2}}{2}\text{\textperiodcentered
}\left\vert \mathrm{ln}\left(  0:-\left\vert K\right\vert ^{-1/2}:\left\vert
K\right\vert ^{-1/2}\text{\textperiodcentered}\frac{e^{\varepsilon
}-e^{-\varepsilon}}{e^{\varepsilon}+e^{-\varepsilon}}:\left\vert K\right\vert
^{-1/2}\right)  \right\vert \\
&  =\frac{\left\vert K\right\vert ^{-1/2}}{2}\text{\textperiodcentered
}\left\vert \mathrm{ln}\left(  x_{c}\left(  A_{c}\right)  :x_{c}\left(
P_{c}\right)  :x_{c}\left(  B_{c}\right)  :x_{c}\left(  Q_{c}\right)  \right)
\right\vert .
\end{align*}

\end{proof}

\begin{exercise}
For $K=-1$, calculate the $K$-distance between the two points given in
$\left(  x_{c},y_{c}\right)  $-coordinates by $\left(  0,0\right)  $ and
$\left(  1/2,0\right)  $\pagebreak
\end{exercise}

\subsection{Areas of hyperbolic lunes}

Finally there is one $K$-area computation that it is convenient to do in Klein
coordinates. Namely suppose that we take the ordinary triangle $T_{c}$ in the
$\left(  x_{c},y_{c}\right)  $-plane with vertices $\left(  0,0\right)  $,
$\left(  \left\vert K\right\vert ^{-1/2}\mathrm{cos}\beta,\left\vert
K\right\vert ^{-1/2}\mathrm{sin}\beta\right)  $ and $\left(  \left\vert
K\right\vert ^{-1/2}\mathrm{cos}\beta,-\left\vert K\right\vert ^{-1/2}%
\mathrm{sin}\beta\right)  $. Notice that two of the three vertices lie on the
edge of the universe of the Klein $K$-disk and that the $K$-angle at the third
vertex is%
\[
\alpha=2\beta.
\]
(In fact $\left(  0,0\right)  $ is the \textit{one} point in the Klein
$K$-disk where $K$-angels \textit{are} faithfully represented. (Why?)) We will
call the interior of this triangle, or any $K$-rigid motion of it, an $\alpha
$\textbf{-lune}. So we wish to compute the $K$-area of an $\alpha$-lune. Since
\textbf{HG} is a $K$-geometry we know from Exercise \ref{79} that this area
$A_{K}\left(  \alpha\right)  $ is given by the formula%
\[
A_{K}\left(  \alpha\right)  =%
%TCIMACRO{\dint \nolimits_{T_{c}}}%
%BeginExpansion
{\displaystyle\int\nolimits_{T_{c}}}
%EndExpansion
\frac{1}{\left(  1-\left\vert K\right\vert \left(  x_{c}^{2}+y_{c}^{2}\right)
\right)  ^{3/2}}dx_{c}dy_{c}.
\]


\begin{exercise}
\label{97}Show that%
\[
A_{K}\left(  \alpha\right)  =\left\vert K\right\vert ^{-1}\left(  \pi
-\alpha\right)  .
\]


Hint: Use the substitution%
\begin{align*}
\underline{x_{c}}  &  =\left\vert K\right\vert ^{1/2}x_{c}\\
\underline{y_{c}}  &  =\left\vert K\right\vert ^{1/2}y_{c}%
\end{align*}
to reduce the computation to the computation in the case that $\left\vert
K\right\vert =1$. Then use polar coordinates to get
\[
A_{K}\left(  \alpha\right)  =\left\vert K\right\vert ^{-1}%
%TCIMACRO{\dint \nolimits_{\theta=-\beta}^{\theta=\beta}}%
%BeginExpansion
{\displaystyle\int\nolimits_{\theta=-\beta}^{\theta=\beta}}
%EndExpansion%
%TCIMACRO{\dint \nolimits_{r=0}^{r=\frac{\mathrm{cos}\beta}{\mathrm{cos}\theta
%}}}%
%BeginExpansion
{\displaystyle\int\nolimits_{r=0}^{r=\frac{\mathrm{cos}\beta}{\mathrm{cos}%
\theta}}}
%EndExpansion
\frac{1}{\left(  1-r^{2}\right)  ^{3/2}}rdrd\theta
\]
Then do the substitution%
\begin{align*}
u  &  =1-r^{2}\\
du  &  =-2rdr
\end{align*}
to compute
\[%
%TCIMACRO{\dint \nolimits_{r=0}^{r=\frac{\mathrm{cos}\beta}{\mathrm{cos}\theta
%}}}%
%BeginExpansion
{\displaystyle\int\nolimits_{r=0}^{r=\frac{\mathrm{cos}\beta}{\mathrm{cos}%
\theta}}}
%EndExpansion
\frac{1}{\left(  1-r^{2}\right)  ^{3/2}}rdr.
\]
In the final step use the substitution%
\[
t=\mathrm{sin}\left(  \theta\right)  .
\]
\pagebreak
\end{exercise}

\section{Stereographic projection in \textbf{HG}}

\subsection{The Poincar\'{e} $K$-disk}

Under stereographic projection, the center of projection is the South Pole
$\left(  0,0,-1\right)  $. So if $K<0$ a point $\left(  x,0,z\right)  $ on the
$K$-geometry goes out the hyperboloid to infinity in the $\left(  x,z\right)
$-plane, the line joining $\left(  0,0,-1\right)  $ to that point becomes
parallel to an asymptote of the hyperbola%
\[
Kx^{2}+z^{2}=1.
\]
So the line approaches a line of slope $\pm\left\vert K\right\vert ^{1/2}$ in
the $\left(  x,z\right)  $-plane, that is the line $z=$ $\pm\left\vert
K\right\vert ^{1/2}x-1$. So the intersection of that line with the line $z=1$
in the $\left(  x,z\right)  $-plane approaches the point with $x=\pm
2\left\vert K\right\vert ^{-1/2}$. Therefore under stereographic projection,
the edge of the universe is given by the circle%
\[
x_{s}^{2}+y_{s}^{2}=4\left\vert K\right\vert ^{-1}.
\]
The interior of this circle, that is, the image of $K$-geometry under
stereographic projection, is called the Poincar\'{e} model of Hyperbolic
Geometry, of course again after a famous geometer, Henr\'{\i} Poincar\'{e}.
Again, since \textbf{HG} is a $K$-geometry, all the rules of Part \ref{III}
apply. So by Exercise \ref{73}b) line in the $K$-geometry are given by circles
of the form%
\[
\left(  x_{s}-\frac{2a}{K}\right)  ^{2}+\left(  y_{s}-\frac{2b}{K}\right)
^{2}=\frac{4\left(  K+a^{2}+b^{2}\right)  }{K^{2}}%
\]
in the Poincar\'{e} $K$-disk. The darker arc below%
\begin{equation}%
%TCIMACRO{\FRAME{itbpF}{1.8836in}{1.4961in}{0in}{}{}{Figure}%
%{\special{ language "Scientific Word";  type "GRAPHIC";
%maintain-aspect-ratio TRUE;  display "USEDEF";  valid_file "T";
%width 1.8836in;  height 1.4961in;  depth 0in;  original-width 11.4726in;
%original-height 9.0987in;  cropleft "0";  croptop "1";  cropright "1";
%cropbottom "0";  tempfilename 'MXAJC00Q.png';tempfile-properties "XPR";}}}%
%BeginExpansion
{\includegraphics[
natheight=9.098700in,
natwidth=11.472600in,
height=1.4961in,
width=1.8836in
]%
{MXAJC00Q.png}%
}%
%EndExpansion
\label{92}%
\end{equation}
represents the $K$-line%
\[
ax+by+1=0
\]
in the Poincar\'{e} $K$-disk. \pagebreak

\subsection{Stereographic projection preserves angles}

\begin{exercise}
(\textbf{HG}) a) Show that stereographic projection is conformal, that is,
that the measure of $K$-angles between $K$-lines on $K$-geometry is just the
ordinary Euclidean measure of angles formed.by (the circles that are) their
stereographic projections.

Hint: From Exercise \ref{36} we know that, for tangent vectors $V_{1}$ and
$V_{2}$ emanating from the same point on the $K$-geometry,
\begin{align*}
V_{1}\bullet_{K}V_{2}  &  =V_{1}^{s}\bullet_{s}V_{2}^{s}\\
&  =\left(  V_{1}^{s}\right)  \cdot\left(
\begin{array}
[c]{cc}%
\rho^{2} & 0\\
0 & \rho^{2}%
\end{array}
\right)  \cdot\left(  V_{2}^{s}\right)  ^{t}.
\end{align*}


b) For $K=-1$, construct the $K$-line in $\left(  x_{s},y_{s}\right)
$-coordinates that meets the $K$-line
\[
\left(  x_{s}-2\right)  ^{2}+\left(  y_{s}-2\right)  ^{2}=4
\]
perpendicularly in the point $\left(  2-\sqrt{2},2-\sqrt{2}\right)  $.
\end{exercise}

To get a more precise idea of what $K$-lines look like under stereographic
projection, consider the picture $\left(  \ref{92}\right)  $ again. The
equations of the circles in the picture are%
\begin{equation}
x_{s}^{2}+y_{s}^{2}=\frac{4}{\left\vert K\right\vert } \label{93}%
\end{equation}
and%
\begin{equation}
\left(  x_{s}-\frac{2a}{K}\right)  ^{2}+\left(  y_{s}-\frac{2b}{K}\right)
^{2}=\frac{4\left(  K+a^{2}+b^{2}\right)  }{K^{2}} \label{94}%
\end{equation}
Construct a third circle whose diameter is the line segment from $\left(
0,0\right)  $ to $\left(  \frac{2a}{K},\frac{2b}{K}\right)  $, namely the
circle $\left(  x_{s}-\frac{a}{K}\right)  ^{2}+\left(  y_{s}-\frac{b}%
{K}\right)  ^{2}=\left(  \frac{a}{K}\right)  ^{2}+\left(  \frac{b}{K}\right)
^{2}$ which can be rewritten%
\begin{equation}
x_{s}^{2}+y_{s}^{2}-\frac{2a}{K}x_{s}-\frac{2b}{K}y_{s}=0. \label{95}%
\end{equation}


\begin{lemma}
The circles $\left(  \ref{93}\right)  $, $\left(  \ref{94}\right)  $, and
$\left(  \ref{95}\right)  $ all three pass through two common points.
\end{lemma}

\begin{proof}
From $\left(  \ref{93}\right)  $ and $\left(  \ref{94}\right)  $ we get by
addition that%
\[
x_{s}^{2}+y_{s}^{2}+\left(  x_{s}-\frac{2a}{K}\right)  ^{2}+\left(
y_{s}-\frac{2b}{K}\right)  ^{2}=\frac{4}{\left\vert K\right\vert }%
+\frac{4\left(  K+a^{2}+b^{2}\right)  }{K^{2}}.
\]
Simplifying this last equation and dividing both sides by $2$ we obtain the
equation $\left(  \ref{95}\right)  $. So the two points $P^{\prime}$ and
$Q^{\prime}$ in picture $\left(  \ref{92}\right)  $ that satisfy both
equations $\left(  \ref{93}\right)  $ and $\left(  \ref{94}\right)  $ also
satisfy equation $\left(  \ref{95}\right)  $.
\end{proof}

The Lemma tells us that that the angle formed by the segments $\overline
{\left(  0,0\right)  P^{\prime}}$ and $\overline{P^{\prime}\left(  \frac
{2a}{K},\frac{2b}{K}\right)  }$ is a right angle since it is an inscribed
angle in the circle $\left(  \ref{95}\right)  $ whose associated central angle
is a diameter of that circle. But $\overline{\left(  0,0\right)  P^{\prime}}$
is a radius of circle $\left(  \ref{93}\right)  $ and so $\overline{P^{\prime
}\left(  \frac{2a}{K},\frac{2b}{K}\right)  }$ is tangent to circle $\left(
\ref{93}\right)  $. Similarly $\overline{P^{\prime}\left(  \frac{2a}{K}%
,\frac{2b}{K}\right)  }$ is a radius of circle $\left(  \ref{94}\right)  $ and
so $\overline{\left(  0,0\right)  P^{\prime}}$ is tangent to circle $\left(
\ref{94}\right)  $. So we conclude the following Theorem.

\begin{theorem}
(\textbf{HG}) In the Poincar\'{e} model for $K$-geometry, the $K$-lines are
represesented by circular arcs that meet the edge of the universe
perpendicularly.\pagebreak
\end{theorem}

\subsection{Infinite triangles in the Poincar\'{e} $K$-disk}

By Exercise \ref{73}b) and \ref{73}c), lines in \textbf{HG} become circles
under stereographic projection unless the line in \textbf{HG} passes through
the North Pole (in which case it corresponds to a line through $\left(
x_{s},y_{s}\right)  =\left(  0,0\right)  $ in the $\left(  x_{s},y_{s}\right)
$-plane). Suppose a hyperbolic triangle $T$ corresponds to a region $T_{s}$ in
$\left(  x_{s},y_{s}\right)  $-coordinates and the vertices of $T$ correspond
to $\left(  x_{s},y_{s}\right)  =\left(  -2,0\right)  $, $\left(  x_{s}%
,y_{s}\right)  =\left(  2,0\right)  $, and $\left(  x_{s},y_{s}\right)
=\left(  0,2\right)  $. So one side of $T_{s}$ lies on the line $y_{s}=0$.

\begin{exercise}
(\textbf{HG}) a) Use Exercise \ref{73}b) to compute the equations for the
other two sides of $T_{s}$.

b) In the $\left(  x_{s},y_{s}\right)  $-plane, draw $T_{s}$ as accurately as
you can when $K=-\frac{1}{4}$, then when $K=-1$.
\end{exercise}

The area of a hyperbolic triangle $T$ is given by the formula%
\[%
%TCIMACRO{\dint \nolimits_{T_{s}}}%
%BeginExpansion
{\displaystyle\int\nolimits_{T_{s}}}
%EndExpansion
\frac{1}{\left(  1+\frac{K}{4}\left(  x_{s}^{2}+y_{s}^{2}\right)  \right)
^{2}}dx_{s}dy_{s}.
\]
However we do not as yet have a way to calculate the area numerically for any
given triangle $T$. The last topic in this book will remedy that situation.
Analogously to the case of spherical triangles, we start from the fact that we
do know the area of $\alpha$-lunes. From Exercise \ref{97} the $K$-area of one
an $\alpha$-lune with vertex at $\left(  0,0\right)  $ in $\left(  x_{c}%
,y_{c}\right)  $-coordinates is%
\[
\left\vert K\right\vert ^{-1}\left(  \pi-\alpha\right)  .
\]
Since rotation of the $\left(  x_{c},y_{c}\right)  $-plane around $\left(
0,0\right)  $ is a $K$-rigid motion, this formula holds for any $K$-lune with
vertex at $\left(  0,0\right)  $. Now represent the \textit{same} lunes in the
$\left(  x_{s},y_{s}\right)  $-plane. Below is a picture in the $\left(
x_{s},y_{s}\right)  $-plane of some of these $K$-lunes.%
\begin{equation}%
%TCIMACRO{\FRAME{itbpF}{1.6137in}{1.4978in}{0in}{}{}{Figure}%
%{\special{ language "Scientific Word";  type "GRAPHIC";  display "USEDEF";
%valid_file "T";  width 1.6137in;  height 1.4978in;  depth 0in;
%original-width 8.0877in;  original-height 8.0877in;  cropleft "0";
%croptop "1";  cropright "1";  cropbottom "0";
%tempfilename 'MXAJC00R.png';tempfile-properties "XPR";}}}%
%BeginExpansion
{\includegraphics[
natheight=8.087700in,
natwidth=8.087700in,
height=1.4978in,
width=1.6137in
]%
{MXAJC00R.png}%
}%
%EndExpansion
\label{98}%
\end{equation}


\begin{exercise}
(\textbf{HG}) Use Exercise \ref{97} to show that the area in the picture
$\left(  \ref{98}\right)  $ that lies in the union of the $\alpha$-lune and
the $\beta$-lune but does not lie in the $\left(  \alpha+\beta\right)  $- lune
has $K$-area $\left\vert K\right\vert ^{-1}\pi$.
\end{exercise}

\begin{definition}
(\textbf{HG}) An infinite $K$-triangle is the figure given in stereographic
projection coordinates by the stereographic projection of three $K$-lines such
that any two meet the edge of the universe in a common point.
\end{definition}

\begin{exercise}
a) (\textbf{HG}) Use Exercise \ref{99} to show that the area of (the interior
of) any infinte triangle has $K$-area%
\[
\left\vert K\right\vert ^{-1}\text{\textperiodcentered}\pi.
\]
For example, if $K=-1$ we have%
\[%
%TCIMACRO{\FRAME{itbpF}{1.6769in}{1.6769in}{0in}{}{}{Figure}%
%{\special{ language "Scientific Word";  type "GRAPHIC";
%maintain-aspect-ratio TRUE;  display "USEDEF";  valid_file "T";
%width 1.6769in;  height 1.6769in;  depth 0in;  original-width 10.3734in;
%original-height 10.3734in;  cropleft "0";  croptop "1";  cropright "1";
%cropbottom "0";  tempfilename 'MXAJC00S.png';tempfile-properties "XPR";}}}%
%BeginExpansion
{\includegraphics[
natheight=10.373400in,
natwidth=10.373400in,
height=1.6769in,
width=1.6769in
]%
{MXAJC00S.png}%
}%
%EndExpansion
\]


b) Use a) to give a formula for the $K$-area of any infinite $n$-gon in
\textbf{HG}, that is, a figure described by a set of $n$ disjoint $K$-lines
that is the limit of a family of finite $n$-gons, all of whose vertices have
gone to infinity. In particular, what is the area of any infinite hexagon?

Hint: Divide the infinite $n$-gon into infinite triangles.\pagebreak
\end{exercise}

\subsection{Areas of polygons in \textbf{HG}}

Consider the picture below in the Poincar\'{e} model for \textbf{HG}. Find six
lunes that cover an infinte hexagon. Notice that the six lunes cover the
shaded hyperbolic triangle two extra times.%

\[%
%TCIMACRO{\FRAME{itbpF}{2.9966in}{1.6734in}{0in}{}{}{Figure}%
%{\special{ language "Scientific Word";  type "GRAPHIC";
%maintain-aspect-ratio TRUE;  display "USEDEF";  valid_file "T";
%width 2.9966in;  height 1.6734in;  depth 0in;  original-width 16.2637in;
%original-height 9.0546in;  cropleft "0";  croptop "1";  cropright "1";
%cropbottom "0";  tempfilename 'MXAJC00T.png';tempfile-properties "XPR";}}}%
%BeginExpansion
{\includegraphics[
natheight=9.054600in,
natwidth=16.263700in,
height=1.6734in,
width=2.9966in
]%
{MXAJC00T.png}%
}%
%EndExpansion
\]


\begin{exercise}
(\textbf{HG}) a) Use the picture and remarks just above to explain why the
$K$-area of the hyperbolic triangle is%
\[
\left\vert K\right\vert ^{-1}\text{\textperiodcentered}\left(  \pi-\left(
\alpha+\beta+\delta\right)  \right)  .
\]


b) Use a) to give a formula for the $K$-area of a hyperbolic $n$-gon.
\end{exercise}


\end{document}
