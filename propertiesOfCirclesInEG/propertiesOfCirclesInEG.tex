\documentclass{ximera}

\usepackage{microtype}
\usepackage{tikz}
\usepackage{tkz-euclide}
\usetkzobj{all}
\tikzstyle geometryDiagrams=[ultra thick,color=blue!50!black]

\renewcommand{\epsilon}{\varepsilon}



\prerequisites{euclideanGeometry}
\outcomes{circles}

\title{Properties of circles in \textbf{EG}}

\begin{document}
\begin{abstract}
In this activity we study circles in Euclidean geometry.
\end{abstract}
\maketitle


\subsection*{Basics}

Our final topic before leaving Euclidean geometry is circles. We
include this partly for its own interest, and partly because the
properties we visit here will be useful later on. Again we explore the
topic through a sequence of exercises (with hints to their solutions
to ease the way). We begin with perhaps the most basic fact of all
about circles in \textbf{EG}.

\begin{exercise}
The circle of radius $1$ has (interior) area $\pi$. Use this to reason
to the fact that the circle of radius $1$ has circumference $2\pi$.%
\begin{image}
\begin{tikzpicture}[geometryDiagrams]
\draw (0,0) circle (2cm);
\draw[thin] (0,0)--(2,0);
\draw[thin] (0,0)--({2*cos(10)},{2*sin(10)});
\draw[thin] (0,0)--({2*cos(20)},{2*sin(20)});
\draw[thin] (0,0)--({2*cos(30)},{2*sin(30)});
\draw[thin] (0,0)--({2*cos(40)},{2*sin(40)});
\draw[thin] (0,0)--({2*cos(50)},{2*sin(50)});
\draw[thin] (0,0)--({2*cos(60)},{2*sin(60)});
\draw[thin] (0,0)--({2*cos(70)},{2*sin(70)});
\draw[thin] (0,0)--({2*cos(80)},{2*sin(80)});
\draw[thin] (0,0)--({2*cos(90)},{2*sin(90)});
\draw[thin] (0,0)--({2*cos(100)},{2*sin(100)});
\draw[thin] (0,0)--({2*cos(110)},{2*sin(110)});
\draw[thin] (0,0)--({2*cos(120)},{2*sin(120)});
\draw[thin] (0,0)--({2*cos(130)},{2*sin(130)});
\draw[thin] (0,0)--({2*cos(140)},{2*sin(140)});
\draw[thin] (0,0)--({2*cos(150)},{2*sin(150)});
\draw[thin] (0,0)--({2*cos(160)},{2*sin(160)});
\draw[thin] (0,0)--({2*cos(170)},{2*sin(170)});
\draw[thin] (0,0)--({2*cos(180)},{2*sin(180)});
\draw[thin] (0,0)--({2*cos(190)},{2*sin(190)});
\draw[thin] (0,0)--({2*cos(200)},{2*sin(200)});
\draw[thin] (0,0)--({2*cos(210)},{2*sin(210)});
\draw[thin] (0,0)--({2*cos(220)},{2*sin(220)});
\draw[thin] (0,0)--({2*cos(230)},{2*sin(230)});
\draw[thin] (0,0)--({2*cos(240)},{2*sin(240)});
\draw[thin] (0,0)--({2*cos(250)},{2*sin(250)});
\draw[thin] (0,0)--({2*cos(260)},{2*sin(260)});
\draw[thin] (0,0)--({2*cos(270)},{2*sin(270)});
\draw[thin] (0,0)--({2*cos(280)},{2*sin(280)});
\draw[thin] (0,0)--({2*cos(290)},{2*sin(290)});
\draw[thin] (0,0)--({2*cos(300)},{2*sin(300)});
\draw[thin] (0,0)--({2*cos(310)},{2*sin(310)});
\draw[thin] (0,0)--({2*cos(320)},{2*sin(320)});
\draw[thin] (0,0)--({2*cos(330)},{2*sin(330)});
\draw[thin] (0,0)--({2*cos(340)},{2*sin(340)});
\draw[thin] (0,0)--({2*cos(350)},{2*sin(350)});
\end{tikzpicture}
\end{image}

Hint: Approximate a rectangle by rearranging the slices in the picture.
Compute the area of the ``rectangle.''
\end{exercise}



Next we turn to some facts about chords in circles and angels inscribed in circles.

\begin{exercise}
On the circle with center $O$ below,
\begin{image}
\begin{tikzpicture}[geometryDiagrams]
\coordinate (O) at (0,0);
\coordinate (A) at ({2*cos(40)},{2*sin(40)});
\coordinate (B) at ({2*cos(350)},{2*sin(350)});
\coordinate (X) at ({2*cos(170)},{2*sin(170)});

\draw[thin] (A)--(O);
\draw[thin] (A)--(X);
\draw[thin] (X)--(B);
\draw (0,0) circle (2cm);

\tkzLabelPoints[above](O)
\tkzLabelPoints[left](X)
\tkzLabelPoints[right](A,B)

\end{tikzpicture}
\end{image}
show that%
\[
\angle AXB=(1/2)(\angle AOB).
\]
Hint: $\triangle OAX$ is isosceles.
\end{exercise}



\begin{exercise}
On the circle with center $O$ below,
\begin{image}
\begin{tikzpicture}[geometryDiagrams]
\coordinate (O) at (0,0);
\coordinate (A) at ({2*cos(40)},{2*sin(40)});
\coordinate (B) at ({2*cos(290)},{2*sin(290)});
\coordinate (X) at ({2*cos(170)},{2*sin(170)});

\draw[thin] (A)--(O);
\draw[thin] (A)--(X);
\draw[thin] (X)--(B);
\draw[thin] (O)--(B);
\draw (0,0) circle (2cm);

\tkzLabelPoints[above](O)
\tkzLabelPoints[left](X)
\tkzLabelPoints[right](A)
\tkzLabelPoints[below](B)

\end{tikzpicture}
\end{image}
show that%
\[
\angle AXB=(1/2)(\angle AOB).
\]


Hint: Draw the diameter through $O$ and $X$ and add.
\end{exercise}




\begin{exercise}
On the circle with center $O$ below,
\begin{image}
\begin{tikzpicture}[geometryDiagrams]
\coordinate (O) at (0,0);
\coordinate (A) at ({2*cos(60)},{2*sin(60)});
\coordinate (B) at ({2*cos(20)},{2*sin(20)});
\coordinate (X) at ({2*cos(170)},{2*sin(170)});

\draw[thin] (A)--(O);
\draw[thin] (A)--(X);
\draw[thin] (X)--(B);
\draw[thin] (O)--(B);
\draw (0,0) circle (2cm);

\tkzLabelPoints[below](O)
\tkzLabelPoints[left](X)
\tkzLabelPoints[above](A)
\tkzLabelPoints[right](B)

\end{tikzpicture}
\end{image}
show that
\[
\angle AXB=(1/2)(\angle AOB).
\]


Hint: Draw the diameter through $O$ and $X$ and subtract.
\end{exercise}



We can summarize the results of the last three exercises into the following theorem.

\begin{theorem}
\label{43}The measure of any angle inscribed in a circle is one-half of the
measure of the corresponding central angle.
\end{theorem}

\begin{exercise}
Use similar triangles and the previous exercises to show that
\[
\left\vert AX\right\vert \cdot \left\vert
XB\right\vert =\left\vert A'X\right\vert\cdot \left\vert
XB'\right\vert
\]
in the figure below.%
\begin{image}
\begin{tikzpicture}[geometryDiagrams]
\coordinate (X) at (-.02,1.02);
\coordinate (B') at ({2*cos(60)},{2*sin(60)});
\coordinate (B) at ({2*cos(20)},{2*sin(20)});
\coordinate (A) at ({2*cos(140)},{2*sin(140)});
\coordinate (A') at ({2*cos(190)},{2*sin(190)});

\draw[thin] (A)--(B);
\draw[thin] (A')--(B');
\draw (0,0) circle (2cm);

\tkzLabelPoints[left](A,A')
\tkzLabelPoints[right](B)
\tkzLabelPoints[above](B')
\tkzLabelPoints[above](X)

\end{tikzpicture}
\end{image}
Hint: Draw $\overline{AB'}$ and $\overline{A'B}$.
\end{exercise}


\begin{exercise}
Use similar triangles and the previous exercises to show that
\[
\left\vert AX\right\vert\cdot\left\vert XB\right\vert
=\left\vert A'X\right\vert \cdot \left\vert XB'\right\vert
\]
in the figure below.%
\begin{image}
\begin{tikzpicture}[geometryDiagrams]
\coordinate (X) at (4.51,.22);
\coordinate (B') at ({2*cos(0)},{2*sin(0)});
\coordinate (B) at ({2*cos(20)},{2*sin(20)});
\coordinate (A) at ({2*cos(140)},{2*sin(140)});
\coordinate (A') at ({2*cos(190)},{2*sin(190)});

\draw[thin] (A)--(X);
\draw[thin] (A')--(X);
\draw (0,0) circle (2cm);

\tkzLabelPoints[left](A,A')
\tkzLabelPoints[above right](B)
\tkzLabelPoints[below right](B')
\tkzLabelPoints[right](X)

\end{tikzpicture}
\end{image}

Hint: Draw $\overline{AB^{\prime}}$ and $\overline{A^{\prime}B}$.
\end{exercise}




\begin{exercise}
Show that, given any three non-collinear points in the Euclidean
plane, there is a unique circle passing through the three points.

Hint: Show that the center of the circle must be the intersection of
the perpendicular bisectors of any two of the sides of the triangle
whose vertices are the three given points.
\end{exercise}

But how about four points in the plane, no three of which are
collinear?

\begin{exercise}
\begin{enumerate}
\item Draw four points in the Euclidean plane, no $3$ of which are collinear, that cannot lie on a single circle.
\item Draw four points in the Euclidean plane that do lie on a single
circle.
\end{enumerate}
\end{exercise}

The issue we will explore in the next two sections is the question of
finding a numerical condition about the four points that tells us
exactly when they all lie on a single circle. For that, we will need a
very famous mathematical relationship, one very closely related to the
notion of perspective in painting. That is, how do you faithfully
render depth on a flat canvas? This relationship is called
the \textit{cross-ratio} of the four points.

\subsection*{Cross-ratio of points on the number line}

We begin by studying the cross-ratio of four points on a line. Start
with the set of points on the real number line with coordinate $t$ and
add one extra point called $t=\infty$. Call the resulting set
$\overline{\mathbb{R}}$. You could think of the resulting set as the
set of all lines through the origin in $\mathbb{R}^{2}$ by assigning
to each line the real number that is its slope and to the $y$-axis the
slope $\infty$.

\begin{exercise}\hfil
\begin{enumerate}
\item Show that the transformation%
\[
\left(  x,y\right)  \mapsto\left(  \underline{x},\underline{y}\right)
=\left(  x,y\right)  \cdot\left(
\begin{array}
[c]{cc}%
d & b\\
c & a
\end{array}
\right)
\]
is a one-to-one and onto (linear) transformation of $\mathbb{R}^{2}$ as long as
\begin{equation}
\left\vert
\begin{array}
[c]{cc}%
d & b\\
c & a
\end{array}
\right\vert \neq0. \label{52}%
\end{equation}

\item For the transformation in the previous Exercise, show that every line through the origin in $\left( x,y\right) $-space is sent to a line
through the origin in $\left( \underline{x},\underline{y}\right)
$-space. The slope $t$ of the line through $\left( 0,0\right) $ and
$\left( x,y\right) $ is of course $t=\frac{y}{x}$. What is the
slope \underline{$t$} of the line through
$\left( \underline{0},\underline{0}\right) $ and $\left(
\underline{x},\underline{y}\right)  $? Show that%
\begin{equation}
\underline{t}=\frac{at+b}{ct+d} \label{41}%
\end{equation}
\end{enumerate}
\begin{solution}
\answer[free-response]{
To show the map is one-to-one, we will provide an explicit inverse
map, namely:
\[
\left(  \underline{x},\underline{y}\right)\mapsto \left(  \underline{x},\underline{y}\right) 
\frac{1}{ad-bc}\begin{pmatrix}
a &-b \\
-c& d
\end{pmatrix}
\]
Now apply the given transformation and then its inverse to see
\[
(x,y) \cdot \begin{pmatrix}
d & b \\
c & d
\end{pmatrix}\frac{1}{ad-bc}\begin{pmatrix}
a &-b \\
-c& d
\end{pmatrix} = (x,y).
\]
In fact, the inverse map shows that the original map is onto as well,
 as for any element of $\mathbb{R}^2$, we can find the preimage of
 this element using the inverse.  

To answer the second part, consider the line
\[
\l(u) = u\cdot (x,y).
\]
Now 
\[
\underline{\l}(u) = u\cdot (cy+dx,ay+bx)
\]
and this line has slope
\[
\frac{ay+bx}{cy+dx}.
\]
On the other hand, 
\begin{align*}
\frac{at+b}{ct+d} &= \frac{\frac{ay}{x} + b}{\frac{cy}{x}+d} \\
&= \frac{ay+bx}{cy+dx}.
\end{align*}
This is what we needed to show.
}\end{solution}
\end{exercise}




\begin{definition}
Functions $\left( \ref{41}\right) $ for which the condition $\left(
\ref{52}\right)  $ holds are called \textbf{linear fractional transformations}.
\end{definition}


\begin{exercise}
Show that a linear fractional transformation%
\begin{align*}
\overline{\mathbb{R}} &\rightarrow\overline{\mathbb{R}}\\
t &\mapsto\underline{t} =\frac{at+b}{ct+d}%
\end{align*}
is one-to-one and onto. What is its inverse function? (Your answer should show that
the inverse function is also a linear fractional transformation.)

Hint: By algebra solve for $t$ in terms of $\underline{t}$. Then graph%
\[
\underline{t}=\frac{at+b}{ct+d}%
\]
in the $\left(  t,\underline{t}\right)  $-plane. If $c=0$ show that the graph
is a straight line with non-zero slope and%
\[
\infty\mapsto\infty.
\]
If $c\neq0$, show that the graph has exactly one horizontal asymptote where
$t\mapsto\infty$ and one vertical asymptote where $\underline{t}\mapsto\infty$.

\begin{solution}
\answer[free-response]{
To show this map is one-to-one and onto, we will again provide an
explicit inverse map:
\[
\underline{t} \mapsto \frac{d \underline{t}-b}{-c \underline{t} +a}
\]
If $c= 0$, this is a line with slope $d/a$. Note since $ad-bc \ne 0$,
we see that this slope is neither $0$ nor undefined. Moreover, by the construction of $\overline{\mathbb{R}}$, we see that $\infty\mapsto\infty$.

If $c\ne0$ we have that
\[
\lim_{t\to \infty} \frac{at+b}{ct+d}  = a/c
\]
and this is the horizontal asymptote. On the other hand, 
\[
\lim_{\underline{t}\to\infty} \frac{d \underline{t}-b}{-c \underline{t} +a} = -d/c
\]
this corresponds to the vertical asymptote. 
}
\end{solution}
\end{exercise}







\begin{exercise}
Show that the set of linear fractional transformations form a group
under the operation of composition of functions. That is, check
associativity, identity element and existence of inverses.
\begin{solution}
\answer[free-response]{
To show that the set of linear fractional transformations form a group, we must show that this set is 
\begin{enumerate}
\item closed under function composition, 
\item that functional composition is an associative operation on linear fractional transformations, 
\item that there is an identity element, and 
\item that every linear fractional transformation has an inverse that is also a linear fractional transformation. 
\end{enumerate}
We will start by showing that the set of linear fractional
transformations is closed under functional composition. Write
\begin{align*}
f:t &\mapsto\frac{at+b}{ct+d}\\
g:t &\mapsto\frac{mt+n}{pt+q}
\end{align*}
where
\[
\begin{array}{|cc|}
d & b\\
c& a
\end{array} \ne 0\qquad\text{and}\qquad
\begin{array}{|cc|}
q & n\\
p& m
\end{array} \ne 0.
\]
Now
\[
f\circ g : t \mapsto \frac{(am+bp)t+(an+bq)}{(cm+dp)t + (cn+dq)}
\]
and note that
\begin{align*}
\begin{array}{|cc|}
cn+dq & an+bq\\
cm+dp& am+bp
\end{array}  &= (cn+dq)(am+bp)- (an+bq)(cm+dp)\\
&= (bc - ad)(np-mq) \\
&\ne 0.
\end{align*}
Hence we see that the composition of linear fractional transformations
is a linear fractional transformation.  

Now we will show that composition is an associative operation. Note,
functional composition is \textbf{always} an associative
operation. However, it is a good exercise to actually show this---so
we will. Write
\[
f:t \mapsto\frac{at+b}{ct+d},\qquad
g:t \mapsto\frac{mt+n}{pt+q},\qquad
h:t \mapsto\frac{rt+s}{ut+v}.
\]
where
\[
\begin{array}{|cc|}
d & b\\
c& a
\end{array} \ne 0,\qquad
\begin{array}{|cc|}
q & n\\
p& m
\end{array} \ne 0,\qquad
\begin{array}{|cc|}
v & s\\
u& r
\end{array} \ne 0.
\]
Write
\[
f\circ g : t \mapsto \frac{(am+bp)t+(an+bq)}{(cm+dp)t + (cn+dq)}
\]
and so
\[
(f\circ g) \circ h : t\mapsto \frac{(amr+bpr+anu+bqu)t+(ams+bps+anv+bqv)}{(cmr+dpr+cnu+dqu)t+(cms+dps+cnv+dqv)}. 
\]
Now
\[
g\circ h : t \mapsto \frac{(mr+nu)t+(ms+nv)}{(pr+qu)t + (ps+qv)}
\]
and
\[
f\circ (g \circ h) : t\mapsto \frac{(amr+bpr+anu+bqu)t+(ams+bps+anv+bqv)}{(cmr+dpr+cnu+dqu)t+(cms+dps+cnv+dqv)}. 
\]
From our work above we see that $(f\circ g)\circ h = f\circ(g\circ
 h)$.  

The identity element is
\[
t\mapsto \frac{a\cdot t + 0}{0\cdot t + a} = t,
\]
where $a$ is a nonzero real number.  Finally, each linear fractional
transformation has an inverse---we found this in the previous problem.
}
\end{solution}
\end{exercise}



\begin{exercise}
\label{59}Show that, for any three distinct points $t_{2},t_{3}$ and $t_{4}$,
the function of $t$ given by the formula%
\[
\underline{t}=\frac{t_{3}-t_{4}}{t_{3}-t_{2}}\frac{t-t_{2}}{t-t_{4}}%
=\frac{t-t_{2}}{t_{3}-t_{2}}\div\frac{t-t_{4}}{t_{3}-t_{4}}%
\]
takes $t_{2}$ to $0$, takes $t_{3}$ to $1$ and takes $t_{4}$ to
$\infty$. Show that this function is a linear fractional
transformation, that is, a function of the form
$\left( \ref{41}\right) $ for which the condition $\left(
\ref{52}\right)$ holds.
\begin{solution}
\answer[free-response]{
Write
\begin{align*}
t_2 &\mapsto \frac{t_3-t_4}{t_3-t_2} \cdot \frac{t_2-t_2}{t_2-t_4} = 0.\\
t_3 &\mapsto \frac{t_3-t_4}{t_3-t_2} \cdot \frac{t_3-t_2}{t_3-t_4} = 1.\\
t_4 &\mapsto \frac{t_3-t_4}{t_3-t_2} \cdot \frac{t_4-t_2}{t_4-t_4} = \infty.
\end{align*}
With a slight bit of algebra, we can see that this is a linear fractional transformation:
\[
\frac{t_{3}-t_{4}}{t_{3}-t_{2}}\frac{t-t_{2}}{t-t_{4}} = \frac{t_3t - t_2t_3-t_4t +t_2t_4}{t_3t-t_3t_4-t_2t+t_2t_4} = \frac{(t_3-t_4)t + (t_2t_4-t_2t_3)}{(t_3-t_2)t + (t_2t_4-t_3t_4)}
\]
though we must check the determinant condition:
\begin{align*}
\begin{array}{|cc|}
(t_2t_4 - t_3t_4) & (t_2t_4 - t_2t_3) \\
(t_3-t_2) & (t_3-t_4)
\end{array}
&=
\begin{array}{|cc|}
(t_2 - t_3)t_4 & (t_4 - t_3)t_2 \\
(t_3-t_2) & (t_3-t_4)
\end{array}\\
&=(t_2-t_3)(t_3-t_4)t_4 - (t_3-t_2)(t_4-t_3)t_2\\
&= (t_2-t_3)(t_3-t_4)t_4 - (t_2-t_3)(t_3-t_4)t_2\\
&= (t_2-t_3)(t_3-t_4)(t_4 -t_2).
\end{align*}
Since $t_2$, $t_3$, and $t_4$ are distinct, this cannot be
zero.
}
\end{solution}
\end{exercise}





\begin{exercise}
\label{57}Show that any linear fractional transformation $\left(
\ref{41}\right)$ that leaves $0$, $1$, and $\infty$ fixed is the identity map.
\begin{solution}
\answer[free-response]{
Consider an arbitrary linear fractional transformation
\[
f:t \mapsto \frac{at +b}{ct+d}.
\]
If $f(0)=0$, then we see that $b=0$ and that $d\ne 0$. Proceeding
along with this information if $f(1) = 1$, then we see that
\[
\frac{a}{c+d} = 1
\]
If $f(\infty) = \infty$, then we see that $c=0$. Hence, $a=d\ne
0$. This is the identity map.  }
\end{solution}
\end{exercise}







\begin{exercise}
\label{42}Suppose that you are given a function $\left(  \ref{41}\right)  $
and four points $t_{1},t_{2},t_{3}$ and $t_{4}$. Let
\[
\underline{t_{i}}=\frac{at_{i}+b}{ct_{i}+d}%
\]
for $i=1,2,3,4$. Show that%
\[
\frac{\underline{t_{1}}-\underline{t_{2}}}{\underline{t_{3}}-\underline{t_{2}%
}}\div\frac{\underline{t_{1}}-\underline{t_{4}}}{\underline{t_{3}}%
-\underline{t_{4}}}=\frac{t_{1}-t_{2}}{t_{3}-t_{2}}\div\frac{t_{1}-t_{4}%
}{t_{3}-t_{4}}.
\]
[MJG,288]

Hint: Just write out the formula for each side and do the high school
algebra.  There is a fancier way that uses that the set of linear
fractional transformations form a group whose operation is
composition. It goes like this. Use Exercise \ref{59} to show that the
inverse of the linear fractional transformation
\[
t\mapsto\frac{t-t_{2}}{t_{3}-t_{2}}\div\frac{t-t_{4}}{t_{3}-t_{4}}%
\]
followed by%
\[
t\mapsto\underline{t}%
\]
and then followed by
\[
t\mapsto\frac{t-\underline{t_{2}}}{\underline{t_{3}}-\underline{t_{2}}}%
\div\frac{t-\underline{t_{4}}}{\underline{t_{3}}-\underline{t_{4}}}%
\]
fixes $0$, $1$, and $\infty$ and so is the identity transformation by Exercise
\ref{57}. So%
\[
t\mapsto\frac{t-t_{2}}{t_{3}-t_{2}}\div\frac{t-t_{4}}{t_{3}-t_{4}}%
\]
is the same transformation as%
\[
t\mapsto\frac{\underline{t}-\underline{t_{2}}}{\underline{t_{3}}%
-\underline{t_{2}}}\div\frac{\underline{t}-\underline{t_{4}}}{\underline
{t_{3}}-\underline{t_{4}}}.
\]
\begin{solution}
\answer[free-response]{
We will use the fancier method of a solution. Start by writing
\begin{align*}
\frac{t-t_2}{t_3-t_2} \div \frac{t-t_4}{t_3-t_4}\mapsto t\mapsto \underline{t}\mapsto \frac{\underline{t}-\underline{t_{2}}}{\underline{t_{3}}%
-\underline{t_{2}}}\div\frac{\underline{t}-\underline{t_{4}}}{\underline
{t_{3}}-\underline{t_{4}}}.
\end{align*}
From our work above, we know this is a linear fractional
transformation. Moreover, from our work above, we can see that this
map fixes $0$, $1$, and $\infty$, hence this map is the identity
map. The upshot of this is that the map
\[
t\mapsto \frac{t-t_2}{t_3-t_2} \div \frac{t-t_4}{t_3-t_4}
\]
is now seen to be identical to the composition of maps
\[
t\mapsto \underline{t}\mapsto \frac{\underline{t}-\underline{t_{2}}}{\underline{t_{3}}%
-\underline{t_{2}}}\div\frac{\underline{t}-\underline{t_{4}}}{\underline
{t_{3}}-\underline{t_{4}}}.
\]
Evaluating these maps at $t_1$ gives the desired result. 
}
\end{solution}
\end{exercise}

\begin{definition}
\label{44}The cross-ratio $\left(  t_{1}:t_{2}:t_{3}:t_{4}\right)  $ of four
(ordered) points $t_{1},t_{2},t_{3}$ and $t_{4}$ is defined by%
\[
\left(  t_{1}:t_{2}:t_{3}:t_{4}\right)  =\frac{t_{1}-t_{2}}{t_{3}-t_{2}}%
\div\frac{t_{1}-t_{4}}{t_{3}-t_{4}}.
\]

\end{definition}

Exercise \ref{42} shows that if four points are moved by any function $\left(
\ref{41}\right)  $ the cross-ratio $\left(  \underline{t_{1}}:\underline
{t_{2}}:\underline{t_{3}}:\underline{t_{4}}\right)  $ of the output four
points is the same as the cross-ratio $\left(  t_{1}:t_{2}:t_{3}:t_{4}\right)
$ of the original four points.












\subsection*{Cross-ratio of points on a circle}

\begin{exercise}\label{46}
In the diagram
\begin{image}
\begin{tikzpicture}[geometryDiagrams]
\coordinate (O) at ({2*cos(120)},{2*sin(120)});
\coordinate (A) at ({2*cos(190)},{2*sin(190)});
\coordinate (B) at ({2*cos(250)},{2*sin(250)});
\coordinate (C) at ({2*cos(330)},{2*sin(330)});

\draw (0,0) circle (2cm);
\draw (O) -- (C);
\draw (A) -- (B);
\draw (O) -- (B);
\draw (B) -- (C);
\draw (O) -- (A);


\tkzMarkAngle[size=.9cm,thin](A,O,B)
\tkzLabelAngle[pos = 0.7](A,O,B){$\alpha$}
\tkzMarkAngle[arc=ll,size=0.9cm,thin](B,O,C)
\tkzLabelAngle[pos = 0.7](B,O,C){$\beta$}

\tkzLabelPoints[above](O)
\tkzLabelPoints[left](A)
\tkzLabelPoints[below](B)
\tkzLabelPoints[right](C)

\end{tikzpicture}
\end{image}
show that%
\[
\frac{\left\vert AB\right\vert }{\left\vert CB\right\vert }=\frac
{\sin\alpha}{\sin\beta}=\frac{\sin\left(  \angle
AOB\right)  }{\sin\left(  \angle COB\right)  }.
\]


Hint: Notice that by Theorem \ref{43}
\[
m\left(  \angle BAO\right)  +m\left(\angle OCB\right)  =180^{\circ}%
\]
so that%
\[
\sin\left(  \angle BAO\right)  =\sin\left(\angle  OCB\right)  .
\]
Now use the Law of Sines.
\begin{solution}
\answer[free-response]{
By the Law of Sines we have that
\[
\frac{|AB|}{\sin\alpha} = \frac{|OB|}{\sin(\angle OAB)}\qquad\text{and}\qquad\frac{|BC|}{\sin\beta} = \frac{|OB|}{\sin(\angle OCB)}.
\]
Write 
\begin{align*}
m\left(  \angle BAO\right)  +m\left(\angle OCB\right)  &=180^{\circ}\\
m\left(  \angle BAO\right)  &=180^{\circ}-m\left(\angle OCB\right)  \\
\sin\left(\angle BAO\right)  &=\sin\left(180^{\circ}-m\left(\angle OCB\right)\right)  \\
\sin\left(\angle BAO\right)  &=\sin\left(180^{\circ}\right)\cos\left(\angle OCB\right) - \cos\left(180^{\circ}\right)\sin\left(\angle OCB\right)\\ 
\sin\left(\angle BAO\right)  &=\sin\left(\angle OCB\right).
\end{align*}
So  now we have that 
\[
\frac{|AB|}{\sin\alpha} = \frac{|OB|}{\sin(\angle OAB)}= \frac{|OB|}{\sin(\angle OCB)}=\frac{|BC|}{\sin\beta}. 
\]
Using algebra we find 
\[
\frac{\left\vert AB\right\vert }{\left\vert CB\right\vert }=\frac
{\sin\alpha}{\sin\beta}=\frac{\sin\left(  \angle
AOB\right)  }{\sin\left(  \angle COB\right)  }.
\]
}
\end{solution}
\end{exercise}

\begin{exercise}
\label{47} Show that if, in the above figure, $B$ moves along the
circle to the other side of $C$, it is still true that%
\[
\frac{\left\vert AB\right\vert }{\left\vert CB\right\vert }=\frac
{\sin\left(  \angle AOB\right)  }{\sin\left(  \angle
COB\right)  }%
\]
\begin{solution}
\answer[free-response]{
By the Law of Sines we have that
\[
\frac{|AB|}{\sin(\angle AOB)} = \frac{|OB|}{\sin(\angle OAB)}\qquad\text{and}\qquad\frac{|CB|}{\sin(\angle COB)} = \frac{|OB|}{\sin(\angle OCB)}.
\]
However, 
\[
\angle OAB = \angle OCB
\]
So now we have that
\[
\frac{|AB|}{\sin(\angle AOB)} =\frac{|CB|}{\sin(\angle COB)}. 
\]
Using algebra we find 
\[
\frac{\left\vert AB\right\vert }{\left\vert CB\right\vert }=\frac
{\sin\left(  \angle AOB\right)  }{\sin\left(  \angle
COB\right)  }.
\]
}
\end{solution}
\end{exercise}






\begin{exercise}\label{48}
In the diagram
\begin{image}
\begin{tikzpicture}[geometryDiagrams]
\coordinate (O) at ({2*cos(120)},{2*sin(120)});
\coordinate (A) at ({2*cos(190)},{2*sin(190)});
\coordinate (B) at ({2*cos(250)},{2*sin(250)});
\coordinate (C) at ({2*cos(330)},{2*sin(330)});

\coordinate (A') at (-3.21,-3);
\coordinate (B') at (-.59,-3);
\coordinate (C') at (3.73,-3);

\draw (0,0) circle (2cm);
\draw (O) -- (C');
\draw (A) -- (B);
\draw (O) -- (B');
\draw (B) -- (C);
\draw (O) -- (A');

\draw (-4,-3) -- (5,-3);


\tkzMarkAngle[size=.9cm,thin](A,O,B)
\tkzLabelAngle[pos = 0.7](A,O,B){$\alpha$}
\tkzMarkAngle[arc=ll,size=0.9cm,thin](B,O,C)
\tkzLabelAngle[pos = 0.7](B,O,C){$\beta$}

\tkzMarkAngle[size=0.7cm,thin](B',A',A)
\tkzLabelAngle[pos = 0.4](B',A',A){$\gamma$}

\tkzMarkAngle[size=0.9cm,thin](C,C',B')
\tkzLabelAngle[pos = 0.7](C,C',B'){$\delta$}

\tkzLabelPoints[above](O)
\tkzLabelPoints[left](A)
\tkzLabelPoints[below left](B)
\tkzLabelPoints[below](A',B',C')
\tkzLabelPoints[right](C)

\end{tikzpicture}
\end{image}
show that%
\[
\frac{\left\vert A'B'\right\vert }{\left\vert C^{\prime
}B'\right\vert }=\frac{\sin\alpha}{\sin\beta}\div
\frac{\sin\gamma}{\sin\delta}=\frac{\sin\left(  \angle
A'OB'\right)  }{\sin\left(  \angle C^{\prime
}OB'\right)  }\div\frac{\sin\left(  \angle B'%
A'O\right)  }{\sin\left(  \angle B'C'O\right)
}.
\]
[MJG,266-267]
\begin{solution}
\answer[free-response]{
By the Law of Sines we have that
\[
\frac{|A'B'|}{\sin \alpha} = \frac{|OB'|}{\sin \gamma}\qquad\text{and}\qquad\frac{|C'B'|}{\sin \beta} = \frac{|OB'|}{\sin\delta}.
\]
We now see that
\[
|A'B'| = \frac{|OB'|\sin\alpha}{\sin \gamma}\qquad\text{and}\qquad|C'B'| = \frac{|OB'|\sin\beta}{\sin\delta}.
\]
Finally, we see directly that 
\[
\frac{\left\vert A'B'\right\vert }{\left\vert C^{\prime
}B'\right\vert }=\frac{\sin\alpha}{\sin\beta}\div
\frac{\sin\gamma}{\sin\delta}=\frac{\sin\left(  \angle
A'OB'\right)  }{\sin\left(  \angle C^{\prime
}OB'\right)  }\div\frac{\sin\left(  \angle B'%
A'O\right)  }{\sin\left(  \angle B'C'O\right)
}.
\]
}
\end{solution}
\end{exercise}


\begin{exercise}\label{49}
Show that if, in the previous figure, $B'$ moves
along the line to the other side of $C'$, it is still true that%
\[
\frac{\left\vert A'B'\right\vert }{\left\vert C^{\prime
}B'\right\vert }=\frac{\sin\left(  \angle A'%
OB'\right)  }{\sin\left(  \angle C'OB'\right)
}\div\frac{\sin\left(  \angle B'A'O\right)
}{\sin\left(  \angle B'C'O\right)  }.
\]
\begin{solution}
\answer[free-response]{
By the Law of Sines we have that
\[
\frac{|A'B'|}{\sin(\angle A'OB')} = \frac{|OB'|}{\sin(\angle B'A'O)}\qquad\text{and}\qquad\frac{|C'B'|}{\sin(\angle C'OB')} = \frac{|OB'|}{\sin(\angle B'C'O)}.
\]
We now see that
\[
|A'B'| = \frac{|OB'|\sin(\angle A'OB')}{\sin(B'A'O)}\qquad\text{and}\qquad|C'B'| = \frac{|OB'|\sin(\angle C'OB')}{\sin(\angle B'C'O)}.
\]
Again, we see directly that 
\[
\frac{\left\vert A'B'\right\vert }{\left\vert C^{\prime
}B'\right\vert }=\frac{\sin\left(  \angle A'%
OB'\right)  }{\sin\left(  \angle C'OB'\right)
}\div\frac{\sin\left(  \angle B'A'O\right)
}{\sin\left(  \angle B'C'O\right)  }.
\]
}
\end{solution}
\end{exercise}

These last two Exercises allow us to define the cross-ratio of four
points on a circle.

\begin{definition}
For a sequence of four (ordered) points $A,B,C,$ and $D$ on a circle,
we define%
\[
\left(  A:B:C:D\right)  =\frac{\left\vert AB\right\vert }{\left\vert
CB\right\vert }\div\frac{\left\vert AD\right\vert }{\left\vert CD\right\vert }%
\]
which we call the cross-ratio of the ordered sequence of the four points.
Similarly for a sequence of four (ordered) points $A',B^{\prime
},C',$ and $D'$ on a line, we define%
\[
\left(  A':B':C':D'\right)  =\frac{\left\vert
A'B'\right\vert }{\left\vert C'B'\right\vert
}\div\frac{\left\vert A'D'\right\vert }{\left\vert C^{\prime
}D'\right\vert }%
\]
which we call the cross-ratio of the ordered sequence of the four points.
\end{definition}

Notice that Definition \ref{44} is just a refinement of the definition
of $\left( A':B':C':D'\right) $ just above. In Definition \ref{44} we
are keeping track of the signs of the terms in the quotients whereas
$\left( A':B':C':D'\right)$ is always non-negative.

\begin{exercise}\label{50}
Show that, in the figure%
\begin{image}
\begin{tikzpicture}[geometryDiagrams]
\coordinate (O) at ({2*cos(120)},{2*sin(120)});
\coordinate (A) at ({2*cos(190)},{2*sin(190)});
\coordinate (B) at ({2*cos(250)},{2*sin(250)});
\coordinate (C) at ({2*cos(310)},{2*sin(310)});
\coordinate (D) at ({2*cos(340)},{2*sin(340)});

\coordinate (A') at (-3.21,-3);
\coordinate (B') at (-.59,-3);
\coordinate (C') at (2.31,-3);
\coordinate (D') at (4.64,-3);

\draw (0,0) circle (2cm);
\draw (O) -- (C');
\draw (A) -- (B);
\draw (C) -- (D);
\draw (O) -- (B');
\draw (B) -- (C);
\draw (O) -- (A');
\draw (O) -- (D');

\draw (-4,-3) -- (6,-3);



\tkzLabelPoints[above](O)
\tkzLabelPoints[left](A)
\tkzLabelPoints[below left](B)
\tkzLabelPoints[below](A',B',C',D')
\tkzLabelPoints[right](C)
\tkzLabelPoints[above right](D)

\end{tikzpicture}
\end{image}
we have the equality%
\[
\left(  A:B:C:D\right)  =\left(A':B':C':D'\right)  .
\]


Hint: Use Exercises \ref{46}--\ref{49}.

What happens in if we move $B$ to the other side of $C$?
\begin{solution}
\answer{
By our previous work, 
\begin{align*}
(A':B':C':D') &=\frac{|A'B'|}{|C'B'|}\div \frac{|A'D'|}{|C'D'|}\\
&=\left(\frac{\sin(\angle A'OB')}{\sin(\angle C'OB')} \div \frac{\sin(\angle B'A'O)}{\sin(\angle B'C'O)}\right) \div \left(\frac{\sin(\angle A'OD')}{\sin(\angle C'OD')}\div\frac{\sin(\angle D'A'O)}{\sin(\angle D'C'O)}\right).
\end{align*}
Since 
\begin{align*}
m(\angle A'OB') &= m(\angle AOB),\\
m(\angle C'OB') &= m(\angle COB),\\
m(\angle A'OD') &= m(\angle AOD),\\
m(\angle C'OD') &= m(\angle COD),\\
m(\angle B'A'O) &= m(\angle D'A'O),
\end{align*}
and as we have shown several times now, 
\[
\sin(\angle B'C'O) = \sin(\angle D'C'O)
\]
as 
\[
m(\angle B'C'O) + m(\angle D'C'O) = 180^\circ.
\]
With all of these substitutions, our ``large'' expression above becomes
\[
\frac{\sin(\angle AOB)}{\sin(\angle COB)} \div \frac{\sin(\angle AOD)}{\sin(\angle COD)} = (A:B:C:D). 
\]
Since none of the formulas change when $B$ is moved to the other side
of $C$, we are done.  }
\end{solution}
\end{exercise}

We say that ``Cross-ratio is invariant under stereographic
projection.''

\subsection*{Ptolemy's Theorem}

Given any three non-collinear points in the Euclidean plane, there is
one and only one circle that passes through the three points. (How do
you construct it?) You can easily convince yourself with a few
examples that, given four non-collinear points $A,B,C$ and $D$ in the
plane, it is not always true that there is a circle that passes
through all four. A famous theorem of classical Euclidean geometry
gives the condition that there is a circle that passes through all
four.

\begin{theorem}[Ptolemy] If the ordered sequence of points $A,B,C$ and $D$ lies on a circle,
\begin{image}
\begin{tikzpicture}[geometryDiagrams]
\coordinate (A) at ({2*cos(190)},{2*sin(190)});
\coordinate (B) at ({2*cos(250)},{2*sin(250)});
\coordinate (C) at ({2*cos(310)},{2*sin(310)});
\coordinate (D) at ({2*cos(340)},{2*sin(340)});

\draw (0,0) circle (2cm);
\draw (A) -- (B);
\draw (C) -- (D);
\draw (A) -- (D);
\draw (B) -- (C);

\tkzLabelPoints[left](A)
\tkzLabelPoints[below left](B)
\tkzLabelPoints[right](C)
\tkzLabelPoints[above right](D)

\end{tikzpicture}
\end{image}
then%
\[
\left\vert AC\right\vert \cdot\left\vert BD\right\vert
=\left\vert AD\right\vert \cdot\left\vert BC\right\vert
+\left\vert AB\right\vert \cdot\left\vert CD\right\vert
.
\]
That is, the product of the diagonals of the quadrilateral $ABCD$ is the sum
of the products of pairs of opposite sides.
\end{theorem}

\begin{proof}
We need to check that%
\[
\left\vert AC\right\vert \cdot\left\vert BD\right\vert
=\left\vert AD\right\vert \cdot\left\vert BC\right\vert
+\left\vert AB\right\vert \cdot\left\vert CD\right\vert
\]
or, what is the same, we need to check that%
\[
\frac{\left\vert AC\right\vert \cdot\left\vert
BD\right\vert }{\left\vert AD\right\vert \cdot\left\vert
BC\right\vert }=1+\frac{\left\vert AB\right\vert \text{\textperiodcentered
}\left\vert CD\right\vert }{\left\vert AD\right\vert \text{\textperiodcentered
}\left\vert BC\right\vert }.
\]
That is, we need to check that
\[
\left(  A:C:B:D\right)  =1+\left(  A:B:C:D\right)  .
\]
But by Exercise \ref{50} this is the same as checking that%
\[
\left(  A':C':B':D'\right)  =1+\left(
A':B':C':D'\right)
\]
for the projection of the four points onto a line from a point $O$ on the
circle. But that is the same thing as showing that
\[
\frac{\left\vert A'C'\right\vert \text{\textperiodcentered
}\left\vert B'D'\right\vert }{\left\vert A'D^{\prime
}\right\vert \cdot\left\vert B'C^{\prime
}\right\vert }=1+\frac{\left\vert A'B'\right\vert
\cdot\left\vert C'D'\right\vert
}{\left\vert A'D'\right\vert \cdot \left\vert B'C'\right\vert }%
\]
which is the same thing as showing that%
\[
\left\vert A'C'\right\vert \text{\textperiodcentered
}\left\vert B'D'\right\vert =\left\vert A'D^{\prime
}\right\vert \cdot\left\vert B'C^{\prime
}\right\vert +\left\vert A'B'\right\vert
\cdot\left\vert C'D'\right\vert .
\]
Now check this last equality by high-school algebra.
\end{proof}

\end{document}
