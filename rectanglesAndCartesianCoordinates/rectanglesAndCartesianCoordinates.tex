\documentclass{ximera}

\usepackage{microtype}
\usepackage{tikz}
\usepackage{tkz-euclide}
\usetkzobj{all}
\tikzstyle geometryDiagrams=[ultra thick,color=blue!50!black]

\renewcommand{\epsilon}{\varepsilon}



%% \prerequisites{geometry}
%% \outcome{euclideanGeometry,parallelPostulate}

\title{Rectangles and cartesian coordinates}
\begin{document}
\begin{abstract}
In this activity, we explore some consequences of Euclid's fifth
axiom.
\end{abstract}
\maketitle



\section*{Euclid's fifth axiom, the parallel postulate}

We are finally ready to introduce Euclid's fifth and final axiom, the
so-called \textit{Parallel Postulate}.

\begin{axiom}[E5]
Through a point not on a line there passes a unique parallel line.
\end{axiom}

\textbf{NG} together with E5 is called euclidean geometry
(\textbf{EG}). As mentioned above, we will see later that there is
another geometry called \textit{hyperbolic geometry} (\textbf{HG})
that satisfies all the postulates of \textbf{NG} but not E5. In it,
the sum of the interior angles of a triangle will \textit{always} be
less than $180^\circ$! [MJG,134]

\begin{problem}
\label{24} Show that if two parallel lines are cut by a
transversal line, then alternate interior angles are equal.

\begin{hint}
Draw a picture and seek a contradiction.
\end{hint}
\begin{freeResponse}
Seeking a contradiction, suppose that we have two parallel lines cut
by a transversal line where the alternate interior angles are not
equal:
\begin{image}
\begin{tikzpicture}[geometryDiagrams]
\coordinate (A) at (0,4.5);
\coordinate (B) at (10,4.5);
\coordinate (C) at (0,1.5);
\coordinate (D) at (10,1.5);
\coordinate (E) at (1.5,0);
\coordinate (F) at (3.5,6);
\coordinate (G) at (2,1.5);
\coordinate (H) at (3,4.5);
\draw (A)--(B);
\draw (C)--(D);
\draw[thin] (E)--(F);

\tkzMarkAngle[size=0.6cm,thin](D,G,H)
\tkzLabelAngle[pos = 0.4](D,G,H){$\alpha$}

\tkzMarkAngle[size=0.5cm,thin](A,H,G)
\tkzLabelAngle[pos = -0.3](A,H,G){$\epsilon$}


%\draw[thin,step=.5cm] (0,0) grid (10,5);
\end{tikzpicture}
\end{image}
Without loss of generality, we may assume that $\epsilon<
\alpha$. From the vertex of the angle with measure $\epsilon$, add a
new line that crosses the tranversal line with angle $\alpha$. By E5,
this new line must cross the lower of the parallel lines:
\begin{image}
\begin{tikzpicture}[geometryDiagrams]
\coordinate (A) at (0,4.5);
\coordinate (B) at (10,4.5);
\coordinate (C) at (0,1.5);
\coordinate (D) at (10,1.5);
\coordinate (E) at (1.5,0);
\coordinate (F) at (3.5,6);
\coordinate (G) at (2,1.5);
\coordinate (H) at (3,4.5);
\coordinate (I) at (0,5.79);
\draw (A)--(B);
\draw (C)--(D);
\draw[thin] (E)--(F);
\draw[thin,dashed] (I)--(D);

\tkzMarkAngle[size=0.6cm,thin](D,G,H)
\tkzLabelAngle[pos = 0.4](D,G,H){$\alpha$}

\tkzMarkAngle[size=0.5cm,thin](I,H,G)
\tkzLabelAngle[pos = -0.3](I,H,G){$\alpha$}

\tkzMarkAngle[size=0.5cm,thin](G,H,D)
\tkzLabelAngle[pos = 0.3](G,H,D){$\beta$}
\end{tikzpicture}
\end{image}
Now we have constructed a triangle where two angle sum to $180^\circ$,
as we have seen, this is impossible. Hence if two parallel lines are
cut by a transversal line, then alternate interior angles are equal.
\end{freeResponse}

\end{problem}

\begin{definition} 
We will call the set of points on a line which lie on one side of a
given point a \textbf{ray}.  We call the given point the
\textbf{origin} of the ray.
\end{definition}

\begin{definition}
We call two rays in the plane \textbf{parallel} if they lie on
parallel lines and they both lie on the same side of the transversal
line passing through their origins.
\end{definition}

\begin{definition} An \textbf{angle} in the plane is the union of two ordered
rays with common origin and choice of one of the two connected regions
into which the union of the rays divides the plane. We often denote
angles by $\angle BAC$ where $A$ is the common origin and $B$ a point
along one of the rays, called the initial ray, and $C$ is a point
along the other ray, called the final ray.
\end{definition}


The choice of the region is either clear from the context or
explicitly given.

\begin{problem}
Given an angle $\angle BAC$ show by drawings the two regions into
which it divides the plane. Show how the (signed) measure of the angle
depends on:
\begin{enumerate}
\item Which region you pick as the interior/exterior. 
\item Which ray is the initial ray and which is the final ray of the
  angle.
\end{enumerate}
\begin{freeResponse}
We present a summary in the table below:
\begin{center}
\begin{tabular}{|c|c|}\hline
%\begin{image}
\begin{tikzpicture}[geometryDiagrams]
\coordinate (A) at (0,0);
\coordinate (B) at (2,2);
\coordinate (C) at (2.83,0);
\tkzDrawVector (A,B)
\tkzDrawVector (A,C)

\tkzLabelPoints[below](A,C)
\tkzLabelPoints[above](B)

\coordinate (initRay) at (1.42,0);
\node[below] at (initRay) {initial ray};

\coordinate (int) at (1.6,1);
\node[below] at (int) {interior};

\tkzMarkAngle[size=0.6cm,thin](C,A,B)
\end{tikzpicture}
%\end{image}
&
%\begin{image}
\begin{tikzpicture}[geometryDiagrams]
\coordinate (A) at (0,0);
\coordinate (B) at (2,2);
\coordinate (C) at (2.83,0);
\tkzDrawVector (A,B)
\tkzDrawVector (A,C)

\tkzLabelPoints[below](A,C)
\tkzLabelPoints[above](B)

\coordinate (initRay) at (1,1);
\node[above,rotate=45] at (initRay) {initial ray};

\coordinate (int) at (1.6,1);
\node[below] at (int) {interior};

\tkzMarkAngle[size=0.6cm,thin](C,A,B)
\end{tikzpicture}
%\end{image} 
\\

$\angle BAC$ is positive & $\angle BAC$ is negative\\\hline

%\begin{image}
\begin{tikzpicture}[geometryDiagrams]
\coordinate (A) at (0,0);
\coordinate (B) at (2,2);
\coordinate (C) at (2.83,0);
\tkzDrawVector (A,B)
\tkzDrawVector (A,C)

\tkzLabelPoints[below](A,C)
\tkzLabelPoints[above](B)

\coordinate (initRay) at (1,1);
\node[above,rotate=45] at (initRay) {initial ray};

\coordinate (ext) at (1.6,1);
\node[below] at (ext) {exterior};

\tkzMarkAngle[size=0.6cm,thin](B,A,C)
\end{tikzpicture}
%\end{image}
&
%\begin{image}
\begin{tikzpicture}[geometryDiagrams]
\coordinate (A) at (0,0);
\coordinate (B) at (2,2);
\coordinate (C) at (2.83,0);
\tkzDrawVector (A,B)
\tkzDrawVector (A,C)

\tkzLabelPoints[below](A,C)
\tkzLabelPoints[above](B)

\coordinate (initRay) at (1.42,0);
\node[below] at (initRay) {initial ray};

\coordinate (ext) at (1.6,1);
\node[below] at (ext) {exterior};

\tkzMarkAngle[size=0.6cm,thin](B,A,C)
\end{tikzpicture}\\
$\angle BAC$ is positive & $\angle BAC$ is negative\\ \hline

%\end{image}
\end{tabular}
\end{center}
\end{freeResponse}

\end{problem}

\begin{problem}
Show that angles $\angle BAC$ and $\angle B'A'C'$ in the
  euclidean plane are equal (have the same measure) if corresponding
  rays are parallel. Hence if $\angle B'A'C'$ can be rotated around
  $A'$ to obtain an angle $\angle B''A'C''$ with corresponding rays
  parallel to those of $\angle BAC$, then $\angle BAC$ and $\angle
  B'A'C'$ are equal.

\begin{hint}
Draw your angles and extend their legs so that you have two sets of
parallel sides.
\end{hint}
\begin{freeResponse}
Start with the given angles and extend their legs so that we now have two sets of parallel lines:
\begin{image}
\begin{tikzpicture}[geometryDiagrams]
\coordinate (A) at (2,3.5);
\coordinate (B) at (4,5.5);
\coordinate (C) at (4.83,3.5);
\tkzDrawVector (A,B)
\tkzDrawVector (A,C)

\coordinate (D) at (4.5,1.5);
\coordinate (E) at (6.5,3.5);
\coordinate (F) at (7.33,1.5);
\tkzDrawVector (D,E)
\tkzDrawVector (D,F)

\coordinate (G) at (8,3.5);
\coordinate (H) at (8,5);

\tkzLabelPoints[below](A,C)
\tkzLabelPoints[above](B)

\node[below,black] at (D) {$A'$};
\node[above,black] at (E) {$B'$};
\node[below,black] at (F) {$C'$}; 

\draw[thin] (-1,3.5)--(G);
\draw[thin] (-1,1.5)--(8,1.5);
\draw[thin] (3,0)--(H);
\draw[thin] (-1,.5)--(B);

\tkzLabelAngle[pos=0.4](C,A,B){$\alpha$}
\tkzMarkAngle[size=0.6cm,thin](C,A,B)

\tkzMarkAngle[size=0.7cm,thin](F,D,E)
\tkzLabelAngle[pos=0.5](E,D,F){$\alpha'$}

\tkzMarkAngle[size=0.8cm,thin](A,E,D)
\tkzLabelAngle[pos=-0.5](A,E,D){$\alpha''$}
%\draw[thin,step=.5cm] (0,0) grid (10,5);
\end{tikzpicture}
\end{image}
By our previous work with alternate interior angles,
\[
\alpha = \alpha'' \qquad\text{and}\qquad \alpha'' = \alpha',
\]
hence $\alpha = \alpha'$.
\end{freeResponse}
\end{problem}

\begin{problem}
Use the `uniqueness' assertion in E5 together with what we have
established about neutral geometry to show that in \textbf{EG} the sum
of the interior angles of any triangle is $180^\circ$.

\begin{hint}
Make two parallel lines, the first being the base of a given triangle
and the second being the unique parallel line that passes through the
vertex opposite to the base.
\end{hint}
\begin{freeResponse}
Start by drawing two parallel lines, the first containing the base of a
given triangle and the second passing through the vertex opposite to
the base:
\begin{image}
\begin{tikzpicture}[geometryDiagrams]
\coordinate (A) at (0,2);
\coordinate (B) at (2,5);
\coordinate (C) at (6.5,2);
\coordinate (F) at (9,2);
\coordinate (G) at (-1,2);
\coordinate (H) at (9,5);
\coordinate (I) at (-1,5);

\draw (A)--(B)--(C)--cycle;
\draw[thin] (G)--(F);
\draw[thin] (H)--(I);

\tkzLabelPoints[above](B)
\tkzLabelPoints[below](A,C)

\tkzMarkAngle[size=0.6cm,thin](A,B,C)
\tkzLabelAngle[pos = 0.35](A,B,C){$\beta$}

\tkzMarkAngle[size=0.6cm,thin](C,A,B)
\tkzLabelAngle[pos = 0.35](C,A,B){$\alpha$}

\tkzMarkAngle[size=0.9cm,thin](B,C,A)
\tkzLabelAngle[pos = 0.7](B,C,A){$\gamma$}

%\draw[step=.5cm] (0,0) grid (10,5);
\end{tikzpicture}
\end{image}
As we have seen, using E5 we can assert that alternate interior angles
are equal, hence we may further decorate our diagram as follows: 
\begin{image}
\begin{tikzpicture}[geometryDiagrams]
\coordinate (A) at (0,2);
\coordinate (B) at (2,5);
\coordinate (C) at (6.5,2);
\coordinate (F) at (9,2);
\coordinate (G) at (-1,2);
\coordinate (H) at (9,5);
\coordinate (I) at (-1,5);

\draw (A)--(B)--(C)--cycle;
\draw[thin] (G)--(F);
\draw[thin] (H)--(I);

\tkzLabelPoints[above](B)
\tkzLabelPoints[below](A,C)

\tkzMarkAngle[size=0.7cm,thin](A,B,C)
\tkzLabelAngle[pos = 0.35](A,B,C){$\beta$}

\tkzMarkAngle[size=0.6cm,thin](C,A,B)
\tkzLabelAngle[pos = 0.35](C,A,B){$\alpha$}

\tkzMarkAngle[size=0.7cm,thin](I,B,A)
\tkzLabelAngle[pos = -0.35](I,B,A){$\alpha$}

\tkzMarkAngle[size=0.9cm,thin](B,C,A)
\tkzLabelAngle[pos = 0.7](B,C,A){$\gamma$}

\tkzMarkAngle[size=0.7cm,thin](C,B,H)
\tkzLabelAngle[pos = 0.5](C,B,H){$\gamma$}
\end{tikzpicture}
\end{image}
Hence we see $\alpha + \beta+ \gamma = 180^\circ$.
\end{freeResponse}
\end{problem}


\begin{problem}
Show that in \textbf{EG} the sum of the interior angles of a
quadrilateral is $360^\circ$.
\begin{freeResponse}
As we did before, we note that given any quadrilateral, even a
nonconvex quadrilateral, we can break it into two triangles by
connecting appropriate opposite vertices:
\begin{image}
\begin{tikzpicture}[geometryDiagrams]
\coordinate (A) at (2,5);
\coordinate (B) at (.5,0);
\coordinate (C) at (2.5,2.5);
\coordinate (D) at (6,2);
\draw (A)--(B)--(C)--(D)--cycle;
\draw[thin,dashed] (A)--(C);
%\draw[step=.5cm,thin] (0,0) grid (10,5);
\end{tikzpicture}
\end{image}
Labeling angles, we see:
\begin{image}
\begin{tikzpicture}[geometryDiagrams]
\coordinate (A) at (2,5);
\coordinate (B) at (.5,0);
\coordinate (C) at (2.5,2.5);
\coordinate (D) at (6,2);
\draw (A)--(B)--(C)--(D)--cycle;
\draw[thin,dashed] (A)--(C);

\tkzMarkAngle[size=1.1cm,thin](B,A,C)
\tkzLabelAngle[pos = 0.8](B,A,C){$\alpha$}

\tkzMarkAngle[size=1.2cm,thin](C,B,A)
\tkzLabelAngle[pos = 0.8](C,B,A){$\beta$}

\tkzMarkAngle[size=.5cm,thin](A,C,B)
\tkzLabelAngle[pos = -0.3](A,C,B){$\gamma$}

\tkzMarkAngle[size=1.2cm,thin](C,A,D)
\tkzLabelAngle[pos = 0.8](C,A,D){$\alpha'$}

\tkzMarkAngle[size=1.3cm,thin](A,D,C)
\tkzLabelAngle[pos = 1](A,D,C){$\beta'$}

\tkzMarkAngle[size=.6cm,thin](D,C,A)
\tkzLabelAngle[pos = 0.3](D,C,A){$\gamma'$}
%\draw[step=.5cm,thin] (0,0) grid (10,5);
\end{tikzpicture}
\end{image}
Now we have that the sum of the interior angles of this quadrilateral
is
\begin{align*}
\alpha + \beta + \gamma + \alpha'+\beta' + \gamma' &= 180^\circ + 180^\circ\\
&= 360^\circ. 
\end{align*}
\end{freeResponse}
\end{problem}

\begin{problem}\label{112} 
Now we will show in \textbf{EG} that given any positive real numbers
$a$ and $b$, there exist rectangles with adjacent side of lengths $a$
and $b$. Your task is to fill-in the details of the proof below.

Start by constructing lines $\l_1$ and $\l_2$ with $\l_1$
perpendicular to $\l_2$ at point $A$. On $\l_1$ add point $B$ so that
$|AB|=a$. Next construct $\l_3$ perpendicular to $\l_1$ through
$B$. On $\l_3$ add point $C$ such that $|BC|=b$. Finally add $\l_4$
through $C$ so that $\l_4$ is perpendicular to $\l_3$. 
\begin{image}
\begin{tikzpicture}[geometryDiagrams]
\coordinate (A) at (2,1);
\coordinate (B) at (8,1);
\coordinate (C) at (8,4);
\coordinate (D) at (2,4);
\coordinate (E) at (0,1);
\coordinate (F) at (10,1);
\coordinate (G) at (2,0);
\coordinate (H) at (2,5);
\coordinate (I) at (8,0);
\coordinate (J) at (8,5);
\coordinate (K) at (0,4);
\coordinate (L) at (10,4);

\draw (E)--(F);
\draw (G)--(H);
\draw (I)--(J);
\draw (K)--(L);

\node[left] at (E) {$\l_1$};
\node[below] at (G) {$\l_2$};
\node[below] at (I) {$\l_3$};
\node[left] at (K) {$\l_4$};

\node[below] at (5,1) {$a$};
\node[right] at (8,2.5) {$b$};
%\node[above] at (5,4) {$a'$};
%\node[left] at (2,2.5) {$b'$};

\tkzMarkRightAngle(B,A,D)
\tkzMarkRightAngle(C,B,A)
\tkzMarkRightAngle(D,C,B)

\tkzLabelPoints[below left](A)
\tkzLabelPoints[below right](B)
\tkzLabelPoints[above right](C)
%\tkzLabelPoints[above left](D)

%\draw[step=.5cm,thin] (0,0) grid (10,5);
\end{tikzpicture}
\end{image}


Explain why $\l_3$ is parallel to $\l_2$.

\begin{freeResponse}
Since alternate interior angles are equal
\begin{image}
\begin{tikzpicture}[geometryDiagrams]
\coordinate (A) at (2,1);
\coordinate (B) at (8,1);
\coordinate (C) at (8,4);
\coordinate (D) at (2,4);
\coordinate (E) at (0,1);
\coordinate (F) at (10,1);
\coordinate (G) at (2,0);
\coordinate (H) at (2,5);
\coordinate (I) at (8,0);
\coordinate (J) at (8,5);
\coordinate (K) at (0,4);
\coordinate (L) at (10,4);

\draw (E)--(F);
\draw (G)--(H);
\draw (I)--(J);
\draw[thin] (K)--(L);

\node[left] at (E) {$\l_1$};
\node[below] at (G) {$\l_2$};
\node[below] at (I) {$\l_3$};
\node[left] at (K) {$\l_4$};

\node[below] at (5,1) {$a$};
\node[right] at (8,2.5) {$b$};
%\node[above] at (5,4) {$a'$};
%\node[left] at (2,2.5) {$b'$};

\tkzMarkRightAngle(B,A,D)
\tkzMarkRightAngle[thin](C,B,A)
\tkzMarkRightAngle[thin](D,C,B)
\tkzMarkRightAngle(I,B,F)
%\tkzMarkRightAngle[thin](A,D,C)

\tkzLabelPoints[below left](A)
\tkzLabelPoints[below right](B)
\tkzLabelPoints[above right](C)
%\tkzLabelPoints[above left](D)

%draw[step=.5cm,thin] (0,0) grid (10,5);
\end{tikzpicture}
\end{image}
we see that $\l_3$ is parallel to $\l_2$. 
\end{freeResponse}


Explain why $\l_4$ intersects $\l_2$.

\begin{freeResponse}
  Seeking a contradiction, suppse that $\l_4$ does not intersect
  $\l_2$. This would mean that $\l_4$ is parallel to $\l_2$. However,
  both $\l_4$ and $\l_3$ pass through point $C$ and now both are
  parallel to $\l_2$, this violates E5.
\end{freeResponse}


Call the intersection of $\l_2$ and $\l_4$ point $D$ and label the two
remaining sides $a'$ and $b'$. Explain why $\angle ADC = 90^\circ$.

\begin{freeResponse}
Since the sum of the interior angles of any quadrilateral is
$360^\circ$ and our quadrilateral has three angles of measure
$90^\circ$, we see that the fourth angle must also have a measure of
$90^\circ$.
\end{freeResponse}



Explain why $\l_1$ is parallel to $\l_4$.

\begin{freeResponse}
Since alternate interior angles are equal
\begin{image}
\begin{tikzpicture}[geometryDiagrams]
\coordinate (A) at (2,1);
\coordinate (B) at (8,1);
\coordinate (C) at (8,4);
\coordinate (D) at (2,4);
\coordinate (E) at (0,1);
\coordinate (F) at (10,1);
\coordinate (G) at (2,0);
\coordinate (H) at (2,5);
\coordinate (I) at (8,0);
\coordinate (J) at (8,5);
\coordinate (K) at (0,4);
\coordinate (L) at (10,4);

\draw (E)--(F);
\draw[thin] (G)--(H);
\draw (I)--(J);
\draw (K)--(L);

\node[left] at (E) {$\l_1$};
\node[below] at (G) {$\l_2$};
\node[below] at (I) {$\l_3$};
\node[left] at (K) {$\l_4$};

\node[below] at (5,1) {$a$};
\node[right] at (8,2.5) {$b$};
\node[above] at (5,4) {$a'$};
\node[left] at (2,2.5) {$b'$};

\tkzMarkRightAngle[thin](B,A,D)
\tkzMarkRightAngle(C,B,A)
\tkzMarkRightAngle[thin](D,C,B)
\tkzMarkRightAngle(B,C,L)
\tkzMarkRightAngle[thin](A,D,C)

\tkzLabelPoints[below left](A)
\tkzLabelPoints[below right](B)
\tkzLabelPoints[above right](C)
\tkzLabelPoints[above left](D)

%draw[step=.5cm,thin] (0,0) grid (10,5);
\end{tikzpicture}
\end{image}
we see that $\l_1$ is parallel to $\l_4$. 
\end{freeResponse}



Finally, add a segment to our figure and use a triangle congruence
theorem to explain why $a=a'$ and $b = b'$.

\begin{freeResponse}
Add segment $\bar{AC}$ to our figure:
\begin{image}
\begin{tikzpicture}[geometryDiagrams]
\coordinate (A) at (2,1);
\coordinate (B) at (8,1);
\coordinate (C) at (8,4);
\coordinate (D) at (2,4);
\coordinate (E) at (0,1);
\coordinate (F) at (10,1);
\coordinate (G) at (2,0);
\coordinate (H) at (2,5);
\coordinate (I) at (8,0);
\coordinate (J) at (8,5);
\coordinate (K) at (0,4);
\coordinate (L) at (10,4);

\draw (E)--(F);
\draw (G)--(H);
\draw (I)--(J);
\draw (K)--(L);
\draw (A)--(C);

\node[left] at (E) {$\l_1$};
\node[below] at (G) {$\l_2$};
\node[below] at (I) {$\l_3$};
\node[left] at (K) {$\l_4$};

\node[below] at (5,1) {$a$};
\node[right] at (8,2.5) {$b$};
\node[above] at (5,4) {$a'$};
\node[left] at (2,2.5) {$b'$};

\tkzMarkRightAngle(A,D,C)
\tkzMarkRightAngle(C,B,A)

\tkzLabelPoints[below left](A)
\tkzLabelPoints[below right](B)
\tkzLabelPoints[above right](C)
\tkzLabelPoints[above left](D)

\tkzMarkAngle[size=.4cm,thin](C,A,D)
\tkzMarkAngle[size=.4cm,thin](A,C,B)
\tkzMarkAngle[arc=ll,size=.6cm,thin](B,A,C)
\tkzMarkAngle[arc=ll,size=.6cm,thin](D,C,A)

%draw[step=.5cm,thin] (0,0) grid (10,5);
\end{tikzpicture}
\end{image}
Since alternate interior angles are equal, we have that
\[
\angle ACB \cong \angle CAD\qquad\text{and}\qquad \angle CAB \cong \angle DCA.
\]
Now by ASA
\[
\triangle ACB \cong \triangle CAD.
\]
Hence we see that $a = a'$ and $b = b'$. This completes the proof that
$ABCD$ is a rectangle.
\end{freeResponse}
\end{problem}



\begin{problem}
Show that there is a cartesian coordinate system on \textbf{EG}. This
means you must show that there is a bijection between the points in
\textbf{EG} and elements of $\R^2$, the set of pairs of real numbers.

\begin{freeResponse}
Pick any point of \textbf{EG}, call it $A$. Our map will take this
point to $(0,0)$. By our previous problem, we can construct an
$a\times b$ rectangle $ABCD$ in \textbf{EG}.
\begin{image}
\begin{tikzpicture}[geometryDiagrams]
\coordinate (A) at (2,1);
\coordinate (B) at (8,1);
\coordinate (C) at (8,4);
\coordinate (D) at (2,4);
\coordinate (E) at (0,1);
\coordinate (F) at (10,1);
\coordinate (G) at (2,0);
\coordinate (H) at (2,5);
\coordinate (I) at (8,0);
\coordinate (J) at (8,5);
\coordinate (K) at (0,4);
\coordinate (L) at (10,4);

\draw (E)--(F);
\draw (G)--(H);
\draw (I)--(J);
\draw (K)--(L);

\node[below] at (5,1) {$a$};
\node[right] at (8,2.5) {$b$};
\node[above] at (5,4) {$a$};
\node[left] at (2,2.5) {$b$};

\tkzMarkRightAngle(B,A,D)
\tkzMarkRightAngle(C,B,A)
\tkzMarkRightAngle(D,C,B)
\tkzMarkRightAngle(A,D,C)

\tkzLabelPoints[below left](A)
\tkzLabelPoints[below right](B)
\tkzLabelPoints[above right](C)
\tkzLabelPoints[above left](D)

%draw[step=.5cm,thin] (0,0) grid (10,5);
\end{tikzpicture}
\end{image}
Our map will take the the point $C$ to $(a,b)$. On the other hand, if
$B$ was placed to the left of $A$, our map takes $C$ to the point
$(-a,b)$. If $D$ was placed below $A$, then our map takes $C$ to
$(a,-b)$. Finally if $B$ was placed to the left of $A$ and $D$ was
placed below $A$, then our map takes $C$ to $(-a,-b)$.  By
construction, this map is one-to-one and onto and hence shows that we
have a cartesian coordinate system in \textbf{EG}.
\end{freeResponse}

\end{problem}


\section*{The distance formula in \textbf{EG}}

It is the existence of a cartesian coordinate system in \textbf{EG} that
allows us to define distance between points%
\[
d\left( (a_{1},b_{1}), (a_{2},b_{2}) \right)
=\sqrt{(a_{2}-a_{1})^{2}+(b_{2}-b_{1})^{2}}
\]
and so gives rigorous mathematical meaning to a concept that the
ancient Greeks were never able to describe precisely, namely the
similarity of figures in \textbf{EG}. For that we will require the
notion of a \textit{dilation} or \textit{magnification} in
\textbf{EG}. We need a cartesian coordinate system to describe
dilation precisely, a reality backed up by the fact that similarities
do not exist in \textbf{HG} or \textbf{SG}. (Try drawing two triangles
that are similar but not congruent on a perfectly spherical balloon!)

\begin{problem}
State and prove the Pythagorean theorem in \textbf{EG}.

\begin{hint}
In the cartesian plane, construct a square with vertices
\[
\left(0,0\right),\qquad \left(a+b,0\right),\qquad \left(0,a+b\right), \qquad \left( a+b,a+b\right).
\] 
Inside that square, construct the square with vertices 
\[
\left(a,0\right),\qquad \left(a+b,a\right),\qquad \left(b,a+b\right), \qquad \left(0,b\right).
\]
\end{hint}
\begin{freeResponse}
Consider the square in the cartesian plane with vertices
\[
\left(0,0\right),\qquad \left(a+b,0\right),\qquad \left(0,a+b\right),
\qquad \left( a+b,a+b\right)
\]
with a second quadrilateral with vertices
\[
\left(a,0\right),\qquad \left(a+b,a\right),\qquad \left(b,a+b\right), \qquad \left(0,b\right).
\]
\begin{image}
\begin{tikzpicture}[geometryDiagrams]
\coordinate (A) at (0,0);
\coordinate (B) at (0,4);
\coordinate (C) at (4,4);
\coordinate (D) at (4,0);
\coordinate (E) at (1,0);
\coordinate (F) at (0,3);
\coordinate (G) at (3,4);
\coordinate (H) at (4,1);

\draw (A)--(B)--(C)--(D)--cycle;
\draw (E)--(F)--(G)--(H)--cycle;

\node[below left] at (A) {$(0,0)$};
\node[above left] at (B) {$(0,a+b)$};
\node[above right] at (C) {$(a+b,a+b)$};
\node[below right] at (D) {$(a+b,0)$};

\node[below] at (E) {$(a,0)$};
\node[left] at (F) {$(0,b)$};
\node[above] at (G) {$(b,a+b)$};
\node[right] at (H) {$(a+b,a)$};

%draw[step=.5cm,thin] (0,0) grid (10,5);
\end{tikzpicture}
\end{image}
We should explain why the smaller quadrilateral is in fact a
square. First note that each side has the same length, as by SAS the
sides are made of congruent triangles. Second note that each of the
interior angles are equal to each other, and since the sum of the
interior angles of a quadrilateral is $360^\circ$, each interior angle
must be $90^\circ$.

Now rearrange the figure above as follows:
\begin{image}
\begin{tikzpicture}[geometryDiagrams]
\coordinate (A) at (0,0);
\coordinate (B) at (0,4);
\coordinate (C) at (4,4);
\coordinate (D) at (4,0);
\coordinate (E) at (1,0);
\coordinate (F) at (0,3);
\coordinate (G) at (1,4);
\coordinate (H) at (4,3);
\coordinate (I) at (1,3);

\draw (A)--(B)--(C)--(D)--cycle;
\draw (E)--(F);
\draw (I)--(C);
\draw (E)--(G);
\draw (F)--(H);

\node[below left] at (A) {$(0,0)$};
\node[above left] at (B) {$(0,a+b)$};
\node[above right] at (C) {$(a+b,a+b)$};
\node[below right] at (D) {$(a+b,0)$};

\node[below] at (E) {$(a,0)$};
\node[left] at (F) {$(0,b)$};
\node[above] at (G) {$(b,a+b)$};
\node[right] at (H) {$(a+b,a)$};

%draw[step=.5cm,thin] (0,0) grid (10,5);
\end{tikzpicture}
\end{image}
Comparing both figures and labeling the ``long side'' $c$, we see:
\begin{image}
\begin{tikzpicture}[geometryDiagrams]
%% First square
\coordinate (A) at (0,0);
\coordinate (B) at (0,4);
\coordinate (C) at (4,4);
\coordinate (D) at (4,0);
\coordinate (E) at (1,0);
\coordinate (F) at (0,3);
\coordinate (G) at (3,4);
\coordinate (H) at (4,1);

\draw (A)--(B)--(C)--(D)--cycle;
\draw (E)--(F)--(G)--(H)--cycle;

\node[right] at (.5,1.5) {$c$};
\node[above] at (2.5,.5) {$c$};
\node[left] at (3.5,2.5) {$c$};
\node[below] at (1.5,3.5) {$c$};

%% Second square
\coordinate (AA) at (6,0);
\coordinate (BB) at (6,4);
\coordinate (CC) at (10,4);
\coordinate (DD) at (10,0);
\coordinate (EE) at (7,0);
\coordinate (FF) at (6,3);
\coordinate (GG) at (7,4);
\coordinate (HH) at (10,3);
\coordinate (II) at (7,3);

\draw (AA)--(BB)--(CC)--(DD)--cycle;
\draw (EE)--(FF);
\draw (II)--(CC);
\draw (EE)--(GG);
\draw (FF)--(HH);

\node[below] at (6.5,3) {$a$};
\node[right] at (7,3.5) {$a$};
\node[left] at (6,3.5) {$a$};
\node[above] at (6.5,4) {$a$};

\node[below] at (8.5,3) {$b$};
\node[left] at (10,1.5) {$b$};
\node[right] at (7,1.5) {$b$};
\node[above] at (8.5,0) {$b$};
\end{tikzpicture}
\end{image}
Since both figures have the same area, they will have the same area
when all of the triangles are removed:
\begin{image}
\begin{tikzpicture}[geometryDiagrams]
%% First square
\coordinate (E) at (1,0);
\coordinate (F) at (0,3);
\coordinate (G) at (3,4);
\coordinate (H) at (4,1);

\draw (E)--(F)--(G)--(H)--cycle;

\node[right] at (.5,1.5) {$c$};
\node[above] at (2.5,.5) {$c$};
\node[left] at (3.5,2.5) {$c$};
\node[below] at (1.5,3.5) {$c$};

%% Second square
\coordinate (BB) at (6,4);
\coordinate (DD) at (10,0);
\coordinate (EE) at (7,0);
\coordinate (FF) at (6,3);
\coordinate (GG) at (7,4);
\coordinate (HH) at (10,3);
\coordinate (II) at (7,3);

\draw (EE)--(GG)--(BB)--(FF)--(HH)--(DD)--cycle;

\node[below] at (6.5,3) {$a$};
\node[right] at (7,3.5) {$a$};
\node[left] at (6,3.5) {$a$};
\node[above] at (6.5,4) {$a$};

\node[below] at (8.5,3) {$b$};
\node[left] at (10,1.5) {$b$};
\node[right] at (7,1.5) {$b$};
\node[above] at (8.5,0) {$b$};
\end{tikzpicture}
\end{image}
Hence we see that given a right triangle, 
\begin{image}
\begin{tikzpicture}[geometryDiagrams]
\coordinate (A) at (0,2);
\coordinate (B) at (0,5);
\coordinate (C) at (6.5,2);
\tkzMarkRightAngle(C,A,B)
\tkzDefMidPoint(A,B) \tkzGetPoint{a}
\tkzDefMidPoint(A,C) \tkzGetPoint{b}
\tkzDefMidPoint(B,C) \tkzGetPoint{c}
\draw (A)--(B)--(C)--cycle;
\tkzLabelPoints[above](c)
\tkzLabelPoints[below](b)
\tkzLabelPoints[left](a)
\end{tikzpicture}
\end{image}
we have that $a^2 + b^2 = c^2.$
\end{freeResponse}

\end{problem}

\begin{problem}
Use the Pythagorean theorem to justify the euclidean distance formula,
\[
d\left( (a_{1},b_{1}), (a_{2},b_{2}) \right)
=\sqrt{(a_{2}-a_{1})^{2}+(b_{2}-b_{1})^{2}}.
\]

\begin{freeResponse}
Plot the points $(a_1,b_1)$ and $(a_2,b_2)$.
\begin{image}
\begin{tikzpicture}[geometryDiagrams]
	\begin{axis}[
            xmin=0, xmax=5,ymin=0,ymax=5,
            axis lines =left, xlabel=$x$, ylabel=$y$,
            every axis y label/.style={at=(current axis.above origin),anchor=south},
            every axis x label/.style={at=(current axis.right of origin),anchor=west},
            xtick={1,3}, xticklabels={$a_1$,$a_2$},
            ytick={2,3}, yticklabels={$b_1$,$b_2$},
            axis on top,
          ]       
          \addplot[penColor,dashed] plot coordinates {(1,0) (1,2)};
          \addplot[penColor,dashed] plot coordinates {(0,2) (1,2)};
          \addplot[penColor,dashed] plot coordinates {(3,0) (3,3)};
          \addplot[penColor,dashed] plot coordinates {(0,3) (3,3)};
          \addplot[penColor,only marks,mark=*] coordinates{(1,2)};  %% closed hole          
          \addplot[penColor,only marks,mark=*] coordinates{(3,3)};  %% closed hole          
        \end{axis}
\end{tikzpicture}
\end{image}
We may now construct a right triangle with sides of length $(a_2-a_1)$
and $(b_2-b_1)$ whose hypotenuse is the shortest path between the two points. 
\begin{image}
\begin{tikzpicture}[geometryDiagrams]
	\begin{axis}[
            xmin=0, xmax=5,ymin=0,ymax=5,
            axis lines =left, xlabel=$x$, ylabel=$y$,
            every axis y label/.style={at=(current axis.above origin),anchor=south},
            every axis x label/.style={at=(current axis.right of origin),anchor=west},
            xtick={1,3}, xticklabels={$a_1$,$a_2$},
            ytick={2,3}, yticklabels={$b_1$,$b_2$},
            axis on top,
          ]       
          \node at (axis cs:2,1.75) {$(a_2-a_1)$}; 
          \node at (axis cs:3.5,2.5) {$(b_2-b_1)$}; 
          \addplot[dashed] plot coordinates {(1,2) (3,3)};
          \addplot[penColor] plot coordinates {(1,2) (3,2)};
          \addplot[penColor] plot coordinates {(3,2) (3,3)};
          \addplot[penColor,only marks,mark=*] coordinates{(1,2)};  %% closed hole          
          \addplot[penColor,only marks,mark=*] coordinates{(3,3)};  %% closed hole          
        \end{axis}
\end{tikzpicture}
\end{image}
By the Pythagorean theorem, the length of this path is given by 
\[
\sqrt{(a_{2}-a_{1})^{2}+(b_{2}-b_{1})^{2}}.
\]
\end{freeResponse}

\end{problem}

\begin{problem}
Summarize the results from this section. In particular, indicate which
results follow from the others.
\begin{freeResponse}
\end{freeResponse}
\end{problem}


\end{document}
