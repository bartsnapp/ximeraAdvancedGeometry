\shortdescription{In this activity we explore some consequences of Euclid's fifth axiom.}
\activitytitle{Rectangles and cartesian coordinates}
\prerequisites{geometry}
\outcomes{euclideanGeometry,parallelPostulate}



\subsection*{Euclid's fifth axiom, the parallel postulate}

We are finally ready to introduce Euclid's fifth and final axiom, the
so-called \textit{Parallel Postulate}.

\begin{axiom}
(E5) Through a point not on a line there passes a unique parallel line.
\end{axiom}

\textbf{NG} together with E5 is called Euclidean geometry (\textbf{EG}). As
mentioned above, we will see later that there is another geometry
(\textbf{HG}) that satisfies all the postulates of \textbf{NG} but not E5. In
it, the sum of the interior angles of a triangle will \textit{always} be less
than $180^\circ$! [MJG,134]

\begin{question}
\label{24} Show that, if two parallel lines are cut by a
transversal line, opposite interior angles are equal.
\end{question}

We will call the set of points on a line which lie on one side of a given
point a \textit{ray}. We call the given point the \textit{origin} of the ray.

We call two rays in the plane parallel if they lie on parallel lines and they
both lie on the same side of the transversal line passing through their origins.

Strictly speaking, an angle in the plane is the union of two ordered rays with
common origin and choice of one of the two connected regions into which the
union of the rays divides the plane. We often denote angles by $\angle BAC$
where $A$ is the common origin and $B$ a point along one of the rays, called
the initial ray, and $C$ is a point along the other ray, called the final ray.
The choice of the region is either clear from the context or explicitly given.

\begin{question}\hfil
\begin{enumerate}
\item Given an angle $\angle BAC$ show by drawings the two regions into
which it divides the plane. Show how the (signed) measure of the angle
depends on which region you pick and on which is the initial ray and
which is the final ray of the angle.
\item Show that angles $\angle BAC$ and $\angle B^{\prime}A^{\prime}C^{\prime}$
in the Euclidean plane are equal (i.e. have the same measure) if
corresponding rays are parallel. Hence if $\angle
B^{\prime}A^{\prime}C^{\prime}$ can be rotated around $A^{\prime}$ to
obtain an angle $\angle B^{\prime\prime }A^{\prime}C^{\prime\prime}$
with corresponding rays parallel to those of $\angle BAC$, then
$\angle BAC$ and $\angle B^{\prime}A^{\prime}C^{\prime}$ are equal.
\end{enumerate}
\end{question}

\begin{question}
Use the `uniqueness' assertion in E5 together with what we have
established about Neutral Geometry to show that in \textbf{EG} the sum
of the interior angles of any triangle is $180^\circ$.
\end{question}

\begin{question}
Show that in \textbf{EG} the sum of the interior angles of a
quadrilateral is $360^\circ$.
\end{question}

\begin{question}\label{112} 
Show in \textbf{EG} that, given any positive real numbers $a$ and $b$,
there exist rectangles with adjacent side of lengths $a$ and $b$.

Hint: To show that opposite sides are of equal length, suppose
not. For example, suppose the top of the rectangle has length
$a^{\prime}$ and the bottom has length $a$ and, for example,
$a^{\prime}>a$. Mark the point at length $a$ along the top, starting
at the left-hand vertex. Connect that point to the right-hand bottom
vertex. Find a triangle that contradicts the fact that in \textbf{NG}
an exterior angle is greater than either remote interior angles.
\end{question}

\begin{question}
Show that there is a cartesian coordinate system on \textbf{EG}, that
is, the set of points of \textbf{EG} are in one-to-one correspondence
with the set of pairs of real numbers.
\end{question}

\pagebreak

\subsection*{The distance formula in \textbf{EG}}

It is the existence of a cartesian coordinate system in \textbf{EG} that
allows us to define distance between points%
\begin{equation}
d\left(  \left(  a_{1},b_{1}\right)  ,\left(  a_{2},b_{2}\right)  \right)
=\sqrt{\left(  a_{2}-a_{1}\right)  ^{2}+\left(  b_{2}-b_{1}\right)  ^{2}}
\label{113}%
\end{equation}
and so gives rigorous mathematical meaning to a concept that the ancient
Greeks were never able to describe precisely, namely the similarity of figures
in \textbf{EG}. For that we will require the notion of a dilation or
magnification in \textbf{EG}. And we need a cartesian coordinate system to
describe dilation precisely, a reality backed up by the fact that similarities
do not exist in \textbf{HG} or \textbf{SG}. (Try drawing two triangles that
are similar but not congruent on a perfectly spherical balloon!)

\begin{question}
State the Pythagorean theorem in \textbf{EG} and use Question
\ref{112} to prove it.

Hint: In the cartesian plane, construct a square with vertices 
\[
\left(0,0\right),\qquad \left(a+b,0\right),\qquad \left(0,a+b\right), \qquad \left( a+b,a+b\right).
\] 
Inside that square, construct the square with vertices 
\[
\left(a,0\right),\qquad \left(a+b,a\right),\qquad \left(b,a+b\right), \qquad \left(0,b\right).
\]
Show that the area of the big square is the area of the little square
plus the area of $4$ right triangles, each of area $\frac{ab}{2}$.
\end{question}

\begin{question}
Use the Pythagorean theorem to justify the Euclidean distance formula,
see $\left(\ref{113}\right)$.
\end{question}

\pagebreak

\subsection*{Dilations in \textbf{EG}}

\begin{definition}
A \textbf{dilation} is a one-to-one onto transformation of the
cartesian plane to itself that
\begin{enumerate}
\item fixes one point called the center of the dilation,
\item takes each line through the fixed point to itself,
\item multiples all distances by a fixed positive real number called the
magnification factor of the dilation.
\end{enumerate}
\end{definition}

\begin{definition}
Given a point $\left(  x_{0},y_{0}\right)  $ in the (cartesian) plane and a
positive real number $r$, we define a mapping $D$ with center $\left(
x_{0},y_{0}\right)  $ and magnification factor $r$ by the formula%
\begin{equation}
D\left(  x,y\right)  =\left(  x_{0},y_{0}\right)  +r\left(  x-x_{0}%
,y-y_{0}\right)  . \label{23}%
\end{equation}
We will also denote the output $D\left(  x,y\right)  $ of the dilation as
$\left(  \underline{x},\underline{y}\right)  $.
\end{definition}

\begin{question}
Using cartesian coordinates for the plane, show that the mapping
$D$ defined in $\left(  \ref{23}\right)  $ is a dilation with magnification
factor $r$ and center $\left(  x_{0},y_{0}\right)  $.
\end{question}

\begin{question}
Show (using several-variable calculus if you wish) that a
dilation with magnification factor $r$ multiplies all areas by a factor of
$r^{2}$.
\end{question}

\begin{question}\hfil
\begin{enumerate}
\item Show that the inverse mapping of a dilation is again a
dilation with the same center but with magnification factor $r^{-1}$.
\item Show that a dilation takes lines to lines.
\end{enumerate}
Hint: For a)\ solve for $\left(  x,y\right)  $ in terms of $\left(
\underline{x},\underline{y}\right)  $. For b) write the equation%
\[
ax+by=c
\]
for the given line. Then substitute for $\left(  x,y\right)  $ its expression
in terms of $\left(  \underline{x},\underline{y}\right)$.
\end{question}

\begin{question}
Show that a dilation takes any line to a line parallel (or
equal) to itself.

Hint: Compute slopes.
\end{question}

\begin{question}
Show that a dilation by a factor of $r$ takes any vector to $r$
times itself.

Hint: Realize the vector as the difference of two points.
\end{question}

\begin{question}
Show that a dilation of the plane preserves angles.

Hint: Use the dot product of vectors emanating from the same point to measure
angles
\end{question}

\newpage

\subsection*{Similarity in \textbf{EG}}

\begin{definition}
Two triangles are \textbf{similar} if there is a dilation of the
plane that takes one to a triangle which is congruent to the other. We write%
\[
\triangle ABC\sim\triangle A^{\prime}B^{\prime}C^{\prime}%
\]
to denote that these two triangles are similar (where the order of the
vertices tells us which vertices correspond).
\end{definition}

\begin{question}
\begin{enumerate}
\item Show that, if two triangles are similar, then corresponding
sides are proportional.

Hint: You have to start from the supposition that the two triangles satisfy
the definition of similar triangles.

\item Show that, if corresponding sides of two triangles are proportional, then
the two triangles are similar.

Hint: You have to start from the supposition that corresponding sides of the
two triangles are proportional and use SSS to show that there is a dilation of
$\triangle ABC$ is congruent to $\triangle A^{\prime}B^{\prime}C^{\prime}$.
\end{enumerate}
\end{question}

\begin{question}
\begin{enumerate}
\item Show that, if two triangles are similar, then corresponding
angles are equal.

Hint: You have to start from the supposition that the two triangles satisfy
the definition of similar triangles.

\item Show that, if corresponding angles of two triangles are equal, then the two
triangles are similar.

Hint: You have to start from the supposition that corresponding angles of the
two triangles are equal, then use a dilation with $r=|A^{\prime}B^{\prime
}|/|AB|$ and ASA to show that the dilation of one triangle that is congruent
to the other.
\end{enumerate}
\end{question}

\begin{question}
\label{39} Show that two triangles are similar if corresponding
sides are parallel.

Hint: Use the fact that angles are equal if corresponding rays are
parallel.
\end{question}

\begin{question}
Show that two triangles are similar if corresponding sides are perpendicular.

Hint: Extend one of the rays of the first angle until it crosses the
corresponding ray of the second angle.
\end{question}

\pagebreak

\subsection*{Concurrence theorems in \textbf{EG}, Ceva's theorem}

Before leaving (plane) Euclidean Geometry, we will visit two more of
its many sets of memorable properties, one's that you may or may not
have seen in high school. The first of these comes under the name of
concurrence theorems--these theorems relate the measures of the three
sides (or angles) of a triangle to the measure of quantities
constructed from those sides by some uniform rule.

\begin{question}
\label{25} Denote the measure or area of a triangle $\triangle
ABC$ as $\left\vert \triangle AFC\right\vert $. Show that, in the
diagram below,
\[
\frac{\left\vert AF\right\vert }{\left\vert FB\right\vert }=\frac{\left\vert
\triangle AFC\right\vert }{\left\vert \triangle CFB\right\vert }%
=\frac{\left\vert \triangle AFX\right\vert }{\left\vert \triangle
XFB\right\vert }.
\]
\[
\begin{tikzpicture}[geometryDiagrams]
\tkzDefPoint(0,0){A}
\tkzDefPoint(5,3){C}
\tkzDefPoint(8,0){B}
\tkzDefPoint(4.79,1.33){X}
\tkzDefPoint(4.63,0){F}
\tkzDefPoint(6.26,1.74){D}
\tkzDefPoint(3.27,1.96){E}
\draw (A)--(B)--(C)--cycle;
\draw (A)--(D);
\draw (F)--(C);
\draw (B)--(E);


\tkzLabelPoints[above](C)
\tkzLabelPoints[above right](X)
\tkzLabelPoints[below](A,B,F)
\tkzLabelPoints[above left](E)
\tkzLabelPoints[right](D)
%\draw[step=.5cm] (0,0) grid (10,5);
\end{tikzpicture}
\]
\end{question}

\begin{question}
\label{26} Use Question \ref{25} to show by pure algebra that%
\begin{equation}
\frac{\left\vert AF\right\vert }{\left\vert FB\right\vert }=\frac{\left\vert
\triangle AXC\right\vert }{\left\vert \triangle CXB\right\vert }. \label{27}%
\end{equation}

\end{question}

\begin{question}
\label{28}  For three concurrent segments $\overline{AD}$,
$\overline{BE}$ and $\overline{CF}$ as given in Question \ref{25}, use
Question \ref{26} to show that%
\[
\frac{\left\vert AF\right\vert }{\left\vert FB\right\vert }%
\text{\textperiodcentered}\frac{\left\vert BD\right\vert }{\left\vert
DC\right\vert }\text{\textperiodcentered}\frac{\left\vert CE\right\vert
}{\left\vert EA\right\vert }=1.
\]


Hint: Use Question \ref{25} repeadetly. 
%% Hint: Use Question \ref{27} with side $\overline{BC}$ replacing
%% $\overline{AB}$, and $\left( \ref{27}\right)$ with side
%% $\overline{CA}$ replacing $\overline{AB}$.
\end{question}

This last result with its converse, which you will show in the next Question,
is called \textit{Ceva's Theorem}. [MJG,287-288]

\begin{question}
Show the converse of the result in Question \ref{28}, namely
that, if%
\[
\frac{\left\vert AF\right\vert }{\left\vert FB\right\vert }%
\text{\textperiodcentered}\frac{\left\vert BD\right\vert }{\left\vert
DC\right\vert }\text{\textperiodcentered}\frac{\left\vert CE\right\vert
}{\left\vert EA\right\vert }=1
\]
then the lines $AD$, $BE$, and $CF$ pass through a common point.

Hint: Suppose that they do not pass through a common point.
\[
\begin{tikzpicture}[geometryDiagrams]
\tkzDefPoint(0,0){A}
\tkzDefPoint(5,3){C}
\tkzDefPoint(8,0){B}
\tkzDefPoint(5,0){F}
\tkzDefPoint(6.26,1.74){D}
\tkzDefPoint(2.75,1.65){E}
\draw (A)--(B)--(C)--cycle;
\draw (A)--(D);
\draw (F)--(C);
\draw (B)--(E);


\tkzLabelPoints[above](C)
\tkzLabelPoints[below](A,B,F)
\tkzLabelPoints[above left](E)
\tkzLabelPoints[right](D)
%\draw[step=.5cm] (0,0) grid (10,5);
\end{tikzpicture}
\]
Notice that if, for example, $F$ moves along the segment
$\overline{AB}$, then $\frac{\left\vert AF\right\vert }{\left\vert
FB\right\vert }$ is a strictly increasing function of $\left\vert
AF\right\vert $. Now use Question \ref{28} to determine a position
$F^{\prime}$ for $F$ along the segment $\overline{AB}$ at which
\[
\frac{\left\vert AF^{\prime}\right\vert }{\left\vert F^{\prime}B\right\vert
}\text{\textperiodcentered}\frac{\left\vert BD\right\vert }{\left\vert
DC\right\vert }\text{\textperiodcentered}\frac{\left\vert CE\right\vert
}{\left\vert EA\right\vert }=1.
\]
\end{question}

\begin{question}
A \textbf{median} of a triangle is a line segment from a vertex
to the midpoint of the opposite side. Show that the medians of any triangle
meet in a common point.

Hint: Use Ceva's Theorem.
\end{question}

\begin{question}
Use Ceva's theorem to show that the three altitudes of a
triangle are concurrent.
\[
\begin{tikzpicture}[geometryDiagrams]
\tkzDefPoint(0,0){A}
\tkzDefPoint(5,4){C}
\tkzDefPoint(6.5,0){B}
\tkzDefPoint(5,0){F}
\tkzDefPoint(5.54,2.56){D}
\tkzDefPoint(3.98,3.18){E}
\draw (A)--(B)--(C)--cycle;
\draw (A)--(D);
\draw (F)--(C);
\draw (B)--(E);


\tkzLabelPoints[above](C)
\tkzLabelPoints[below](A,B,F)
\tkzLabelPoints[above left](E)
\tkzLabelPoints[right](D)
%\draw[step=.5cm] (0,0) grid (10,5);
\end{tikzpicture}
\]
Hint: Use all three similarities of the form $\triangle
CEB\sim\triangle CDA$ and then apply Ceva's theorem.
\end{question}



