\documentclass{ximera}

\usepackage{microtype}
\usepackage{tikz}
\usepackage{tkz-euclide}
\usetkzobj{all}
\tikzstyle geometryDiagrams=[ultra thick,color=blue!50!black]

\renewcommand{\epsilon}{\varepsilon}



\title{Rigid motions in central projection}

\begin{document}
\begin{abstract}
  Here we study rigid motions in central projection coordinates.
\end{abstract}
\maketitle

\section{Rigid motions in central projection coordinates}

Suppose now we have a $K$-rigid motion
\[
\left(  \underline{x},\underline{y},\underline{z}\right)  =\left(
x,y,z\right)  \cdot M
\]
of $K$-geometry, given by a $K$-orthogonal matrix%
\[
M=\begin{bmatrix}
m_{11} & m_{12} & m_{13}\\
m_{21} & m_{22} & m_{23}\\
m_{31} & m_{32} & m_{33}
\end{bmatrix}.
\]
Let's convert this $K$-rigid motion to a rigid motion in central
projection coordinates. This new rigid motion will not necessarily be
a mapping defined by a matrix, so we'll have to use some new notation.


\begin{image}
\begin{tikzpicture}
  \node (xyz) {$\begin{bmatrix}x & y & z\end{bmatrix}\in\mathbb{R}^3$};
  \node (xyzBar) [node distance=2cm,below of=xyz] {$\begin{bmatrix}\underline{x} & \underline{y} & \underline{z}\end{bmatrix}$};
  \node (xcyc1) [node distance=4cm,right of=xyz] {$(x_c,y_c,1)\in\mathbb{R}^2\times\{1\}$};
  \node (xcyc1Bar) [node distance=4cm,right of=xyzBar] {$(\underline{x_c},\underline{y_c},1)$};
  \node (xcyc) [node distance=4cm,right of=xcyc1] {$(x_c,y_c)\in\mathbb{R}^2$};
  \node (xcycBar) [node distance=4cm,right of=xcyc1Bar] {$(\underline{x_c},\underline{y_c})$};
  \draw[|->] (xcyc1) to node[above] {$\cdot \lambda$} (xyz);
  \draw[|->] (xcyc1Bar) to node [above] {$\cdot\underline{\lambda}$} (xyzBar);
  \draw[|->] (xcyc) to (xcyc1);
  \draw[|->] (xcycBar) to (xcyc1Bar);

  \draw[|->] (xyz) to node[right] {$\cdot M$} (xyzBar);
  \draw[|->] (xcyc1) to node[right] {$\mu_{c'}=?$} (xcyc1Bar);
  \draw[|->] (xcyc) to node[right] {$\mu_{c}=?$} (xcycBar);
\end{tikzpicture}
\end{image}

\begin{problem}
  Using the diagram above, explain why
  \begin{align*}
(\underline{x_{c}},\underline{y_{c}})  &= \mu_{c}(x_{c},y_{c}) \\
  & =\left(\frac{m_{11}x_{c}+m_{21}y_{c}+m_{31}}{m_{13}x_{c}+m_{23}y_{c}+m_{33}},
  \frac{m_{12}x_{c}+m_{22}y_{c}+m_{32}}{m_{13}x_{c}+m_{23}y_{c}+m_{33}
}\right).
  \end{align*}
  \begin{freeResponse}
    Let's follow this diagram around, starting at the right-most
    corner. We want to start with $(x_c,y_c)$ and end with
    $(\underline{x_c},\underline{y_c}) = \mu_c(x_c,y_c)$. Since $\R^2$
    embeds canonically into $\R^2\times\{1\}$, we move to the
    upper-middle position almost immediately. Now, via multiplication
    by $\lambda$, $(x_c,y_c,1)$ maps to $(x,y,z)$. From here we may use
    matrix multiplication
    \[
    \begin{bmatrix}
      \underline{x} & \underline{y} & \underline{z}
    \end{bmatrix}
    =
    \begin{bmatrix}
    x & y & z
    \end{bmatrix}
    \cdot\begin{bmatrix}
    m_{11} & m_{12} & m_{13}\\
    m_{21} & m_{22} & m_{23}\\
    m_{31} & m_{32} & m_{33}
    \end{bmatrix}.
    \]
    Expanded out, this is
    \[
    \begin{bmatrix}
      m_{11}x+m_{21}y+m_{31}z & m_{12}x+m_{22}y+m_{32}z & 
      m_{13}x+m_{23}y+m_{33}z
    \end{bmatrix}.
    \]
    We are now at the bottom left-hand corner of our diagram. To move
    to the lower position, we multiply by $\underline{\lambda}^{-1}$, which
    is equivalent to dividing by $\underline{z}$. Hence our element
    becomes
     \[
    \begin{bmatrix}
      \frac{m_{11}x+m_{21}y+m_{31}z}{m_{13}x+m_{23}y+m_{33}z} &
      \frac{m_{12}x+m_{22}y+m_{32}z}{m_{13}x+m_{23}y+m_{33}z} & 1
    \end{bmatrix}.
    \]
    To write this in terms of $x_c$ and $y_c$, we must divide each
    numerator and denominator by $z$
    \[
    \begin{bmatrix}
      \frac{m_{11}(x/z)+m_{21}(y/z)+m_{31}}{m_{13}(x/z)+m_{23}(y/z)+m_{33}}
      &
      \frac{m_{12}(x/z)+m_{22}(y/z)+m_{32}}{m_{13}(x/z)+m_{23}(y/z)+m_{33}}
      & 1
    \end{bmatrix}.
    \]
    Now since $x/z = x_c$ and $y/z =y_c$, we may pull this back to
    $\R^2$ in the lower left-hand corner and write
    \[
    \mu_c(x_c,y_c) = \left(
    \frac{m_{11}x_c+m_{21}y_c+m_{31}}{m_{13}x_c+m_{23}y_c+m_{33}},
    \frac{m_{12}x_c+m_{22}y_c+m_{32}}{m_{13}x_c+m_{23}y_c+m_{33}}
    \right).
    \]
  \end{freeResponse}
\end{problem}


\section{From $K$-rigid motions to rigid motions in central projection}

Now let's use our new tool to convert $K$-rigid motions to rigid motions in central projection. 



\begin{problem}
  Assuming $K > 0$, consider the $K$-rigid motion of the $R$-sphere
  \[
  M_\theta=
  \begin{bmatrix}
    \cos\theta & \sin\theta & 0\\
    -\sin\theta & \cos\theta & 0\\
    0 & 0 & 1
  \end{bmatrix}.
  \]
  Can you describe geometrically what this mapping is doing to the
  points in central projection?
\end{problem}


\begin{problem}
  Assuming $K > 0$, consider the $K$-rigid motion of the $R$-sphere
  \[
  M_\theta=
  \begin{bmatrix}
    \cos\theta & \sin\theta & 0\\
    -\sin\theta & \cos\theta & 0\\
    0 & 0 & 1
  \end{bmatrix}.
  \]
  Convert this to a rigid motion in central projection.
\end{problem}


\begin{problem}
  Assuming $K > 0$, consider the $K$-rigid motion of the $R$-sphere
  \[
  N_\psi=
  \begin{bmatrix}
    \cos\psi & 0 & R^{-1}\cdot\sin\psi\\
    0 & 1 & 0\\
    -R\cdot\sin\psi & 0 & \cos\psi
  \end{bmatrix}.
  \]
Can you describe geometrically what this mapping is doing to the
points in central projection?  
\end{problem}


\begin{problem}
  Assuming $K > 0$, consider the $K$-rigid motion of the $R$-sphere
  \[
  N_\psi=
  \begin{bmatrix}
    \cos\psi & 0 & R^{-1}\cdot\sin\psi\\
    0 & 1 & 0\\
    -R\cdot\sin\psi & 0 & \cos\psi
  \end{bmatrix}.
  \]
  Convert this to a rigid motion in central projection.
\end{problem}


\begin{problem}
  Assuming $K < 0$, consider the $K$-rigid motion of the $K$-surface
  \[
  N_\psi=
  \begin{bmatrix}
    \cosh\psi & 0 & |K|^{1/2}\cdot\sinh\psi\\
    0 & 1 & 0\\
    |K|^{-1/2}\cdot\sinh\psi & 0 & \cosh\psi
  \end{bmatrix}
  \]
  Can you describe geometrically what this mapping is doing to the
  points in central projection?
\end{problem}

  
\begin{problem}
  Assuming $K < 0$, consider the $K$-rigid motion of the $K$-surface
  \[
  N_\psi=
  \begin{bmatrix}
    \cosh\psi & 0 & |K|^{1/2}\cdot\sinh\psi\\
    0 & 1 & 0\\
    |K|^{-1/2}\cdot\sinh\psi & 0 & \cosh\psi
  \end{bmatrix}
  \]
  Convert this to a rigid motion in central projection.
\end{problem}


\begin{problem}
Summarize the results from this section. In particular, indicate which
results follow from the others.
\begin{freeResponse}
\end{freeResponse}
\end{problem}


\end{document}
