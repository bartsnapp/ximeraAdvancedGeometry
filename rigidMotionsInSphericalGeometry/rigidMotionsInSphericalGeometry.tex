\documentclass{ximera}

\usepackage{microtype}
\usepackage{tikz}
\usepackage{tkz-euclide}
\usetkzobj{all}
\tikzstyle geometryDiagrams=[ultra thick,color=blue!50!black]

\renewcommand{\epsilon}{\varepsilon}



\title{Rigid motions in spherical geometry}
\begin{document}
\begin{abstract}
Here we begin to extend the idea of congruence to spherical geometry.
\end{abstract}
\maketitle

\subsection*{Rigid motions in euclidean coordinates}

If we have a curve $\hat{X}\left(  t\right)  =\left(  \hat{x}\left(  t\right)
,\hat{y}\left(  t\right)  ,\hat{z}\left(  t\right)  \right)  $ on the
$R$-sphere, we have seen that we measure its length $L$ by the formula%
\begin{equation}
L=%
%TCIMACRO{\dint \nolimits_{b}^{e}}%
%BeginExpansion
{\displaystyle\int\nolimits_{b}^{e}}
%EndExpansion
\l\left(  t\right)  \d t \label{54}%
\end{equation}
where
\begin{equation}
\l\left(  t\right)  ^{2}=\frac{d\hat{X}}{dt}\bullet\frac{d\hat{X}}{dt}%
=\frac{dX}{dt}\bullet_{K}\frac{dX}{dt} \label{53}%
\end{equation}
and that we measure angles $\theta$ between tangent vectors $\hat{V}_{1}$ and
$\hat{V}_{2}$ at a point on the $R$-sphere by the formula%
\[
\theta=\arccos \left(  \frac{\hat{V}_{1}\bullet\hat{V}_{2}}{\left\vert
\hat{V}_{1}\right\vert \cdot \left\vert \hat{V}%
_{2}\right\vert }\right)  =\arccos \left(  \frac{V_{1}\bullet_{K}V_{2}%
}{\left\vert V_{1}\right\vert _{K}\cdot \left\vert
V_{2}\right\vert _{K}}\right)
\]
where%
\[
\left\vert V\right\vert _{K}^{2}=V\bullet_{K}V.
\]


Let $\hat{M}$ denote an invertible $3\times3$ matrix. We begin by noting the
condition that a transformation%
\[
\left(  \underline{\hat{x}},\underline{\hat{y}},\underline{\hat{z}}\right)
=\left(  \hat{x},\hat{y},\hat{z}\right)  \cdot\hat{M}%
\]
preserve the length of any curve lying on the $R$-sphere. Rewriting the
transformation as%
\begin{equation}
\underline{\hat{X}}=\hat{X}\cdot\hat{M} \label{trans1}%
\end{equation}
the formulas $\left(  \ref{54}\right)  $ and $\left(  \ref{53}\right)  $ show
that all we have to worry about is that%
\[
\frac{d\underline{\hat{X}}}{dt}\bullet\frac{d\underline{\hat{X}}}{dt}%
=\frac{d\hat{X}}{dt}\bullet\frac{d\hat{X}}{dt}%
\]
for all values $t$ of the parameter of the curve. But%
\begin{equation}
\frac{d\underline{\hat{X}}}{dt}=\frac{d\hat{X}}{dt}\cdot\hat{M} \label{trans2}%
\end{equation}
by the product rule since $\hat{M}$ is a constant matrix. So the
transformation given by the matrix $\hat{M}$ will preserve the length of any
path and will preserve the measure of any angle if%
\begin{equation}
\frac{d\hat{X}}{dt}\cdot\hat{M}\cdot\left(  \frac{d\hat{X}}{dt}\cdot\hat
{M}\right)  ^{t}=\frac{d\hat{X}}{dt}\cdot\left(  \frac{d\hat{X}}{dt}\right)
^{t}. \label{trans3}%
\end{equation}


\begin{exercise}\hfil
\label{trans} 
\begin{enumerate}
\item Show that the transformation $\left(
\ref{trans1}\right)  $ takes the $R$-sphere to itself if%
\[
\hat{M}\cdot\hat{M}^{t}=I
\]
where $I$ is the $3\times3$ identity matrix. (A matrix $\hat{M}$ satisfying
this condition is called an orthogonal matrix.)

\item Show that $\left(  \ref{trans3}\right)  $ also holds if $\hat{M}$ is orthogonal.
\end{enumerate}
\end{exercise}

\begin{exercise}
 Show that the matrix%
\[
\left(
\begin{array}
[c]{ccc}%
\cos \theta & \sin \theta & 0\\
-\sin \theta & \cos \theta & 0\\
0 & 0 & 1
\end{array}
\right)
\]
is orthogonal. Describe geometrically what this transformation is doing to the
$R$-sphere. [DS,316ff]
\end{exercise}

\begin{exercise}
 Show that the matrix%
\[
\left(
\begin{array}
[c]{ccc}%
\cos \varphi & 0 & \sin \varphi\\
0 & 1 & 0\\
-\sin \varphi & 0 & \cos \varphi
\end{array}
\right)
\]
is orthogonal. Describe geometrically what this transformation is doing to the
$R$-sphere.
\end{exercise}

\begin{definition}
A distance-preserving transformation of a space or geometry is called a
\textbf{rigid motion} or a \textbf{congruence}.
\end{definition}

So, by Exercise \ref{trans}, every orthogonal matrix $\hat{M}$ corresponds to
a rigid motion of the $R$-sphere. (It can be shown that every rigid motion of
the $R$-sphere is given by an orthogonal matrix, but we will not treat that
subtlety in this book.)

\begin{exercise}
 Show that the set of orthogonal matrices $\hat{M}$ form a group.
That is, show that
\begin{enumerate}
\item the product of two orthogonal matrices is orthogonal,

\item the identity matrix is orthogonal,

\item the inverse matrix $\hat{M}^{-1}$ of an orthogonal matrix $\hat{M}$ is orthogonal.

Hint: Write%
\[
\hat{M}\cdot\hat{M}^{-1}=I=\hat{M}\cdot\hat{M}^{t}%
\]
and use matrix multiplication to reduce to showing that the transpose of an
orthogonal matrix is orthogonal.) [MJG,311]

\end{enumerate}
\end{exercise}

Our conclusion from this exercise is that the set of rigid motions of the
euclidean $R$-sphere form a group.

\subsection*{Orthogonal and $K$-orthogonal matrices}

Recalling again the fact that the euclidean $R$-sphere is a $K$-geometry with
$K=1/R^{2}$ we should compare the $K$-orthogonal transformations $M$ in
Definition \ref{88} with the orthogonal ones just above.

\begin{exercise}
 Referring to Definition \ref{88}, show that the transformations%
\[
\left(  \underline{\hat{x}},\underline{\hat{y}},\underline{\hat{z}}\right)
=\left(  \hat{x},\hat{y},\hat{z}\right)  \cdot\hat{M}%
\]
and%
\[
\left(  \underline{x},\underline{y},\underline{z}\right)  =\left(
x,y,z\right)  \cdot M
\]
give the same rigid motion of the $R$-sphere if
\[
M=\left(
\begin{array}
[c]{ccc}%
1 & 0 & 0\\
0 & 1 & 0\\
0 & 0 & R
\end{array}
\right)  \cdot\hat{M}\cdot\left(
\begin{array}
[c]{ccc}%
1 & 0 & 0\\
0 & 1 & 0\\
0 & 0 & R^{-1}%
\end{array}
\right)  .
\]

\end{exercise}

\begin{exercise}
 Show that the matrix%
\[
\left(
\begin{array}
[c]{ccc}%
\cos \theta & \sin \theta & 0\\
-\sin \theta & \cos \theta & 0\\
0 & 0 & 1
\end{array}
\right)
\]
is both orthogonal and $K$-orthogonal and gives the same transformation of the
euclidean $R$-sphere. Describe geometrically what this transformation is doing
to the $R$-sphere.
\end{exercise}

\begin{exercise}
 Show that the matrix%
\[
\hat{M}=\left(
\begin{array}
[c]{ccc}%
\cos \varphi & 0 & \sin \varphi\\
0 & 1 & 0\\
-\sin \varphi & 0 & \cos \varphi
\end{array}
\right)
\]
is orthogonal and that the matrix%
\[
M=\left(
\begin{array}
[c]{ccc}%
\cos \varphi & 0 & R^{-1}\cdot \sin %
\varphi\\
0 & 1 & 0\\
-R\cdot \sin \varphi & 0 & \cos \varphi
\end{array}
\right)
\]
is the $K$-orthogonal matrix describing the same transformation of the
$R$-sphere. Describe geometrically what this transformation is doing to the
$R$-sphere.
\end{exercise}


\end{document}
