\documentclass{ximera}

\usepackage{microtype}
\usepackage{tikz}
\usepackage{tkz-euclide}
\usetkzobj{all}
\tikzstyle geometryDiagrams=[ultra thick,color=blue!50!black]

\renewcommand{\epsilon}{\varepsilon}



\title{Rigid motions in $(x,y,z)$-coordinates}
\begin{document}
\begin{abstract}
  Here we dig deeper to understand our different coordinates.
\end{abstract}
\maketitle

We now wish to figure out how to convert a transformation
\[
\begin{bmatrix}

\underline{\hat{x}} & \underline{\hat{y}} & \underline{\hat{z}}%
\end{bmatrix}
  =\begin{bmatrix}
\hat{x} & \hat{y} & \hat{z}%
\end{bmatrix}
          \cdot\hat{M} \label{100}%
\]
from euclidean space to $(x,y,z)$-coordinates. Recall to convert a
point from $(x,y,z)$-coordinates to euclidean coordinates, we write
\[
\begin{bmatrix}
\hat{x} & \hat{y} & \hat{z}%
\end{bmatrix}
=\begin{bmatrix}
x & y & Rz
\end{bmatrix}
=\begin{bmatrix}
x & y & z
\end{bmatrix}
\cdot\begin{bmatrix}
1 & 0 & 0\\
0 & 1 & 0\\
0 & 0 & R
\end{bmatrix}.
\]
So we have the diagram%
\begin{image}
\begin{tikzpicture}
  \node (xyzHat) {$\begin{bmatrix}\hat{x} & \hat{y} & \hat{z}\end{bmatrix}\in\mathbb{R}^3$};
  \node (xyzHatBar) [node distance=2cm,below of=xyzHat] {$\begin{bmatrix}\underline{\hat{x}} & \underline{\hat{y}} & \underline{\hat{z}}\end{bmatrix}$};
  \node (xyz) [node distance=6cm,right of=xyzHat] {$\begin{bmatrix}x & y & z\end{bmatrix}\in\mathbb{R}^3$};
  \node (xyzBar) [node distance=6cm,right of=xyzHatBar] {$\begin{bmatrix}\underline{x} & \underline{y} & \underline{z}\end{bmatrix}$};
  \draw[->] (xyz) to node[above] {$\cdot\left[\begin{smallmatrix}1 & 0 & 0\\ 0 & 1 & 0\\ 0 & 0 & R\end{smallmatrix}\right]$} (xyzHat);
  \draw[->] (xyz) to node[right] {$\cdot M = \text{?}$} (xyzBar);
  \draw[->] (xyzHat) to node[right] {$\cdot\hat{M}$} (xyzHatBar);
  \draw[->] (xyzBar) to node [above] {$\cdot\left[\begin{smallmatrix}1 & 0 & 0\\ 0 & 1 & 0\\ 0 & 0 & R\end{smallmatrix}\right]$} (xyzHatBar);
\end{tikzpicture}
\end{image}
\begin{problem}
Using the diagram above, explain why%
\[
\begin{bmatrix}
\underline{x} & \underline{y} & \underline{z}%
\end{bmatrix}
=\begin{bmatrix}
x & y & z
\end{bmatrix}
  \cdot\begin{bmatrix}
%
1 & 0 & 0\\
0 & 1 & 0\\
0 & 0 & R
\end{bmatrix}
  \cdot\hat{M}\cdot\begin{bmatrix}
%
1 & 0 & 0\\
0 & 1 & 0\\
0 & 0 & R^{-1}%
\end{bmatrix}.
\]

\end{problem}

%% So, if we let%
%% \[
%% M=\begin{bmatrix}
%% %
%% 1 & 0 & 0\\
%% 0 & 1 & 0\\
%% 0 & 0 & R
%% \end{bmatrix}
%%   \cdot\hat{M}\cdot\begin{bmatrix}
%% %
%% 1 & 0 & 0\\
%% 0 & 1 & 0\\
%% 0 & 0 & R^{-1}%
%% \end{bmatrix}
%%   ,
%% \]
%% then%
%% \begin{equation}
%% \begin{bmatrix}
%% %
%% \underline{x} & \underline{y} & \underline{z}%
%% \end{bmatrix}
%%   =\begin{bmatrix}
%% %
%% x & y & z
%% \end{bmatrix}
%%   \cdot M, \label{15}%
%% \end{equation}
%% that is $M$ is the matrix that gives the transformation $\left(
%% \ref{12}\right)  $ in $\left(  x,y,z\right)  $-coordinates.

%% So how would we check whether a transformation given in $\left(  x,y,z\right)
%% $-coordinates by a matrix $M$ preserves distances in euclidean space?
%% Again, starting from $\left(  \ref{16}\right)  $ this is just a substitution
%% problem:%
%% \begin{gather*}
%% \hat{M}\cdot\hat{M}^\transpose=I\\
%% M=\begin{bmatrix}
%% %
%% 1 & 0 & 0\\
%% 0 & 1 & 0\\
%% 0 & 0 & R
%% \end{bmatrix}
%%   \cdot\hat{M}\cdot\begin{bmatrix}
%% %
%% 1 & 0 & 0\\
%% 0 & 1 & 0\\
%% 0 & 0 & R^{-1}%
%% \end{bmatrix}
%%  \\
%% \begin{bmatrix}
%% %
%% 1 & 0 & 0\\
%% 0 & 1 & 0\\
%% 0 & 0 & R^{-1}%
%% \end{bmatrix}
%%   \cdot M\cdot\begin{bmatrix}
%% %
%% 1 & 0 & 0\\
%% 0 & 1 & 0\\
%% 0 & 0 & R
%% \end{bmatrix}
%%   =\hat{M}%
%% \end{gather*}


\begin{problem}
Show that the condition that a transformation $M$ in
$\left(x,y,z\right)$-coordinates preserves distances in euclidean
space is the condition that%
\[
M\cdot\begin{bmatrix}
1 & 0 & 0\\
0 & 1 & 0\\
0 & 0 & K^{-1}
\end{bmatrix}
  \cdot M^\transpose=\begin{bmatrix}
1 & 0 & 0\\
0 & 1 & 0\\
0 & 0 & K^{-1}
\end{bmatrix}. 
\]

\end{problem}

This is the condition (in $\left(  x,y,z\right)  $-coordinates) which affirms
that the transformation which takes the path $\left(  x(t),y(t),z(t)\right)  $
to the path $\left(  x(t),y(t),z(t)\right)  \cdot M$ preserves lengths of
tangent vectors at corresponding points. Therefore, by integrating, the
(total) length of the curve $\left\{  \left(  x(t),y(t),z(t)\right)  \cdot
M:b\leq t\leq e\right\}  $ is the same as the total length of the curve
$\left\{  \left(  x(t),y(t),z(t)\right)  :b\leq t\leq e\right\}  $.

\begin{problem}
Verify that your condition above is the correct condition by showing
that any $3\times3$ matrix $M$ that satisfying your condition also satisfies%
\[
\left(   V  \cdot M\right)  \bullet_{K}\left(   V
\cdot M\right)  =V\bullet_{K}V
\]
where%
\[
V=X_{2}-X_{1}.
\]
That is, the transformation given in $\left( x,y,z\right)
$-coordinates by a matrix $M$ that satisfies your condition preserves
the $K$-dot product.
\end{problem}





\begin{definition}
A $K$-distance-preserving transformation of $K$-geometry is called a
$K$\textbf{-rigid motion} or a $K$\textbf{-congruence}.
\end{definition}

With this definition, and our work above, we make a new definition:


\begin{definition}
A $3\times3$ matrix $M$ is called \dfn{$\boldsymbol{K}$-orthogonal} if
\[
M\cdot\begin{bmatrix}
1 & 0 & 0\\
0 & 1 & 0\\
0 & 0 & K^{-1}%
\end{bmatrix}  \cdot M^\transpose=\begin{bmatrix}
1 & 0 & 0\\
0 & 1 & 0\\
0 & 0 & K^{-1}%
\end{bmatrix}.
\]
\end{definition}



\begin{problem}
  For $K\ne 0$ show that, if $M$ is $K$-orthagonal, then the transformation
  \[
  \begin{bmatrix}
    \underline{x} & \underline{y} & \underline{z}
  \end{bmatrix}
  =
  \begin{bmatrix}
    x & y & z
  \end{bmatrix}
  \cdot M.
  \]
  takes the set of points $\left(x,y,z\right)$ such
  that
\[
1 = K\left(x^2 + y^2\right) +z^2
\]
to the set of points
$\left(\underline{x},\underline{y},\underline{z}\right)$
such that
\[
1=K\left(\underline{x}^2 + \underline{y}^{2}\right) + \underline{z}^{2}.
\]
That is, $M$ gives a one-to-one and onto mapping of $K$-geometry to
itself.
\begin{hint}
  Write the equation
  \[
  1=K\left(\underline{x}^2 + \underline{y}^{2}\right) + \underline{z}^{2}.
  \]
  as
\[
\begin{bmatrix}
\underline{x} & \underline{y} & \underline{z}%
\end{bmatrix}  \cdot\begin{bmatrix}
1 & 0 & 0\\
0 & 1 & 0\\
0 & 0 & K^{-1}%
\end{bmatrix}  \cdot
\begin{bmatrix}
\underline{x}\\
\underline{y}\\
\underline{z}%
\end{bmatrix}  =\frac{1}{K}.
\]
\end{hint}
\end{problem}




\begin{problem}
For $K\neq0$, show that the set of $K$-orthogonal matrices $M$ form a
group.  That is, show that
\begin{enumerate}
\item multiplication of $K$-orthogonal matrices is associative, 
\item the product of two $K$-orthogonal matrices is $K$-orthogonal,
\item the identity matrix is $K$-orthogonal,
\item the inverse matrix $M^{-1}$ of a $K$-orthogonal matrix $M$ is $K$-orthogonal.
\end{enumerate}
\end{problem}

\begin{problem}
  Convert the orthogonal matrix
  \[
  \hat{M}=\begin{bmatrix}
  \cos\theta & \sin\theta & 0\\
  -\sin\theta & \cos\theta & 0\\
  0 & 0 & 1
  \end{bmatrix}
  \]
  into a $K$-orthogonal matrix. Are you surprised? Why or why not.
\end{problem}

\begin{problem}
  Convert the orthogonal matrix
  \[
  \hat{M}=\begin{bmatrix}
  \cos\psi & 0 & \sin\psi\\
  0 & 1 & 0\\
  -\sin\psi & 0 & \cos\psi
  \end{bmatrix}
  \]
  into a $K$-orthogonal matrix. Are you surprised? Why or why not.
\end{problem}

















\subsection*{Why use $K$-coordinates?}

We have seen that we could measure the usual euclidean lengths of
curves $\hat{X}(t)$ in terms of the formulas $X(t)$ in $K$-warped
space using the $K$-dot product. The short reason for this is that
\[
\frac{d\hat{X}(t)}{dt}\bullet\frac{d\hat{X}(t)}{dt}=\frac{dX(t)}{dt}\bullet_{K}\frac{dX(t)}{dt}%
\]
where%
\[
\frac{dX(t)}{dt}\bullet_{K}\frac{dX(t)}%
{dt}=\left(\frac{dX(t)}{dt}\right)\cdot\begin{bmatrix}
1 & 0 & 0\\
0 & 1 & 0\\
0 & 0 & K^{-1}%
\end{bmatrix}  \cdot\left(  \frac{dX(  t)  }{dt}\right)  ^\transpose.
\]
In other words, the usual geometry of the sphere of radius $R$ is
simply the geometry of the set
\[
\{(x,y,z)\in\R^3:1 = K\left(x^2 + y^2\right) +z^2\}
\]
with $K=1/R^{2}$ and with lengths (and areas) given by the $K$-dot
product. Said another way, we can do all of spherical geometry in
$(x,y,z)$-coordinates. All we need is the set defined by the relation
\[
1 = K\left(x^2 + y^2\right) +z^2
\]
and the $K$-dot product. But the set definied by the equation above
continues to exist even if $K=0$ or $K<0$, and the $K$-dot product
formula continues to make sense even if $K<0$. In short we have the
following table:

\[
{\renewcommand{\arraystretch}{2.7}
  \begin{tabular}{|c||c|c|c|}\hline
    & \begin{minipage}{2cm}\begin{center}Spherical ($K>0$)\end{center}\end{minipage} & \begin{minipage}{2cm}\begin{center}Euclidean ($K=0$)\end{center}\end{minipage} & \begin{minipage}{2cm}\begin{center}Hyperbolic ($K<0$)\end{center}\end{minipage}\\\hline\hline
    \begin{minipage}{2cm}\begin{center}Surface in \\ euclidean space\end{center}\end{minipage} & $\hat{x}^{2}+\hat{y}^{2}+\hat{z}^{2}=R^{2}$ & DNE  & DNE \\\hline
    \begin{minipage}{2cm}\begin{center}Euclidean dot product\end{center}\end{minipage} & $\hat{V}\cdot \hat{V}^\transpose$ & DNE  & DNE\\\hline
     \begin{minipage}{2cm}\begin{center}Surface in $K$-warped space\end{center}\end{minipage} & \multicolumn{3}{c|}{$1=K\left(  x^{2}+y^{2}\right)  +z^{2}$}\\\hline
    \begin{minipage}{2cm}\begin{center}$K$-dot product\end{center}\end{minipage} & $V_{1}\left[\begin{smallmatrix}1 & 0 & 0\\ 0 & 1 & 0\\ 0 & 0 & K^{-1}\end{smallmatrix}\right] V_{2}^\transpose$ &  DNE & $V_{1}\left[\begin{smallmatrix}1 & 0 & 0\\ 0 & 1 & 0\\ 0 & 0 & K^{-1}\end{smallmatrix}\right]V_{2}^\transpose$\\\hline
\end{tabular}}
\]



This table tells us that `there is something else out there,' that is,
some other type of two-dimensional geometry beyond plane geometry and
spherical geometry. But the gap in the bottom row of the table is a
bit disturbing. If we can't express the usual dot-product in plane
geometry as the $K$-dot product for $K=0$, we can't pass smoothly from
spherical through plane geometry to hyperbolic geometry using
$(x,y,z)$-coordinates. We now examine two ways to produce coordinates
uniformly for spherical, plane and hyperbolic geometry that overcome
this difficulty.



\begin{problem}
Summarize the results from this section. In particular, indicate which
results follow from the others.
\begin{freeResponse}
\end{freeResponse}
\end{problem}


\end{document}
