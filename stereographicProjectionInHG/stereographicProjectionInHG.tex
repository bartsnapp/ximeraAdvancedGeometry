\documentclass{ximera}

\title{Stereographic projection in HG}

\begin{document}
\begin{abstract}
Now we explore the Poincar\'e disk.
\end{abstract}
\maketitle

\subsection*{The Poincar\'{e} $K$-disk}

Under stereographic projection, the center of projection is the South Pole
$\left(  0,0,-1\right)  $. So if $K<0$ a point $\left(  x,0,z\right)  $ on the
$K$-geometry goes out the hyperboloid to infinity in the $\left(  x,z\right)
$-plane, the line joining $\left(  0,0,-1\right)  $ to that point becomes
parallel to an asymptote of the hyperbola%
\[
Kx^{2}+z^{2}=1.
\]
So the line approaches a line of slope $\pm\left\vert K\right\vert ^{1/2}$ in
the $\left(  x,z\right)  $-plane, that is the line $z=$ $\pm\left\vert
K\right\vert ^{1/2}x-1$. So the intersection of that line with the line $z=1$
in the $\left(  x,z\right)  $-plane approaches the point with $x=\pm
2\left\vert K\right\vert ^{-1/2}$. Therefore under stereographic projection,
the edge of the universe is given by the circle%
\[
x_{s}^{2}+y_{s}^{2}=4\left\vert K\right\vert ^{-1}.
\]
The interior of this circle, that is, the image of $K$-geometry under
stereographic projection, is called the Poincar\'{e} model of Hyperbolic
Geometry, of course again after a famous geometer, Henr\'{\i} Poincar\'{e}.
Again, since \textbf{HG} is a $K$-geometry, all the rules of Part \ref{III}
apply. So by Exercise \ref{73}b) line in the $K$-geometry are given by circles
of the form%
\[
\left(  x_{s}-\frac{2a}{K}\right)  ^{2}+\left(  y_{s}-\frac{2b}{K}\right)
^{2}=\frac{4\left(  K+a^{2}+b^{2}\right)  }{K^{2}}%
\]
in the Poincar\'{e} $K$-disk. The darker arc below%
\begin{equation}%
%TCIMACRO{\FRAME{itbpF}{1.8836in}{1.4961in}{0in}{}{}{Figure}%
%{\special{ language "Scientific Word";  type "GRAPHIC";
%maintain-aspect-ratio TRUE;  display "USEDEF";  valid_file "T";
%width 1.8836in;  height 1.4961in;  depth 0in;  original-width 11.4726in;
%original-height 9.0987in;  cropleft "0";  croptop "1";  cropright "1";
%cropbottom "0";  tempfilename 'MXAJC00Q.jpg';tempfile-properties "XPR";}}}%
%BeginExpansion
{\includegraphics[
natheight=9.098700in,
natwidth=11.472600in,
height=1.4961in,
width=1.8836in
]%
{MXAJC00Q.jpg}%
}%
%EndExpansion
\label{92}%
\end{equation}
represents the $K$-line%
\[
ax+by+1=0
\]
in the Poincar\'{e} $K$-disk. 

\subsection*{Stereographic projection preserves angles}

\begin{exercise}\hfil
\begin{enumerate}
\item Show that stereographic projection is conformal, that is, that
  the measure of $K$-angles between $K$-lines on $K$-geometry is just
  the ordinary Euclidean measure of angles formed.by (the circles that
  are) their stereographic projections.

Hint: From Exercise \ref{36} we know that, for tangent vectors $V_{1}$ and
$V_{2}$ emanating from the same point on the $K$-geometry,
\begin{align*}
V_{1}\bullet_{K}V_{2}  &  =V_{1}^{s}\bullet_{s}V_{2}^{s}\\
&  =\left(  V_{1}^{s}\right)  \cdot\left(
\begin{array}
[c]{cc}%
\rho^{2} & 0\\
0 & \rho^{2}%
\end{array}
\right)  \cdot\left(  V_{2}^{s}\right)  ^{t}.
\end{align*}


\item For $K=-1$, construct the $K$-line in $\left( x_{s},y_{s}\right)
  $-coordinates that meets the $K$-line
\[
\left(  x_{s}-2\right)  ^{2}+\left(  y_{s}-2\right)  ^{2}=4
\]
perpendicularly in the point $\left(  2-\sqrt{2},2-\sqrt{2}\right)  $.
\end{enumerate}
\end{exercise}

To get a more precise idea of what $K$-lines look like under stereographic
projection, consider the picture $\left(  \ref{92}\right)  $ again. The
equations of the circles in the picture are%
\begin{equation}
x_{s}^{2}+y_{s}^{2}=\frac{4}{\left\vert K\right\vert } \label{93}%
\end{equation}
and%
\begin{equation}
\left(  x_{s}-\frac{2a}{K}\right)  ^{2}+\left(  y_{s}-\frac{2b}{K}\right)
^{2}=\frac{4\left(  K+a^{2}+b^{2}\right)  }{K^{2}} \label{94}%
\end{equation}
Construct a third circle whose diameter is the line segment from $\left(
0,0\right)  $ to $\left(  \frac{2a}{K},\frac{2b}{K}\right)  $, namely the
circle $\left(  x_{s}-\frac{a}{K}\right)  ^{2}+\left(  y_{s}-\frac{b}%
{K}\right)  ^{2}=\left(  \frac{a}{K}\right)  ^{2}+\left(  \frac{b}{K}\right)
^{2}$ which can be rewritten%
\begin{equation}
x_{s}^{2}+y_{s}^{2}-\frac{2a}{K}x_{s}-\frac{2b}{K}y_{s}=0. \label{95}%
\end{equation}


\begin{lemma}
The circles $\left(  \ref{93}\right)  $, $\left(  \ref{94}\right)  $, and
$\left(  \ref{95}\right)  $ all three pass through two common points.
\end{lemma}

\begin{proof}
From $\left(  \ref{93}\right)  $ and $\left(  \ref{94}\right)  $ we get by
addition that%
\[
x_{s}^{2}+y_{s}^{2}+\left(  x_{s}-\frac{2a}{K}\right)  ^{2}+\left(
y_{s}-\frac{2b}{K}\right)  ^{2}=\frac{4}{\left\vert K\right\vert }%
+\frac{4\left(  K+a^{2}+b^{2}\right)  }{K^{2}}.
\]
Simplifying this last equation and dividing both sides by $2$ we obtain the
equation $\left(  \ref{95}\right)  $. So the two points $P^{\prime}$ and
$Q^{\prime}$ in picture $\left(  \ref{92}\right)  $ that satisfy both
equations $\left(  \ref{93}\right)  $ and $\left(  \ref{94}\right)  $ also
satisfy equation $\left(  \ref{95}\right)  $.
\end{proof}

The Lemma tells us that that the angle formed by the segments $\overline
{\left(  0,0\right)  P^{\prime}}$ and $\overline{P^{\prime}\left(  \frac
{2a}{K},\frac{2b}{K}\right)  }$ is a right angle since it is an inscribed
angle in the circle $\left(  \ref{95}\right)  $ whose associated central angle
is a diameter of that circle. But $\overline{\left(  0,0\right)  P^{\prime}}$
is a radius of circle $\left(  \ref{93}\right)  $ and so $\overline{P^{\prime
}\left(  \frac{2a}{K},\frac{2b}{K}\right)  }$ is tangent to circle $\left(
\ref{93}\right)  $. Similarly $\overline{P^{\prime}\left(  \frac{2a}{K}%
,\frac{2b}{K}\right)  }$ is a radius of circle $\left(  \ref{94}\right)  $ and
so $\overline{\left(  0,0\right)  P^{\prime}}$ is tangent to circle $\left(
\ref{94}\right)  $. So we conclude the following Theorem.

\begin{theorem}
 In the Poincar\'{e} model for $K$-geometry, the $K$-lines are
represesented by circular arcs that meet the edge of the universe
perpendicularly.
\end{theorem}

\subsection*{Infinite triangles in the Poincar\'{e} $K$-disk}

By Exercise \ref{73}b) and \ref{73}c), lines in \textbf{HG} become circles
under stereographic projection unless the line in \textbf{HG} passes through
the North Pole (in which case it corresponds to a line through $\left(
x_{s},y_{s}\right)  =\left(  0,0\right)  $ in the $\left(  x_{s},y_{s}\right)
$-plane). Suppose a hyperbolic triangle $T$ corresponds to a region $T_{s}$ in
$\left(  x_{s},y_{s}\right)  $-coordinates and the vertices of $T$ correspond
to $\left(  x_{s},y_{s}\right)  =\left(  -2,0\right)  $, $\left(  x_{s}%
,y_{s}\right)  =\left(  2,0\right)  $, and $\left(  x_{s},y_{s}\right)
=\left(  0,2\right)  $. So one side of $T_{s}$ lies on the line $y_{s}=0$.

\begin{exercise}\hfil
\begin{enumerate}
\item Use the fact that the equation for lines in stereographic
  projection of $K$-geometry are given by
\[
\left(  x_{s}-\frac{2a}{K}\right)  ^{2}+\left(  y_{s}-\frac{2b}{K}\right)
^{2}=\frac{4\left(  K+a^{2}+b^{2}\right)  }{K^{2}}.
\]
to compute the equations for the other two sides of $T_{s}$.

\item In the $\left( x_{s},y_{s}\right) $-plane, draw $T_{s}$ as
  accurately as you can when $K=-\frac{1}{4}$, then when $K=-1$.
\end{enumerate}
\end{exercise}

The area of a hyperbolic triangle $T$ is given by the formula%
\[%
%TCIMACRO{\dint \nolimits_{T_{s}}}%
%BeginExpansion
{\displaystyle\int\nolimits_{T_{s}}}
%EndExpansion
\frac{1}{\left(  1+\frac{K}{4}\left(  x_{s}^{2}+y_{s}^{2}\right)  \right)
^{2}}dx_{s}dy_{s}.
\]
However we do not as yet have a way to calculate the area numerically for any
given triangle $T$. The last topic in this book will remedy that situation.
Analogously to the case of spherical triangles, we start from the fact that we
do know the area of $\alpha$-lunes. From Exercise \ref{97} the $K$-area of one
an $\alpha$-lune with vertex at $\left(  0,0\right)  $ in $\left(  x_{c}%
,y_{c}\right)  $-coordinates is%
\[
\left\vert K\right\vert ^{-1}\left(  \pi-\alpha\right)  .
\]
Since rotation of the $\left(  x_{c},y_{c}\right)  $-plane around $\left(
0,0\right)  $ is a $K$-rigid motion, this formula holds for any $K$-lune with
vertex at $\left(  0,0\right)  $. Now represent the \textit{same} lunes in the
$\left(  x_{s},y_{s}\right)  $-plane. Below is a picture in the $\left(
x_{s},y_{s}\right)  $-plane of some of these $K$-lunes.%
\begin{equation}%
%TCIMACRO{\FRAME{itbpF}{1.6137in}{1.4978in}{0in}{}{}{Figure}%
%{\special{ language "Scientific Word";  type "GRAPHIC";  display "USEDEF";
%valid_file "T";  width 1.6137in;  height 1.4978in;  depth 0in;
%original-width 8.0877in;  original-height 8.0877in;  cropleft "0";
%croptop "1";  cropright "1";  cropbottom "0";
%tempfilename 'MXAJC00R.jpg';tempfile-properties "XPR";}}}%
%BeginExpansion
{\includegraphics[
natheight=8.087700in,
natwidth=8.087700in,
height=1.4978in,
width=1.6137in
]%
{MXAJC00R.jpg}%
}%
%EndExpansion
\label{98}%
\end{equation}


\begin{exercise}
 Use Exercise \ref{97} to show that the area in the picture
$\left(  \ref{98}\right)  $ that lies in the union of the $\alpha$-lune and
the $\beta$-lune but does not lie in the $\left(  \alpha+\beta\right)  $- lune
has $K$-area $\left\vert K\right\vert ^{-1}\pi$.
\end{exercise}

\begin{definition}
 An infinite $K$-triangle is the figure given in stereographic
projection coordinates by the stereographic projection of three $K$-lines such
that any two meet the edge of the universe in a common point.
\end{definition}

\begin{exercise}
a)  Use Exercise \ref{99} to show that the area of (the interior
of) any infinte triangle has $K$-area%
\[
\left\vert K\right\vert ^{-1}\cdot\pi.
\]
For example, if $K=-1$ we have%
\[%
%TCIMACRO{\FRAME{itbpF}{1.6769in}{1.6769in}{0in}{}{}{Figure}%
%{\special{ language "Scientific Word";  type "GRAPHIC";
%maintain-aspect-ratio TRUE;  display "USEDEF";  valid_file "T";
%width 1.6769in;  height 1.6769in;  depth 0in;  original-width 10.3734in;
%original-height 10.3734in;  cropleft "0";  croptop "1";  cropright "1";
%cropbottom "0";  tempfilename 'MXAJC00S.jpg';tempfile-properties "XPR";}}}%
%BeginExpansion
{\includegraphics[
natheight=10.373400in,
natwidth=10.373400in,
height=1.6769in,
width=1.6769in
]%
{MXAJC00S.jpg}%
}%
%EndExpansion
\]


b) Use a) to give a formula for the $K$-area of any infinite $n$-gon in
\textbf{HG}, that is, a figure described by a set of $n$ disjoint $K$-lines
that is the limit of a family of finite $n$-gons, all of whose vertices have
gone to infinity. In particular, what is the area of any infinite hexagon?

Hint: Divide the infinite $n$-gon into infinite triangles.
\end{exercise}

\subsection*{Areas of polygons in \textbf{HG}}

Consider the picture below in the Poincar\'{e} model for \textbf{HG}. Find six
lunes that cover an infinte hexagon. Notice that the six lunes cover the
shaded hyperbolic triangle two extra times.%

\[%
%TCIMACRO{\FRAME{itbpF}{2.9966in}{1.6734in}{0in}{}{}{Figure}%
%{\special{ language "Scientific Word";  type "GRAPHIC";
%maintain-aspect-ratio TRUE;  display "USEDEF";  valid_file "T";
%width 2.9966in;  height 1.6734in;  depth 0in;  original-width 16.2637in;
%original-height 9.0546in;  cropleft "0";  croptop "1";  cropright "1";
%cropbottom "0";  tempfilename 'MXAJC00T.jpg';tempfile-properties "XPR";}}}%
%BeginExpansion
{\includegraphics[
natheight=9.054600in,
natwidth=16.263700in,
height=1.6734in,
width=2.9966in
]%
{MXAJC00T.jpg}%
}%
%EndExpansion
\]


\begin{exercise}\hfil
\begin{enumerate}
\item Use the picture and remarks just above to explain why the
$K$-area of the hyperbolic triangle is%
\[
\left\vert K\right\vert ^{-1}\cdot\left(  \pi-\left(
\alpha+\beta+\delta\right)  \right)  .
\]
\item Use a) to give a formula for the $K$-area of a hyperbolic $n$-gon.
\end{enumerate}
\end{exercise}






\end{document}
