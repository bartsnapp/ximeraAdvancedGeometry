\documentclass{ximera}

\title{Stereographic projection in SG}

\begin{document}
\begin{abstract}
Here we see that stereographic projection preserves angles and we
explore and what this implies. 
\end{abstract}
\maketitle


\subsection*{Stereographic projection preserves angles}

We now turn to a simple way to make a map of the $R$-sphere in such a
way that the measure of any angle on the map is exactly the same as
the measure of the corresponding angle on the $R$-sphere. The map
coordinates that do the job are our old friends (or enemies?) the
\textit{stereographic projection coordinates}.

\begin{exercise}\hfil
\begin{enumerate}
\item Compute the stereographic projection of a point on the
  $R$-sphere in $\left( \hat{x},\hat{y},\hat{z}\right) $-space to the
  plane $\hat{z}=R$.

\item Show that the coordinates $\left( \hat{x}_{s},\hat{y}_{s}\right)
  $ of the stereographic projection of a point on the $R$-sphere in
  $\left( \hat{x},\hat{y},\hat{z}\right) $-space to the plane
  $\hat{z}=R$ are the same as the coordinates $\left(
  x_{s},y_{s}\right) $ of the stereographic projection of the
  corresponding point in $\left( x,y,z\right) $-coordinates to the
  plane $z=1$.

Hint: Reduce to showing that%
\[
R\left(  \frac{2\hat{x}}{\hat{z}+R},\frac{2\hat{y}}{\hat{z}+R}\right)
=\left(  \frac{2x}{z+1},\frac{2y}{z+1}\right)  .
\]
\end{enumerate}
\end{exercise}

\begin{exercise}\hfil
\begin{enumerate}
\item Show that stereographic projection is conformal, that is, that
  the angle between two paths through a point on the $R$-sphere in
  $\left( \hat{x},\hat{y},\hat{z}\right)$-space is the same as the
  usual (euclidean) angle between the corresponding two paths through
  the corresponding point in the $\left( x_{s},y_{s}\right) $-plane.

Hint: From a previous exercise, we know that for tangent vectors
$\hat{V}_{1}$ and $\hat{V}_{2}$ emanating from the same point on the
$R$-sphere
\begin{align*}
\hat{V}_{1}\bullet\hat{V}_{2}  &  =V_{1}\bullet_{K}V_{2}\\
&  =V_{1}^{s}\bullet_{s}V_{2}^{s}\\
&  =\left(  V_{1}^{s}\right)  \cdot\left(
\begin{array}
[c]{cc}%
\rho^{2} & 0\\
0 & \rho^{2}%
\end{array}
\right)  \cdot\left(  V_{2}^{s}\right)  ^{t}
\end{align*}
where $\rho =\frac{z+1}{2}$. 



\item Draw a picture of an angle between two paths through a point on
  the Euclidean $R$-sphere and the stereographic projection of that
  angle onto the plane $\hat{z}=R$. Try to give an intuitive geometric
  explanation for why it should have the same measure as the original
  angle.
\end{enumerate}
\end{exercise}

\subsection*{Areas of spherical triangles in stereographic projection
coordinates}

We know that lines in \textbf{SG} become circles under stereographic
projection unless the line in \textbf{SG} passes through the North
Pole (in which case it corresponds to a line through $\left(
x_{s},y_{s}\right) =\left( 0,0\right) $ in the $\left(
x_{s},y_{s}\right) $-plane). Specifically, we know that if we intersect 
$K$-geometry
\[
\left\{  \left(  x,y,z\right)  \in\mathbb{R}^{3}:1=K\left(  x^{2}%
+y^{2}\right)  +z^{2}\right\}
\]
with the plane
\[
ax+by+z = 0
\]
we find the circle
\[
\left(  x_{s}-\frac{2a}{K}\right)  ^{2}+\left(  y_{s}-\frac{2b}{K}\right)
^{2}=\frac{4\left(  K+a^{2}+b^{2}\right)  }{K^{2}}.
\]


Suppose a spherical triangle $T$ corresponds to a region $T_{s}$ in
$\left( x_{s},y_{s}\right) $-coordinates and the vertices of $T$
correspond to $\left( x_{s},y_{s}\right) =\left( -2,0\right) $,
$\left( x_{s} ,y_{s}\right) =\left( 2,0\right) $, and $\left(
x_{s},y_{s}\right) =\left( 0,2\right) $. So one side of $T_{s}$ lies
on the line $y_{s}=0$.

\begin{exercise}\label{st}\hfil
\begin{enumerate}
\item Compute the equations for the other two sides of $T_{s}$.
\item In the $\left( x_{s},y_{s}\right) $-plane, draw $T_{s}$ as
  accurately as you can when $K=4$, then then when $K=\frac{1}{4}$.
\item Compute the area of $T$ in both cases in b).
Hint: You will need the radian measure of the angle at each of the
vertices of $T_{s}$. Why? To calculate these angles, calculate
$\frac{dy_{s}}{dx_{s}}$ by implicit differentiation of the equations
in a), then take $\arctan\left( \frac{dy_{s}}{dx_{s}}\right)
$ in radians. Your job will be easier if you notice that the
$y_{s}$-axis divides $T_{s}$ into two congruent isosceles triangles.
\end{enumerate}
\end{exercise}

\begin{exercise}
Explain why we know from an exercise before that in all of
the cases in the previous exercise %\ref{st}
the area of the spherical triangle $T$
is also given by the formula%
\[%
\int_{T_{s}}
\frac{1}{\left(  1+\frac{K}{4}\left(  x_{s}^{2}+y_{s}^{2}\right)  \right)
^{2}}dx_{s}dy_{s}.
\]

\end{exercise}



\end{document}
