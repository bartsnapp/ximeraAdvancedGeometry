\documentclass{ximera}

\usepackage{microtype}
\usepackage{tikz}
\usepackage{tkz-euclide}
\usetkzobj{all}
\tikzstyle geometryDiagrams=[ultra thick,color=blue!50!black]

\renewcommand{\epsilon}{\varepsilon}



\title{Surface area and volume of the $R$-sphere in euclidean $3$-space}
\begin{document}
\begin{abstract}
In this activity we begin to explore spherical geometry.
\end{abstract}
\maketitle


We are now going to study the geometry of the $R$-sphere in euclidean
$3$-space. This is the sphere of radius $R$ in normal $3$-space. We
will use the coordinates $(\hat{x},\hat{y},\hat{z})$ to denote this
space. We will show why there is a factor of $1/3$ in many formulas
for volumes in $3$-dimensional euclidean geometry, just like there is
a factor of $1/2$ in many formulas for areas in $2$-dimensional
euclidean geometry.



\section{Circumference and area of circles}

We begin with perhaps the most basic fact of all about circles in
\textbf{EG}. We will use calculus to compute the perimeter of a unit
circle.


\begin{problem}
Consider a unit circle with $n$ inscribed triangles (with one vertex
at the center of the circle), here we have shown the $n$th triangle:
\begin{image}
\begin{tikzpicture}[geometryDiagrams]
\draw [domain=-20:65] plot ({4*cos(\x)}, {4*sin(\x)});
\coordinate (A) at (4,0);
\coordinate (O) at (0,0);
\coordinate (C) at (2.83,2.83);
\draw[thin] (A)--(O)--(C)--(A);
\node[left] at (3.41,1.41) {$b_n$};
\node[right] at (4,0) {$(1,0)$};
\node[left] at (0,0) {$(0,0)$};
\node[above] at (C) {$P$};
\tkzMarkAngle[size=0.7cm,thin](A,O,C)
\tkzLabelAngle[pos = 0.5](A,O,C){$\theta$}
\end{tikzpicture}
\end{image}
Find $P$ and $\theta$, and $b_n$.
\end{problem}


\begin{problem}
Explain what the limit
\[
\lim_{n\to\infty} n\cdot b_n
\]
computes and compute the limit.
\begin{hint}
You might want to use a half-angle formula
\[
\sin(\alpha/2) = \sqrt{\frac{1-\cos(\alpha)}{2}}.
\]
\end{hint}
\end{problem}



\begin{problem}
The circle of radius $1$ has circumference $2\pi$. Use this to reason
that the circle of radius $1$ has area $\pi$.
\begin{image}
\begin{tikzpicture}[geometryDiagrams]
\draw (0,0) circle (2cm);
\foreach \coeff in {0,10,20,...,350}
         {
           \draw[thin] (0,0)--({2*cos(\coeff)},{2*sin(\coeff)});
         }
\end{tikzpicture}
\end{image}

\begin{hint}
Approximate a rectangle by rearranging the slices in the picture.
Compute the area of the ``rectangle.''
\end{hint}
\begin{freeResponse}
Slice up the circle as indicated and rearrange into the following
``rectangle.'' 
\begin{image}
\begin{tikzpicture}[geometryDiagrams] % r= 2cm
  \foreach \coeff in {0,1,2,...,17}
   {
     \draw[thin] ({2*\coeff *sin(5)},0)--({(2*\coeff+ 2)* sin(5)},{2*cos(5)});
     \draw[thin] ({(2*\coeff + 2)*sin(5)},{2*cos(5)})--({(2+2*\coeff)*sin(5)},0); %one too many!
   }
   \draw[thin] ({2*18 *sin(5)},0)--({(2*18+ 2)* sin(5)},{2*cos(5)});
   \draw (0,0)--({2*18 *sin(5)},0);
   \draw ({2*sin(5)},2)--({2*19 *sin(5)},2);
\end{tikzpicture}
\end{image}
Note, while this is not truly a rectangle, as the slices become finer,
and finer, the shape will more, and more, closely approximate a
rectangle. This rectangle has perimeter $2+2\pi$, and hence is a
$1\times \pi$ rectangle. Thus the area of the rectangle is $\pi$ as is
the area of the initial circle.
\end{freeResponse}
\end{problem}


Not content to give just one explanation, let's give another. The
techniques we use in this explanation will be used in later problems.

\begin{problem}
Again, consider the circle of radius $1$ with circumference $2\pi$.
Suppose we inscribe $n$ triangles (with one vertex at the center of
the circle), here we have shown the $n$th triangle:
\begin{image}
\begin{tikzpicture}[geometryDiagrams]
\draw [domain=-20:65] plot ({4*cos(\x)}, {4*sin(\x)});
\coordinate (A) at (4,0);
\coordinate (B) at (3.41,1.41);
\coordinate (O) at (0,0);
\coordinate (C) at (2.83,2.83);
\draw[thin] (A)--(O)--(C)--(A);
\draw[thin] (B)--(O);
\node at (2,1.1) {$h_n$};
\node[below] at (3.1,2) {$b_n$};

\tkzMarkRightAngle[thin](A,B,O)

\end{tikzpicture}
\end{image}
In each triangle, label $b_n$ and $h_n$, where $b_n$ is conceptualized
as a ``base'' and $h_n$ is a ``height.'' Compute
\[
\lim_{n\to \infty} h_n
\]
and explain your reasoning.
\begin{freeResponse}
As $n$ goes to infinity, $h_n$ goes to the radius length of the radius
of the circle. Hence
\[
\lim_{n\to \infty} h_n = 1.
\]
\end{freeResponse}
\end{problem}

\begin{problem}
Now compute 
\[
\lim_{n\to \infty} n\cdot b_n
\]
and explain your reasoning.
\begin{freeResponse}
As $n$ goes to infinity, $n\cdot b_n$ goes to the circumference of the circle. Hence 
\[
\lim_{n\to \infty} n\cdot b_n = 2\pi.
\]
\end{freeResponse}
\end{problem}

\begin{problem}
Consider the limit
\[
\lim_{n\to\infty} \frac{n\cdot b_n\cdot h_n}{2}.
\]
Explain what this limit is computing, and compute this limit.
\begin{freeResponse}
This limit is computing the sum of the areas of the triangles as $n$
goes to infinity. This will be the area of the circle of radius
$1$. From using substitution from our work above, we see
\begin{align*}
\lim_{n\to\infty} \frac{n\cdot b_n\cdot h_n}{2} &= \frac{1\cdot 2\pi}{2}\\
&=\pi.
\end{align*}
\end{freeResponse}
\end{problem}


\section{Surface area of spheres}

To compute the surface area of the sphere of radius $R$ in $3$-dimensional
euclidean space, we will show that its surface area is equal to the surface
area of something we can lay out flat. The argument for this goes way back to
the great physicist and mathematician, Archimedes of Alexandria, in the $2$nd
century B.C. To follow his argument, we have to begin by computing the area of
a `lamp shade' or `collar.' We think of a circular collar as in the figure
below%
\begin{image}
\begin{tikzpicture}
    \draw (-1,0) arc (180:360:1cm and 0.5cm);
    \draw (-1,0) arc (180:0:1cm and 0.5cm);
    \draw (-2,-3) arc (180:370:2cm and 1cm);
    \draw[dashed] (-2,-3) arc (180:10:2cm and 1cm);
    \draw(-2,-2.9)  -- (-1,0);
    \draw(2,-2.9)   -- (1,0);
    \node [left] at (2,-1) {$s$};
    \shade[left color=blue!5!white,right color=blue!60!white,opacity=0.3] (-1,0) arc (180:360:1cm and 0.5cm) -- (2,-3) arc (360:180:2cm and 1cm) -- cycle;
    \shade[left color=blue!5!white,right color=blue!60!white,opacity=0.3] (0,0) circle (1cm and 0.5cm);
\end{tikzpicture}
\end{image}
as approximated by an arrangement of trapezoids. To achieve this, we
approximate the bottom circle of the collar by an inscribed regular $n$-gon
whose vertices are the points of intersection with the slant lines in the
figure. Similarly approximate the top circle by an inscribed regular $n$-gon
positioned directly above the bottom one, again with vertices given by the
points of intersection with the slant lines. Complete a side of the bottom
$n$-gon and the side of the top $n$-gon directly above it to a trapezoid by
adjoining the two slant lines in the figure that connect endpoints. Let
\begin{itemize}
\item $b_{n}$ denote the length of a side of the bottom regular
  $n$-gon and let 
\item $t_{n}$ denote the length of a side of the top $n$-gon. 
\end{itemize}
Then the trapezoid has area
\[
\left(  \frac{b_{n}+t_{n}}{2}\right)  \cdot h_{n}%
\]
where $h_{n}$ is the vertical height of the trapezoid. The collar is
approximated by the union of these $n$ trapezoids, so the area of the collar
is approximated by the sum of the areas of the $n$ congruent trapezoids,
namely%
\[
n\cdot \left(  \frac{b_{n}+t_{n}}{2}\right)
\cdot h_{n}=\left(  \frac{n\cdot %
b_{n}+n\cdot t_{n}}{2}\right) \cdot h_{n}.
\]
As $n$ goes to infinity, the area of the approximation approaches the area of
the collar. But if 
\begin{itemize}
\item $c_{b}$ is the circumference of the bottom circle and 
\item $c_{t}$ is the circumference of the top circle and 
\item $s$ is the slant height of the collar as shown in the above
  figure,
\end{itemize}
then
\begin{align*}
\lim_{n\to \infty} n\cdot b_{n}  &  =c_{b}\\
\lim_{n\to \infty} n \cdot t_{n}  &  =c_{t}\\
\lim_{n\to \infty} h_{n}  &  =s.
\end{align*}


\begin{problem}
Let
\begin{itemize}
\item $r_b$ be the radius of the circle defining the base of the collar,
\item $r_t$ be the radius of the circle defining the top of the collar,
\item $r_a$ be the average of $r_b$ and $r_t$,
\item $s$ be the slant height of the collar.
\end{itemize}
Explain why the area of the collar is%
\[
\pi \cdot \left( r_{b}+r_{t}\right)\cdot s =2\pi\cdot r_{a}\cdot s
\]
\end{problem}


\begin{theorem}
 The surface area of the sphere of radius $R$ is the same as the
surface area of the label of the smallest can into which the sphere will fit.%
\pgfdeclareradialshading{ballshading}{\pgfpoint{-10bp}{10bp}}
 {color(0bp)=(gray!40!white); 
 color(9bp)=(gray!75!white);
 color(18bp)=(gray!70!blue); 
 color(25bp)=(gray!50!blue); 
 color(50bp)=(black)}
\begin{image}
\begin{pgfpicture}
\pgfellipse[]{\pgfxy(0,-2)}{\pgfxy(0,.7)}{\pgfxy(2,0)}
\pgfstroke
  \pgfpathcircle{\pgfpoint{0cm}{0cm}}{2cm}
  \pgfshadepath{ballshading}{20}
  \pgfusepath{}
 \pgfxyline(2,2)(2,-2)
 \pgfxyline(-2,2)(-2,-2)
\pgfellipse[]{\pgfxy(0,2)}{\pgfxy(0,.7)}{\pgfxy(2,0)}
\pgfstroke
\end{pgfpicture}
\end{image}
Namely the surface area of the sphere of radius $R$ is
\[
2\pi R\cdot 2R=4\pi R^{2}.
\]

\end{theorem}

\begin{problem}
Show why the above theorem is true.

\begin{hint}
Slice the picture above into $n$ horizontal slices. Approximate the
piece of the surface of the sphere between the $i$th pair of
successive slices by a collar $C_{i}$. Let $a\left( C_{i}\right) $
denote the area of $C_{i}$, let $r_{i}$ denote its average radius and
Explain why the surface area of the sphere is
\[
\lim_{n\to \infty}\sum_{i=1}^{n} a(C_i).
\]
Explain further how we can conclude that the area of the sphere is
given by
\[%
\lim_{n\to \infty}\sum_{i=1}^{n} 2\pi\cdot r_{i}\cdot s_{i}.
\]

\end{hint}
\begin{hint}
Let $h_{i}$ denote the vertical height of the label on the can between
the $i$th pair of successive slices. Explain why the area of the label is exactly%
\[%
\sum_{i=1}^{n} 2\pi\cdot h_{i}.
\]
\end{hint}
\begin{hint}
Explain why the relationship between each $r_{i}$, $s_{i}$ and $h_{i}$
is given by the picture below.
\begin{image}
\begin{tikzpicture}[geometryDiagrams]
\coordinate (A) at (0,1);
\coordinate (B) at (4,1);
\coordinate (C) at (4,-1);
\coordinate (D) at (.5,2);
\coordinate (E) at (-.5,0);
\coordinate (F) at (-.5,2);
\draw (A)--(B)--(C)--cycle;
\draw (D)--(E)--(F)--cycle;
\node[above] at (2,1) {$r_i$};
\node[above] at (2,0) {$R$};
\node[left] at (-.5,1) {$h_i$};
\node[left,above] at (-.2,1) {$s_i$};
\end{tikzpicture}
\end{image}
Now use facts about similar triangles to explain why
\[
r_{i}\cdot s_{i}=h_{i}\cdot R.
\]
\end{hint}
\end{problem}




\section{Volumes of pyramids and spheres}

We will start by looking at the volumes of pyramids.

\begin{problem} Show that an $r\times r\times r$ cube can be constructed
from three equal pyramids with an $r\times r$ square base. Conclude
that the volume of each pyramid is $1/3$ the volume of the cube,
namely
\[
\frac{r^{3}}{3}.
\]

\begin{hint}
Suppose the cube had a hollow interior and infinitely thin
faces. Put your (infinitely tiny) eye at one vertex of the cube and
look inside. How many faces of the cube can you see?
\end{hint}

\begin{freeResponse}
\begin{image}
\begin{tikzpicture}
\coordinate (A) at (0,0,0);
\coordinate (B) at (4,0,0);
\coordinate (C) at (4,4,0);
\coordinate (D) at (0,4,0);
\coordinate (A') at (0,0,4);
\coordinate (B') at (4,0,4);
\coordinate (C') at (4,4,4);
\coordinate (D') at (0,4,4);



\draw[fill=red,opacity=0.3] (A)--(B)--(C)--(D)--cycle; % base
\draw[fill=blue,opacity=0.3] (A')--(B')--(C')--(D')--cycle; % base
%% \draw[fill=orange,opacity=0.3] (B1)--(B2)--(peak);
%% \draw[fill=yellow,opacity=0.3] (B2)--(B3)--(peak);
%% \draw[fill=green,opacity=0.3] (B3)--(B4)--(peak);
%% \draw[fill=blue,opacity=0.3] (B4)--(B5)--(peak);
%% \draw[fill=purple,opacity=0.3] (B5)--(B1)--(peak);


\draw (A)--(B)--(C)--(D)--cycle;
\draw (A)--(A');
\draw (B)--(B');
\draw (C)--(C');
\draw (D)--(D');
\draw (A')--(B')--(C')--(D')--cycle;

\draw[ultra thick] (A)--(A')--(B')--(B)--cycle;
\draw[ultra thick] (A')--(D);
\draw[ultra thick] (B)--(D);
\draw[ultra thick, red] (B')--(D);
\draw[ultra thick, red] (A)--(D);
\end{tikzpicture}
\end{image}
\end{freeResponse}
\end{problem}

We next want to show why all pyramids with an $r\times r$ square base
and vertical altitude $r$ have the same volume. That is, if we put the
vertex of the pyramid anywhere in a plane parallel to the base and at
distance $r$, the volume is unchanged.

This fact is an example of \textit{Cavalieri's Principle}: Shearing a
figure parallel to a fixed direction does not change the
$n$-dimensional measure of an object in euclidean $n$-space. Think of
a stack of (very thin) books.

\begin{problem}
Show that Cavalieri's Principle is true for the pyramid using several
variable calculus.

\begin{hint}
Put the base of the pyramid $P$ so that its vertices are $\left(
0,0\right) $, $\left( r,0\right) $, $\left( 0,r\right) $ and $\left(
r,r\right) $ in $3$-dimensional euclidean space. Consider the
transformation%
\[
\left(  \underline{\hat{x}},\underline{\hat{y}},\underline{\hat{z}}\right)
=
\begin{bmatrix}  \hat{x} & \hat{y} & \hat{z}\end{bmatrix}
\begin{bmatrix}
1 & 0 & 0\\
0 & 1 & 0\\
a & b & 1
\end{bmatrix}
\]
now use calculus.
\end{hint}

\begin{hint}
Recall that if $P$ is a region in $3$-space, 
\[
\int_{\underline{P}} f(\hat{x},\hat{y},\hat{z})  \d \hat{x}d\hat{y}d\hat{z} = 
\int_{P} f(\underline{\hat{x}},\underline{\hat{y}},\underline{\hat{z}})) \left|\det J(\underline{\hat{x}},\underline{\hat{y}},\underline{\hat{z}})\right| \d \hat{x}d\hat{y}d\hat{z},
\]
where
\[
J(\underline{\hat{x}},\underline{\hat{y}},\underline{\hat{z}}) =
{\renewcommand\arraystretch{2}
\begin{bmatrix}
\pp[\underline{\hat{x}}]{\hat{x}} & \pp[\underline{\hat{x}}]{\hat{y}} & \pp[\underline{\hat{x}}]{\hat{z}} \\
\pp[\underline{\hat{y}}]{\hat{x}} & \pp[\underline{\hat{y}}]{\hat{y}} & \pp[\underline{\hat{y}}]{\hat{z}} \\
\pp[\underline{\hat{z}}]{\hat{x}} & \pp[\underline{\hat{z}}]{\hat{y}} & \pp[\underline{\hat{z}}]{\hat{z}} 
\end{bmatrix}}.
\]
\end{hint}



\end{problem}

\section{Magnification principle}

\textit{Magnification principle:} If an object in euclidean $n$-space is
magnified by factors of $r_{1},\ldots,r_{n}$, its $n$-dimensional
measure is multiplied by $r_{1}\cdots r_{n}$.

\begin{problem}
Use this magnification principle to justify the volume formula%
\[
(1/3)B\cdot h
\]
for any pyramid with rectangular base of area $B$ and vertical altitude $h$.
\end{problem}

\begin{problem}
Prove the magnification principle using several variable calculus.

\begin{hint}
Consider the transformation%
\[
\left(  \underline{\hat{x}_{1}},\dots,\underline{\hat{x}_{n}}\right)
=
\begin{bmatrix}
\hat{x}_{1} & \dots & \hat{x}_{n}
\end{bmatrix}  
\begin{bmatrix}
r_{1} & 0   & 0 & \dots & 0 & 0\\
0     & r_2 & 0 & \dots & 0 & 0\\
\hdotsfor{6}\\
0     & 0   & 0 & \dots & 0 & r_{n}%
\end{bmatrix}
\]
and use calculus.
\end{hint}
\end{problem}

Now suppose we have any pyramid%
\begin{image}
\begin{tikzpicture}
%\draw[thick,-stealth] (0,0,0)--(0,0,6); 
%\draw[thick,-stealth] (0,0,0)--(0,6,0);
%\draw[-stealth] (0,0,0)--(6,0,0);
\coordinate (B1) at (0,0,0);
\coordinate (B2) at (0,0,2);
\coordinate (B3) at (2,0,1);
\coordinate (B4) at (4,0,2);
\coordinate (B5) at (3,0,0);
\coordinate (peak) at (1,3,1);

\draw[fill=red,opacity=0.3] (B1)--(B2)--(B3)--(B4)--(B5)--cycle; % base
\draw[fill=orange,opacity=0.3] (B1)--(B2)--(peak);
\draw[fill=yellow,opacity=0.3] (B2)--(B3)--(peak);
\draw[fill=green,opacity=0.3] (B3)--(B4)--(peak);
\draw[fill=blue,opacity=0.3] (B4)--(B5)--(peak);
\draw[fill=purple,opacity=0.3] (B5)--(B1)--(peak);

\draw[ultra thick] (B2)--(B3)--(B4)--(B5)--(peak)--(B4)--(B3)--(peak)--(B2);
\end{tikzpicture}
\end{image}
with any shaped base of area $B$ and any vertical altitude $h$. Approximate
the base as close as you want (i.e.\ $\varepsilon$-close) by tiling its interior
with rectangles. Let $t$ denote the sum of the areas of these rectangles.
Approximate the base as close as you want (i.e.\ $\varepsilon$-close) by
covering it entirely with rectangles. Let $T$ denote the sum of the areas of
these rectangles. Why is the area $B$ of the base of the pyramid caught
between $B-\varepsilon$ and $B+\varepsilon$?

\begin{problem}
Show that the volume $V$ of the pyramid is caught between $\left(
1/3\right) \cdot t\cdot h$ and $\left( 1/3\right) \cdot T\cdot
h$. Argue that, given any positive real number $\varepsilon$, however
small, the volume $V$ of the pyramid is caught between $\left(
1/3\right) \cdot \left( B-\varepsilon\right)\cdot h$ and $\left(
1/3\right) \cdot \left( B+\varepsilon\right) \cdot h$. Finally, show that
\[
V=\left(  1/3\right)  \cdot B\cdot h.
\]
\end{problem}

\section{Relation between volume and surface area of a sphere}

Think of a disco-ball. Its surface is approximately a sphere, but that
surface is made up of tiny flat mirrors.

\begin{problem}\hfil
\begin{enumerate}
\item Explain why you can think of the disco-ball as being made up of
  pyramids, with each pyramid having base one of the tiny mirrors and
  vertex at the interior point $O$ at the center of the disco-ball.
\item Argue that the volume of the disco-ball is $\left( 1/3\right) $
  times the distance $h$ from a mirror to $O$ times the sum of the
  areas of all the mirrors.
\end{enumerate}
\end{problem}

\begin{problem}
Argue that, as the mirrors are made to be smaller and smaller,
\begin{enumerate}
\item the sum of the areas of the mirrors approaches the surface area of a sphere,
\item the distance $h$ approaches the radius $R$ of that sphere,
\item the volume of the disco-ball approaches the volume of the sphere.
\end{enumerate}
Conclude that, for a sphere of radius $R$ in euclidean $3$-space, the relation
between the volume $V$ of the sphere and the surface area $S$ of the sphere is
given by the formula%
\[
V=\frac{R\cdot S}{3}.
\]
\end{problem}

\begin{problem}
Use the work above to deduce a formula for the volume of a sphere.
\end{problem}




\begin{problem}
Summarize the results from this section. In particular, indicate which
results follow from the others.
\begin{freeResponse}
\end{freeResponse}
\end{problem}


\end{document}
