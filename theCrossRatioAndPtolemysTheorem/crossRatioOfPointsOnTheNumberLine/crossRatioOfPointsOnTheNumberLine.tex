\documentclass{ximera}

\usepackage{microtype}
\usepackage{tikz}
\usepackage{tkz-euclide}
\usetkzobj{all}
\tikzstyle geometryDiagrams=[ultra thick,color=blue!50!black]

\renewcommand{\epsilon}{\varepsilon}



%% \prerequisites{euclideanGeometry}
%% \outcome{circles}

\title{Cross-ratio of points on the number line}

\begin{document}
\begin{abstract}
In this activity, we do some basic projective geometry and learn
Ptolemy's Theorem.
\end{abstract}
\maketitle

In this section, we will study some basic ideas of \textit{projective
  geometry}. The main topic, one closely related to the notion of
perspective in painting, is called the \textit{cross-ratio}.

In particular, we will use the cross-ratio to prove a famous
mathematical relationship, \textit{Ptolemy's Theorem}. We will then
explore several corollaries of Ptolemy's Theorem. Before we start, we
will need the \textit{Law of Sines}.

\begin{problem}
State and prove the Law of Sines. 
\begin{hint}
Draw an arbitrary triangle, and add an altitude, call it $h$. 
\end{hint}
\begin{freeResponse}
\end{freeResponse}
\end{problem}



\section*{Cross-ratio of points on the number line}

We begin by studying the cross-ratio of four points on a line. Start
with the set of points on the real number line with coordinate $t$ and
add one extra point called $t=\infty$. Call the resulting set
$\overline{\mathbb{R}}$. You could think of the resulting set as the
set of all lines through the origin in $\mathbb{R}^{2}$ by assigning
to each line the real number that is its slope and to the $y$-axis the
slope $\infty$.

\begin{problem}\hfil
\begin{enumerate}
\item Show that the transformation%
\[
\left(  x,y\right)  \mapsto\left(  \underline{x},\underline{y}\right)
=\left(  x,y\right)  \cdot\left(
\begin{array}
[c]{cc}%
d & b\\
c & a
\end{array}
\right)
\]
is a one-to-one and onto (linear) transformation of $\mathbb{R}^{2}$ as long as
\begin{equation}
\left\vert
\begin{array}
[c]{cc}%
d & b\\
c & a
\end{array}
\right\vert \neq0. \label{52}%
\end{equation}

\item For the transformation in the previous Problem, show that every line through the origin in $\left( x,y\right) $-space is sent to a line
through the origin in $\left( \underline{x},\underline{y}\right)
$-space. The slope $t$ of the line through $\left( 0,0\right) $ and
$\left( x,y\right) $ is of course $t=\frac{y}{x}$. What is the
slope \underline{$t$} of the line through
$\left( \underline{0},\underline{0}\right) $ and $\left(
\underline{x},\underline{y}\right)  $? Show that%
\begin{equation}
\underline{t}=\frac{at+b}{ct+d} \label{41}%
\end{equation}
\end{enumerate}
\begin{freeResponse}
To show the map is one-to-one, we will provide an explicit inverse
map, namely:
\[
\left(  \underline{x},\underline{y}\right)\mapsto \left(  \underline{x},\underline{y}\right) 
\frac{1}{ad-bc}\begin{pmatrix}
a &-b \\
-c& d
\end{pmatrix}
\]
Now apply the given transformation and then its inverse to see
\[
(x,y) \cdot \begin{pmatrix}
d & b \\
c & d
\end{pmatrix}\frac{1}{ad-bc}\begin{pmatrix}
a &-b \\
-c& d
\end{pmatrix} = (x,y).
\]
In fact, the inverse map shows that the original map is onto as well,
 as for any element of $\mathbb{R}^2$, we can find the preimage of
 this element using the inverse.  

To answer the second part, consider the line
\[
\l(u) = u\cdot (x,y).
\]
Now 
\[
\underline{\l}(u) = u\cdot (cy+dx,ay+bx)
\]
and this line has slope
\[
\frac{ay+bx}{cy+dx}.
\]
On the other hand, 
\begin{align*}
\frac{at+b}{ct+d} &= \frac{\frac{ay}{x} + b}{\frac{cy}{x}+d} \\
&= \frac{ay+bx}{cy+dx}.
\end{align*}
This is what we needed to show.
\end{freeResponse}
\end{problem}




\begin{definition}
Functions $\left( \ref{41}\right) $ for which the condition $\left(
\ref{52}\right)  $ holds are called \textbf{linear fractional transformations}.
\end{definition}


\begin{problem}
Show that a linear fractional transformation%
\begin{align*}
\overline{\mathbb{R}} &\rightarrow\overline{\mathbb{R}}\\
t &\mapsto\underline{t} =\frac{at+b}{ct+d}%
\end{align*}
is one-to-one and onto. What is its inverse function? (Your answer should show that
the inverse function is also a linear fractional transformation.)

\begin{hint}
By algebra solve for $t$ in terms of $\underline{t}$. Then graph%
\[
\underline{t}=\frac{at+b}{ct+d}%
\]
in the $\left(  t,\underline{t}\right)  $-plane. If $c=0$ show that the graph
is a straight line with non-zero slope and%
\[
\infty\mapsto\infty.
\]
If $c\neq0$, show that the graph has exactly one horizontal asymptote where
$t\mapsto\infty$ and one vertical asymptote where $\underline{t}\mapsto\infty$.
\end{hint}
\begin{freeResponse}
To show this map is one-to-one and onto, we will again provide an
explicit inverse map:
\[
\underline{t} \mapsto \frac{d \underline{t}-b}{-c \underline{t} +a}
\]
If $c= 0$, this is a line with slope $d/a$. Note since $ad-bc \ne 0$,
we see that this slope is neither $0$ nor undefined. Moreover, by the construction of $\overline{\mathbb{R}}$, we see that $\infty\mapsto\infty$.

If $c\ne0$ we have that
\[
\lim_{t\to \infty} \frac{at+b}{ct+d}  = a/c
\]
and this is the horizontal asymptote. On the other hand, 
\[
\lim_{\underline{t}\to\infty} \frac{d \underline{t}-b}{-c \underline{t} +a} = -d/c
\]
this corresponds to the vertical asymptote. 
\end{freeResponse}
\end{problem}







\begin{problem}
Show that the set of linear fractional transformations form a group
under the operation of composition of functions. That is, check
associativity, identity element and existence of inverses.
\begin{freeResponse}
To show that the set of linear fractional transformations form a group, we must show that this set is 
\begin{enumerate}
\item closed under function composition, 
\item that functional composition is an associative operation on linear fractional transformations, 
\item that there is an identity element, and 
\item that every linear fractional transformation has an inverse that is also a linear fractional transformation. 
\end{enumerate}
We will start by showing that the set of linear fractional
transformations is closed under functional composition. Write
\begin{align*}
f:t &\mapsto\frac{at+b}{ct+d}\\
g:t &\mapsto\frac{mt+n}{pt+q}
\end{align*}
where
\[
\begin{array}{|cc|}
d & b\\
c& a
\end{array} \ne 0\qquad\text{and}\qquad
\begin{array}{|cc|}
q & n\\
p& m
\end{array} \ne 0.
\]
Now
\[
f\circ g : t \mapsto \frac{(am+bp)t+(an+bq)}{(cm+dp)t + (cn+dq)}
\]
and note that
\begin{align*}
\begin{array}{|cc|}
cn+dq & an+bq\\
cm+dp& am+bp
\end{array}  &= (cn+dq)(am+bp)- (an+bq)(cm+dp)\\
&= (bc - ad)(np-mq) \\
&\ne 0.
\end{align*}
Hence we see that the composition of linear fractional transformations
is a linear fractional transformation.  

Now we will show that composition is an associative operation. Note,
functional composition is \textbf{always} an associative
operation. However, it is a good problem to actually show this---so
we will. Write
\[
f:t \mapsto\frac{at+b}{ct+d},\qquad
g:t \mapsto\frac{mt+n}{pt+q},\qquad
h:t \mapsto\frac{rt+s}{ut+v}.
\]
where
\[
\begin{array}{|cc|}
d & b\\
c& a
\end{array} \ne 0,\qquad
\begin{array}{|cc|}
q & n\\
p& m
\end{array} \ne 0,\qquad
\begin{array}{|cc|}
v & s\\
u& r
\end{array} \ne 0.
\]
Write
\[
f\circ g : t \mapsto \frac{(am+bp)t+(an+bq)}{(cm+dp)t + (cn+dq)}
\]
and so
\[
(f\circ g) \circ h : t\mapsto \frac{(amr+bpr+anu+bqu)t+(ams+bps+anv+bqv)}{(cmr+dpr+cnu+dqu)t+(cms+dps+cnv+dqv)}. 
\]
Now
\[
g\circ h : t \mapsto \frac{(mr+nu)t+(ms+nv)}{(pr+qu)t + (ps+qv)}
\]
and
\[
f\circ (g \circ h) : t\mapsto \frac{(amr+bpr+anu+bqu)t+(ams+bps+anv+bqv)}{(cmr+dpr+cnu+dqu)t+(cms+dps+cnv+dqv)}. 
\]
From our work above we see that $(f\circ g)\circ h = f\circ(g\circ
 h)$.  

The identity element is
\[
t\mapsto \frac{a\cdot t + 0}{0\cdot t + a} = t,
\]
where $a$ is a nonzero real number.  Finally, each linear fractional
transformation has an inverse---we found this in the previous problem.
\end{freeResponse}
\end{problem}



\begin{problem}
\label{59}Show that, for any three distinct points $t_{2},t_{3}$ and $t_{4}$,
the function of $t$ given by the formula%
\[
\underline{t}
=\frac{(t-t_2)(t_3-t_4)}{(t-t_4)(t_3-t_2)}
\]
takes $t_{2}$ to $0$, takes $t_{3}$ to $1$ and takes $t_{4}$ to
$\infty$. Show that this function is a linear fractional
transformation, that is, a function of the form
$\left( \ref{41}\right) $ for which the condition $\left(
\ref{52}\right)$ holds.
\begin{freeResponse}
Write
\begin{align*}
t_2 &\mapsto \frac{(t_2-t_2)(t_3-t_4)}{(t_2-t_4)(t_3-t_2)} = 0.\\
t_3 &\mapsto \frac{(t_3-t_{2})(t_{3}-t_{4})}{(t_3-t_{4})(t_{3}-t_{2})}= 1.\\
t_4 &\mapsto \frac{(t_4-t_{2})(t_{3}-t_{4})}{(t_4-t_{4})(t_{3}-t_{2})} = \infty.
\end{align*}
With a slight bit of algebra, we can see that this is a linear fractional transformation:
\[
\frac{(t-t_{2})(t_{3}-t_{4})}{(t-t_{4})(t_{3}-t_{2})} = \frac{t_3t - t_2t_3-t_4t +t_2t_4}{t_3t-t_3t_4-t_2t+t_2t_4} = \frac{(t_3-t_4)t + (t_2t_4-t_2t_3)}{(t_3-t_2)t + (t_2t_4-t_3t_4)}
\]
though we must check the determinant condition:
\begin{align*}
\begin{array}{|cc|}
(t_2t_4 - t_3t_4) & (t_2t_4 - t_2t_3) \\
(t_3-t_2) & (t_3-t_4)
\end{array}
&=
\begin{array}{|cc|}
(t_2 - t_3)t_4 & (t_4 - t_3)t_2 \\
(t_3-t_2) & (t_3-t_4)
\end{array}\\
&=(t_2-t_3)(t_3-t_4)t_4 - (t_3-t_2)(t_4-t_3)t_2\\
&= (t_2-t_3)(t_3-t_4)t_4 - (t_2-t_3)(t_3-t_4)t_2\\
&= (t_2-t_3)(t_3-t_4)(t_4 -t_2).
\end{align*}
Since $t_2$, $t_3$, and $t_4$ are distinct, this cannot be
zero.
\end{freeResponse}
\end{problem}





\begin{problem}
\label{57}Show that any linear fractional transformation $\left(
\ref{41}\right)$ that leaves $0$, $1$, and $\infty$ fixed is the identity map.
\begin{freeResponse}
Consider an arbitrary linear fractional transformation
\[
f:t \mapsto \frac{at +b}{ct+d}.
\]
If $f(0)=0$, then we see that $b=0$ and that $d\ne 0$. Proceeding
along with this information if $f(1) = 1$, then we see that
\[
\frac{a}{c+d} = 1
\]
If $f(\infty) = \infty$, then we see that $c=0$. Hence, $a=d\ne
0$. This is the identity map.  \end{freeResponse}
\end{problem}







\begin{problem}
\label{42}Suppose that you are given a function $\left(  \ref{41}\right)  $
and four points $t_{1},t_{2},t_{3}$ and $t_{4}$. Let
\[
\underline{t_{i}}=\frac{at_{i}+b}{ct_{i}+d}%
\]
for $i=1,2,3,4$. Show that%
\[
\frac{(\underline{t_1}-\underline{t_{2}})(\underline{t_{3}}-\underline{t_{4}})}{(\underline{t_1}-\underline{t_{4}})(\underline{t_{3}}-\underline{t_{2}})}=
\frac{(t_1-t_{2})(t_{3}-t_{4})}{(t_1-t_{4})(t_{3}-t_{2})}.
\]
[MJG,288]

\begin{hint}
Just write out the formula for each side and do the high school
algebra.  There is a fancier way that uses that the set of linear
fractional transformations form a group whose operation is
composition. It goes like this. Use a previous problem to show that
the inverse of the linear fractional transformation
\[
t\mapsto\frac{(t-t_2)(t_3-t_4)}{(t-t_4)(t_3-t_2)}
\]
followed by%
\[
t\mapsto\underline{t}%
\]
and then followed by
\[
t\mapsto\frac{(t-\underline{t_2})(\underline{t_3}-\underline{t_4})}{(t-\underline{t_4})(\underline{t_3}-\underline{t_2})}
\]
fixes $0$, $1$, and $\infty$ and so is the identity transformation by a previous problem. So%
\[
t\mapsto\frac{(t-t_2)(t_3-t_4)}{(t-t_4)(t_3-t_2)}
\]
is the same transformation as%
\[
t\mapsto\frac{(\underline{t}-\underline{t_2})(\underline{t_3}-\underline{t_4})}{(\underline{t}-\underline{t_4})(\underline{t_3}-\underline{t_2})}.
\]
\end{hint}
\begin{freeResponse}
We will use the fancier method of a solution. Start by writing
\begin{align*}
\frac{(t-t_2)(t_3-t_4)}{(t-t_4)(t_3-t_2)}
\mapsto t\mapsto \underline{t}\mapsto 
\frac{(\underline{t}-\underline{t_2})(\underline{t_3}-\underline{t_4})}{(\underline{t}-\underline{t_4})(\underline{t_3}-\underline{t_2})}.
\end{align*}
From our work above, we know this is a linear fractional
transformation. Moreover, from our work above, we can see that this
map fixes $0$, $1$, and $\infty$, hence this map is the identity
map. The upshot of this is that the map
\[
t\mapsto\frac{(t-t_2)(t_3-t_4)}{(t-t_4)(t_3-t_2)}
\]
is now seen to be identical to the composition of maps
\[
t\mapsto \underline{t}\mapsto \frac{(\underline{t}-\underline{t_2})(\underline{t_3}-\underline{t_4})}{(\underline{t}-\underline{t_4})(\underline{t_3}-\underline{t_2})}.
\]
Evaluating these maps at $t_1$ gives the desired result. 
\end{freeResponse}
\end{problem}

\begin{definition}
\label{44}Given four ordered points $t_{1},t_{2},t_{3}$ and $t_{4}$, the \textbf{cross-ratio} $\left(t_{1}:t_{2}:t_{3}:t_{4}\right)$ is defined by
\[
\left( t_{1}:t_{2}:t_{3}:t_{4}\right)
=\frac{t_{1}-t_{2}}{t_{3}-t_{2}}% \div\frac{t_{1}-t_{4}}{t_{3}-t_{4}}.
\]

\end{definition}

A previous problem shows that if four points are moved by any function $\left(
\ref{41}\right)  $ the cross-ratio $\left(  \underline{t_{1}}:\underline
{t_{2}}:\underline{t_{3}}:\underline{t_{4}}\right)  $ of the output four
points is the same as the cross-ratio $\left(  t_{1}:t_{2}:t_{3}:t_{4}\right)
$ of the original four points.












\section*{Cross-ratio of points on a circle}

\begin{problem}\label{46}
In the diagram
\begin{image}
\begin{tikzpicture}[geometryDiagrams]
\coordinate (O) at ({2*cos(120)},{2*sin(120)});
\coordinate (A) at ({2*cos(190)},{2*sin(190)});
\coordinate (B) at ({2*cos(250)},{2*sin(250)});
\coordinate (C) at ({2*cos(330)},{2*sin(330)});

\draw (0,0) circle (2cm);
\draw (O) -- (C);
\draw (A) -- (B);
\draw (O) -- (B);
\draw (B) -- (C);
\draw (O) -- (A);


\tkzMarkAngle[size=.9cm,thin](A,O,B)
\tkzLabelAngle[pos = 0.7](A,O,B){$\alpha$}
\tkzMarkAngle[arc=ll,size=0.9cm,thin](B,O,C)
\tkzLabelAngle[pos = 0.7](B,O,C){$\beta$}

\tkzLabelPoints[above](O)
\tkzLabelPoints[left](A)
\tkzLabelPoints[below](B)
\tkzLabelPoints[right](C)

\end{tikzpicture}
\end{image}
show that%
\[
\frac{\left\vert AB\right\vert }{\left\vert CB\right\vert }=\frac
{\sin\alpha}{\sin\beta}=\frac{\sin\left(  \angle
AOB\right)  }{\sin\left(  \angle COB\right)  }.
\]


\begin{hint}
Notice that
\[
m\left(  \angle BAO\right)  +m\left(\angle OCB\right)  =180^{\circ}%
\]
so that%
\[
\sin\left(  \angle BAO\right)  =\sin\left(\angle  OCB\right)  .
\]
Now use the Law of Sines.
\end{hint}
\begin{freeResponse}
By the Law of Sines we have that
\[
\frac{|AB|}{\sin\alpha} = \frac{|OB|}{\sin(\angle OAB)}\qquad\text{and}\qquad\frac{|BC|}{\sin\beta} = \frac{|OB|}{\sin(\angle OCB)}.
\]
Write 
\begin{align*}
m\left(  \angle BAO\right)  +m\left(\angle OCB\right)  &=180^{\circ}\\
m\left(  \angle BAO\right)  &=180^{\circ}-m\left(\angle OCB\right)  \\
\sin\left(\angle BAO\right)  &=\sin\left(180^{\circ}-m\left(\angle OCB\right)\right)  \\
\sin\left(\angle BAO\right)  &=\sin\left(180^{\circ}\right)\cos\left(\angle OCB\right) - \cos\left(180^{\circ}\right)\sin\left(\angle OCB\right)\\ 
\sin\left(\angle BAO\right)  &=\sin\left(\angle OCB\right).
\end{align*}
So  now we have that 
\[
\frac{|AB|}{\sin\alpha} = \frac{|OB|}{\sin(\angle OAB)}= \frac{|OB|}{\sin(\angle OCB)}=\frac{|BC|}{\sin\beta}. 
\]
Using algebra we find 
\[
\frac{\left\vert AB\right\vert }{\left\vert CB\right\vert }=\frac
{\sin\alpha}{\sin\beta}=\frac{\sin\left(  \angle
AOB\right)  }{\sin\left(  \angle COB\right)  }.
\]
\end{freeResponse}
\end{problem}

\begin{remark}
Note that if, in the above figure, $B$ moves along the
circle to the other side of $C$, it is still true that%
\[
\frac{\left\vert AB\right\vert }{\left\vert CB\right\vert }=\frac
{\sin\left(  \angle AOB\right)  }{\sin\left(  \angle
COB\right)  }%
\]
%% \begin{freeResponse}
%% By the Law of Sines we have that
%% \[
%% \frac{|AB|}{\sin(\angle AOB)} = \frac{|OB|}{\sin(\angle OAB)}\qquad\text{and}\qquad\frac{|CB|}{\sin(\angle COB)} = \frac{|OB|}{\sin(\angle OCB)}.
%% \]
%% However, 
%% \[
%% \angle OAB = \angle OCB
%% \]
%% So now we have that
%% \[
%% \frac{|AB|}{\sin(\angle AOB)} =\frac{|CB|}{\sin(\angle COB)}. 
%% \]
%% Using algebra we find 
%% \[
%% \frac{\left\vert AB\right\vert }{\left\vert CB\right\vert }=\frac
%% {\sin\left(  \angle AOB\right)  }{\sin\left(  \angle
%% COB\right)  }.
%% \]
%% \end{freeResponse}
\end{remark}






\begin{problem}\label{48}
In the diagram
\begin{image}
\begin{tikzpicture}[geometryDiagrams]
\coordinate (O) at ({2*cos(120)},{2*sin(120)});
\coordinate (A) at ({2*cos(190)},{2*sin(190)});
\coordinate (B) at ({2*cos(250)},{2*sin(250)});
\coordinate (C) at ({2*cos(330)},{2*sin(330)});

\coordinate (A') at (-3.21,-3);
\coordinate (B') at (-.59,-3);
\coordinate (C') at (3.73,-3);

\draw (0,0) circle (2cm);
\draw (O) -- (C');
\draw (A) -- (B);
\draw (O) -- (B');
\draw (B) -- (C);
\draw (O) -- (A');

\draw (-4,-3) -- (5,-3);


\tkzMarkAngle[size=.9cm,thin](A,O,B)
\tkzLabelAngle[pos = 0.7](A,O,B){$\alpha$}
\tkzMarkAngle[arc=ll,size=0.9cm,thin](B,O,C)
\tkzLabelAngle[pos = 0.7](B,O,C){$\beta$}

\tkzMarkAngle[size=0.7cm,thin](B',A',A)
\tkzLabelAngle[pos = 0.4](B',A',A){$\gamma$}

\tkzMarkAngle[size=0.9cm,thin](C,C',B')
\tkzLabelAngle[pos = 0.7](C,C',B'){$\delta$}

\tkzLabelPoints[above](O)
\tkzLabelPoints[left](A)
\tkzLabelPoints[below left](B)
\tkzLabelPoints[below](A',B',C')
\tkzLabelPoints[right](C)

\end{tikzpicture}
\end{image}
show that%
\[
\frac{\left\vert A'B'\right\vert }{\left\vert C^{\prime
}B'\right\vert }=\frac{\sin\alpha}{\sin\beta}\div
\frac{\sin\gamma}{\sin\delta}=\frac{\sin\left(  \angle
A'OB'\right)  }{\sin\left(  \angle C^{\prime
}OB'\right)  }\div\frac{\sin\left(  \angle B'%
A'O\right)  }{\sin\left(  \angle B'C'O\right)
}.
\]
[MJG,266-267]
\begin{freeResponse}
By the Law of Sines we have that
\[
\frac{|A'B'|}{\sin \alpha} = \frac{|OB'|}{\sin \gamma}\qquad\text{and}\qquad\frac{|C'B'|}{\sin \beta} = \frac{|OB'|}{\sin\delta}.
\]
We now see that
\[
|A'B'| = \frac{|OB'|\sin\alpha}{\sin \gamma}\qquad\text{and}\qquad|C'B'| = \frac{|OB'|\sin\beta}{\sin\delta}.
\]
Finally, we see directly that 
\[
\frac{\left\vert A'B'\right\vert }{\left\vert C^{\prime
}B'\right\vert }=\frac{\sin\alpha}{\sin\beta}\div
\frac{\sin\gamma}{\sin\delta}=\frac{\sin\left(  \angle
A'OB'\right)  }{\sin\left(  \angle C^{\prime
}OB'\right)  }\div\frac{\sin\left(  \angle B'%
A'O\right)  }{\sin\left(  \angle B'C'O\right)
}.
\]
\end{freeResponse}
\end{problem}


\begin{problem}\label{49}
Show that if, in the previous figure, $B'$ moves
along the line to the other side of $C'$, it is still true that%
\[
\frac{\left\vert A'B'\right\vert }{\left\vert C^{\prime
}B'\right\vert }=\frac{\sin\left(  \angle A'%
OB'\right)  }{\sin\left(  \angle C'OB'\right)
}\div\frac{\sin\left(  \angle B'A'O\right)
}{\sin\left(  \angle B'C'O\right)  }.
\]
\begin{freeResponse}
By the Law of Sines we have that
\[
\frac{|A'B'|}{\sin(\angle A'OB')} = \frac{|OB'|}{\sin(\angle B'A'O)}\qquad\text{and}\qquad\frac{|C'B'|}{\sin(\angle C'OB')} = \frac{|OB'|}{\sin(\angle B'C'O)}.
\]
We now see that
\[
|A'B'| = \frac{|OB'|\sin(\angle A'OB')}{\sin(B'A'O)}\qquad\text{and}\qquad|C'B'| = \frac{|OB'|\sin(\angle C'OB')}{\sin(\angle B'C'O)}.
\]
Again, we see directly that 
\[
\frac{\left\vert A'B'\right\vert }{\left\vert C^{\prime
}B'\right\vert }=\frac{\sin\left(  \angle A'%
OB'\right)  }{\sin\left(  \angle C'OB'\right)
}\div\frac{\sin\left(  \angle B'A'O\right)
}{\sin\left(  \angle B'C'O\right)  }.
\]
\end{freeResponse}
\end{problem}

These last two Problems allow us to define the cross-ratio of four
points on a circle.

\begin{definition}
For a sequence of four (ordered) points $A,B,C,$ and $D$ on a circle,
we define%
\[
\left(  A:B:C:D\right)  =\frac{\left\vert AB\right\vert }{\left\vert
CB\right\vert }\div\frac{\left\vert AD\right\vert }{\left\vert CD\right\vert }%
\]
which we call the \textbf{cross-ratio of the ordered sequence of four
  points on a circle}.  Similarly for a sequence of four (ordered) points
$A',B^{\prime },C',$ and $D'$ on a line, we define%
\[
\left(  A':B':C':D'\right)  =\frac{\left\vert
A'B'\right\vert }{\left\vert C'B'\right\vert
}\div\frac{\left\vert A'D'\right\vert }{\left\vert C^{\prime
}D'\right\vert }%
\]
which we call the \textbf{cross-ratio of the ordered sequence of the
  four points on a line}.
\end{definition}

Notice that the original definition of a \textit{cross-ratio} is just
a refinement of the definition of $\left( A':B':C':D'\right)$ just
above. In the original definition we are keeping track of the signs of
the terms in the quotients whereas $\left( A':B':C':D'\right)$ is
always non-negative.

\begin{problem}\label{50}
Show that, in the figure%
\begin{image}
\begin{tikzpicture}[geometryDiagrams]
\coordinate (O) at ({2*cos(120)},{2*sin(120)});
\coordinate (A) at ({2*cos(190)},{2*sin(190)});
\coordinate (B) at ({2*cos(250)},{2*sin(250)});
\coordinate (C) at ({2*cos(310)},{2*sin(310)});
\coordinate (D) at ({2*cos(340)},{2*sin(340)});

\coordinate (A') at (-3.21,-3);
\coordinate (B') at (-.59,-3);
\coordinate (C') at (2.31,-3);
\coordinate (D') at (4.64,-3);

\draw (0,0) circle (2cm);
\draw (O) -- (C');
\draw (A) -- (B);
\draw (C) -- (D);
\draw (O) -- (B');
\draw (B) -- (C);
\draw (O) -- (A');
\draw (O) -- (D');

\draw (-4,-3) -- (6,-3); % line projecting on



\tkzLabelPoints[above](O)
\tkzLabelPoints[left](A)
\tkzLabelPoints[below left](B)
\tkzLabelPoints[below](A',B',C',D')
\tkzLabelPoints[right](C)
\tkzLabelPoints[above right](D)

\end{tikzpicture}
\end{image}
we have the equality%
\[
\left(  A:B:C:D\right)  =\left(A':B':C':D'\right)  .
\]


\begin{hint}
Use the previous problems. 
\end{hint}

%What happens in if we move $B$ to the other side of $C$?
\begin{freeResponse}
By our previous work, 
\begin{align*}
(A':B':C':D') &=\frac{|A'B'|}{|C'B'|}\div \frac{|A'D'|}{|C'D'|}\\
&=\left(\frac{\sin(\angle A'OB')}{\sin(\angle C'OB')} \div \frac{\sin(\angle B'A'O)}{\sin(\angle B'C'O)}\right) \div \left(\frac{\sin(\angle A'OD')}{\sin(\angle C'OD')}\div\frac{\sin(\angle D'A'O)}{\sin(\angle D'C'O)}\right).
\end{align*}
Since 
\begin{align*}
m(\angle A'OB') &= m(\angle AOB),\\
m(\angle C'OB') &= m(\angle COB),\\
m(\angle A'OD') &= m(\angle AOD),\\
m(\angle C'OD') &= m(\angle COD),\\
m(\angle B'A'O) &= m(\angle D'A'O),
\end{align*}
and as we have shown, 
\[
\sin(\angle B'C'O) = \sin(\angle D'C'O)
\]
as 
\[
m(\angle B'C'O) + m(\angle D'C'O) = 180^\circ.
\]
With all of these substitutions, our ``large'' expression above becomes
\[
\frac{\sin(\angle AOB)}{\sin(\angle COB)} \div \frac{\sin(\angle AOD)}{\sin(\angle COD)} = (A:B:C:D). 
\]
This completes the proof. 
\end{freeResponse}
\end{problem}


We say that ``Cross-ratio is invariant under stereographic
projection.''







\section*{Ptolemy's Theorem}

You can easily convince yourself with a few examples that, given four
non-collinear points $A,B,C$ and $D$ in the plane, it is not always
true that there is a circle that passes through all four. A famous
theorem of classical Euclidean geometry gives a condition when there
is a circle that passes through all four.

\begin{theorem}[Ptolemy] If the ordered sequence of points $A,B,C$ and $D$ lies on a circle,
\begin{image}
\begin{tikzpicture}[geometryDiagrams]
\coordinate (A) at ({2*cos(190)},{2*sin(190)});
\coordinate (B) at ({2*cos(250)},{2*sin(250)});
\coordinate (C) at ({2*cos(310)},{2*sin(310)});
\coordinate (D) at ({2*cos(340)},{2*sin(340)});

\draw (0,0) circle (2cm);
\draw (A) -- (B);
\draw (C) -- (D);
\draw (A) -- (D);
\draw (B) -- (C);

\draw[thin] (A) -- (C);
\draw[thin] (B) -- (D);

\tkzLabelPoints[left](A)
\tkzLabelPoints[below left](B)
\tkzLabelPoints[right](C)
\tkzLabelPoints[above right](D)

\end{tikzpicture}
\end{image}
then%
\[
\left\vert AC\right\vert \cdot\left\vert BD\right\vert
=\left\vert AD\right\vert \cdot\left\vert BC\right\vert
+\left\vert AB\right\vert \cdot\left\vert CD\right\vert
.
\]
That is, the product of the diagonals of the quadrilateral $ABCD$ is the sum
of the products of pairs of opposite sides.
\end{theorem}

\begin{proof}
We need to check that%
\[
\left\vert AC\right\vert \cdot\left\vert BD\right\vert
=\left\vert AD\right\vert \cdot\left\vert BC\right\vert
+\left\vert AB\right\vert \cdot\left\vert CD\right\vert
\]
or, what is the same, we need to check that%
\[
\frac{\left\vert AC\right\vert \cdot\left\vert
BD\right\vert }{\left\vert AD\right\vert \cdot\left\vert
BC\right\vert }=1+\frac{\left\vert AB\right\vert \text{\textperiodcentered
}\left\vert CD\right\vert }{\left\vert AD\right\vert \text{\textperiodcentered
}\left\vert BC\right\vert }.
\]
That is, we need to check that
\[
\left(  A:C:B:D\right)  =1+\left(  A:B:C:D\right)  .
\]
But by a previous problem this is the same as checking that%
\[
\left(  A':C':B':D'\right)  =1+\left(
A':B':C':D'\right)
\]
for the projection of the four points onto a line from a point $O$ on the
circle. But that is the same thing as showing that
\[
\frac{\left\vert A'C'\right\vert \text{\textperiodcentered
}\left\vert B'D'\right\vert }{\left\vert A'D^{\prime
}\right\vert \cdot\left\vert B'C^{\prime
}\right\vert }=1+\frac{\left\vert A'B'\right\vert
\cdot\left\vert C'D'\right\vert
}{\left\vert A'D'\right\vert \cdot \left\vert B'C'\right\vert }%
\]
which is the same thing as showing that%
\[
\left\vert A'C'\right\vert \text{\textperiodcentered
}\left\vert B'D'\right\vert =\left\vert A'D^{\prime
}\right\vert \cdot\left\vert B'C^{\prime
}\right\vert +\left\vert A'B'\right\vert
\cdot\left\vert C'D'\right\vert .
\]
\end{proof}


\begin{problem}
Check the last equality in the proof of Ptolemy's Theorem using high
school-algebra.
\begin{freeResponse}
\end{freeResponse}
\end{problem}

\begin{problem}
Use Ptolemy's Theorem to give another proof of the Pythagorean Theorem.
\begin{hint}
Let the four points in Ptolemy's Theorem form a rectangle.
\end{hint}
\begin{freeResponse}
\end{freeResponse}
\end{problem}


\begin{problem}
Prove the addition formula for sine:
\[
\sin(\alpha + \beta) = \sin(\alpha)\cos(\beta) +
\cos(\alpha)\sin(\beta)
\]
\begin{hint}
Consider this setup, 
\begin{image}
\begin{tikzpicture}[geometryDiagrams]
\coordinate (A) at ({2*cos(80)},{2*sin(80)});
\coordinate (B) at ({2*cos(180)},{2*sin(180)});
\coordinate (C) at ({2*cos(270)},{2*sin(270)});
\coordinate (D) at ({2*cos(0)},{2*sin(0)});

\draw (0,0) circle (2cm);
\draw (A) -- (B);
\draw (C) -- (D);
\draw (A) -- (D);
\draw (B) -- (C);

\draw[thin] (A) -- (C);
\draw[thin] (B) -- (D);

\tkzLabelPoints[above](A)
\tkzLabelPoints[left](B)
\tkzLabelPoints[below](C)
\tkzLabelPoints[right](D)

\tkzMarkAngle[size=0.7cm,thin](D,B,A)
\tkzLabelAngle[pos = 0.5](D,B,A){$\alpha$}

\tkzMarkAngle[arc=ll,size=0.8cm,thin](C,B,D)
\tkzLabelAngle[pos = 0.6](C,B,D){$\beta$}

\end{tikzpicture}
\end{image}
where $\bar{BD}$ is a unit diameter for the circle.
\end{hint}
\end{problem}


\begin{problem}
Prove the subtraction formula for sine:
\[
\sin(\alpha - \beta) = \sin(\alpha)\cos(\beta) -
\cos(\alpha)\sin(\beta)
\]
\begin{hint}
Consider this setup, 
\begin{image}
\begin{tikzpicture}[geometryDiagrams]
\coordinate (A) at ({2*cos(110)},{2*sin(110)});
\coordinate (B) at ({2*cos(180)},{2*sin(180)});
\coordinate (C) at ({2*cos(0)},{2*sin(0)});
\coordinate (D) at ({2*cos(65)},{2*sin(65)});

\draw (0,0) circle (2cm);
\draw (A) -- (B);
\draw (C) -- (D);
\draw (A) -- (D);
\draw (B) -- (C);

\draw[thin] (A) -- (C);
\draw[thin] (B) -- (D);

\tkzLabelPoints[above](A)
\tkzLabelPoints[left](B)
\tkzLabelPoints[right](C)
\tkzLabelPoints[above](D)

\tkzMarkAngle[size=0.7cm,thin](C,B,A)
\tkzLabelAngle[pos = 0.5](C,B,A){$\alpha$}

\tkzMarkAngle[arc=ll,size=1.1cm,thin](C,B,D)
\tkzLabelAngle[pos = 0.9](C,B,D){$\beta$}

\end{tikzpicture}
\end{image}
where $\bar{BC}$ is a unit diameter for the circle.
\end{hint}
\end{problem}

Both of the above formulas were very important to Ptolemy. You see,
Ptolemy wrote a book called the \textit{Almagest}. The
\textit{Almagest} was a book that contained all the current
information about the stars, including information on how a potential
reader could reproduce their observations and conclusions. A key
technical hurtle that needed to be resolved was the computation of
sine and cosine. The two formulas above, were the key in this endeavor.


%% Given any three non-collinear points in the Euclidean plane, there is
%% one and only one circle that passes through the three points. 

%% \begin{problem}
%% Show that, given any three non-collinear points in the Euclidean
%% plane, there is a unique circle passing through the three points.

%% \begin{hint}
%% Show that the center of the circle must be the intersection of
%% the perpendicular bisectors of any two of the sides of the triangle
%% whose vertices are the three given points.
%% \end{hint}

%% \begin{freeResponse}
%% \end{freeResponse}

%% \end{problem}

%% But how about four points in the plane, no three of which are
%% collinear?

%% \begin{problem}
%% \begin{enumerate}\hfil
%% \item Draw four points in the Euclidean plane, no $3$ of which are collinear, that cannot lie on a single circle.
%% \item Draw four points in the Euclidean plane that do lie on a single
%% circle.
%% \end{enumerate}
%% \end{problem}




\begin{problem}
Summarize the results from this section. In particular, indicate which
results follow from the others.
\begin{freeResponse}
\end{freeResponse}
\end{problem}









\end{document}
