\documentclass{ximera}

\title{The curvature $K$ becomes negative}

\begin{document}
\begin{abstract}
Here we start to focus on hyperbolic geometry.
\end{abstract}
\maketitle

\subsection*{The world sheet and the light cone}

We now turn to the case in which the radius $R$ of the Euclidean
$R$-sphere goes to infinity and beyond! Of course that doesn't make
any sense in $\left( \hat{x},\hat{y},\hat{z}\right) $-space but if we
look at the $R$-sphere in $\left( x,y,z\right) $-coordinates, it makes
perfect sense because there the equation of the $R$-sphere is%
\begin{equation}
K\left(  x^{2}+y^{2}\right)  +z^{2}=1 \label{81}%
\end{equation}
for $K=\frac{1}{R^{2}}$ so that $R$ going to infinity means that $K$
goes to $0$ and `beyond' simply means that $K$ becomes negative. We
have seen that all we need to have a geometry with lengths, angles,
areas and congruences is to have a smooth set and a dot-product
between vectors tangent to that set. Now if $K$ becomes negative, our
geometry becomes a hyperboloid of two sheets (obtained by rotating a
hyperbola in the $\left( x,z\right) $-plane with major axis the
$z$-axis around that axis). So that we have a connected universe, we
will only consider the `top' sheet (where $z>0$) as our
$K$-geometry. (In special relativity, this sheet might be called
something like the `world sheet.') If, instead of rotating a hyperbola
around the $z$-axis we rotate the asymptotes of the hyperbola around
the $z$-axis, we obtain a cone given by the equation%
\[
K\left(  x^{2}+y^{2}\right)  +z^{2}=0.
\]
(Again this might be called something like the `light cone.')

There is one potential problem we need to worry about when $K<0$, and it is
regarding the length of tangent vectors. Namely, our formulas for lengths
invove taking square roots of dot products of tangent vectors with themselves,
so those dot-products had better be positive (and only zero if the tangent
vector itself is the zero-vector.) Our $K$-dot product is given by the formula%
\[
\left(
\begin{array}
[c]{ccc}%
a & b & c
\end{array}
\right)  \cdot\left(
\begin{array}
[c]{ccc}%
1 & 0 & 0\\
0 & 1 & 0\\
0 & 0 & K^{-1}%
\end{array}
\right)  \cdot\left(
\begin{array}
[c]{c}%
a\\
b\\
c
\end{array}
\right)  .
\]
So, when $K<0$, it seems entirely possible that some tangent vector $V$ has
the property that $V\bullet_{K}V<0$. (Indeed that will always happen if $c$ is
sufficiently big and $x$ and $y$ are sufficiently small.
[MJG,241-242]

\subsection*{Non-zero tangent vectors in \textbf{HG} have positive length}

\begin{exercise}
Suppose that $V$ emanates from $\left(  0,0,0\right)  $ in $\left(
x,y,z\right)  $-space.
\begin{enumerate}
\item Show that $V\bullet_{K}V=0$ if and only if $V$ points in a direction of the
light cone.

\item Show that $V\bullet_{K}V<0$ if and only if $V$ points in a direction inside
the light cone.

%% Hint: Use that the (Euclidean) angle $\theta$ that the light cone makes with
%% the plane $z=0$ is given by taking any point $\left(  x,y,z\right)  $ on the
%% light cone with $z>0$ and computing%
%% \[
%% \mathrm{tan}\left(  \theta\right)  =\frac{z}{\sqrt{x^{2}+y^{2}}}=\left\vert
%% K\right\vert ^{1/2}.
%% \]
%% Compute the angle that $V$ makes with the plane $z=0$ in a similar way.
\end{enumerate}
Hint: Use the gradient. 
\end{exercise}

Now our world sheet lies \textit{inside} the light cone but tangent vectors to
it point \textit{outside} the light cone. That is what saves our $K$-dot
product, as we see in the next Lemma.

\begin{lemma}
Let $V=\left(  a,b,c\right)  $ denote a vector that is tangent
to our $K$-geometry, that is, to the set $\left(  \ref{81}\right)  $. Then%
\[
V\bullet_{K}V\geq0
\]
and $V\bullet_{K}V=0$ if and only if $V=0$.
\end{lemma}

\begin{proof}
If $c=0$, then the assertion of the Lemma is obviously true. So we can assume
$c\neq0$. Notice, since $V$ is assumed to be a tangent vector at $\left(
x,y,z\right)  $, this means that $\left(  x,y,z\right)  $ is not the North
Pole so that
\[
x^{2}+y^{2}>0.
\]


Next replacing $V$ with $\frac{1}{c}\left(  V\right)  $ just multiplies
$V\bullet_{K}V$ by $\frac{1}{c^{2}}$ so it suffices to consider the case in
which%
\[
V=\left(  a,b,1\right)
\]
and we must show that%
\[
\left(  a^{2}+b^{2}\right)  +K^{-1}>0.
\]
Since $V$ is tangent to our $K$-geometry at some point $\left(  x,y,z\right)
$, we know by Exercise \ref{82} that $\left(  x,y,z\right)  \bullet_{K}V=0$,
that is,%
\[
ax+by+\frac{z}{K}=0.
\]
On the other hand%
\[
K\left(  x^{2}+y^{2}\right)  +z^{2}=1.
\]
Substituting this becomes%
\[
K\left(  x^{2}+y^{2}\right)  +K^{2}\left(  ax+by\right)  ^{2}=1.
\]
On the other hand%
\begin{align*}
\left(  ay-bx\right)  ^{2}  &  \geq0\\
\left(  ay\right)  ^{2}+\left(  bx\right)  ^{2}  &  \geq2abxy
\end{align*}
so that%
\begin{gather*}
K\left(  x^{2}+y^{2}\right)  +K^{2}\left(  \left(  ax\right)  ^{2}+\left(
by\right)  ^{2}+\left(  ay\right)  ^{2}+\left(  bx\right)  ^{2}\right)
\geq1\\
K\left(  x^{2}+y^{2}\right)  +K^{2}\left(  a^{2}+b^{2}\right)  \left(
x^{2}+y^{2}\right)  \geq1\\
K^{-1}+\left(  a^{2}+b^{2}\right)  \geq\frac{1}{K^{2}\left(  x^{2}%
+y^{2}\right)  }.
\end{gather*}

\end{proof}


\end{document}
