\shortdescription{Here we dig deeper to understand our different coordinates.}
\activitytitle{Uniform coordinates for the two-dimensional geometries}


\subsection*{The two-dimensional geometries in $\left(x,y,z\right)
$-coordinates}


At the beginning of this book we changed the coordinates on Euclidean
three-space so that the equations for the sphere of radius $R$ became%
\begin{equation}
1=K\left(  x^{2}+y^{2}\right)  +z^{2} \label{29}%
\end{equation}
where%
\[
K=\frac{1}{R^{2}}.
\]
In these new $\left(  x,y,z\right)  $-coordinates, the set of points $\left(
x,y,z\right)  $ satisfying $\left(  \ref{29}\right)  $ when $K>0$ matched up
in $1-1$ fashion with the $R$-sphere%
\[
\left\{  \left(  \hat{x},\hat{y},\hat{z}\right)  \in\mathbb{R}^{3}:\hat{x}%
^{2}+\hat{y}^{2}+\hat{z}^{2}=R^{2}\right\}
\]
in the usual coordinates $\left(  \hat{x},\hat{y},\hat{z}\right)  $ of
$3$-dimensional Euclidean space.

If we have a curve $\left(  \hat{x}\left(  t\right)  ,\hat{y}\left(  t\right)
,\hat{z}\left(  t\right)  \right)  $ lying in the $R$-sphere in Euclidean
space, then for all $t\in\left[  b,e\right]  $,%
\[
\hat{x}\left(  t\right)  ^{2}+\hat{y}\left(  t\right)  ^{2}+\hat{z}\left(
t\right)  ^{2}=R^{2}.
\]
Differentiating both sides with respect to $t$ we obtain%
\[
2\hat{x}\left(  t\right)  \frac{d\hat{x}}{dt}+2\hat{y}\left(  t\right)
\frac{d\hat{y}}{dt}+2\hat{z}\left(  t\right)  \frac{d\hat{z}}{dt}=0
\]
which we can rewrite as%
\[
\left(  \hat{x}\left(  t\right)  ,\hat{y}\left(  t\right)  ,\hat{z}\left(
t\right)  \right)  \bullet\frac{d\hat{X}\left(  t\right)  }{dt}=0.
\]
[DS,105ff] Said another way, vectors $\hat{V}$ are tangent to the $R$-sphere
at $\hat{X}\left(  t\right)  $ if and only if%
\[
\hat{X}\left(  t\right)  \bullet\hat{V}=0.
\]
[DS,106,109]

Repeating the same calculation in $\left(  x,y,z\right)  $-coordinates, the
corresponding curve $\left(  x\left(  t\right)  ,y\left(  t\right)  ,z\left(
t\right)  \right)  $ lies in the set $\left(  \ref{29}\right)  $ so that%
\begin{align*}
1  &  =K\left(  x\left(  t\right)  ^{2}+y\left(  t\right)  ^{2}\right)
+z\left(  t\right)  ^{2}\\
0  &  =K\left(  2x\left(  t\right)  \frac{dx}{dt}+2y\left(  t\right)
\frac{dy}{dt}\right)  +2z\left(  t\right)  \frac{dz}{dt}.
\end{align*}
That is, a vector $V=\left(  a,b,c\right)  $ is tangent to the set $\left(
\ref{29}\right)  $ if and only if%
\begin{equation}
\left(  x\left(  t\right)  ,y\left(  t\right)  ,z\left(  t\right)  \right)
\cdot\left(
\begin{array}
[c]{ccc}%
2K & 0 & 0\\
0 & 2K & 0\\
0 & 0 & 2
\end{array}
\right)  \cdot V^{t}=0. \label{56}%
\end{equation}


\begin{exercise}
\label{82}For $K\neq0$, show that the condition $\left(  \ref{56}\right)  $ on
$V$ is exactly the same condition as%
\[
\left(  x\left(  t\right)  ,y\left(  t\right)  ,z\left(  t\right)  \right)
\bullet_{K}V=0.
\]

\end{exercise}

We will call the set of $\left(  x,y,z\right)  $ satisfying $\left(
\ref{29}\right)  $ $K$-geometry. Its tangent vectors at a point $\left(
x,y,z\right)  $ in the set are the vectors $V=\left(  a,b,c\right)  $ such
that%
\[
\left(  x,y,z\right)  \bullet_{K}V=0.
\]


If you get nervous using these weird coordinates to compute things
that are clearer in $\left( \hat{x},\hat{y},\hat{z}\right)
$-coordinates, just go through each construction with the special case
of $K=1$ first.  In that special case
\[
\left(  x,y,z\right)  =\left(  \hat{x},\hat{y},\hat{z}\right)
\]
and your calculations reduce to the usual ones on the unit sphere in ordinary
Euclidean $3$-space.

\subsection*{Rigid motions in $\left(  x,y,z\right)  $-coordinates}

We are now going to study $K$-geometry using only $\left(  x,y,z\right)
$-coordinates. If we have a curve $X\left(  t\right)  =\left(  x\left(
t\right)  ,y\left(  t\right)  ,z\left(  t\right)  \right)  $ on the surface
given in $K$-coordinates as%
\begin{equation}
1=K\left(  x^{2}+y^{2}\right)  +z^{2}, \label{86}%
\end{equation}
we have seen that we measure its length $L$ by the formula%
\begin{equation}
L=%
%TCIMACRO{\dint \nolimits_{b}^{e}}%
%BeginExpansion
{\displaystyle\int\nolimits_{b}^{e}}
%EndExpansion
l\left(  t\right)  dt \label{60}%
\end{equation}
where
\begin{equation}
l\left(  t\right)  ^{2}=\frac{dX}{dt}\bullet_{K}\frac{dX}{dt} \label{61}%
\end{equation}
and that we measure angles $\theta$ between tangent vectors $V_{1}$ and
$V_{2}$ at a point on the surface by the formula%
\[
\theta=\arccos\left(  \frac{V_{1}\bullet_{K}V_{2}}{\left\vert
V_{1}\right\vert _{K}\cdot\left\vert V_{2}\right\vert
_{K}}\right)
\]
where%
\[
\left\vert V\right\vert _{K}^{2}=V\bullet_{K}V.
\]


We now want to explore the condition that a transformation%
\[
\left(  \underline{x},\underline{y},\underline{z}\right)  =\left(
x,y,z\right)  \cdot M
\]
take the surface $\left(  \ref{86}\right)  $ to itself and preserve the length
of any curve $\left(  x\left(  t\right)  ,y\left(  t\right)  ,z\left(
t\right)  \right)  $ lying on the surface. Rewriting the transformation as%
\[
\left(  \underline{X}\right)  =\left(  X\right)  \cdot M
\]
the formulas $\left(  \ref{60}\right)  $ and $\left(  \ref{61}\right)  $ show
that all we have to worry about is that%
\[
\frac{d\underline{X}}{dt}\bullet_{K}\frac{d\underline{X}}{dt}=\frac{dX}%
{dt}\bullet_{K}\frac{dX}{dt}%
\]
for all values $t$ of the parameter of the curve. But%
\[
\left(  \frac{d\underline{X}}{dt}\right)  =\left(  \frac{dX}{dt}\right)  \cdot
M
\]
by the product rule since $M$ is a constant matrix. So the transformation
given by the matrix $\hat{M}$ will preserve the length of any path and will
preserve the measure of any angle if%
\begin{equation}
\left(  \frac{dX}{dt}\right)  \cdot M\cdot\left(
\begin{array}
[c]{ccc}%
1 & 0 & 0\\
0 & 1 & 0\\
0 & 0 & K^{-1}%
\end{array}
\right)  \cdot\left(  \left(  \frac{dX}{dt}\right)  \cdot M\right)
^{t}=\left(  \frac{dX}{dt}\right)  \cdot\left(
\begin{array}
[c]{ccc}%
1 & 0 & 0\\
0 & 1 & 0\\
0 & 0 & K^{-1}%
\end{array}
\right)  \cdot\left(  \frac{dX}{dt}\right)  ^{t}. \label{89}%
\end{equation}


\begin{exercise}\hfil
\label{87}
\begin{enumerate}
\item  Show that this last equality is always true if%
\begin{equation}
M\cdot\left(
\begin{array}
[c]{ccc}%
1 & 0 & 0\\
0 & 1 & 0\\
0 & 0 & K^{-1}%
\end{array}
\right)  \cdot M^{t}=\left(
\begin{array}
[c]{ccc}%
1 & 0 & 0\\
0 & 1 & 0\\
0 & 0 & K^{-1}%
\end{array}
\right)  . \label{new89}%
\end{equation}

\item Show that, if $M$ satisfies the identity \ref{new89},then the
transformation $\left(  \underline{x},\underline{y},\underline{z}\right)
=\left(  x,y,z\right)  \cdot M$ takes the set of points $\left(  x,y,z\right)
$ such that%
\[
1=K\left(  x^{2}+y^{2}\right)  +z^{2},
\]
to the set of points $\left(  \underline{x},\underline{y},\underline
{z}\right)  $ such that%
\[
1=K\left(  \underline{x}^{2}+\underline{y}^{2}\right)  +\underline{z}^{2}.
\]
That is, $M$ gives a $1-1$, onto mapping of $K$-geometry to itself.

Hint: For $K\neq0$, write the equation $1=K\left(  \underline{x}%
^{2}+\underline{y}^{2}\right)  +\underline{z}^{2}$ in matrix notation as%
\[
\left(
\begin{array}
[c]{ccc}%
\underline{x} & \underline{y} & \underline{z}%
\end{array}
\right)  \cdot\left(
\begin{array}
[c]{ccc}%
1 & 0 & 0\\
0 & 1 & 0\\
0 & 0 & K^{-1}%
\end{array}
\right)  \cdot\left(
\begin{array}
[c]{c}%
\underline{x}\\
\underline{y}\\
\underline{z}%
\end{array}
\right)  =\frac{1}{K}.
\]
\end{enumerate}
\end{exercise}

\begin{definition}
\label{88}A $3\times3$ matrix $M$ is called $K$-orthogonal if
\[
M\cdot\left(
\begin{array}
[c]{ccc}%
1 & 0 & 0\\
0 & 1 & 0\\
0 & 0 & K^{-1}%
\end{array}
\right)  \cdot M^{t}=\left(
\begin{array}
[c]{ccc}%
1 & 0 & 0\\
0 & 1 & 0\\
0 & 0 & K^{-1}%
\end{array}
\right)  .
\]

\end{definition}

\begin{definition}
A $K$-distance-preserving transformation of $K$-geometry is called a
$K$\textbf{-rigid motion} or a $K$\textbf{-congruence}.
\end{definition}

So $K$-orthogonal matrices give $K$-rigid motions.

\begin{exercise}
For $K\neq0$, show that the set of $K$-orthogonal matrices $M$ form a group.
That is, show that
\begin{enumerate}
\item the product of two $K$-orthogonal matrices is $K$-orthogonal,

\item the identity matrix is $K$-orthogonal,

\item the inverse matrix $M^{-1}$ of a $K$-orthogonal matrix $M$ is $K$-orthogonal.
Hint: Write%
\[
M\cdot M^{-1}=I=M\cdot\left(
\begin{array}
[c]{ccc}%
1 & 0 & 0\\
0 & 1 & 0\\
0 & 0 & K^{-1}%
\end{array}
\right)  \cdot M^{t}\cdot\left(
\begin{array}
[c]{ccc}%
1 & 0 & 0\\
0 & 1 & 0\\
0 & 0 & K
\end{array}
\right)
\]
and use matrix multiplication to reduce to showing that
\[
\left(
\begin{array}
[c]{ccc}%
1 & 0 & 0\\
0 & 1 & 0\\
0 & 0 & K^{-1}%
\end{array}
\right)  \cdot M^{t}\cdot\left(
\begin{array}
[c]{ccc}%
1 & 0 & 0\\
0 & 1 & 0\\
0 & 0 & K
\end{array}
\right)
\]
is $K$-orthogonal.
\end{enumerate}
\end{exercise}

\subsection*{Why use $K$-coordinates?}

We saw that we could measure the usual Euclidean lengths of curves $\hat
{X}\left(  t\right)  $ on the usual Euclidean $R$-sphere just in terms of the
formulas $X\left(  t\right)  $ for their paths in $\left(  x,y,z\right)
$-coordinates using the $K$-dot product, since lengths depended only on
lengths of tangent vectors and%
\[
\frac{d\hat{X}\left(  t\right)  }{dt}\bullet\frac{d\hat{X}\left(  t\right)
}{dt}=\frac{dX\left(  t\right)  }{dt}\bullet_{K}\frac{dX\left(  t\right)
}{dt}%
\]
where%
\[
\frac{dX\left(  t\right)  }{dt}\bullet_{K}\frac{dX\left(  t\right)  }%
{dt}=\left(  \frac{dX\left(  t\right)  }{dt}\right)  \cdot\left(
\begin{array}
[c]{ccc}%
1 & 0 & 0\\
0 & 1 & 0\\
0 & 0 & K^{-1}%
\end{array}
\right)  \cdot\left(  \frac{dX\left(  t\right)  }{dt}\right)  ^{t}.
\]
In other words, the usual geometry of the sphere of radius $R$ is simply the
geometry of the set $\left(  \ref{29}\right)  $ with $K=1/R^{2}$ and with
lengths (and areas) given by the $K$-dot product. Said another way, we can do
all of spherical geometry in $\left(  x,y,z\right)  $-coordinates. All we need
is the set $\left(  \ref{29}\right)  $ and the $K$-dot product. But the set
$\left(  \ref{29}\right)  $ continues to exist even if $K=0$ or $K<0$, and the
$K$-dot product formula continues to make sense even if $K<0$. In short we
have the following table:
\begin{equation}%
{\renewcommand{\arraystretch}{1.6}
\begin{tabular}
[c]{c|c|c}%
Spherical ($K>0$) & Euclidean ($K=0$) & Hyperbolic ($K<0$)\\\hline
$\hat{x}^{2}+\hat{y}^{2}+\hat{z}^{2}=R^{2}$ &  & \\\hline
$\hat{V}\bullet\hat{V}$ &  & \\\hline
$1=K\left(  x^{2}+y^{2}\right)  +z^{2}$ & $1=K\left(  x^{2}+y^{2}\right)
+z^{2}$ & $1=K\left(  x^{2}+y^{2}\right)  +z^{2}$\\ \hline
$V_{1}\bullet_{K}V_{2}$ &  & $V_{1}\bullet_{K}V_{2}$%
\end{tabular}}
\ \ \ \ \label{66}%
\end{equation}
This table tells us that `there is something else out there,' that is, some
other type of two-dimensional geometry beyond plane geometry and spherical
geometry. But the gap in the bottom row of the table is a bit disturbing. If
we can't express the usual dot-product in plane geometry as the $K$-dot
product for $K=0$, we can't pass smoothly from spherical through plane
geometry to hyperbolic geometry using $\left(  x,y,z\right)  $-coordinates. We
now examine two ways to produce coordinates uniformly for spherical, plane and
hyperbolic geometry that overcome this difficulty.
